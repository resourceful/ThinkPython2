% LaTeX source for ``Think Python: How to Think Like a Computer Scientist''
% Copyright (c)  2015  Allen B. Downey.

% License: Creative Commons Attribution-NonCommercial 3.0 Unported License.
% http://creativecommons.org/licenses/by-nc/3.0/
%

%\documentclass[10pt,b5paper]{book}
\documentclass[10pt]{book}
\usepackage[width=5.5in,height=8.5in,hmarginratio=3:2,vmarginratio=1:1]{geometry}

% for some of these packages, you might have to install
% texlive-latex-extra (in Ubuntu)

\usepackage[T1]{fontenc}
\usepackage{textcomp}
\usepackage{mathpazo}
\usepackage{url}
\usepackage{fancyhdr}
\usepackage{graphicx}
\usepackage{amsmath}
\usepackage{amsthm}
%\usepackage{amssymb}
\usepackage{exercise}                        % texlive-latex-extra
\usepackage{makeidx}
\usepackage{setspace}
\usepackage{hevea}                           
\usepackage{upquote}
\usepackage{appendix}
\usepackage[bookmarks]{hyperref}
\usepackage{kotex}

\title{Think Python}
\author{Allen B. Downey}
\newcommand{\thetitle}{Think Python: How to Think Like a Computer Scientistn}
\newcommand{\theversion}{2nd Edition, Version 2.2.20}
\newcommand{\thedate}{}

% these styles get translated in CSS for the HTML version
\newstyle{a:link}{color:black;}
\newstyle{p+p}{margin-top:1em;margin-bottom:1em}
\newstyle{img}{border:0px}

% change the arrows
\setlinkstext
  {\imgsrc[ALT="Previous"]{back.png}}
  {\imgsrc[ALT="Up"]{up.png}}
  {\imgsrc[ALT="Next"]{next.png}}

\makeindex

\newif\ifplastex
\plastexfalse

\begin{document}

\frontmatter

% PLASTEX ONLY
\ifplastex
    \usepackage{localdef}
    \maketitle

\newcount\anchorcnt
\newcommand*{\Anchor}[1]{%
  \@bsphack%
    \Hy@GlobalStepCount\anchorcnt%
    \edef\@currentHref{anchor.\the\anchorcnt}% 
    \Hy@raisedlink{\hyper@anchorstart{\@currentHref}\hyper@anchorend}% 
    \M@gettitle{}\label{#1}% 
    \@esphack%
}


\else
% skip the following for plastex

%\newtheorem{exercise}{Exercise}[chapter]
\newtheorem{exercise}{문제}[chapter]

% LATEXONLY

\input{latexonly}

\begin{latexonly}

\renewcommand{\blankpage}{\thispagestyle{empty} \quad \newpage}

%\blankpage
%\blankpage

% TITLE PAGES FOR LATEX VERSION

%-half title--------------------------------------------------
\thispagestyle{empty}

\begin{flushright}
\vspace*{2.0in}

\begin{spacing}{3}
{\huge Think Python}\\
{\Large How to Think Like a Computer Scientist}
\end{spacing}

\vspace{0.25in}

\theversion

\thedate

\vfill

\end{flushright}

%--verso------------------------------------------------------

\blankpage
\blankpage
%\clearemptydoublepage
%\pagebreak
%\thispagestyle{empty}
%\vspace*{6in}

%--title page--------------------------------------------------
\pagebreak
\thispagestyle{empty}

\begin{flushright}
\vspace*{2.0in}

\begin{spacing}{3}
{\huge Think Python}\\
{\Large How to Think Like a Computer Scientist}
\end{spacing}

\vspace{0.25in}

\theversion

\thedate

\vspace{1in}


{\Large
Allen Downey\\
}


\vspace{0.5in}

{\Large Green Tea Press}

{\small Needham, Massachusetts}

%\includegraphics[width=1in]{figs/logo1.pdf}
\vfill

\end{flushright}


%--copyright--------------------------------------------------
\pagebreak
\thispagestyle{empty}

{\small
Copyright \copyright ~2015 Allen Downey.


\vspace{0.2in}

\begin{flushleft}
Green Tea Press       \\
9 Washburn Ave        \\
Needham MA 02492
\end{flushleft}

Permission is granted to copy, distribute, and/or modify this document
under the terms of the Creative Commons Attribution-NonCommercial 3.0 Unported
License, which is available at \url{http://creativecommons.org/licenses/by-nc/3.0/}.

The original form of this book is \LaTeX\ source code.  Compiling this
\LaTeX\ source has the effect of generating a device-independent
representation of a textbook, which can be converted to other formats
and printed.

The \LaTeX\ source for this book is available from
\url{http://www.thinkpython2.com}

\vspace{0.2in}

} % end small

\end{latexonly}


% HTMLONLY

\begin{htmlonly}

% TITLE PAGE FOR HTML VERSION

{\Large \thetitle}

{\large Allen B. Downey}

\theversion

\thedate

\setcounter{chapter}{-1}

\end{htmlonly}

\fi
% END OF THE PART WE SKIP FOR PLASTEX


\chapter{Preface}

\section*{The strange history of this book}

In January 1999 I was preparing to teach an introductory programming
class in Java.  I had taught it three times and I was getting
frustrated.  The failure rate in the class was too high and, even for
students who succeeded, the overall level of achievement was too low.

One of the problems I saw was the books.  
They were too big, with too much unnecessary detail about Java, and
not enough high-level guidance about how to program.  And they all
suffered from the trap door effect: they would start out easy,
proceed gradually, and then somewhere around Chapter 5 the bottom would
fall out.  The students would get too much new material, too fast,
and I would spend the rest of the semester picking up the pieces.

Two weeks before the first day of classes, I decided to write my
own book.  My goals were:

\begin{itemize}

\item Keep it short.  It is better for students to read 10 pages
than not read 50 pages.

\item Be careful with vocabulary.  I tried to minimize jargon
and define each term at first use.

\item Build gradually. To avoid trap doors, I took the most difficult
topics and split them into a series of small steps. 

\item Focus on programming, not the programming language.  I included
the minimum useful subset of Java and left out the rest.

\end{itemize}

I needed a title, so on a whim I chose {\em How to Think Like
a Computer Scientist}.

My first version was rough, but it worked.  Students did the reading,
and they understood enough that I could spend class time on the hard
topics, the interesting topics and (most important) letting the
students practice.

I released the book under the GNU Free Documentation License,
which allows users to copy, modify, and distribute the book.
\index{GNU Free Documentation License}
\index{Free Documentation License, GNU}

What happened next is the cool part.  Jeff Elkner, a high school
teacher in Virginia, adopted my book and translated it into
Python.  He sent me a copy of his translation, and I had the
unusual experience of learning Python by reading my own book.
As Green Tea Press, I published the first Python version in 2001.
\index{Elkner, Jeff}

In 2003 I started teaching at Olin College and I got to teach
Python for the first time.  The contrast with Java was striking.
Students struggled less, learned more, worked on more interesting
projects, and generally had a lot more fun.
\index{Olin College}

Since then I've continued to develop the book,
correcting errors, improving some of the examples and
adding material, especially exercises.

The result is this book, now with the less grandiose title
{\em Think Python}.  Some of the changes are:

\begin{itemize}

\item I added a section about debugging at the end of each chapter.
  These sections present general techniques for finding and avoiding
  bugs, and warnings about Python pitfalls.

\item I added more exercises, ranging from short tests of
  understanding to a few substantial projects.  Most exercises
  include a link to my solution.

\item I added a series of case studies---longer examples with
  exercises, solutions, and discussion.
  
\item I expanded the discussion of program development plans
  and basic design patterns.

\item I added appendices about debugging and analysis of algorithms.

\end{itemize}

The second edition of {\em Think Python} has these new features:

\begin{itemize}

\item The book and all supporting code have been updated to Python 3.

\item I added a few sections, and more details on the web, to help
beginners get started running Python in a browser, so you don't have
to deal with installing Python until you want to.

\item For Chapter~\ref{turtle} I switched from my own turtle graphics
  package, called Swampy, to a more standard Python module, {\tt
    turtle}, which is easier to install and more powerful.

\item I added a new chapter called ``The Goodies'', which introduces
some additional Python features that are not strictly necessary, but
sometimes handy.

\end{itemize}

I hope you enjoy working with this book, and that it helps
you learn to program and think like
a computer scientist, at least a little bit.


Allen B. Downey \\

Olin College \\


\section*{Acknowledgments}

Many thanks to Jeff Elkner, who
translated my Java book into Python, which got this project
started and introduced me to what has turned out to be my
favorite language.
\index{Elkner, Jeff}

Thanks also to Chris Meyers, who contributed several sections
to {\em How to Think Like a Computer Scientist}.
\index{Meyers, Chris}

Thanks to the Free Software Foundation for developing
the GNU Free Documentation License, which helped make
my collaboration with Jeff and Chris possible, and Creative
Commons for the license I am using now.
\index{GNU Free Documentation License}
\index{Free Documentation License, GNU}
\index{Creative Commons}

Thanks to the editors at Lulu who worked on
{\em How to Think Like a Computer Scientist}.

Thanks to the editors at O'Reilly Media who worked on
{\em Think Python}.

Thanks to all the students who worked with earlier
versions of this book and all the contributors (listed
below) who sent in corrections and suggestions.


\section*{Contributor List}

\index{contributors}
More than 100 sharp-eyed and thoughtful readers have sent in
suggestions and corrections over the past few years.  Their
contributions, and enthusiasm for this project, have been a
huge help.

If you have a suggestion or correction, please send email to 
{\tt feedback@thinkpython.com}.  If I make a change based on your
feedback, I will add you to the contributor list
(unless you ask to be omitted).

If you include at least part of the sentence the
error appears in, that makes it easy for me to search.  Page and
section numbers are fine, too, but not quite as easy to work with.
Thanks!

\begin{itemize}

\small
\item Lloyd Hugh Allen sent in a correction to Section 8.4.

\item Yvon Boulianne sent in a correction of a semantic error in
Chapter 5.

\item Fred Bremmer submitted a correction in Section 2.1.

\item Jonah Cohen wrote the Perl scripts to convert the
LaTeX source for this book into beautiful HTML.

\item Michael Conlon sent in a grammar correction in Chapter 2
and an improvement in style in Chapter 1, and he initiated discussion
on the technical aspects of interpreters.

\item Benoit Girard sent in a
correction to a humorous mistake in Section 5.6.

\item Courtney Gleason and Katherine Smith wrote {\tt horsebet.py},
which was used as a case study in an earlier version of the book.  Their
program can now be found on the website.

\item Lee Harr submitted more corrections than we have room to list
here, and indeed he should be listed as one of the principal editors
of the text.

\item James Kaylin is a student using the text. He has submitted
numerous corrections.

\item David Kershaw fixed the broken {\tt catTwice} function in Section
3.10.

\item Eddie Lam has sent in numerous corrections to Chapters 
1, 2, and 3.
He also fixed the Makefile so that it creates an index the first time it is
run and helped us set up a versioning scheme.  

\item Man-Yong Lee sent in a correction to the example code in
Section 2.4.  

\item David Mayo pointed out that the word ``unconsciously"
in Chapter 1 needed
to be changed to ``subconsciously".

\item Chris McAloon sent in several corrections to Sections 3.9 and
3.10.

\item Matthew J. Moelter has been a long-time contributor who sent
in numerous corrections and suggestions to the book.  

\item Simon Dicon Montford reported a missing function definition and
several typos in Chapter 3.  He also found errors in the {\tt increment}
function in Chapter 13.

\item John Ouzts corrected the definition of ``return value"
in Chapter 3.

\item Kevin Parks sent in valuable comments and suggestions as to how
to improve the distribution of the book.

\item David Pool sent in a typo in the glossary of Chapter 1, as well
as kind words of encouragement.

\item Michael Schmitt sent in a correction to the chapter on files
and exceptions.

\item Robin Shaw pointed out an error in Section 13.1, where the
printTime function was used in an example without being defined.

\item Paul Sleigh found an error in Chapter 7 and a bug in Jonah Cohen's
Perl script that generates HTML from LaTeX.

\item Craig T. Snydal is testing the text in a course at Drew
University.  He has contributed several valuable suggestions and corrections.

\item Ian Thomas and his students are using the text in a programming
course.  They are the first ones to test the chapters in the latter half
of the book, and they have made numerous corrections and suggestions.

\item Keith Verheyden sent in a correction in Chapter 3.

\item Peter Winstanley let us know about a longstanding error in
our Latin in Chapter 3.

\item Chris Wrobel made corrections to the code in the chapter on
file I/O and exceptions. 

\item Moshe Zadka has made invaluable contributions to this project.
In addition to writing the first draft of the chapter on Dictionaries, he
provided continual guidance in the early stages of the book.

\item Christoph Zwerschke sent several corrections and
pedagogic suggestions, and explained the difference between {\em gleich}
and {\em selbe}.

\item James Mayer sent us a whole slew of spelling and
typographical errors, including two in the contributor list.

\item Hayden McAfee caught a potentially confusing inconsistency
between two examples.

\item Angel Arnal is part of an international team of translators
working on the Spanish version of the text.  He has also found several
errors in the English version.

\item Tauhidul Hoque and Lex Berezhny created the illustrations
in Chapter 1 and improved many of the other illustrations.

\item Dr. Michele Alzetta caught an error in Chapter 8 and sent
some interesting pedagogic comments and suggestions about Fibonacci
and Old Maid.

\item Andy Mitchell caught a typo in Chapter 1 and a broken example
in Chapter 2.

\item Kalin Harvey suggested a clarification in Chapter 7 and
caught some typos.

\item Christopher P. Smith caught several typos and helped us
update the book for Python 2.2.

\item David Hutchins caught a typo in the Foreword.

\item Gregor Lingl is teaching Python at a high school in Vienna,
Austria.  He is working on a German translation of the book,
and he caught a couple of bad errors in Chapter 5.

\item Julie Peters caught a typo in the Preface.

\item Florin Oprina sent in an improvement in {\tt makeTime},
a correction in {\tt printTime}, and a nice typo.

\item D.~J.~Webre suggested a clarification in Chapter 3.

\item Ken found a fistful of errors in Chapters 8, 9 and 11.

\item Ivo Wever caught a typo in Chapter 5 and suggested a clarification
in Chapter 3.

\item Curtis Yanko suggested a clarification in Chapter 2.

\item Ben Logan sent in a number of typos and problems with translating
the book into HTML.

\item Jason Armstrong saw the missing word in Chapter 2.

\item Louis Cordier noticed a spot in Chapter 16 where the code
didn't match the text.

\item Brian Cain suggested several clarifications in Chapters 2 and 3.

\item Rob Black sent in a passel of corrections, including some
changes for Python 2.2.

\item Jean-Philippe Rey at Ecole Centrale
Paris sent a number of patches, including some updates for Python 2.2
and other thoughtful improvements.

\item Jason Mader at George Washington University made a number
of useful suggestions and corrections.

\item Jan Gundtofte-Bruun reminded us that ``a error'' is an error.

\item Abel David and Alexis Dinno reminded us that the plural of
``matrix'' is ``matrices'', not ``matrixes''.  This error was in the
book for years, but two readers with the same initials reported it on
the same day.  Weird.

\item Charles Thayer encouraged us to get rid of the semi-colons
we had put at the ends of some statements and to clean up our
use of ``argument'' and ``parameter''.

\item Roger Sperberg pointed out a twisted piece of logic in Chapter 3.

\item Sam Bull pointed out a confusing paragraph in Chapter 2.

\item Andrew Cheung pointed out two instances of ``use before def''.

\item C. Corey Capel spotted the missing word in the Third Theorem
of Debugging and a typo in Chapter 4.

\item Alessandra helped clear up some Turtle confusion.

\item Wim Champagne found a brain-o in a dictionary example.

\item Douglas Wright pointed out a problem with floor division in
{\tt arc}.

\item Jared Spindor found some jetsam at the end of a sentence.

\item Lin Peiheng sent a number of very helpful suggestions.

\item Ray Hagtvedt sent in two errors and a not-quite-error.

\item Torsten H\"{u}bsch pointed out an inconsistency in Swampy.

\item Inga Petuhhov corrected an example in Chapter 14.

\item Arne Babenhauserheide sent several helpful corrections.

\item Mark E. Casida is is good at spotting repeated words.

\item Scott Tyler filled in a that was missing.  And then sent in
a heap of corrections.

\item Gordon Shephard sent in several corrections, all in separate
emails.

\item Andrew Turner {\tt spot}ted an error in Chapter 8.

\item Adam Hobart fixed a problem with floor division in {\tt arc}.

\item Daryl Hammond and Sarah Zimmerman pointed out that I served
up {\tt math.pi} too early.  And Zim spotted a typo.

\item George Sass found a bug in a Debugging section.

\item Brian Bingham suggested Exercise~\ref{exrotatepairs}.

\item Leah Engelbert-Fenton pointed out that I used {\tt tuple}
as a variable name, contrary to my own advice.  And then found
a bunch of typos and a ``use before def''.

\item Joe Funke spotted a typo.

\item Chao-chao Chen found an inconsistency in the Fibonacci example.

\item Jeff Paine knows the difference between space and spam.

\item Lubos Pintes sent in a typo.

\item Gregg Lind and Abigail Heithoff suggested Exercise~\ref{checksum}.

\item Max Hailperin has sent in a number of corrections and
  suggestions.  Max is one of the authors of the extraordinary {\em
    Concrete Abstractions}, which you might want to read when you are
  done with this book.

\item Chotipat Pornavalai found an error in an error message.

\item Stanislaw Antol sent a list of very helpful suggestions.

\item Eric Pashman sent a number of corrections for Chapters 4--11.

\item Miguel Azevedo found some typos.

\item Jianhua Liu sent in a long list of corrections.

\item Nick King found a missing word.

\item Martin Zuther sent a long list of suggestions.

\item Adam Zimmerman found an inconsistency in my instance
of an ``instance'' and several other errors.

\item Ratnakar Tiwari suggested a footnote explaining degenerate
triangles.

\item Anurag Goel suggested another solution for \verb"is_abecedarian"
and sent some additional corrections.  And he knows how to
spell Jane Austen.

\item Kelli Kratzer spotted one of the typos.

\item Mark Griffiths pointed out a confusing example in Chapter 3.

\item Roydan Ongie found an error in my Newton's method.

\item Patryk Wolowiec helped me with a problem in the HTML version.

\item Mark Chonofsky told me about a new keyword in Python 3.

\item Russell Coleman helped me with my geometry.

\item Nam Nguyen found a typo and pointed out that I used the Decorator
pattern but didn't mention it by name.

\item St\'{e}phane Morin sent in several corrections and suggestions.

\item Paul Stoop corrected a typo in \verb+uses_only+.

\item Eric Bronner pointed out a confusion in the discussion of the
order of operations.

\item Alexandros Gezerlis set a new standard for the number and
quality of suggestions he submitted.  We are deeply grateful!

\item Gray Thomas knows his right from his left.

\item Giovanni Escobar Sosa sent a long list of corrections and
suggestions.

\item Daniel Neilson corrected an error about the order of operations.

\item Will McGinnis pointed out that {\tt polyline} was defined
differently in two places.

\item Frank Hecker pointed out an exercise that was under-specified, and
some broken links.

\item Animesh B helped me clean up a confusing example.

\item Martin Caspersen found two round-off errors.

\item Gregor Ulm sent several corrections and suggestions.

\item Dimitrios Tsirigkas suggested I clarify an exercise.

\item Carlos Tafur sent a page of corrections and suggestions.

\item Martin Nordsletten found a bug in an exercise solution.

\item Sven Hoexter pointed out that a variable named {\tt input}
shadows a build-in function.

\item Stephen Gregory pointed out the problem with {\tt cmp}
in Python 3.

\item Ishwar Bhat corrected my statement of Fermat's last theorem.

\item Andrea Zanella translated the book into Italian, and sent a
number of corrections along the way.

\item Many, many thanks to Melissa Lewis and Luciano Ramalho for
  excellent comments and suggestions on the second edition.

\item Thanks to Harry Percival from PythonAnywhere for his help
getting people started running Python in a browser.

\item Xavier Van Aubel made several useful corrections in the second
edition.

\item William Murray corrected my definition of floor division.

% ENDCONTRIB

In addition, people who spotted typos or made corrections include
Czeslaw Czapla,
Richard Fursa, Brian McGhie, Lokesh Kumar Makani, Matthew Shultz, Viet
Le, Victor Simeone, Lars O.D. Christensen, Swarup Sahoo, Alix Etienne,
Kuang He, Wei Huang, Karen Barber, and Eric Ransom.




\end{itemize}

\normalsize
\clearemptydoublepage

% TABLE OF CONTENTS
\begin{latexonly}

\tableofcontents

\clearemptydoublepage

\end{latexonly}

% START THE BOOK
\mainmatter

\chapter{프로그래밍의 정도}
%The way of the program

이 책의 목적은 컴퓨터 과학자처럼 생각하는 방법을 알려주는 것이다.  그
과정에는 수학과 공학 그리고 자연 과학의 몇 가지 대표적인 요소들을
통합한다.  수학자들처럼 컴퓨터 과학자들도 아이디어(구체적으로는
연산들)를 표현하기 위해 형식을 갖춘 언어를 사용한다.  공학자들처럼
무엇인가를 설계하고 요소들을 조립하여 시스템을 만들기도 하며 대안들을
평가하여 타협점을 찾는다.  과학자들처럼 복잡한 시스템들의 동작을
관찰하고 가설을 세워서 예측치를 실험한다.  \index{problem solving}


컴퓨터 과학자에게 있어서 가장 중요한 기술은 {\bf 문제 해결}이다.  문제
해결이라는 것은 문제를 정의한 후 해답을 찾기 위해 창의적으로 생각하여
얻은 해답을 정확하고 명료하게 표현하는 것이다.  나중에 알게 되겠지만,
프로그래밍을 배우는 과정은 문제 해결 능력을 키우는 최고의 기회이다.
그래서, 이 장의 제목을 ``프로그래밍의 정도''라 지었다.


어떤 수준에서는 프로그래밍을 배우게 될 것이다.  그 자체로도 매우 유용한
기술이다.  또 다른 수준에서는 프로그래밍이 수단이 될 것이다.  계속
따라오다 보면 프로그래밍이 수단이 된다는 것이 명확해질 것이다.


\section{프로그램이란 무엇인가?}
%What is a program?

{\bf 프로그램}이란 명령들의 순서로서 어떤 방식으로 연산할지
정한다.  연산은 수식을 풀어내는 것이나 다항식의 해를 구하는 것처럼
수학적일 수도 있지만 문서의 어떤 글을 찾거나 교체한다거나 이미지 처리나
동영상을 재생하는 것처럼 상징적인 연산일 수도 있다. \index{program} 


세부적인 것은 언어마다 다르겠지만 몇 가지 기본적인 명령들은 거의 모든
언어에 나타난다:


\begin{description}
\item[입력(input):] 키보드나 파일, 네트워크나 다른 장치로부터 데이터를
  받는다.
\item[출력(output):] 데이터를 화면에 표시하거나 파일에 저장 또는
  네트워크로 전송 등을 한다.
\item[계산(math):] 덧셈과 곱셉과 같은 기본적인 수학적 연산을 한다.
\item[조건부 실행(conditional execution):] 조건을 확인하여 적절한
  코드를 실행한다.
\item[반복(repetition):] 대체적으로 약간의 변형을 포함하는 어떤 행동을
  반복적으로 수행한다.
\end{description}

믿거나 말거나, 지금 열거한 내용이 거의 전부이다. 지금까지 사용했었던
모든 프로그램이 얼마나 복잡하든 이러한 명령들로 이루어져있다.  그렇기
때문에 프로그래밍이란 것은 크고 복잡한 어떤 작업을 이런 기본 명령들로
동작 가능한 단위들로 작게 자르는 일이라고 보면 된다.



\section{Python 실행하기}
%Running Python}

Python을 시작하는데 있어서 도전거리 중의 하나는 Python과 관련
소프트웨어를 컴퓨터에 설치해야 할 수도 있다는 것이다.  현재 사용하는
운영체제와 익숙하다면, 특히 명령줄 기반의 인터페이스와 익숙하다면, Python을 설치하는데 전혀 어려움이 없을 것이다. 하지만, 초보자라면 시스템 관리와 프로그래밍을 동시에 배워야 한다는 것이 고통스러울 수 있다. 
\index{running Python}
\index{Python!running}

이 문제를 해결하기 위해서 Python을 브라우저에서 시작하기를 추천한다. 나중에는 Python과 익숙해졌을 즈음에 컴퓨터에 Python을 설치하는 것을 권하도록 하겠다. 
\index{Python in a browser}

Python을 실행할 수 있는 여러 웹 페이지들이 있다. 이미 자주 사용해왔던 사이트가 있다면 그것을 사용해도 된다. 만약 없었다면 PhythonAnywher라는 곳을 추천한다. 이 URL(\url{http://tinyurl.com/thinkpython2e})에 시작하기 위한 상세한 정보를 설명해 놓았다.
\index{PythonAnywhere}

Python에는 두 개의 버전이 있다. Python 2와 Python 3 이다. 둘은 매우 유사하여 하나를 배우면 다른 하나로 이동하는 것은 쉽다. 사실 초보자 입장에서는 별반 차이가 없다. 이 책은 Phython 3로 작성되어 있지만, Python 2에 대하여 메모를 남겨 놓기는 했다. 
\index{Python 2}

Pthon {\bf 인터프리터(interpreter)}는 Python 코드를 읽고 실행하는
프로그램이다.  환경에 따라 인터프리터를 아이콘을 클릭하거나 명령줄에
{\tt phtyon}을 입력해서 시작할 수도 있다.  시작하면 다음과 같은 출력을
보게 될 것이다. 
\index{interpreter}

\begin{verbatim}
Python 3.4.0 (default, Jun 19 2015, 14:20:21) 
[GCC 4.8.2] on linux
Type "help", "copyright", "credits" or "license" for more information.
>>> 
\end{verbatim}
%
처음 세 줄은 인터프리터에 대한 정보와 사용 중인 운영체제를 나타내기
때문에 환경에 따라 다르게 나타날 수 있다. 하지만, 버전은 확인해야
한다. 이 예제에서는 {\tt 3.4.0}으로 되어 있고, 3으로 시작한다. 즉,
Python 3을 실행 중이라는 것을 뜻한다. 만약 2로 시작한다면 (예상한
것처럼) Python 2를 실행중이다.

마지막 줄은 {\bf 프롬프트(prompt)}로서 인터프리터가 코드를 입력받을
준비가 되었다는 것을 뜻한다.  코드 한 줄을 쓰고 엔터를 치면
인터프리터가 결과를 보여준다.
\index{prompt}

\begin{verbatim}
>>> 1 + 1
2
\end{verbatim}
%
이제 시작할 준비가 되었다.  이제부터는 Python 인터프리터를 시작하는
방법과 코드를 실행하는 방법을 안다고 가정하겠다.


\section{최초의 프로그램}
\label{hello}
\index{Hello, World}

전통적으로 새로운 언어를 사용하여 작성하는 최초의 프로그램을 ``Hello,
World!''라고 부른다.  ``Hello, World!''라고 표시하는 것이 전부인 프로그램이기 때문이다. Python으로 만든 이 프로그램은 다음과 같다. 


\begin{verbatim}
>>> print('Hello, World!')
\end{verbatim}
%
이것은 {\bf print 문}의 예제이다. 실제 종이 인쇄하는 것이 아니고
화면에 결과를 표시한다.  이 경우에는 결과는 다음의 단어들이다.

\begin{verbatim}
Hello, World!
\end{verbatim}
%
프로그램에 사용된 따옴표가 표시될 글씨의 시작과 끝을 나타내며 결과에는
표시되지 않는다.
\index{quotation mark}
\index{print statement}
\index{statement!print}

괄호는 {\tt print}가 함수라는 것을 나타낸다.  함수에 대해서는
\ref{funcchap} 장에서 다룰 것이다.
\index{function} \index{print function}

Python 2에서는 \texttt{print}문은 함수가 아니라서 괄호를 사용하지
않는다는 것이 약간 다르다.
\index{Python 2}

\begin{verbatim}
>>> print 'Hello, World!'
\end{verbatim}
%
이러한 차이는 차차 이해하게 될 것이다. 하지만 이 정도만 알아도
시작하는데는 충분하다.


\section{Arithmetic operators}
\index{operator!arithmetic}
\index{arithmetic operator}

``Hello, World'' 다음으로 해 볼 것은 산수이다.  Python은 덧셈과 곱셉의
연산을 나타내는 특수한 기호들을 {\bf 연산자(operators)}를 제공한다.

{\tt +}와 {\tt -} 그리고 {\tt *} 연산자는 덧셈과 뺄셈 그리고 곱셉을
나타낸다. 다음의 예제를 살펴보자.

\begin{verbatim}
>>> 40 + 2
42
>>> 43 - 1
42
>>> 6 * 7
42
\end{verbatim}
%
{\tt /} 연산자로 나눗셈을 할 수 있다.

\begin{verbatim}
>>> 84 / 2
42.0
\end{verbatim}
%
이 계산의 결과가 왜 {\tt 42}이 아니고 {\tt 42.0}인지 궁금할
것이다. 그것에 대해서는 다음 절에서 설명하도록 하겠다.

마지막으로 {\tt **} 연산자는 거듭제곱을 계산할 때 사용된다. 그 수를
지수로 사용한다.

\begin{verbatim}
>>> 6**2 + 6
42
\end{verbatim}
%
어떤 언어들에서는 \verb"^"가 거듭제곱 연산에 사용되기도 하지만
Python에서는 그 기호는 비트단위의 연산자인 \texttt{XOR}을
나타낸다. 비트단위 연산자들이 익숙하지 않다면 연산 결과에 놀라워할
것이다.

\begin{verbatim}
>>> 6 ^ 2
4
\end{verbatim}
%
이 책에서는 비트단위 연산자들에 대해서는 다루지 않겠지만, 더 알고
싶다면 \url{http://wiki.python.org/moin/BitwiseOperators}에서 읽을 수
있다.
\index{bitwise operator}
\index{operator!bitwise}


\section{값과 형}
%Values and types
\index{value}
\index{type}
\index{string}

{\bf 값(value)}은 프로그램이 다루는 글자와 숫자와 같은 기본적인 것 중
하나이다.  지금까지 본 것 중에는 {\tt 2}, {\tt 42.0} 그리고
\verb"'Hello, World!'"가 있다.

이 값들은 서로 다른 {\bf 형(type)}에 속한다: {\tt 2}은 {\bf
  정수(integer)}형, {\tt 42.0}은 {\bf 부동 소수점(floating-point)}형의
숫자이다.  \verb"'Hello, World!'"는 {\bf 문자열(string)}형으로 불리는데
마치 글자들이 줄에 매달린 듯하기 때문이다.
\index{integer}
\index{floating-point}

만약에 값이 어떤 형인지 알고 싶다면, 인터프리터가 알려줄 수 있다:

\begin{verbatim}
>>> type(2)
<class 'int'>
>>> type(42.0)
<class 'float'>
>>> type('Hello, World!')
<class 'str'>
\end{verbatim}
%
이 결과들을 보면 ``클래스(class)''라는 단어가 종류 또는 범주를 나타내는
의미로 사용되고 있음을 알 수 있다.  형이라는 것은 값의 종류이다.
\index{class}

놀랍지않게도, 정수는 {\tt int} 형에 속해있고 문자열은 {\tt str}에
속했으며 부동 소수점은 {\tt float}에 속해 있다.
\index{type}
\index{string type}
\index{type!str}
\index{int type}
\index{type!int}
\index{float type}
\index{type!float}

그러면 \verb"'2'"나 \verb"'42.0'"과 같은 값들은 어떤 형을 갖을까?
숫자처럼 보이지만 따옴표 안에 있기 때문에 문자열 형에 속한다.
\index{quotation mark}

\begin{verbatim}
>>> type('2')
<class 'str'>
>>> type('42.0')
<class 'str'>
\end{verbatim}
%
확인한 것처럼 문자열이다.

매우 큰 정수를 입력할 때 {\tt 1,000,000}처럼 숫자 사이에 쉼표를 넣고 싶은 유혹이 일 수도 있다.  이와 같은 표현은 유효한 {\em 정수}형은 아니기는 하지만, Python에서 유효한 표현이기는 하다:

\begin{verbatim}
>>> 1,000,000
(1, 0, 0)
\end{verbatim}
%
이 결과는 우리가 기대한 것이 전혀 아니다!  Python은 {\tt 1,000,000}을
쉼표로 구분된 수열로 해석한다.  이와 같은 수열에 대해서는
이후에 더 자세히 살펴보도록 하겠다.
\index{sequence}

%This is the first example we have seen of a semantic error: the code
%runs without producing an error message, but it doesn't do the
%``right'' thing.
%\index{semantic error}
%\index{error!semantic}
%\index{error message}
% TODO: use this as an example of a semantic error later



\section{형식 언어 그리고 자연어}
%Formal and natural languages
\index{formal language}
\index{natural language}
\index{language!formal}
\index{language!natural}

{\bf 자연어(Natural language)}는 사람들이 말하는 영어, 스페인어와
프랑스어와 같은 언어이다.  사람들에 의해 설계된 것은 아니지만(사람들이
언어에 대해 체계를 정립하기는하였지만) 언어는 자연적으로 발달되었다.

{\bf 형식 언어(Formal language)}는 특정 응용처를 위해서 사람들이 설계한
것이다.  예를 들어 수학자들이 사용하는 표기법이 형식 언어의 일종으로
숫자와 기호들의 관계를 나타내기에 좋다.  화학자들은 형식언어를 사용하여
분자들의 화학 구조를 나타낸다.  그리고 가장 중요한 것은:

\begin{quote}
{\bf 프로그래밍 언어는 연산을 표현하기 위해 설계된 형식 언어이다. }
\end{quote}

형식 언어는 문장의 구조를 결정하는 엄격한 {\bf 문법(syntax)} 규칙을 갖고 있다. 
예를 들어 수학에서 $3 + 3 = 6$이라는 문장은 올바른 문법으로 작성되었지만 $3 + = 3 \$ 6$은 그렇지 않다. 화학에서는 $H_2O$는 문법적으로 올바른 식이지만 $_2Zz$은 아니다. 
\index{syntax}

문법 규칙은 {\bf 토큰(token)}과 구조의 두 형태를 갖고 있다.  토큰은
언어의 기본적인 구성 요소로서 단어나 숫자 그리고 화학에서의 원소들과
같은 것이다.  $3 += 3 \$ 6$라는 식에서의 문제점은 \( \$ \)이 (최소한
내가 알기로는) 수학에서 유효한 토큰은 아니라는 것이다.  유사하게
$_2Zz$도 유효하지 않다. $Zz$를 약어로 갖는 원소가 없기 때문이다.
\index{token}
\index{structure}

두 번째 종류의 문법 규칙은 토큰들을 결합하는 것과 관련이 있다. $3 +=
3$는 규칙에 어긋난다. 이 식에서 $+$와 $=$이 유효한 토큰이기는 하지만
연이어서 두 개를 사용할 수가 없다.  화학 식에서도 마찬가지로 아래 첨자는
이름 뒤에 나타나지 그 앞에 나타나지 않는다.

이 문장은 @ 구조적으로 잘 정의되었z만 유효하지 않은 t*ken이
사용되었다. 이 문장의 토큰은 모두 유효하지만 갖고 있다 구조를 잘못된.

모국어로 문장을 읽거나 형식 언어로 문장을 읽을 때에는 그 구조를
파악해야 한다(자연어를 파악하는 과정은 무의식적으로 일어나기는 한다).
프로그래밍에서 구조를 파악하는 과정을 {\bf 파싱(parsing)}이라고
부른다.
\index{parse}

형식 언어와 자연어 사이에는 토큰과 구조 그리고 문법과 같은 많은 공통적 기능들이 있지만 차이점도 분명하다:
\index{ambiguity}
\index{redundancy}
\index{literalness}

\begin{description}

\item[모호성(ambiguity):] 자연어는 모호하기 때문에 맥락과 다른 정보들을
  통해 이해를 한다. 형식 언어의 경우 거의 또는 완전하게 그 모호성의
  없애도록 설계되었기 때문에 문맥과 상관없이 유일한 의미를 갖고 있다.

\item[중복성(redundancy):] 자연어가 갖고 있는 모호성을 없애고 오해를
  줄이기 위해서 엄청난 중복성을 허용하고 있다. 그 결과로 인해
  장황해지기 쉽다. 형식 언어는 훨씬 간결하다.

\item[직역성(literalness):] 자연어는 숙어와 은유적 표현으로
  넘쳐난다. ``The penny doppred(돈 떨어졌다)''라고 말을 하면 실제 뜻은
  돈에 관한 것도 아니고 실제로 무언가가 떨어지는 것도 아니다(이 숙어는
  뜻은 `이제야 알아 들었다'는 것이다).  형식 언어에서는 말하는 그대로가
  의도한 뜻한다.

\end{description}

자라면서 자연어를 구사하였기 때문에 형식 언어에 익숙해지는데 많은 시간이 걸리곤 한다.  자연어와 형식 언어간의 차이는 시와 산문 간의 차이와도 유사하다. 
 \index{poetry} \index{prose}

\begin{description}

\item[시:] 사용되는 단어의 소리도 그 단어의 의미만큼 중요하며, 시 전체
  봐야만 어떤 감정적 반응을 하거나 또는 영향을 느낄 수가 있다.
  애매모호한 표현이 흔할 뿐만 아니라 의도된 것일 때도 있다.

\item[산문:] 단어가 갖고 있는 의미가 좀 더 중요하다. 그렇기 때문에
  구조에 의미가 더 많이 담겨 있다.  산문은 시보다는 분석하기에 더
  쉽기는 하지만 모호할 때가 많다.

\item[프로그램:] 컴퓨터 프로그램의 의미는 모호하지 않아며 문자적으로
  이해할 수 있어서 구조와 토큰을 분석하는 것만으로 모든 것을 이해할 수
  있다.

\end{description}

형식 언어는 자연어보다 좀 더 난해해서 읽는데 오래 걸린다. 또한, 구조가
중요하기 때문에 위에서 아래로 왼쪽에서 오른쪽으로 읽는 것이 늘 최선의
방식은 아니다.  대신에 프로그램을 상상의 공간에서 분석하고 토큰은
구별해 내고 구조를 해석하는 방법을 익혀야 한다.  마지막으로 세부사항이
중요하다.  철자나 구두점의 표기를 잘못하는 것과 같은 작은 실수를
자연어에서는 지나쳐도 문제가 되지 않았겠지만 형식 언어에서는 큰 차이를
만들어 낸다.


\section{디버깅}
%Debugging
\index{debugging}

프로그래머들은 실수를 한다.  황당하게도 프로그래밍 오류를 {\bf
  버그(bug)}라고 부르고 그것들을 해결하는 과정을 {\bf
  디버깅(debugging)}이라고 부른다.
\index{debugging}
\index{bug}

프로그래밍과 특히 디버깅 때문에 감정이 격해질 수가 있다.  매우 어려운
버그로 고생을 하고 있다면 화가나거나 무기력해지기도 또는 당혹스러워질
수도 있다.

사람들이 컴퓨터를 대할 때 아주 자연스럽게 컴퓨터가 마치 사람인양
대한다는 결과들이 있다. 잘 동작한다면 한 팀원으로 여기기도 하고
외고집스럽고 무례할 정도로 말을 안듯는다면 그런 사람을 대하는 것과
동일하게 대한다고 한다(Reeves와 Nass, {\it The Media Equation: How
  People Treat Computers, Television, and New Media Like Real People
  and Places}).
\index{debugging!emotional response}
\index{emotional debugging}

이와 같은 경우에 대응하는 방법들을 미리 생각해 놓는 것이 이후에 그런
일들을 당했을 때 대처하기가 쉬워진다.  한 가지 방법은 컴퓨터를 마치
일정한 능력을, 예를 들어 속도와 정확도, 갖고 있지만 공감 능력과 전체를
보는 시각이 전혀 없는 직원이라고 생각하면 된다.

당신이 할 일은 좋은 관리자가 되는 것이다: 직원의 장점을 파악하고 단점을
완화시켜주는 것이다.  그 후에 할 일은 문제 해결 과정에서 말 안듣는
컴퓨터와 감정적 씨름을 하는 대신 작업을 효율적으로 처리할 수 있는
방법들을 찾아야 한다.

디버깅 방법을 배우는 것은 좌절스러운 경험이 될 수 있겠지만, 이 기술은
프로그래밍 외에도 여러 방면에서 적용할 수 있는 매우 가치 있는
기술이기도 하다.  각 장의 끝에 있는 절에는 디버깅에 대한 조언이 적혀
있다.  도움이 되길 바란다!


\section{용어 해설}

\begin{description}

\item[문제 해결(Problem solving):]  문제를 정의하고 해법을 찾고 표현하는 과정. 
\index{problem solving}

\item[고수준 언어(high-level language):] Python과 같은 언어로 사람이
  읽고 쓰기 쉽게 설계된 언어.
\index{high-level language}

\item[저수준 언어(low-level language):] 컴퓨터가 실행시키기 쉽게 설계된
  프로그램 언어로서 ``기계어'' 또는 ``어셈블리어''라고도 불림.
\index{low-level language}

\item[이식성(portatiblity):]  한 종류의 컴퓨터 이상에서 프로그램이 실행 가능한 성질.
\index{portability}

\item[해석 프로그램(인터프리터/interpreter):]  프로그램을 읽고 실행시키는 프로그램.
\index{interpret}

\item[프롬프트(prompt):] 사용자로부터 입력을 받을 준비가 되었다는 것을
  알려주기 위한 인터프리터가 표시하는 글자들.
\index{prompt}

\item[프로그램(program):] 연산을 지정해놓은 명령어들의 집합.
\index{program}

\item[print 문(print statment):] Python 인터프리터가 화면에 어떤 값을
  표시하도록 하는 명령어.
\index{print statement}
\index{statement!print}

\item[연산자(operator):] 덧셈, 곱셉, 또는 문자열 연결과 같은 간단한
  연산을 뜻하는 특수 기호
\index{operator}

\item[값(value):]  프로그램이 조작하는 숫자나 문자열 데이터의 기본 단위 중 하나
\index{value}

\item[형(type):] 값들의 분류. 정수({\tt int}형), 부동-소수점 숫자({\tt
    float}형)과 문자열({\tt str}형)이 우리가 지금까지 살펴본 데이터의
  형이다.
\index{type}

\item[정수(integer):] 자연수를 나타내는 데이터 형이다. 
\index{integer}

\item[부동 소수점(floating-point):] 분수 부분을 표현하는 숫자들.
\index{floating-point}

\item[문자열(string):] 연속된 글자들을 표현하는 데이터 형.
\index{string}

\item[자연어(natural language):]  사람들이 말하는 자연적으로 발전한 모든 언어.
\index{natural language}

\item[형식 언어(formal language):] 수학적 개념을 표현하기 위해서나
  컴퓨터 프로그램을 표현하기 위한 언어처럼 사람들이 특정 목적을 갖고
  설계한 모든 언어. 모든 프로그래밍 언어는 형식 언어이다.
\index{formal language}

\item[토큰(token):] 자연어의 단어와 대응되는 개념으로 프로그램의 문법적
  구조를 이루는 기본적인 요소
\index{token}

\item[문법(syntax):] 프로그램의 구조를 결정하는 규칙
\index{syntax}

\item[구문 해석(파싱, parse):] 프로그램을 조사하여 문법적 구조를 분석하는 것
\index{parse}

\item[버그(bug):] 프로그램에 있는 오류
\index{bug}

\item[디버깅(debugging):] 버그를 찾아 고치는 과정
\index{debugging}

\end{description}


\section{연습 문제}
%Exercises

\begin{exercise}

  이 책을 컴퓨터 앞에서 읽는 것을 권한다. 예제들을 만날 때마다 시도
  해볼 수 있기 때문이다.

새로운 기능을 실험할 때에는 실수를 많이 할 수록 좋다. 예를 들어
``Hello, world!'' 프로그램에서 둘 중의 하나의 따옴표를 입력하지 않으면
어떤 일이 생길까?  둘 다 입력하지 않으면 어떻게 될까? {\tt print}를
잘못 입력하면 어떻게 될까?  \index{error message}

이런 류의 실험은 읽은 내용을 오랫동안 기억에 남게 할뿐만 아니라 각 오류
메시지가 어떤 의미를 갖는지 알 수 있기 때문에 프로그래밍할 때에도
도움이 되나. 지금 의도적으로 실수를 만들어 내는 것이 나중에 우연히
실수를 범하는 것보다 낫다.

\begin{enumerate}

\item \texttt{print}문에서 괄호 중 하나 또는 둘 다 입력하지 않으면 어떻게 될까?

\item 문자열을 출력한다고 했을 때 따옴표 중 하나 또는 둘 다 입력하지
  않으면 어떻게 될까?

\item 음수를 표현하기 위해 {\tt -2}처럼 빼기 부호를 사용할 수
  있다. 예를 들어 {\tt 2++2}처럼 더하기 부호를 숫자 앞에 넣은 경우에는
  어떻게 될까?

\item 수학에서는 {\tt 02}처럼 숫자 앞에 있는 0들이 있어도 괜찮다. Python에서 동일한 방식으로 입력한다면 어떻게 될까? 

\item 두 값을 입력할 때에 사이에 연산자가 없는 경우는 어떻게 될까? 

\end{enumerate}

\end{exercise}



\begin{exercise}

Python 인터프리터를 실행시키고 계산기처럼 사용해보자.

\begin{enumerate}

\item 42분 42초는 모두 몇 초일까? 

\item 10 킬로미터는 몇 마일일까? 참고로 1.61 킬로미터가 1 마일이다. 

\item 10 킬로미터를 42분 42초만에 달렸다면 평균 보폭언 얼마나
  될까(마일당 시간을 분과 초로 표시)? 평균 속도는 시간당 마일로
  표기하면 몇인가?
\index{calculator}
\index{running pace}

\end{enumerate}

\end{exercise}




\chapter{변수와 수식 그리고 문장}
%Variables, expressions and statements


프로그래밍 언어의 가장 강력한 기능 중 하나는 {\bf 변수(variable)}가
조작 가능하다는 것이다.  변수는 값에 부여하는 이름이다.
\index{variable}


\section{할당문}
%Assignment statements
\label{variables}
\index{assignment statement}
\index{statement!assignment}

{\bf 할당문(assignment statement)}은 새로운 변수를 생성하고 값을 부여한다. 

\begin{verbatim}
>>> message = 'And now for something completely different'
>>> n = 17
>>> pi = 3.141592653589793
\end{verbatim}
%
이 예제에서는 세 개의 변수에 값을 할당했다.  첫 번째는 {\tt
  message}라는 이름의 변수에 문자열을 할당하였다.  두 번째는 {\tt n}에
정수 {\tt 17}을 부여하였고 세 번째에는 $\pi$의 값을 {\tt pi}라는 변수에
할당하였다.
\index{state diagram}
\index{diagram!state}

종이에 변수를 표현하는 방법은 이름과 값을 적고 화살표를 이름에서 값으로
향하게 그리면 된다. 이와 같은 그림을 {\bf 상태도(state diagram)}이라
부른다.  각 변수의 상태를 나타내기 때문이다(변수 마음 상태로
생각해보자)
%수정필요 (think of it as the variable's state of mind).
그림~\ref{fig.state2}는 방금의 예제의 상태도를 보여준다. 


\begin{figure}
\centerline
{\includegraphics[scale=0.8]{figs/state2.pdf}}
\caption{상태도.}
\label{fig.state2}
\end{figure}



\section{변수명}
%Variable names
\index{variable}

일반적으로 프로그래머는 변수에 의미있는 이름을 부여하며, 그 변수가
무엇에 이용되는지 문서화하기도 한다.

이름은 길게 만들고 싶은만큼 길게 만들 수 있다.  글자와 숫자로 이루어질
수 있지만 숫자로 시작될 수는 없다. 대문자를 못 쓰는 것은 아니지만
관례적으로 변수명으로는 소문자만을 쓴다.

밑 줄 표시를, \verb"_", 이름에서 볼 수도 있다 대체적으로
\verb"your_name"이나 \verb"airspeed_of_unladen_swallow"처럼 여러 단어로
이루어진 이름에서 사용된다.
\index{underscore character}

무효한 이름을 변수에 할당하게 되면 문법 오류 메시지가 뜬다:

\begin{verbatim}
>>> 76trombones = 'big parade'
SyntaxError: invalid syntax
>>> more@ = 1000000
SyntaxError: invalid syntax
>>> class = 'Advanced Theoretical Zymurgy'
SyntaxError: invalid syntax
\end{verbatim}
%
{\tt 76trombones}가 잘못된 이유는 숫자로 시작하기 때문이고 {\tt
  more@}에서 오류가 발생한 이유는 변수명으로 쓸 수 없는 글자 {\tt @}를
사용했기 때문이다.  그렇다면 {\tt class}에는 무슨 문제가 있는 걸까?

{\tt class}는 Python이 정의해 놓은 {\bf 키워드(keyword)}이기 때문이다.
인터프리터는 키워드들을 사용하여 프로그램의 구조를 파악하기 때문에
변수명으로 사용할 수 없다.  
\index{keyword}

Python 3에는 다음과 같은 키워드들이 있다:

\begin{verbatim}
False      class      finally    is         return
None       continue   for        lambda     try
True       def        from       nonlocal   while
and        del        global     not        with
as         elif       if         or         yield
assert     else       import     pass
break      except     in         raise
\end{verbatim}
%
이 목록을 외울 필요는 없다.  대부분의 개발 환경은 키워드들을 다른
색으로 표현해 준다.  이 목록에 있는 키워드 중 하나를 변수 명으
사용하려 한다면 알 수 밖에 없다.


\section{수식과 문장}
%Expressions and statements

{\bf 수식(expression)}은 값과 변수 그리고 연산자들의
조합이다. 값으로만 이루어진 그 자체도 수식으로 인식되며 변수도
마찬가지이다. 다음의 유효한 수식들을 살펴보자.
\index{expression}

\begin{verbatim}
>>> 42
42
>>> n
17
>>> n + 25
42
\end{verbatim}
%
프롬프트에 수식을 입력하면 인터프리터는 수식을 {\bf
계산(evaluate)}하여 그 안에 있는 값을 찾는다.  이 예제에서는 {\tt n}은
17이라는 값을 갖고 있고 {\tt n + 25} 는 42라는 값을 갖고 있다.
\index{evaluate}

{\bf 문장(statement)}은 변수를 생성한다거나 값을 표시하는 것과 같이
결과가 있는 코드의 단위이다.
\index{statement}

\begin{verbatim}
>>> n = 17
>>> print(n)
\end{verbatim}
%
첫 줄은 할당문으로 {\tt n}에 값을 부여하고 있고 두 번째 줄은 {\tt
  n}에 할당된 값을 출력하는 \texttt{print}문이다.

문장을 입력하면, 인터프리터는 {\bf 실행(excute)}한다. 실행한다는 의미는
그 문장이 뜻하는 데로 동작한다는 것이다.  일반적으로 문장에는 값이 할당되지 않는다. 
\index{execute}


\section{스크립트 모드}
%Script mode

지금까지 Python을 {\bf 대화식 모드(interactive mode)}로 인터프리터에
입력한 결과를 즉시 확인할 수 있었다.  대화식 모드는 감을 잡고
시작하기에는 유용하지만 여러 줄의 코드를 사용하기에는 불편하다.
\index{interactive mode}

그 대안으로 {\bf 스크립트(script)}라고 부르는 파일에 코드를 저장해 놓고 인터프리터를 
{\bf 스크립트 모드(script mode)}로 동작시켜서 스크립트를 실행시키는 것이다. 
통상적으로 Python 스크립트 파일의 이름은 {\tt .py}로 끝이 난다. 
\index{script}
\index{script mode}

컴퓨터에서 스크립트를 생성하고 실행시키는 방법을 안되면 이제 제대로
시작할 준비가 되었다.  그게 아니라면 \texttt{PythonAnywere}를 다시 한
번 사용하기를 권한다.  스크립트 모드를 실행하는 방법을
\url{http://tinyurl.com/thinkpython2e}에 기록해 놓았다.

Python이 두 가지 모드를 지원하기 때문에 스크립트로 저장하기 전에 대화식
모드로 실험해볼 수 있다.  하지만 대화식 모드와 스크립트 모드간의 차이가
있기 때문에 그렇게 사용하면 혼란스러울 수 있다.
\index{interactive mode}
\index{script mode}

예를 들어 Python을 계산기로 사용하여 다음과 같이 입력하였다고 해보자. 

\begin{verbatim}
>>> miles = 26.2
>>> miles * 1.61
42.182
\end{verbatim}

첫 줄은 {\tt miles}에 값을 할당하였지만, 그 자체로는 가시적인 결과가
보이지 않는다.  두 번째 줄은 수식이기 때문에 인터프리터가 계산하여
결과를 표시한다. 이 결과를 보면 마라톤의 길이는 42 킬로미터 정도된다.

이 똑같은 코드를 스크립트로 저장하여 실행시켜 보면 어떤 결과도 출력되지
않는다.  스크립트 모드에서의 수식 그 자체로는 어떠한 가시적인 결과가
나타나지 않는다.  사실 Python은 수식을 계산하지만 값을 출력하라는
말을 듣기 전까지는 그 결과를 표시하지 않는다.


\begin{verbatim}
miles = 26.2
print(miles * 1.61)
\end{verbatim}

이 같은 동작이 처음에는 이해가 안될 수 있다. 

스크립트는 대체적을 연속된 문장들로 구성되어 있다. 하나 이상의 문장이
있는 경우에 각 문장이 실행될 때마다 하나씩 결과를 나타낸다.

예를 들어, 다음 같은 스크립를 살펴보자.

\begin{verbatim}
print(1)
x = 2
print(x)
\end{verbatim}
%
이 스크립트를 실행하면 다음의 결과를 얻는다. 

\begin{verbatim}
1
2
\end{verbatim}
%
할당문은 결과를 표시하지 않는다. 

이해를 했는지 확인해보기 위해 다음의 문장들을 Python 인터프리터에 입력하여 어떻게 동작하는지 살펴보라. 

\begin{verbatim}
5
x = 5
x + 1
\end{verbatim}

그리고 이 문장들을 스크립트로 저장하여 실행해보자.  결과가 어떻게
되는가?  스크립트의 각 수식을 \texttt{print}문으로 변환하여 다시
실행해보자.



\section{연산의 우선순위}
%Order of operations
\index{order of operations}
\index{PEMDAS}

어떤 수식에 하나 이상의 연산자가 있다면 연산의 순서는 {\bf 연산의
  우선순위(order of operations)}에 의존한다.  수학 연산자의 경우
Python은 일반적인 수학 연산자의 우선순위를 따른다.  한 가지 방법으로
{\bf PEMDAS}라는 두문문자로 우선순위를 기억하면 쉽다.


\begin{itemize}

\item {\bf P}arentheses(괄호)는 가장 높은 우선순위를 갖고 있기 때문에
  괄호를 사용하면 원하는 데로 연산의 순서를 조작할 수가 있다.  수식에서
  괄호는 가장 먼저 계산되기 때문에 {\tt 2 * (3-1)}은 4가 되고 {\tt
    (1+1)**(5-2)}는 8이 된다.  또한 {\tt (minute * 100) / 60}라는
  식처럼 괄호를 사용하여 계산 결과는 바꾸지 않으면서 수식을 좀 더 읽기
  쉽게 만들 수도 있다.

\item {\bf E}xponentiation(거듭제곱)이 그 다음으로 높은 우선 순위를
  갖고 있다.  {\tt 1 + 2**3}의 결과는 9이지 27이 아니며, {\tt 2 *
    3**2}의 결과는 36이 아니라 18이다.


\item {\bf M}ultiplication(곱셉)과 {\bf D}ivision(나눗셈)은 {\bf
    A}ddition(덧셈)과 {\bf S}ubtraction(뺄셈)보다 더 높은 우선순위를
  갖고 있다. {\tt 2*3-1}의 연산 결과는 4가 아니라 5이며 {\tt 6+4/2}의
  결과는 8이지 5가 아니다.


\item 같은 연산 우선순위를 갖는 연산자는 왼쪽부터 오른쪽으로(거듭제곱은 예외) 계산된다. 
{\tt degrees / 2 * pi}라는 식에서는 나눗셈이 먼저 계산되고 그 결과에 {\tt pi}가 곱해진다.  만약 $2 \pi$로 나누기 원한다면 괄호를 쓰거나 {\tt degrees / 2 / pi}로 식을 고치면 된다.

\end{itemize}


일부러 연산 우선순위를 외우려고 엄청나게 노력할 필요는 없다.  수식이
바로 파악이 안된다면 괄호를 써서 그 식이 당연해지도록 바꾸면 되기 때문이다.


\section{문자열 연산}
%String operations
\index{string!operation}
\index{operator!string}


일반적으로 문자열에 수학적 연산을 쓸 수 없다.  설령 문자열이 숫자처럼
보일지라도 그렇게 쓰는 것은 불법이다.


\begin{verbatim}
'2'-'1'    'eggs'/'easy'    'third'*'a charm'
\end{verbatim}
%
그 규칙에는 두 가지 예외가 있다. {\tt +}와 {\tt *}이다. 


{\tt +} 연산자는 {\bf 문자열 연결(string concatenation)}할 때 사용하는
것으로 문자열들을 앞뒤로 붙여준다. 예를 들어 보자:
\index{concatenation}

\begin{verbatim}
>>> first = 'throat'
>>> second = 'warbler'
>>> first + second
throatwarbler
\end{verbatim}
%
{\tt *} 연산자도 역시 문자열을 다룰 때 쓸 수 있다. {\tt *}는 반복할 때
사용된다.  예를 들어 \verb"'Spam'*3"의 결과는
\verb"'SpamSpamSpam'"이다.  이것을 사용할 때는 한 값이 문자열이어야
하고 다른 값은 정수여야 한다.

{\tt +}와 {\tt *}의 문자열에 대한 용법은 수식에서의 덧셈과 곱셉의
용법과 연결지어 생각하면 좋다.  {\tt 4*3}은 {\tt 4+4+4}와 동치이기
때문에 \verb"'Spam'*3"의 결과로 \verb"'Spam'+'Spam'+'Spam'"과 같은
결과를 기대할 것이고, 실제로 같은 결과를 갖고 있다.  물론, 문자열
연결과 반복은 정수의 덧셈과 곱셉과 다른 면이 많이 있다.  정수의 덧셈의
성질 중 문자열 연결의 성질과 다른 것을 생각해볼 수 있겠는가?
\index{commutativity}


\section{주석}
%Comments
\index{comment}


프로그램의 크기가 커지고 복잡해지게되면 읽기가 어려워진다.  형식 언어는
난해하다고 말을 했었다. 그렇기 때문에 코드의 일부분을 보고 무슨 일을
하는지 또는 왜 그 일을 하는지 이해하지 못할 때가 많다.


그렇기 때문에 프로그램에 자연어로 이 프로그램이 어떤 일을 하는지 기록해
놓는 것은 언제나 좋은 생각이다.  이렇게 기록하는 것을 보고 {\bf
  주석(comment)}라고 부른다.  주석은 \verb"#" 기호로 시작한다.

\begin{verbatim}
# 한 시간에 몇 퍼센트가 지났는지 계산
percentage = (minute * 100) / 60
\end{verbatim}
%
이 경우에는 주석이 혼자 한 줄을 다 쓰고 있다.  주석을 그 줄의 끝에
적을 수도 있다.

\begin{verbatim}
percentage = (minute * 100) / 60     # 시간의 퍼센트
\end{verbatim}
%
인터프리터는 {\tt \#} 이후부터 그 줄의 끝까지의 모든 내용을 무시한다.
프로그램 실행에 전혀 영향을 주지 않는다.

주석이 가장 빛을 발하는 순간은 코드 중에 불명확한 기능을 설명할 때이다.  코드를 읽었을 때 그 프로그램이 {\em 무엇}을 하는지는 알아 낼 수 있을 것이다.  그렇기 때문에 {\em 왜} 그 일을 하는지를 설명하는 것이 가장 유익하다. 

다음의 주석은 코드가 하는 일에 대한 중복 설명이라 불필요하다. 

\begin{verbatim}
v = 5     # assign 5 to v
\end{verbatim}
%
반면 다음의 코드는 코드에서 알아 낼 수 없는 유익한 정보를 담고 있다. 

\begin{verbatim}
v = 5     # meters/second 단위의 속도 
\end{verbatim}
%
변수 명을 잘 정하면 주석을 달아야 할 필요가 없어지지만 너무 긴 이름으로
지으면 읽기가 어려워지기 때문에 적당한 지점에서 타협을 해야 한다.


\section{디버깅}
%Debugging
\index{debugging}
\index{bug}


프로그램에는 문법(syntax) 오류, 실행시간(runtime) 오류, 그리고
문맥(semantic) 오류라는 세 가지 종류의 오류가 있을 수 있다.  이 셋을
구분할 줄 아는 것은 그 오류를 추적하는데 큰 도움을 준다.


\begin{description}

\item[문법(Syntax) 오류:] ``문법(Syntax)''은 프로그램의 구조와 그
  구조에 대한 규칙을 뜻한다. 예를 들어 괄호는 두 개가 한 쌍으로
  사용되어야 하기 때문에 {\tt (1+2)}는 유효하지만, {\tt 8)}는 {\bf문법
    오류}로 사용할 수 없다.
   \index{syntax error}\index{error!syntax}
  \index{error message}\index{syntax} 

  만약 프로그램 어딘가에 문법 오류가 있다면 Python은 오류 메시지를
  출력하고 종료하기 때문에 그 이후의 프로그램은 더 실행 할 수 없다.
  프로그래밍에 입문한지 얼마지나지 않은 몇 주 동안은 문법 오류를
  해결하는데 대부분의 시간을 활애할 것이다.  좀 더 연륜이 쌓인다면
  오류의 수가 더 적어 질 것이고 그런 오류를 더 빨리 찾아 낼 수 있게 된다.


\item[실행시간(Runtime) 오류:] 두 번째 오류는 실행시간 오류라고
  불리는데, 그렇게 불리는 이유는 프로그램이 실행되기 전까지는 있는지 알
  수 없기 때문이다.  이러한 오류들은 {\bf 예외(exception)}이라고
  불리기도 한다.  예외적인(그리고 나쁜) 일이 발생했다는 것을 나타내기
  때문이다.  \index{runtime error} \index{error!runtime}
  \index{exception} \index{safe language} \index{language!safe}

  실행시간 오류는 이 책의 처음 몇 장을 다루는 동안에는 보게 되는 간단한
  프로그램에서는 드물게 나타난다.  실제 이런 류의 오류를 만나려면
  시간이 좀 흘러야 할 것이다.


\item[문맥(Semantic) 오류:] 세 번째 종류의 오류는 ``문맥(semantic)''
  오류로서 사용된 문장들의 의미와 관련이 있다.  프로그램에 문맥 오류가
  있다면 오류 메시지 없이 실행이 잘 되지만 그렇다고 제대로 동작하는
  것은 아니다.  엉뚱하게 동작을 한다.  정확하게 말하자면, 프로그래머가
  시킨 일을 그대로 한다.  \index{semantic error}
  \index{error!semantic} \index{error message}

  문맥 오류를 찾아내는 것은 까다로운 일인 이유는 프로그램이 어디서
  어떻게 잘못되었는지를 프로그램의 실행 결과를 통해 역 추적해야하기
  때문이다.

\end{description}


\section{용어 해설}
%Glossary

\begin{description}

\item[변수(variable):]  값을 가리키는 이름. 
\index{variable}

\item[할당(assignment):]  변수에 값을 지정하는 문장. 
\index{assignment}

\item[상태도(state diagram):]  변수와 그 변수가 가리키는 값을 표현한 그림
\index{state diagram}

\item[키워드(keyword):] 프로그래밍 언어에서 미리 예약해 놓은 단어로서
  프로그램을 구문해석하는 과정에 사용된다.  {\tt if}, {\tt def}와 {\tt
    while}과 같은 키워드는 변수 명으로 사용할 수 없다.
\index{keyword}

\item[피연산자(operand):]  연산자를 사용하여 계산할 때 사용되는 값
\index{operand}

\item[수식(expression):] 변수와 연산자 그리고 값들의 조합으로 하나의
  결과를 나타냄.
\index{expression}

\item[계산(evaluate):]  하나의 값을 얻기 위해 연산하여 수식을 간단화하는 것

\item[문장(statement):] 명령이나 어떤 동작을 뜻하는 코드의 일부분.
  지금까지는 봐왔던 문장은 할당하는 것과 \texttt{print}문이 있다.
\index{statement}

\item[실행(execute):]  문장에 적힌 그대로를 수행
\index{execute}

\item[대화식 모드(interactive mode):] 프롬프트에 코드를 입력하여
  Python 인터프리터를 사용하는 방식
\index{interactive mode}

\item[스크립트 모드(script mode):] 스크린트 파일에 저장되어 있는 코드를
  읽어 실행하도록 Python 인터프리터를 사용하는 방식
\index{script mode}

\item[스크립트(script):] 파일에 저장되어 있는 프로그램
\index{script}

\item[연산의 우선순위(order of operations):] 여러 개의 연산자와
  피연산자로 이루어진 수식이 계산되는 순서를 결정하는 규칙
\index{order of operations}

\item[연결(concatenate):]  두 개의 피 연산자를 앞뒤로 합치는 것
\index{concatenation}

\item[주석(comment):] 다른 프로그래머(또는 소스 코드를 읽는 누군가)를
  위해 프로그램에 기록되어 있는 정보. 프로그램의 실행에는 아무런 영향이
  없음. \index{comment}

\item[문법 오류(syntax error):] 프로그램 해석이 불가능하게 만드는
  프로그램 내의 오류(구문해석이 불가능함)
\index{syntax error}

\item[예외(exception):] 프로그램이 실행 중에 발견되는 오류
\index{exception}

\item[문맥(semantics):] 프로그램의 의미
% The meaning of a program.
\index{semantics}

\item[문맥 오류(semantic error):] 프로그램의 의도와 다르게 프로그램이
  동작하게 만드는 오류
\index{semantic error}

\end{description}


\section{연습 문제}
%Exercises

\begin{exercise}

  앞 장에서 했던 조언처럼 새로운 기능을 배우면 대화식 모드에서
  실험해보고 의도적으로 실수를 해서 무엇이 잘못되는가를 살펴봐야 한다.

\begin{itemize}

\item  {\tt n = 42}는 유효하다는 것을 보았다. {\tt 42 = n}은 어떨까?

\item {\tt x = y = 1 }은 유효한가?

\item 어떤 프로그래밍 언어는 세미콜론({\tt ;})으로 문장이 끝난다.
  Python의 문장의 끝을 세미콜론으로 끝을 내면 어떻게 되는가?

\item 문장의 끝을 구두점으로 끝을 내면 어떻게 되는가?

\item 수학에서는 $x$와 $y$의 곱을 표기 할 때 $x y$로 써도 된다.  Python에서 이렇게 쓰면 어떻게 될까?

\end{itemize}

\end{exercise}


\begin{exercise}

Python의 인터프리터를 계산기처럼 사용하여 다음 문제를 실습해보자.
\index{calculator}

\begin{enumerate}

\item 반지름이 $r$인 구의 부피는 $\frac{4}{3} \pi r^3$ 이다.  이때 구의
  반지름이 5라면 부피가 어떻게 되는가?

\item 책의 가격이 27,500원으로 책정되어 있는데, 책 방에서 40\%
  도서할인전을 하고 있다. 한 권일 때는 배송비가 2,500원인데, 추가
  구입을 할 때마다 150원이 추가된다.  60권을 산다면 배송비를 포함한 총
  금액이 얼마인가?


\item 집에서 6시52분에 나와서 1마일을 천천히 8분 15초에 달리고, 그 다음
  3마일은 마일당 7분 12초가 걸렸다. 그리고 1마일은 처음처럼 천천히
  달렸다.  운동을 마치고 집에 도착하면 몇 시에 아침 식사를 할 수
  있을까?
\index{running pace}

\end{enumerate}
\end{exercise}


\chapter{함수}
%Functions
\label{funcchap}

프로그래밍이라는 문맥에서 {\bf 함수(function)}은 연산을 수행하기 위한
일련의 문장들에 이름을 붙여 놓은 것이다.  함수를 정의할 때는 이름과
문장들 지정해야 한다.  그 다음에 함수를 지정한 이름으로
``호출(call)''할 수 있다 .
\index{function}

\section{함수 호출}
%Function calls
\label{functionchap}
\index{function call}

우리는 이미 {\bf 함수 호출(function call)}의 예를 보았다: 

\begin{verbatim}
>>> type(42)
<class 'int'>
\end{verbatim}
%
함수의 이름은 {\tt type}이다.  괄호 안의 표현은 함수의 {\bf
  인자(argument)}라고 불린다.  이 함수의 결과로 인자의 분류를
알려준다.
\index{parentheses!argument in}

흔히 함수가 인자를 ``받아서(take)'' 결과를 ``리턴(return)''한다고
말한다.  그리고 함수의 결과를 {\bf 리턴 값(return value)}이라고
부른다.
\index{argument}
\index{return value}

Python은 어떤 값을 한 분류에서 다른 분류로 변환하는 함수들을 제공한다.
{\tt int} 함수는 어떤 값이든 받아서 정수로 변환할 수 있으면 변환하고
그렇지 못하는 경우는 불평을 한다.
%complain 불평
\index{conversion!type}
\index{type conversion}
\index{int function}
\index{function!int}

\begin{verbatim}
>>> int('32')
32
>>> int('Hello')
ValueError: invalid literal for int(): Hello
\end{verbatim}
%
{\tt int} 함수는 부동 소수점 숫자를 정수로 변환할 수 있지만 반올림을 하지
않는다.  소수점 부분을 잘라 버리기만 한다.

\begin{verbatim}
>>> int(3.99999)
3
>>> int(-2.3)
-2
\end{verbatim}
%
{\tt float} 함수는 정수와 문자열을 부동 소수점 숫자로 변환한다. 
\index{float function}
\index{function!float}

\begin{verbatim}
>>> float(32)
32.0
>>> float('3.14159')
3.14159
\end{verbatim}
%
마지막으로 {\tt str} 함수는 인자를 문자열로 변환한다. 
\index{str function}
\index{function!str}

\begin{verbatim}
>>> str(32)
'32'
>>> str(3.14159)
'3.14159'
\end{verbatim}
%

\section{수학 함수}
%Math functions
\index{math function}
\index{function, math}

Python은 대부분의 익숙한 수학 함수들을 포함하는 수학 모듈을 갖고 있다.
{\bf 모듈(module)}은 관련된 함수들을 모아 놓은 파일이다.
\index{module}
\index{module object}

모듈에 포함된 함수를 사용하기 전에 {\bf import 문}으로 읽어들여야 한다. 

\begin{verbatim}
>>> import math
\end{verbatim}
%
이 문장은 \texttt{math}라고 불리는 {\bf 모듈 객체(module object)}를
생성한다.  모듈 객체를 표시하도록 하면 객체에 관한 정볼르 얻을 수
있다.


\begin{verbatim}
>>> math
<module 'math' (built-in)>
\end{verbatim}
%
모듈 객체는 모듈에 정의된 함수들과 변수들을 포함하고 있다.  모듈에
포함된 함수를 사용하고 싶으면 모듈과 사용하기 원하는 함수의 이름을
구두점(닷, dot)으로 구분하여 지정해야 한다.  이와 같은 형식을 {\bf 닷
  표기법(dot notation)}이라고 부른다.
\index{dot notation}

\begin{verbatim}
>>> ratio = signal_power / noise_power
>>> decibels = 10 * math.log10(ratio)

>>> radians = 0.7
>>> height = math.sin(radians)
\end{verbatim}
%
첫 예제에서는 \verb"math.log10"을 사용하여 \verb"signal_power" 와
\verb"noise_power"의 변수가 정의되어 있다고 가정했을 때의 신호대잡음비를
데시벨로 변환한다.  수학 모듈은 밑 수를 {\tt e}로 하는 {\tt log}도 제공한다. 
\index{log function}
\index{function!log}
\index{sine function}
\index{radian}
\index{trigonometric function}
\index{function, trigonometric}

두 번째 예제에서는 {\tt radians} 변수에 대한 사인 값을 계산한다.  {\tt
  sin}, {\tt cos} 그리고 {\tt tan}와 같은 삼각함수와 관련된 함수들의
인자 값은 이 예제에서 사용한 변수 명처럼 라디안을 쓴다.  도를
라디안으로 변환하려면 180으로 나누고 $\pi$로 곱하면 된다.


\begin{verbatim}
>>> degrees = 45
>>> radians = degrees / 180.0 * math.pi
>>> math.sin(radians)
0.707106781187
\end{verbatim}
%
{\tt math.pi}라고 쓰면 {\tt pi} 변수를 수학 모듈에서 가져온다.  이 값은
소수점 15자리까지 정확한 근사값이다.
\index{pi}

삼각함수 식에 따라 45도의 사인 값은 2분의 루트 2이기 때문에 예제의
결과가 맞는지 비교해 볼 수 있다.
\index{sqrt function}
\index{function!sqrt}

\begin{verbatim}
>>> math.sqrt(2) / 2.0
0.707106781187
\end{verbatim}
%

\section{구성}
%Composition
\index{composition}

지금까지 변수와 수식 그리고 문장이라는 프로그램의 요소들을 독립적으로만
다루었기 때문에 구성 요소들을 화합하여 쓰는 것에 대해서는 다루지
않았다. 

프로그래밍 언어의 가장 유익한 기능 중 하나는 작은 단위의 필요한
부분들을 모아서 {\bf 구성(compose)}할 수 있다는 것이다.  예를 들면,
함수의 인자로 산술 연산자들을 포함하는 어떤 수식이라도 다 받을 수 있다. 


\begin{verbatim}
x = math.sin(degrees / 360.0 * 2 * math.pi)
\end{verbatim}
%
함수 호출도 할 수 있다. 

\begin{verbatim}
x = math.exp(math.log(x+1))
\end{verbatim}
%
거의 모든 곳에 값을 넣을 수 있으며 임의의 수식을 쓸 수 있다.  단, 한
가지 예외가 있다.  할당문의 왼쪽에는 변수의 이름이 있어야 한다.
왼쪽에 다른 어떤 수식이라도 오면 문법 오류가 발생한다 (이것에
대한 예외 상황을 이후에 살펴 보자).


\begin{verbatim}
>>> minutes = hours * 60                 # 맞음
>>> hours * 60 = minutes                 # 틀림!
SyntaxError: can't assign to operator
\end{verbatim}
%
\index{SyntaxError}
\index{exception!SyntaxError}


\section{새로운 함수의 추가}
%Adding new functions

지금까지는 Python이 제공하는 함수들만 사용했었는데, 새로운 함수도 추가
할 수 있다.  {\bf 함수 정의(function definition)}로 새로운 함수에
이름을 부여하고 그 함수를 호출했을 때 실행될 일련의 문장들을 지정할 수
있다.  
\index{function}
\index{function definition}
\index{definition!function}

여기 그 예가 있다.

\begin{verbatim}
def print_lyrics():
    print("I'm a lumberjack, and I'm okay.")
    print("I sleep all night and I work all day.")
\end{verbatim}
%
{\tt def}라는 키워드가 함수 정의를 표시한다.  이 함수의 이름은
\verb"print_lyrics"이다.  함수 이름에 대한 규칙은 변수명에 대한 규칙과
똑같다.  글자와 숫자 그리고 밑줄 표시는 사용 가능하지만 숫자가 첫
글자로 오면 안된다.  키워드로 예약되어 있는 단어들은 함수명으로 쓸 수
없으며 변수와 함수를 동일한 이름으로 짓는 것은 피해야 한다.
\index{def keyword}
\index{keyword!def}
\index{argument}

함수명 다음에 빈 괄호를 사용한 것은 이 함수가 아무런 인자도 사용하지
않는다는 것을 뜻한다.
\index{parentheses!empty}
\index{header}
\index{body}
\index{indentation}
\index{colon}

함수 정의의 첫 줄을 {\bf 헤더(header)}라고 부르고 그 나머지 부분을
{\bf 내용(바디, body)}라고 부른다.  헤더의 끝은 콜론이고 내용은
들여쓰기를 해야 한다.  관례적으로 항상 4개의 공백으로 들여쓰기를 한다.
내용에는 몇 개의 문장이든 포함될 수 있다.
%body 내용


\texttt{print}문에 문자열은 큰따옴표로 싸여있다. 작은 따옴표나
큰따옴표나 동일하다.  대부분의 사람들은 작은 따옴표를 사용하지만,
예외적으로 작은 따옴표(아포스트로피 또는 생략 기호)가 문자열 중에
포함된 경우에만 큰따옴표를 사용한다.

모든 인용부호(작은 따옴표와 큰따옴표)는 대체적으로 키보드의 엔터 키
옆에 있는 ``반듯한 인용부호''를 사용해야 한다.  이 문장에서 쓰고 있는
``휘어진 인용부호''는 Python에서 쓸 수 없다.

대화식 모드에서 함수 정의를 입력하면 인터프리터는 함수의 내용 부분에
점({\tt ...})을 표시하여 정의가 끝나지 않았다는 것을 표시한다.
\index{ellipses}

\begin{verbatim}
>>> def print_lyrics():
...     print("I'm a lumberjack, and I'm okay.")
...     print("I sleep all night and I work all day.")
...
\end{verbatim}
%
함수의 정의를 끝내려면 빈 줄을 입력하면 된다.  이렇게 정의된 함수 객체는
\verb"function"으로 분류된다.
\index{function type}
\index{type!function}

\begin{verbatim}
>>> print(print_lyrics)
<function print_lyrics at 0xb7e99e9c>
>>> type(print_lyrics)
<class 'function'>
\end{verbatim}
%
새로운 함수를 호출하는 문법은 내장된 함수들을 호출하는 방식과 똑같다. 

\begin{verbatim}
>>> print_lyrics()
I'm a lumberjack, and I'm okay.
I sleep all night and I work all day.
\end{verbatim}
%
이렇게 정의된 함수는 다른 함수 내에서 사용될 수 있다.  예를 들어 방금
출력한 후렴구를 반복적으로 호출하도록 \verb"repeat_lyrics"라는 함수를
작성해 보자


\begin{verbatim}
def repeat_lyrics():
    print_lyrics()
    print_lyrics()
\end{verbatim}
%
그리고 \verb"repeat_lyrics"을 호출해보자.

\begin{verbatim}
>>> repeat_lyrics()
I'm a lumberjack, and I'm okay.
I sleep all night and I work all day.
I'm a lumberjack, and I'm okay.
I sleep all night and I work all day.
\end{verbatim}
%
실제 노래가 이렇게 흘러가지는 않는다.


\section{정의와 활용}
%Definitions and uses
\index{function definition}

이전 절에서 사용한 코드를 다시 가져와 보자.  전체 코드는 다음과 같다. 

\begin{verbatim}
def print_lyrics():
    print("I'm a lumberjack, and I'm okay.")
    print("I sleep all night and I work all day.")

def repeat_lyrics():
    print_lyrics()
    print_lyrics()

repeat_lyrics()
\end{verbatim}
%
이 프로그램은 \verb"print_lyrics" 와 \verb"repeat_lyrics"라는 두 개의
함수를 정의하고 있다.  함수 정의는 다른 문장처럼 실행되지만, 실행의
결과로 함수 객체만 생성이 된다. 실제 함수 정의의 내용은 그 함수가
호출되기 전까지는 실행되지 않으며 함수 정의 자체는 어떠한 결과도 출력하지 않는다.
\index{use before def}

예상한 것처럼, 함수를 생성한 후에야 함수를 실행할 수 있다.  다시
말하면, 함수가 호출되기 이전에 함수가 정의되어야 한다.

연습을 해보자. 두 번쨰 함수의 마지막 줄을 이 프로그램의 가장 처음으로
옮겨보자.  프로그램을 실행해서 함수 정의보다 먼저 함수 호출 문장이
나타나면 어떤 오류 메시지를 출력하는지 보자.

함수 호출 문장을 원상복귀 한 후, 이번에는 \verb"repeat_lyrics"
다음에 \verb"print_lyrics"를 배치해보자.  이 상태로 프로그램을
실행시키면 어떻게 되나?



\section{실행의 흐름}
%Flow of execution
\index{flow of execution}

함수를 정의한 후에 사용되도록 만들려면 어떤 순서로 문장들이
실행되는지를 알아야 한다.  이를 {\bf 실행의 흐름(flow of
  execution)}이라고 부른다.


실행의 순서는 언제 프로그램의 첫 문장부터 시작된다.  가장 위에서부터
밑까지 문장은 한 번에 하나씩 실행된다.

함수 정의는 프로그램의 실행의 흐름을 변경시키지 않지만 함수 정의 내의
문장들은 해당 함수가 호출되기 이전에는 실행되지 않는다는 것을 기억해야
한다.

함수 호출은 실행 흐름의 우회로와 같다.  다음 문장을 실행시키는 대신
함수의 내용으로 이동하여 그 안의 문장들을 실행시킨 후 이전에 멈췄던
부분부터 실행한다.

매우 간단하게 들리겠지만, 함수가 또 다른 함수를 호출 할 수 있다는 것을
되새겨 보면 그렇지 않다는 것을 깨달을 것이다.  함수의 중간에서 다른
함수의 문장들을 호출할 수 있다.  그러는 중에 또 다른 함수를 호출할 수도
있다!

다행스러운 것은, Python은 현재 실행 중이던 위치를 파악하는 것에
능숙하다.  함수의 실행이 완료되면 호출했던 함수에서 마지막 실행 중이던
위치를 찾아 계속 진행할 수 있다.  마침내 프로그램의 끝에 도달하면
종료한다.

정리하면, 프로그램을 읽을 때에는 프로그램의 처음부터 한 줄 씩 읽으려 할
필요가 없다.  오히려 실행의 흐름을 따르는 것이 이해하기 더 쉬울 수 있다.


\section{매개 변수와 인자값}
%Parameters and arguments
\label{parameters}
\index{parameter}
\index{function parameter}
\index{argument}
\index{function argument}

어떤 함수들은 인자를 사용한다는 것을 보았다.  예를 들어 {\tt
  math.sin}을 쓸 때 숫자를 인자로 입력했었다.  어떤 함수는 하나 이상의
인자를 받기도 한다.  {\tt math.pow}는 밑수와 지수라는 두 개의 인자를
받는다.

함수 내에서는 전달 받은 인자를 {\bf 매개 변수(파라미터, parameter)}라는
변수로 할당된다.  인자를 받는 어떤 함수의 정의를 살펴보자.
\index{parentheses!parameters in}

\begin{verbatim}
def print_twice(bruce):
    print(bruce)
    print(bruce)
\end{verbatim}
%
이 함수는 전달 받은 인자를 {\tt bruce}라는 매개 변수로 할당한다.
함수가 호출되면 그 매개 변수가 무엇이든 간에 두 번 출력하고 있다.

다음의 함수는 출력 가능한 모든 값에 쓸 수 있다. 

\begin{verbatim}
>>> print_twice('Spam')
Spam
Spam
>>> print_twice(42)
42
42
>>> print_twice(math.pi)
3.14159265359
3.14159265359
\end{verbatim}
%
내장된 함수들에 적용되었던 구성에 관한 규칙들은 프로그래머가 정의한
함수들에도 똑같이 적용된다.   그러므로, \verb"print_twice"의 인자
값으로 어떤 수식이라도 쓸 수 있다.
\index{composition}
\index{programmer-defined function}
\index{function!programmer defined}

\begin{verbatim}
>>> print_twice('Spam '*4)
Spam Spam Spam Spam
Spam Spam Spam Spam
>>> print_twice(math.cos(math.pi))
-1.0
-1.0
\end{verbatim}
%
함수가 호출되기 전에 인자 값을 계산한다.  예에서 사용한 \verb"'Spam'*4"와 
{\tt math.cos(math.pi)}라는 수식은 한 번씩 계산된다.
\index{argument}

변수 역시도 인자로 사용할 수 있다.

\begin{verbatim}
>>> michael = 'Eric, the half a bee.'
>>> print_twice(michael)
Eric, the half a bee.
Eric, the half a bee.
\end{verbatim}
%
인자로 전달한 ({\tt michael})이라는 병수명은 함수를 정의할 때 사용한
매개 변수({\tt bruce})와는 아무런 관계가 없다.  호출되기 전 코드(호출자, caller) 어떤
값이었든 상관없다.  \verb"print_twice" 함수 내에서는 {\tt bruce}라
부른다.
%caller 호출자 


\section{변수와 매개 변수의 지역성}
%Variables and parameters are local
\index{local variable}
\index{variable!local}

함수 내에서 변수를 생성하면 {\bf 지역 또는 로컬(local)} 변수라고
부른다.  함수 내에서만 존재하기 때문이다.  예를 들어 보자.
\index{parentheses!parameters in}

\begin{verbatim}
def cat_twice(part1, part2):
    cat = part1 + part2
    print_twice(cat)
\end{verbatim}
%
이 함수는 두 개의 인자를 받아서 연결하고, 그 결과를 두 번 출력한다.  이
함수 정의를 쓰는 예를 살펴 보자.
\index{concatenation}

\begin{verbatim}
>>> line1 = 'Bing tiddle '
>>> line2 = 'tiddle bang.'
>>> cat_twice(line1, line2)
Bing tiddle tiddle bang.
Bing tiddle tiddle bang.
\end{verbatim}
%
\verb"cat_twice" 가 종료하면 함수 내에서 사용되었던 변수 {\tt cat}은
없어진다.  출력해보려고 하면 예외처리 된다.
\index{NameError}
\index{exception!NameError}

\begin{verbatim}
>>> print(cat)
NameError: name 'cat' is not defined
\end{verbatim}
%
매개 변수도 지역에서만 정의된다. 예를 들어 \verb"print_twice" 밖에서는
{\tt bruce}라는 것은 존재하지 않는다.
\index{parameter}


\section{스택 상태도}
%Stack diagrams
\label{stackdiagram}
\index{stack diagram}
\index{function frame}
\index{frame}

각 변수의 사용 범위 파악에 {\bf 스택 상태도(stack diagram)}을 그려보는
것이 도움이 된다.  상태도와 유사하게 스택 상태도는 각 변수의 값을
나타낸다.  추가적으로 각 변수가 어떤 함수에 포함되는지도 나타낸다.
\index{stack diagram}
\index{diagram!stack}

각 함수는 {\bf 프레임(frame)}이라는 단위로 구분된다.  여기서 프레임은
하나의 상자로 함수의 이름이 곁에 적혀 있고 상자 내부에는 변수와 매개
변수가 적혀 있다.  앞서 본 예제의 스택 상태도는
그림~\ref{fig.stack}에 나타나 있다.

\begin{figure}
\centerline
{\includegraphics[scale=0.8]{figs/stack.pdf}}
\caption{스택 상태도.}
\label{fig.stack}
\end{figure}

어떤 함수가 어떤 함수를 호출 했는지를 알아 볼 수 있도록 프레임들이
스택에 정리되어 있다.  이 예제에서는 \verb"print_twice" 함수는
\verb"cat_twice" 함수에 의해 호출되었으며 \verb"cat_twice" 함수는
\verb"__main__"에 의해 호출이 되었다.  \verb"__main__"라는 함수는
최상위 프레임에게 부여하는 특별한 이름이다.  함수 밖에서 생성된 변수는
\verb"__main__" 함수에 속해 있다.
\index{main}

각 매개 변수는 해당 인자가 갖고 있는 값과 똑같은 값을 갖고 있다.
그렇기 때문에 {\tt part1}가 같고 있는 값은 {\tt line1}이 갖고 있는 값고
똑같으며, {\tt part2}가 같고 있는 값은 {\tt line2}와 동일하다. 그리고,
{\tt bruce}는 {\tt cat}과 똑같다.

함수 호출 중에 오류가 발생하면 Python은 함수의 이름과 호출을 시도한
함수명을 출력한다.  그리고 다시 그 함수를 호출한 상위 프레임의 함수명을
출력한다.  이 과정을 \verb"__main__"에 도달 할 때까지 반복한다.

예를 들어, \verb"print_twice" 내에서 {\tt cat}를 접근하려고 시도 한다면
{\tt NameError}이라는 오류 메시지를 받을 것이다.

\begin{verbatim}
Traceback (innermost last):
  File "test.py", line 13, in __main__
    cat_twice(line1, line2)
  File "test.py", line 5, in cat_twice
    print_twice(cat)
  File "test.py", line 9, in print_twice
    print(cat)
NameError: name 'cat' is not defined
\end{verbatim}
%
이런 식으로 함수의 리스트를 보여주는 것을 {\bf 추적(트레이스백,
  traceback)}이라고 부른다.  오류가 발생했을 때 어떤 프로그램 파일에서
오류가 발생했는지 그리고 실행 중이던 함수와 오류를 일으킨 줄 번호에 대한
정보를 보여준다.  그리고 오류를 일으킨 코드도 보여준다.
\index{traceback}


트레이스백의 함수의 순서는 스택 상태도에서 나타난 프레임의 순서와
똑같다.  현재 실행 중이던 함수가 가장 아래에 표시된다.



\section{열매가 있는 함수들과 비어 있는 함수}
%Fruitful functions and void functions}
\index{fruitful function}
\index{void function}
\index{function, fruitful}
\index{function, void} 

수학 함수들과 같은 우리가 사용해본 함수들은 결과를 리턴한다.  이런
종류의 함수를 구분하는 좋은 이름이 따로 없어서 {\bf 열매가 있는
  함수(fruitful function)}이라고 부른다.  \verb"print_twice"와 같은
류의 다른 함수들은 어떤 동작을 하지만 결과 값을 리턴하지는 않는다.
이런 류는 {\bf 비어 있는 함수(void function)}라고 부른다.

열매가 있는 함수를 호출하면 거의 모든 경우에 리턴받은 결과를 활용하기를
원한다.  예를 들어, 그 결과에 변수를 할당하거나 수식의 일부로
사용하기도 한다.

\begin{verbatim}
x = math.cos(radians)
golden = (math.sqrt(5) + 1) / 2
\end{verbatim}
%
대화식 모드에서 함수를 호출할 때  Python은 결과를 표시한다. 

\begin{verbatim}
>>> math.sqrt(5)
2.2360679774997898
\end{verbatim}
%
스크립트로 실행했을 때, 열매가 있는 함수를 그 자체로만 호출하면 리턴 받은 결과는 영원히 잃어버리고 만다!

\begin{verbatim}
math.sqrt(5)
\end{verbatim}
%
이 스크립트는 루트 5를 계산하지만 결과를 저장하지도 표시하지도 않기
때문에 그렇게 유용하지는 않다.
\index{interactive mode}
\index{script mode}

비어 있는 함수는 화면에 무언가를 표시하거나 다른 어떤 영향이 있는
것처럼 보이지만 리턴할 결과 값이 없다.  만약 결과 값을 변수에
할당하려고 시도한다면 {\tt None}이라는 특수한 값을 돌려 받는다.
\index{None special value}
\index{special value!None}

\begin{verbatim}
>>> result = print_twice('Bing')
Bing
Bing
>>> print(result)
None
\end{verbatim}
%
{\tt None}이라는 값은 \verb"'None'"이라는 문자열과 같은 것이 아니다.
그 자체가 특수한 값을 같은 또 다른 분류이다.

\begin{verbatim}
>>> type(None)
<class 'NoneType'>
\end{verbatim}
%
지금까지 우리가 작성한 모든 함수들은 모두 비어 있는 함수들이었다.  이제
앞으로 다룰 장들에서 열매가 있는 함수들을 작성하기 시작할 것이다.
\index{NoneType type}
\index{type!NoneType}


\section{왜 함수인가?}
%Why functions?
\index{function, reasons for}

프로그램의 내용을 왜 함수들로 나눠야 하는지에 대한 이유가 아직은 명확하지 않을 것이다.  그럴만한 이유는 많이 있다. 

\begin{itemize}

\item 새로운 함수를 생성하면 문장들의 묶음에 이름을 지어줄 수 있다.
  그러면 프로그램을 읽거나 디버깅하기가 쉬워진다.

\item 함수들을 활용하면 반복적인 코드들을 제거할 수 있기 때문에
  프로그램의 길이가 짧아진다.  나중에 반복되는 코드를 수정을 할 일이
  생긴다면 한 곳에서만 수정을 하면 된다. 

\item 긴 프로그램을 함수들로 나누어 놓으면 한 번에 한 부분씩 디버깅을
  할 수 있으며, 디버깅이 완료되면 제대로 동작하는 전체로 다시 모을
  수있다.

\item 잘 설계된 함수들은 여러 프로그램들에 유용하게 사용될 수 있다.
  하나를 잘 작성해 놓고 디버깅을 해 놓으면 다른 곳에서도 그 부분을
  재활용할 수 있다.

\end{itemize}


\section{디버깅}
%Debugging

가장 중요한 기술 중에 하나인 디버깅 기술을 가져야 한다.  좌절스럽게
만들 때도 있지만 디버깅은 지적으로 풍부하고 도전적고 또한 프로그래밍을
흥미롭게 만드는 것이기도 하다.
\index{experimental debugging}
\index{debugging!experimental}

디버깅은 어떤 면에서 탐정이 수사를 펼치는 것과 같다.  현재 발생한
결과를 이끌어 낸 과정과 사건들을 유추할 있는 여러 단서들과 만나게
된다.

디버깅은 실험에 기반한 과학과도 같다.  무엇인 잘못된 것 같단 아이디어가
떠 오르면 프로그램을 수정해보고 다시 시도 해 보면 된다.  가정이 옳다면
수정에 대한 결과를 예측해 볼 수 있게 되고 동작하는 프로그램으로 한 걸음
더 가까이 다가갈 수 있게 된다.  만약 가정이 틀렸다면 다른 방법을
시도해봐야 한다.  셜록 홈즈가 말하듯이 ``불가능한 것들을 하나씩 제거한 뒤에
남은 것이 아무리 불가능해 보일지라도 그게 사실일 것이다.''(A. 코난 도일,
{\em 4개의 서명})
\index{Holmes, Sherlock}
\index{Doyle, Arthur Conan}

어떤 이들에게는 프로그래밍과 디버깅은 똑같은 것이다.  그들에게
프로그래밍이라는 것은 원하는 동작을 할 때까지 프로그램을 조금씩
디버깅해 나가는 것이다.  동작 가능한 프로그램을 먼저 작성한 후에 조금씩
변형을 만들어 내고 디버깅하는 것이 기본 접근 방식이다. 

예를 들어, 리눅스라는 운영체제는 지금은 수 백만 줄의 코드로
이루어져있지만, 최초에는 리누스 토발즈가 인텔 80386 CPU를 사용하는
간단한 프로그램으로 시작되었다.  래리 그린필드에 의하면 ``리누스의 초기
프로젝트 중에 하나는 AAAA를 BBBB로 변환하는 프로그램을 만드는
것이었다.  이게 나중에는 발전하여 리눅스가 되었다.''  ({\em The Linux
  Users' Guide(리눅스 사용자 가이드)} 베타 1판).
\index{Linux}


\section{용어 해설}
%Glossary

\begin{description}

\item[(function):] 이름이 있는 일련의 문장들로 유용한 작업을 수행한다.
  함수는 인자를 받을 수도 안 받을 수도 있으며 실행하였을 때 결과 돌려
  줄 수도 있고 그렇지 않을 수도 있음
\index{function}

\item[함수 정의(function definition):] 새로운 함수를 생성하는 문장으로
  함수의 이름과 매개 변수들을 지정하고 그리고 문장들을 포함함
\index{function definition}

\item[함수 객체(function object):] 함수 정의에 의해 생성되는 값.
  함수의 이름이 함수 객체를 가리키는 변수임 
  \index{function definition}

\item[헤더(header):] 함수 정의의 첫 줄 
\index{header}

\item[내용(바디, body):] 함수 정의 내의 일련의 문장들
\index{body}

\item[매개 변수(parameter):] 함수 내에서 인자로 전달된 값을 가리키는 이름
\index{parameter}

\item[함수 호출(function call):] 함수를 실행시키는 문장.  함수의 이름과
  괄호로 싸여있는 인자들의 목록으로 구성되어 있음
\index{function call}

\item[인자(argument):] 함수가 호출되었을 때 함수에 제공되는 값.  이
  값은 해당 함수에 매개 변수에 할당됨
\index{argument}

\item[지역 변수(local variable):] 함수 내에서 정의된 변수.  지역 변수는
  정의된 함수 내에서만 사용될 수 있음
\index{local variable}

\item[리턴 값(return value):] 함수의 결과.  함수 호출이 수식의 일부로
  사용되었다면 리턴 값은 수식이 사용하는 값이 됨
\index{return value}

\item[열매가 있는 함수(fruitful function):]  결과를 리턴하는 함수
\index{fruitful function}

\item[비어 있는 함수(void function):] {\tt None}을 리턴하는 함수
\index{void function}

\item[{\tt None}:]  비어 있는 함수가 리턴하는 특수한 값
\index{None special value}
\index{special value!None}

\item[모듈(module):] A file that contains a
collection of related functions and other definitions.
\index{module}

\item[읽어들이기 문장(import statement):] 모듈 파일을 읽어서 모듈
  객체를 생성하는 문장
\index{import statement}
\index{statement!import}

\item[모듈 객체(module object):] {\tt import}문으로 생성되는 값으로
  모듈에서 정의한 값들을 사용할 수 있도록 함
\index{module}

\item[닷 표기법(dot notation):] 다른 모듈의 함수를 호출하는 문법으로
  모듈의 이름과 함수의 이름을 점(닷, 구두점)으로 연결하는 표기법
\index{dot notation}

\item[구성(composition):] 더 큰 수식의 일부로 수식을 쓰거나 더 큰
  문장의 일부로 문장을 작성하는 것
\index{composition}

\item[실행의 흐름(flow of execution):]  문장들의 실행 순서
\index{flow of execution}

\item[스택 상태도(stack diagram):] 함수가 사용하고 있는 스택의 상태를
  그림으로 표기하는 방법으로 변수와 각 변수가 가리키는 값을 나타나냄
\index{stack diagram}

\item[프레임(frame):] 스택 상태도의 상자로 함수 호출을 나타냄.  상자는
  지역 변수와 함수의 매개 변수를 포함함.
\index{function frame}
\index{frame}

\item[트레이스백(추적, traceback):] 실행 중인 함수들의 목록으로 예외가
  발생할 때 출력이 됨
\index{traceback}


\end{description}


\section{연습 문제}
%Exercises

\begin{exercise}
\index{len function}
\index{function!len}

\verb"right_justify"(오른쪽 정렬이라는 의미)라는 이름과 문자열 {\tt
  s}를 매개 변수로 사용하는 함수를 작성하여라.  문자열 앞에 충분한
공백을 두어서 문자열의 마지막 글자가 70번째 열에 보이도록 하라.

\begin{verbatim}
>>> right_justify('monty')
                                                                 monty
\end{verbatim}

힌트: 문자열 연결과 반복을 사용하라.  Python은 {\tt len}이라는 내장
함수를 갖고 있다.  이 함수는 문자열의 길이를 리턴한다.
\verb"len('monty')"의 값은 5이다.

\end{exercise}


\begin{exercise}
\index{function object}
\index{object!function}

함수 객체는 값이기 때문에 변수에 할당할 수도 있고 인자로 전달할 수도
있다.  예를 들어, \verb"do_twice"는 함수 객체를 인자로 전달 받아 두 번
호출하는 함수이다.

\begin{verbatim}
def do_twice(f):
    f()
    f()
\end{verbatim}

다음 예제는 \verb"do_twice"를 사용하여 \verb"print_spam"라는 함수를 두 번 호출한다.


\begin{verbatim}
def print_spam():
    print('spam')

do_twice(print_spam)
\end{verbatim}

\begin{enumerate}

\item 이 예제를 스크립트로 작성하여 제대로 동작하는지 확인해보라

\item \verb"do_twice"를 두 개의 인자를 받도록 수정하여라.  함수 객체와
  값을 전달받아 해당 함수는 두 번 호출하고 값은 인자로 전달하도록
  만들라.

\item 이 장의 처음에 다뤘던 \verb"print_twice" 함수의 정의를 복사하여
  스크립트에 포함시켜라

\item 수정한 \verb"do_twice" 함수를 사용하여 \verb"print_twice"를 두 번
  호출하여라.  이 때, \verb"print_twice" 함수의 인자는 \verb"'spam'"을
  사용하여라.

\item 함수 객체와 값을 전달 받아 해당 함수를 4번 호출하고 값은 매개
  변수로 사용하는 \verb"do_four"라는 새로운 함수를 정의하여라.  이
  함수의 내용에는 네 개가 아니라 두 개의 문장만 있어야 한다.

\end{enumerate}

해답: \url{http://thinkpython2.com/code/do_four.py}.

\end{exercise}



\begin{exercise}

  메모: 이 연습 문제는 지금까지 우리가 배운 문장들과 기능들만을
  사용하여 해결해야 한다.

\begin{enumerate}

\item 아래와 같은 격자 무늬를 만드는 함수를 작성하라:
\index{grid}

\begin{verbatim}
+ - - - - + - - - - +
|         |         |
|         |         |
|         |         |
|         |         |
+ - - - - + - - - - +
|         |         |
|         |         |
|         |         |
|         |         |
+ - - - - + - - - - +
\end{verbatim}
%
힌트: 한 줄에 하나 이상의 값을 출력하려면 값을 쉼표로 구분지으면 된다. 

\begin{verbatim}
print('+', '-')
\end{verbatim}
%
기본적으로 {\tt print}는 출력이 끝나면 다음 줄로 넘어간다.  이 같은
동작을 수정하려면 다음과 같은 방법으로 동작을 변경할 수 있다.

\begin{verbatim}
print('+', end=' ')
print('-')
\end{verbatim}
%
이 문장들의 결과는 \verb"'+ -'"이다. 

{\tt print}문에 인자가 없다면 현재 출력 중이던 줄을 종료하고 다음
줄로 넘어간다.

\item 네 개의 행과 열이 있는 격자 무늬를 그리는 함수를 작성하여라. 

\end{enumerate}

해답: \url{http://thinkpython2.com/code/grid.py}.  

출처: 이 예제는 스티브 오우알린의 ({\em Practical C Programming, 3판},
오라일리 출판사, 1997), 연습 문제를 기초로 하였다.

\end{exercise}





\chapter{사례 연구: 인터페이스 설계}
%Case study: interface design
\label{turtlechap}

이 장은 함께 동작하는 함수들을 설계하는 과정을 사례를 들어 살펴보자. 

{\tt turtle}이라는 모듈을 소개하고자 한다.  이 모듈은 거북이 그래픽스를
사용하여 이미지를 생성해 준다.  Python을 설치하면 {\tt turtle} 모듈이
내장되어 있지만 PythonAnywhere을 사용 중이라면 이 모듈을 사용할 수
없다(최소한 이 글을 쓰는 중에는 불가능했다).

Python을 컴퓨터에 설치했다면 여기의 예제들을 실행할 수 있다.  아직 설치
안했다면 이제는 설치를 해보자.  설치하는 방법은
\url{http://tinyurl.com/thinkpython2e}에 나와 있다.

이 장에서 사용한 예제들의 코드는
\url{http://thinkpython2.com/code/polygon.py} 에서 얻을 수 있다.


\section{Turtle 모듈}
%The turtle module
\label{turtle}

{\tt turtle} 모듈이 포함되어 있는지 확인해보고 싶으면 Python을
인터프리터를 열어서 다음을 입력해보자.


\begin{verbatim}
>>> import turtle
>>> bob = turtle.Turtle()
\end{verbatim}

이 코드를 실행하면 거북이를 나타내는 작은 화살표가 포함된 새로운 창이
열린다.  확인했다면 창을 닫자.

{\tt mypolygon.py}이라는 파일을 생성해서 다음의 내용을 입력해보자. 


\begin{verbatim}
import turtle
bob = turtle.Turtle()
print(bob)
turtle.mainloop()
\end{verbatim}
%
{\tt turtle} 모듈 (소문자 't')은 {\tt Turtle}(대문자 'T')이라는 함수를
제공한다.  이 함수는 Turtle 객체를 생성하는데 {\tt bob} 이라는 변수에
할당했다.  {\tt bob}을 출력하면 다음의 내용이 표시된다.

\begin{verbatim}
<turtle.Turtle object at 0xb7bfbf4c>
\end{verbatim}
%
{\tt bob} 이라는 변수는 {\tt turtle} 모듈에 정의되어 있는 {\tt
  Turtle}이라는 객체를 가리킨다.

마지막으로 \verb"mainloop"이 하는 일은 사용자가 무언가를 하기까지 새로
뜬 창을 대기시킨다.  지금의 경우에는 창을 닫는 것 외에는 더 할 수 있는
일이 없기는 한다.

\texttt{Turtle}을 생성했다면 {\bf 메소드(method)}를 호출하여 거북이를
창에서 움직일 수 있다.  메소드는 함수와 비슷한데 사용하는 문법이 약간
다르다.  거북이를 전진시켜 보자.

\begin{verbatim}
bob.fd(100)
\end{verbatim}
%
{\tt fd}라는 메소드는 {\tt bob}이라는 Turtle 객체와 연결되어 있다.
메소드의 호출은 마치 요청하는 것과 같다. {\tt bob}에게 전진 요청을 하는
것이다.

{\tt fd}의 인자는 픽셀단위의 거리이기 때문에 디스플레이 장치에 따라 다를 수 있다. 

\texttt{Turtle} 객체에서 호출 할 수 있는 메소드는 세 가지이다.  {\tt
  bk}는 뒤로 가기, {\tt lk}와 {\tt rk}는 각각 왼쪽과 오른쪽으로
회전하기 메소드이다. {\tt lk}와 {\tt rk}의 인자 값은 각도이다.

각 \texttt{Turtle}은 펜을 갖고 있는데, 이 펜을 들거나 내려 놓을 수
있다.  펜을 내리면 \texttt{Turtle}이 이동한 흔적을 남길 수 있다.  {\tt
  pu}와 {\tt pd}는 각각 펜을 들기와 펜 내리기를 뜻한다.

직각을 그리려면 다음의 줄들을 프로그램에 추가하자(위치는 {\tt bob}을
생성한 줄과 \verb"mainloop"를 호출하는 줄 사이에 삽입하면 된다.

\begin{verbatim}
bob.fd(100)
bob.lt(90)
bob.fd(100)
\end{verbatim}
%
이 프로그램을 실행하면, {\tt bob}이 동쪽으로 가다가 북으로 이동하고, 그
길에 두 개의 줄을 남긴다.

이제 프로그램을 수정하여 정사각형을 그려 보자.  정사각형 그리기를
성공하기 전까지는 다음으로 넘어가지 말자.

%\newpage

\section{간단한 반복}
%Simple repetition
\label{repetition}
\index{repetition}

정사각형을 그리기 위해 아마도 다음과 같이 작성했을 것이다. 

\begin{verbatim}
bob.fd(100)
bob.lt(90)

bob.fd(100)
bob.lt(90)

bob.fd(100)
bob.lt(90)

bob.fd(100)
\end{verbatim}
%
{\tt for}문을 쓰면 훨씬 간결하게 똑같은 일을 해 낼 수 있다.  다음의
예제를 {\tt mypolygon.py}에 추가하고 다시 실행시켜보자.
\index{for loop}
\index{loop!for}
\index{statement!for}

\begin{verbatim}
for i in range(4):
    print('Hello!')
\end{verbatim}
%
그 결과는 다음과 같은 것이다.

\begin{verbatim}
Hello!
Hello!
Hello!
Hello!
\end{verbatim}
%
이 예제가 {\tt for}문을 쓰는 가장 간단한 형태이다.  나중에 좀 더
살펴보자.  지금의 예제만으로도 정사각형을 그리는 프로그램을 더 단순하게
다시 작성할 수 있을 것이다.  성공할 때까지 시도해보자.

여기에 {\tt for}문을 써서 정사각형을 그리는 코드가 있다. 


\begin{verbatim}
for i in range(4):
    bob.fd(100)
    bob.lt(90)
\end{verbatim}
%
{\tt for}문의 문법은 함수 정의와 유사하다.  콜론으로 끝이 나는
헤더와 들여쓰기된 내용으로 구성된다.  내용에는 문장이 몇 개라도 포함될
수 있다.

{\tt for}문은 {\bf 반복문(루프, loop)}이라고도 불린다.  실행의
흐름이 내용을 한 번 다 실행하고 다시 처음으로 돌아가서 실행하기
때문이다.  이 경우에는 내용을 4번 반복한다.
\index{loop}

이 버전은 이전의 정사각형을 그리는 코드와 약간 다르다.  정사각형의
마지막 면을 그리고 난 뒤에 회전을 한 번 더 하기 때문이다.  추가 회전이
있어 시간은 더 들지만 반복문 내에서 동일한 동작을 반복하기 때문에 코드
복잡도는 낮아 지는 장점이 있다.  반복문을 사용하는 버전은 거북이를
처음에 시작했던 위치와 방향으로 다시 돌려놓는 기능도 한다.  

\section{연습 문제}
%Exercises

TurtleWorld를 활용하는 연습 문제들을 풀어 보자.  재미있을 뿐만 아니라
의미도 있다.  하나씩 해결 할 때마다 어떤 의미를 갖는지 생각해보자.

다음 절에는 연습 문제들의 해답이 있으니 다 풀기 전까지는 해답을 보지
말자(최소한 시도는 해보자).


\begin{enumerate}

\item {\tt square}라는 이름의 함수를 작성하고 거북이를 대신할 매개
  변수는 {\tt t}로 하자.  거북이로 정사각형을 그려보자.

  {\tt bob}을 {\tt square}의 인자로 전달하여 호출하도록 수정하고
  프로그램을 다시 실행해보자.


\item {\tt square} 함수에 {\tt length}라는 매개 변수를 추가 해보자.
  정사각형의 변이 {\tt length}가 되도록 수정하고, 함수를 호출하는
  문장도 인자를 두 개를 받도록 수정하자.  프로그램을 재실행해보라.
  {\tt length}에 다양한 값을 넣어 프로그램을 실험해보자.


\item {\tt square} 함수를 복사해서 {\tt polygon}이라는 함수를 만들자.
  그리고 {\tt n}이라는 매개 변수를 추가한 후 함수의 내용에 면이 n인
  다각형을 그리도록 수정하자.  힌트: 다각형의 외각은
  $360/n$도이다.
  \index{polygon function} \index{function!polygon}

\item 거북이({\tt t})와 반지름({\tt r})을 매개 변수로 하는 {\tt
    circle}이라는 함수를 작성해보자.  이 때 {\tt polygon} 함수를
  활용하여 원에 근사한 도형을 그려보자.  {\tt r}의 변경해보면서 함수를
  검사해보자.
  \index{circle function} \index{function!circle}

  힌트: {\tt length * n = 둘레}를 만족하도록 값을 정하면 된다.

\item {\tt circle}보다 좀 더 일반적인 {\tt arc}라는 함수를 만들어
  보자.  이 함수는 추가로 {\tt angle}이라는 매개 변수를 받는다.  {\tt
    angle}는 몇 도 크기의 호를 그릴지 정하는 매개 변수이다.  단위는
  도이다.  {\tt angle=360}이라면 완전한 원을 그리게 된다.
\index{arc function}
\index{function!arc}

\end{enumerate}


\section{캡슐화}
%Encapsulation

첫 번째 연습 문제에서 정사각형 그리기 코드를 함수 정의로 만들고,
거북이를 매개 변수로 하여 함수를 호출하라고 했다.  여기 그 해답이
있다.


\begin{verbatim}
def square(t):
    for i in range(4):
        t.fd(100)
        t.lt(90)

square(bob)
\end{verbatim}
%
가장 안쪽의 문장의 {\tt fd}와 {\tt lt}는 두 번 들여쓰기가 되어 {\tt
  for} 반복문 안에 있다는 것과 반복문은 함수 정의 내에 있다는 것을
나타낸다.  그 다음 줄의 {\tt square(bob)}문은 왼쪽 끝에 있기 때문에
{\tt for} 반복문이나 함수 정의 내에 있지 않다는 것을 뜻한다.

함수 내의 {\tt t}는 {\tt bob}과 동일한 거북이를 가리키기 때문에 {\tt
  t.lt(90)}과 {\tt bob.lt(90)}는 서로 똑같은 결과를 갖는다.  그렇다면
매개 변수를 {\tt bob}이라 부르지 않는 이유는 무엇일까?  그 이유는 {\tt
  t}라고 쓰면 {\tt bob} 외에도 다른 거북이 객체들을 받을 수 있기
때문이다. 두 번째 거북이를 만들고 {\tt square}의 인자로 전달 할 수도
있을 것이다.


\begin{verbatim}
alice = turtle.Turtle()
square(alice)
\end{verbatim}
%
함수 내에 어떤 코드를 넣는 것을 보고 {\bf 캡슐화(encapsulation)}이라고
한다.  캡슐화의 장점 중 하나는 코드에 이름을 붙일 수가 있다는 것이다.
또 다른 장점은 코드의 재활용이다.  함수의 내용을 복사 붙이기 보다 함수
호출이 훨씬 간결하다!  
\index{encapsulation}


\section{일반화}
%Generalization

그 다음 과정은 {\tt square} 함수에 매개 변수 {\tt length}를 추가하는
것이다.  해답은 다음과 같다.


\begin{verbatim}
def square(t, length):
    for i in range(4):
        t.fd(length)
        t.lt(90)

square(bob, 100)
\end{verbatim}
%
함수에 매개 변수를 추가하는 것을 {\bf 일반화(generalization)}이라
한다.  함수를 좀 더 범용적으로 쓸 수 있기 때문이다.  이전의 버전에서는
\texttt{square}는 언제나 동일한 크기를 가졌지만, 이번 버전에서는 길이를
변경할 수 있게 되었다.
\index{generalization}

다음 과정도 마찬가지로 일반화다.  정사각형만 그리는 대신 다각형를
그릴 수 있는 {\tt polygon}을 만드는 것이다.  해답을 살펴보자.

\begin{verbatim}
def polygon(t, n, length):
    angle = 360 / n
    for i in range(n):
        t.fd(length)
        t.lt(angle)

polygon(bob, 7, 70)
\end{verbatim}
%
이 예제에서는 면의 길이기 70인 정칠각형을 그렸다. 

Python 2를 사용 중이라면 정수 나눗셈으로 인해 {\tt angle}의 값이 다를
수 있다.  간단한 해결책으로 {\tt angle = 360.0 / n}으로 계산식을 바꿀
수 있다.  분자가 소수점이기 때문에 결과도 소수점이 된다.
\index{Python 2}

함수가 여러 개의 숫자를 인자로 받는다면 그 값이 어떤 값이 어떤 순서로
써야하는지 잊기 쉽다.  그런 경우에는 이름과 매개 변수를 인자 목록에
함께 적으면 좋다.


\begin{verbatim}
polygon(bob, n=7, length=70)
\end{verbatim}
%
매개 변수의 이름을 ``키워드''로서 포함시켰기 때문에 {\bf 키워드
  인자}라고 부른다({\tt while}이나 {\tt def}와 같은 Python의 키워드와
혼돈하면 안된다.)
\index{keyword argument}
\index{argument!keyword}


이 문법이 프로그램을 좀 더 읽기 쉽게 만든다.  또한 인자와 매개 변수가
함수에서 어떻게 연관있는지를 알려주기도 한다.  함수를 호출하면 인자
값이 매개 변수에 할당된다는 것이 명확해진다.



\section{인터페이스 설계}
%Interface design

그 다음 과정은 반지름을 매개 변수 {\tt r}로 하는 {\tt circle} 함수를
작성하는 것이다.  이 해법은 {\tt polygon} 함수를 이용해 50각형을 그린다.

\begin{verbatim}
import math

def circle(t, r):
    circumference = 2 * math.pi * r
    n = 50
    length = circumference / n
    polygon(t, n, length)
\end{verbatim}
%
첫 줄은 반지름을 {\tt r}로 하는 원의 둘레를 $ 2 \pi r$로 계산한다.
{\tt math.pi}를 쓰기 때문에 {\tt math} 모듈을 불러와야 한다.
{\tt import}문은 관례적으로 항상 스크립트의 첫 줄에 적는다.

근사한 원의 둘레는 {\tt n}개의 선으로 표현하였고, {\tt length}는 그
선의 길이를 나타낸다.  결과적으로 {\tt polygon}은 50각형을 그려서
반지름 {\tt r}인 원을 근사화하고 있다.

이 해법의 한계점은 {\tt n}이 상수라는 것이다.  그렇기 때문에 매우 큰
원을 그릴 때는 다각형의 면이 매우 길어 질 것이고 아주 작은 원이라면
매우 짧은 선을 그리는 시간 낭비를 한다.  이를 해결하려면 {\tt n}을 매개
변수로 받으 일반화를 해야 한다. 인터페이스가 좀 복잡해지겠지만, {\tt
  circle} 함수를 부르는 사용자가 제어권을 갖도록 한다.
\index{interface}

{\bf 인터페이스(interface)}는 함수를 어떻게 쓸 수 있는지 알려주는
요약이다.  어떤 매개 변수가 있고, 함수가 무엇을 하는지 그리고 어떤
결과를 리턴하는지 알려준다.  인터페이스가 ``단순하면'' 불필요한 세부
내용을 신경 쓰지 않고 원하는 것을 얻을 수 있다.

이 예제에서는 그리려는 원의 크기를 정하기 위해 {\tt r}이 인터페이스에
포함되어 있다.  {\tt n}은 원이 어떻게 표현되어야 하는가에 대한 정보를
담고 있기 때문에 적절하지는 않다.

인터페이스를 복잡하게 만드는 대신 {\tt n}의 값을 둘레 {\tt
  circumference}에 비례하도록 적절하게 정하는게 더 낫다.



\begin{verbatim}
def circle(t, r):
    circumference = 2 * math.pi * r
    n = int(circumference / 3) + 3
    length = circumference / n
    polygon(t, n, length)
\end{verbatim}
%
원을 이루는 선의 개수를 {\tt circumference/3}에 가까운 정수로 바꿨다.
한 선의 길이가 대략 3이 되도록 하였다.  원처럼 보일 수 있도록 충분히
작고 효율적일 만큼 충분히 크며 어떤 원에도 무난한 길이이다.


그리고 삼격형보다 작은 다각형은 만들지 않도록 {\tt n}에 3을 더했다. 


\section{리팩터링}
%Refactoring
\label{refactoring}
\index{refactoring}

{\tt circle} 함수를 작성할 때 {\tt polygon} 함수를 재사용할 수 있었던
이유는 다각형으로 원을 근사화할 수 있기 때문이다.  {\tt arc}의 경우는
{\tt polygon}나 {\tt circle}를 그대로 가져다 쓸 수 없다.

대안으로 {\tt polygon}의 사본에서 시작해서 {\tt arc}로 내용을 변형하는
방법이 있다.  그 결과는 다음과 같다.

\begin{verbatim}
def arc(t, r, angle):
    arc_length = 2 * math.pi * r * angle / 360
    n = int(arc_length / 3) + 1
    step_length = arc_length / n
    step_angle = angle / n
    
    for i in range(n):
        t.fd(step_length)
        t.lt(step_angle)
\end{verbatim}
%
이 함수의 하반부는 {\tt polygon}과 유사하게 생겼다.  그렇지만, {\tt
  polygon}의 인터페이스를 변형하기 전에는 쓸 수가 없다.  각도가 {\tt
  polygon} 함수의 세 번째 인자가 되도록 인터페이스를 수정할 수 있을
것이다.  그렇게 되면 {\tt polygon}이라는 이름은 부적절 해진다!  대신
변형된 이 함수의 이름을 {\tt 폴리라인(polyline)}이라 부르자.

\begin{verbatim}
def polyline(t, n, length, angle):
    for i in range(n):
        t.fd(length)
        t.lt(angle)
\end{verbatim}
%
이제 {\tt polygon}와 {\tt arc}가 {\tt polyline}를 사용하도록 변형해보자.

\begin{verbatim}
def polygon(t, n, length):
    angle = 360.0 / n
    polyline(t, n, length, angle)

def arc(t, r, angle):
    arc_length = 2 * math.pi * r * angle / 360
    n = int(arc_length / 3) + 1
    step_length = arc_length / n
    step_angle = float(angle) / n
    polyline(t, n, step_length, step_angle)
\end{verbatim}
%
이제 {\tt arc}를 쓰도록 {\tt circle}를 수정할 수 있다. 

\begin{verbatim}
def circle(t, r):
    arc(t, r, 360)
\end{verbatim}
%
인터페이스를 개선하고 코드의 재사용을 높이기 위한 프로그램의 재배치
과정을 {\bf 리팩터링(refactoring)}이라 부른다.  지금의 경우, {\tt
  arc}와 {\tt polygon}에 유사한 코드 부분이 있었기 때문에 {\tt
  polyline}으로 ``공통 부분을 분리''했다.
\index{refactoring}

미리 계획하여 {\tt polyline}을 작성했더라면 리팩터링 과정이 필요 없었을
것이다.  하지만, 대체적으로 프로젝트 초반부에는 모든 인터페이스를
설계하기에는 정보가 충분하지 않다.  코딩을 시작하면 그제야 문제에 대해
좀 더 이해하게 된다.  때로는 리팩터링이 무언가를 배웠다는 것을
반증해주기도 한다.


\section{개발 계획}
%A development plan
\index{development plan!encapsulation and generalization}

{\bf 개발 계획}은 프로그램을 만드는 과정이다.  이 사례 연구에서 사용한
``캡슐화와 일반화''도 프로그램 개발 과정의 일부다.  우리가 사용한
과정을 정리해보자.

\begin{enumerate}

\item 함수 정의가 없는 작은 프로그램으로부터 시작한다.

\item 동작하는 프로그램을 만들고 나면 서로 상관있는 코드들을 구분한다.
  그 코드를 함수로 캡슐화하고 이름을 붙있다.

\item 함수에 적절한 매개 변수를 추가해서 일반화를 한다. 

\item 동작하는 함수가 될 때까지 과정 1에서 3을 반복한다.  재입력(그리고
  디버깅)을 하지 않도록 동작하는 코드는 복사 붙이기 한다.

\item 리팩터링을 통해 프로그램을 개선할 수 있는지 검토한다.  예를
  들어, 유사한 코드가 여기 저기에 보이면 일반화된 함수로 리팩터링할 수
  있는지 생각해본다.

\end{enumerate}

이 과정에는 몇 가지 단점이 있다. 그 대안들은 이후에 살펴볼 것이다.
그래도 이 방법은 프로그램을 함수로 나누는 어떻게 나눌지 개발 전에 미리
알 수 없는 경우라면 충분히 의미있다.  개발하면서 설계도 같이 할 수 있기
때문이다.


\section{설명 문자열}
%docstring
\label{docstring}
\index{docstring}

함수의 시작 부분에 인터페이스를 설명하는 글을 {\bf 설명
  문자열(docstring)}이라 한다(doc은 documentation의 약어이다.  예를
살펴보자.

\begin{verbatim}
def polyline(t, n, length, angle):
    """Draws n line segments with the given length and
    angle (in degrees) between them.  t is a turtle.
    """    
    for i in range(n):
        t.fd(length)
        t.lt(angle)
\end{verbatim}
%
관례상 모든 설명 문자열은 멀티라인 문자열이라고도 불리는 세 개의
큰따옴표로 싸여있다.  세 개의 큰따옴표는 여러 줄에 걸친 문자열을
표현하는데 사용된다.
\index{quotation mark}
\index{triple-quoted string}
\index{string!triple-quoted}
\index{multiline string}
\index{string!multiline}

이 함수를 사용하는데 간결하기는 하지만 꼭 필요한 정보를 포함하고 있다.
설명 문자열은 함수가 어떤 일을 하는지 요약한다(상세 동작 설명은
제외한다).  매개 변수가 함수의 동작에 미치는 영향과 각 매개 변수는 어떤
데이터 형을 갖는지를 설명한다(당연한 경우는 제외한다).

이런 문서를 남기는 것은 인터페이스 설계에 있어 중요한 부분이다.  잘
설계된 인터페이스라면 간단히 설명할 수 있어야 한다.  어떤 함수를
설명하는데 애를 먹는다면 인터페이스에 개선의 여지가 있다는 말이다.


\section{디버깅}
%Debugging
\index{debugging}
\index{interface}

인터페이스는 함수와 그 함수를 호출한 함수간의 계약과 같다.  호출한
함수는 특정 매개 변수를 전달할 것을 그리고 함수는 그에 맞는 작업을
한다는 것을 약속한다.

예를 들면, {\tt polyline}은 거북이 객체 {\tt t}, 정수 {\tt n}, 양수
{\tt length}, 각도를 나타나는 숫자 {\tt angle}이라는 네 개의 인자가
필요하다.

이러한 요구사항을 {\bf 사전 조건(precondition)}이라고 부른다.  함수
실행 전에 기본적인 조건이 참이어야 하기 때문이다.  반대로 말하면,
함수의 끝에 있는 조건을 {\bf 사후 조건(postcondition)}이라 부른다.
사후 조건은 의도된 함수의 결과와 (예, 원을 이루는 선 그리기), 의도하지
않은 부차적 동작의 결과(거북이를 이동한다거나 다른 변경들)를 포함한다.
\index{precondition}
\index{postcondition}

사전 조건를 잘 지정하는 것은 호출하는 함수의 책임이다.  호출하는 함수가
(문서화를 잘했는데도 불구하고) 전제 조건의 값을 제대로 지정하지 못하면
함수는 정확하게 동작하지 않는다.  함수에 문제가 있어서가 아니라
호출하는 함수에 버그가 있어서 잘못된 결과를 얻은 것이다.

사전 조건이 충족되었더라도 사후 조건이 아니라면 버그는 함수에 있다.
명료한 사전/사후 조건은 디버깅하는데 도움이 된다.



\section{용어 해설}
%Glossary

\begin{description}

\item[메소드(method):] 객체에 속해 있는 함수로 호출하기 위해 닷표기법을
  사용함
\index{method}

\item[반복문(루프, loop):] 반복적으로 실행되는 프로그램의 한 부분
\index{loop}

\item[캡슐화(encapsulation):] 일련의 문장들을 함수 정의로 변환하는 과정
\index{encapsulation}

\item[일반화(generalization):] 불필요하게 구체적인(특정 숫자와 같은)
  것을 좀 더 일반적인(변수나 매개 변수) 것으로 교체하는 과정
\index{generalization}

\item[키워드 인자(keyword argument):] 매개 변수 명을 ``키워드''로
  인자에 포함하는 것
\index{keyword argument}
\index{argument!keyword}

\item[인터페이스(interface):] 함수의 사용법에 대한 설명으로 이름과
  인자에 대한 설명 그리고 리턴 값을 포함함
\index{interface}

\item[리팩터링(refactoring):] 동작하는 프로그램의 함수의 인터페이스와
  코드를 개선하는 과정
\index{refactoring}

\item[개발 계획(development plan):] 프로그램을 작성하는 과정
\index{development plan}

\item[설명 문자열(docstring):] 함수 정의의 시작 부분에 작성된 함수의
  인터페이스에 대한 설명
\index{docstring}

\item[사전 조건(precondition):] 호출한 함수가 해당 함수를 실행하기 전에
  만족해야 할 조건
\index{precondition}

\item[사후 조건(postcondition):] 함수가 종료하기 전에 만족시켜야 할 요구 사항
\index{precondition}

\end{description}


\section{연습 문제}
%Exercises

\begin{exercise}

이 장의 코드를 다음의 주소에서 다운 받자.
\url{http://thinkpython2.com/code/polygon.py}.

\begin{enumerate}

\item {\tt circle(bob, radius)}가 실행 중일 때 프로그램의 스택 상태도를
  그려라.  손으로 계산하거나 {\tt print}문을 코드에 넣어도 된다.
\index{stack diagram}

\item \ref{refactoring}절의 {\tt arc}는 아주 정확한 것은 아니다.  원의
  선형 근사치는 언제나 원 밖에 존재하기 때문이다.  그 결과로 거북이는
  정확한 위치에서 몇 픽셀씩 벗어나게 된다.  여기에 나와 있는 해법은 그
  에러를 줄일 수 있는 방법을 제시한다.  코드를 읽어보고 이해가 가는지
  보라.  그림을 그리면 어떻게 동작하는지 알 수 있을 것이다.

\end{enumerate}

\end{exercise}

\begin{figure}
\centerline
{\includegraphics[scale=0.8]{figs/flowers.pdf}}
\caption{거북이 꽃.}
\label{fig.flowers}
\end{figure}

\begin{exercise}
\index{flower}

일반 함수들을 작성해서 그림~\ref{fig.flowers}의 꽃을 그려보라.


해법: \url{http://thinkpython2.com/code/flower.py},
필요한 것 \url{http://thinkpython2.com/code/polygon.py}.

\end{exercise}

\begin{figure}
\centerline
{\includegraphics[scale=0.8]{figs/pies.pdf}}
\caption{거북이 파이.}
\label{fig.pies}
\end{figure}


\begin{exercise}
\index{pie}

그림~\ref{fig.pies}를 그릴 수 있는 일반 함수들을 작성하라.


해법: \url{http://thinkpython2.com/code/pie.py}.

\end{exercise}

\begin{exercise}
\index{alphabet}
\index{turtle typewriter}
\index{typewriter, turtle}

영문 철자는 수직과 수평 선 그리고 몇 개의 곡선의 기본 요소들로 구성되어
있다. 이 기본 요소들을 최소한으로 사용하여 그릴 수 있는 철자를
설계해보라.  그리고, 그 철자들을 그리는 함수를 작성하라.

\verb"draw_a"와 \verb"draw_b"같은 식으로 각 철자마다 함수를 만들어서
{\tt letters.py} 안에 저장하라.
\url{http://thinkpython2.com/code/typewriter.py}에 ``거북이
타자기(turtle typewriter)''를 다운로드 받아 코드를 시험해볼 수 있다.

\url{http://thinkpython2.com/code/letters.py}에 다운로드 받을 수
있다. \url{http://thinkpython2.com/code/polygon.py}도 필요할 것이다.


\end{exercise}

\begin{exercise}

  \url{http://en.wikipedia.org/wiki/Spiral}에서 나선에 대해 읽은 후
  아르케메데스의 나선(또는 그와 유사한 나선)을 그리는 프로그램을 작성해
  보라.  해답은 \url{http://thinkpython2.com/code/spiral.py}.
\index{spiral}
\index{Archimedian spiral} 

\end{exercise}


\chapter{조건문과 재귀문}
%Conditionals and recursion
프로그램의 상태에 따라 다른 코드를 실행하는 {\tt if}문이 이 장의 핵심
주제이다.  그 전에 내림 나눗셈 연산자와 나머지 연산자를 살펴보자.


\section{내림 나눗셈과 나머지 연산자}
%Floor division and modulus

%{\
{\bf 내림 나눗셈} 연산자 \verb"//"는 두 수를 나누어 얻은 결과 값의 정수
값을 취한다.  예를 들어, 105분 길이의 영화가 있다고 했을 때 몇 시간
짜리 영화인지 궁금할 때가 있다.  이런 경우 일반 나눗셈을 사용하면
부동소수점 값을 얻는다.

\begin{verbatim}
>>> minutes = 105
>>> minutes / 60
1.75
\end{verbatim}

하지만, 일반적으로 시간을 소수점으로 표현하지 않는다. 내림 나눗셈은
소수점 부분을 버리고 정수부분만 리턴한다.


\begin{verbatim}
>>> minutes = 105
>>> hours = minutes // 60
>>> hours
1
\end{verbatim}

나머지를 원하면 60분에서 계산된 시간 값을 제외한 나머지 분을 빼면 된다. 

\begin{verbatim}
>>> remainder = minutes - hours * 60
>>> remainder
45
\end{verbatim}

\index{floor division}
\index{floating-point division}
\index{division!floor}
\index{division!floating-point}
\index{modulus operator}
\index{operator!modulus}

이런 복잡한 계산 대신 나눗셈 후 나머지 값을 리턴하는 {\bf 나머지
  연산자} \verb"%"를 사용할 수 있다.

\begin{verbatim}
>>> remainder = minutes % 60
>>> remainder
45
\end{verbatim}
%
나머지 연산자는 보기보다 매우 유용한다.  한 가지 예를 들어 보면, 나머지
연산자는 어떤 값이 서로 나누어 떨어지는 지 확인할 때 쓸 수있다.  {\tt
  x \% y}가 영이라면 {\tt x}는 {\tt y}로 나누어 떨어진다.
\index{divisibility}

또한, 어떤 숫자의 가장 오른쪽 자리수를 얻을 때 나머지 연산자를 쓸 수
있다.  {\tt x \% 10}을 계산하면 {\tt x}(십진수라 가정)의 1의 자리 값을
얻는다.  마찬가지로 {\tt x \% 100}을 하면 마지막 두 자리 수를 얻는다.

Python 2를 사용 중이라면 나눗셈은 다르게 동작한다.  나눗셈 연산자
\verb"/"은 두 수가 모두 정수이면 내림 나눗셈을하고 두 값 중 하나라도
{\tt 실수} 형면 부동소수점 나눗셈을 한다.
\index{Python 2}


\section{불 표현식}
%Boolean expressions
\index{boolean expression}
\index{expression!boolean}
\index{logical operator}
\index{operator!logical}

{\bf 불 표현식(boolean expression)}은 참또는 거짓을 나타내는
표현식이다.  다음의 예들은 두 피연산자를 비교하는 {\tt ==} 연산자를
사용한다.  두 값이 동일하면 {\tt True(참)}을 다르면 {\tt
  False(거짓)}이다.

\begin{verbatim}
>>> 5 == 5
True
>>> 5 == 6
False
\end{verbatim}
%
이 값은 문자열이 아니고, {\tt bool(불)} 데이터 형에 속한{\tt True}와
{\tt False}는 특별한 값이다.
\index{True special value}
\index{False special value}
\index{special value!True}
\index{special value!False}
\index{bool type}
\index{type!bool}

\begin{verbatim}
>>> type(True)
<class 'bool'>
>>> type(False)
<class 'bool'>
\end{verbatim}
%
{\tt ==} 연산자는 {\bf 관계 연산자(relational operator)} 중 하나이다.
나머지는 다음과 같다.

\begin{verbatim}
      x != y               # x 는 y와 같지 않음
      x > y                # x 는 y보다 큼
      x < y                # x 는 y보다 작음
      x >= y               # x 는 y보다 크거나 같음
      x <= y               # x 는 y보다 작거나 같음 y
\end{verbatim}
%
이 연산자들이 익숙해 보일 수 있지만, Python에서 사용하는 기호가 수학
기호와 다르다.  가장 흔한 실수는 하나의 등호({\tt =})를 써서 두 값이
같은지 비교({\tt ==})하는 것이다.  {\tt =}는 할당 연산자이고 {\tt ==}가
관계 연산자인 것을 기억해야 한다.  그리고 {\tt =<}나 {\tt =>}와 같은
연산자는 없다.

\index{relational operator}
\index{operator!relational}


\section {논리 연산자}
%Logical operators
\index{logical operator} \index{operator!logical} 세 개의 {\bf 논리
  연산자(logical operator)}가 있다. 각각은 {\tt and}, {\tt or}, 그리고
{\tt not}이다.  이 연산자들의 의미는 영어에서의 각 단어의 의미와 같다.
예를 들어, {\tt x > 0 and x < 10} 은 {\tt x}가 0보다 크{\em고} 10보다
10보다 작을 때 참이 된다.
\index{and operator}
\index{or operator}
\index{not operator}
\index{operator!and}
\index{operator!or}
\index{operator!not}

{\tt n\%2 == 0 or n\%3 == 0} 은 {\em 둘 중 하나만} 또는 {\em둘 다} 참일
때 참이 된다.  즉, 2 {\em 또는} 3으로 나누어 지거는 경우이다.

마지막으로 {\tt not} 연산자는 불 표현식의 반대 값을 취한다.  {\tt not
  (x > y)}는 {\tt x > y}가 거짓일 때 참이 된다.  즉, {\tt x}가 {\tt
  y}보다 작거나 같은 경우이다.

정확하게 말하면, 논리 연산자의 피연산자는 불 표현식이어야하지만,
Python이 그렇게 엄격하지는 않다.  0이 아닌 어떤 수이기만 하면 {\tt
  True}으로 인식한다.

\begin{verbatim}
>>> 42 and True
True
\end{verbatim}
%
이러한 유연성이 도움이 될 때도 있지만, 미묘한 부분도 있기 때문에 혼돈이
있을 수도 있다.  이렇게 사용하는 것은 피하는게 좋다(뭘 하는지 정확히
알고 있다면 모를까).


\section{조건부 실행}
%Conditional execution
\label{conditional.execution}

\index{conditional statement}
\index{statement!conditional}
\index{if statement}
\index{statement!if}
\index{conditional execution}

프로그램을 유용하게 만들려면 프로그램이 상황에 맞게 동작할 수 있도록
조건을 검사하는 능력이 반드시 필요하다.  {\bf 조건문(conditional
  statement)}이 바로 그 능력을 갖고 있다.  그 중 가장 간단한 것은 {\tt
  if}문이다.

\begin{verbatim}
if x > 0:
    print('x is positive')
\end{verbatim}
%
{\tt if}문 이후의 불 표현식은 {\bf 조건(condition)}이라 부른다.
참이라면, 의도한 문장이 실행되고 그렇지 않다면 아무 일도 일어 나지
않는다.
\index{condition}
\index{compound statement}
\index{statement!compound}

헤더와 들여쓰기된 내용으로 구성된 함수 정의처럼 {\tt if}문도 헤더와
내용을 갖고 있다.  이런 류의 문장들을 {\bf 복합문(compound
  statement)}라고 부른다.

내용에 포함될 수 있는 문장의 수가 정해지지 않았지만 최소한 하나는
있어야 한다.  때로는 어떤 문장도 없는 내용이 필요할 때도 있다(아직 적지
않은 코드를 대신해서 자리만 차지하는 기호를 쓸 수도 있다).  그 때에는
아무 일도 하지 않는 {\tt pass}문을 쓰면 된다.
\index{pass statement}
\index{statement!pass}

\begin{verbatim}
if x < 0:
    pass          # TODO: 음수 처리해야 함!
\end{verbatim}
%

\section{선택적 실행}
%Alternative execution
\label{alternative.execution}
\index{alternative execution}
\index{else keyword}
\index{keyword!else}

{\tt if}문의 두 번째 형식은 ``선택적 실행(aternative
execution)''이다.  두 가지 대안이 존재하고 조건 검사를 통해 대안 중
하나를 선택하여 실행한다.  문법은 다음과 같다.

\begin{verbatim}
if x % 2 == 0:
    print('x는 짝수')
else:
    print('x는 음수')
\end{verbatim}
%
{\tt x}를 2로 나눈 나머지가 0이라면 {\tt x}가 짝수이고, 프로그램은
걸맞는 메시지를 표시한다.  조건이 거짓이라면, 두 번째 부분의 문장이
실행된다.  조건은 참이거나 거짓이기 때문에 둘 중에 정확히 하나만
선택적으로 실행된다.  실행의 흐름이 분리되므로 이러한 종류의 선택지를
{\bf 분기(branch)}라고 부른다.  \index{branch}



\section{연쇄 조건문}
%Chained conditionals
\index{chained conditional}
\index{conditional!chained}

두 개 이상의 가능성이 있는 경우 두개 이상의 분기가 필요하다.  이런 류의
연산을 {\bf 연쇄 조건문(chained conditional)}이다.


\begin{verbatim}
if x < y:
    print('x is less than y')
elif x > y:
    print('x is greater than y')
else:
    print('x and y are equal')
\end{verbatim}
%
{\tt elif}는 ``else if''의 약어이다.  다시 한 번 말하지만, 선택된
분기가 실행된다.  {\tt elif}문이 몇 개가 있던 상관없다.  {\tt else}
절을 만나면 거기가 마지막이지만, 꼭 필요한 것은 아니다.
\index{elif keyword}
\index{keyword!elif}

\begin{verbatim}
if choice == 'a':
    draw_a()
elif choice == 'b':
    draw_b()
elif choice == 'c':
    draw_c()
\end{verbatim}
%
각 조건은 순차적으로 검사된다.  처음 조건이 거짓이면 다음 조건을
검사하는 식이다.  참인 조건을 만나면 해당 분기가 실행되고 {\tt if}문은
종료한다.  여러 개의 조건이 참이 될 수는 있는지 첫 번째 참이 조건만
실행이 된다.



\section{중첩된 조건문}
%Nested conditionals
\index{nested conditional}
\index{conditional!nested}

조건문은 다른 조건문 내에 포함될 수 있다.  전 절에서 봤던 예제를 다음과
같이 적을 수도 있다.

\begin{verbatim}
if x == y:
    print('x and y are equal')
else:
    if x < y:
        print('x is less than y')
    else:
        print('x is greater than y')
\end{verbatim}
%
처음 만나는 조건문은 분기가 두개다.  첫 번째 분기는 간단한
\texttt{printf}문이다.  두 번째 분기는 또 다른 {\tt if}문을 포함하고
있다.  내부의 {\tt if}문도 두 개의 분기를 갖고 있다.  각 분기의 내용도
간단한 문장이지만, 이 문장들도 또 다른 조건문이었을 수도 있다.

문장 들여쓰기를 하면 구조가 시각적으로 명확해지긴 하지만, {\bf 중첩
  조건문(nested conditionals)}을 사용하면 읽기가 어려워진다.  가능하면
쓰지 않는 방법을 찾는 것이 좋다.

논리 연산자를 쓰면 중첩된 조건문이 간단해지는 경우가 많다.  다음의
코드를 하나의 조건문으로 바꿔보자.


\begin{verbatim}
if 0 < x:
    if x < 10:
        print('x is a positive single-digit number.')
\end{verbatim}
%
{\tt print}문은 두 개의 조건문을 모두 통과 했을 때만 실행이 된다.  이와
같은 경우에 {\tt and} 연산자를 쓰면 동일한 효과를 볼 수 있다.

\begin{verbatim}
if 0 < x and x < 10:
    print('x is a positive single-digit number.')
\end{verbatim}

이와 같은 조건문에 Python은 더 간단한 표기법을 제공한다. 

\begin{verbatim}
if 0 < x < 10:
    print('x is a positive single-digit number.')
\end{verbatim}


\section{재귀문}
%Recursion
\label{recursion}
\index{recursion}

어느 한 함수가 또 다른 함수를 부르는 것은 전혀 이상하지 않다.  그렇기
때문에 어떤 함수가 자기 자신을 부르는 것도 가능하다.  함수가 자신을
부르는 것이 유용한 것인지 확신이 서지 않을 수도 있다.  나중에
알게되겠지만,  프로그램이 할 수 있는 가장 마법같은 일 중
하나가 재귀적 호출이다. 예로 다음의 함수를 살펴보자.


\begin{verbatim}
def countdown(n):
    if n <= 0:
        print('발사!')
    else:
        print(n)
        countdown(n-1)
\end{verbatim}
%
{\tt n}이 0이거나 음수이면 ``발사!''고 표시하고 그 외의 경우에는 {\tt
  n}을 화면에 적고 {\tt n-1}을 인자로하여 자기 자신({\tt countdown})을
호출한다.

다음과 같이 함수를 호출하면 어떻게 될까? 

\begin{verbatim}
>>> countdown(3)
\end{verbatim}
%
{\tt countdown}은 {\tt n=3}에서 실행된다.  {\tt n}은 0보다 크기 때문에
3을 출력하고 자신을 호출한다.


\begin{quote}
{\tt countdown}은 {\tt n=2}에서 실행된다.  {\tt n}은 0보다 크기 때문에
2을 출력하고 자신을 호출한다.


\begin{quote}
{\tt countdown}은 {\tt n=1}에서 실행된다.  {\tt n}은 0보다 크기 때문에
1을 출력하고 자신을 호출한다.


\begin{quote} 
{\tt countdown}은 {\tt n=0}에서 실행된다.  {\tt n}은
  0이기 때문에 ``발사!''라고 화면에 표시하고 리턴한다.
\end{quote}

\end{quote}

{\tt n=2}을 받은 {\tt countdown}이 리턴한다.
\end{quote}

{\tt n=3}을 받은 {\tt countdown}이 리턴한다.

그리고 나면, \verb"__main__"으로 돌아온다.  최종적으로 다음과 같이
표시된다.  
\index{main}

\begin{verbatim}
3
2
1
발사!
\end{verbatim}
%
함수가 자기 자신을 호출하는 것을 보고 {\bf 재귀적(recursive)}이라고
한다.  {\bf 재귀문(recursion)}은 함수를 재귀적으로 실행시키는 과정을
말한다.
\index{recursion}
\index{function!recursive}

또 다른 예로, 어떤 문자열을 {\tt n}번 표시하는 함수를 작성해보자.

\begin{verbatim}
def print_n(s, n):
    if n <= 0:
        return
    print(s)
    print_n(s, n-1)
\end{verbatim}
%
{\tt n <= 0}이면 {\bf 리턴문(return statement)}으로 함수가 종료된다.
실행의 흐름은 호출한 함수에게 즉시 리턴된다. 그리고 함수의 나머지
부분은 실행되지 않는다.
\index{return statement}
\index{statement!return}

함수의 나머지 부분은 {\tt countdown}과 유사하다.  {\tt s}를 표시하고
$n-1$번 {\tt s}를 출력하도록 자기 자신을 다시 호출한다.  결과로
표시되는 줄의 수가 {\tt 1 + (n - 1)}이기 때문에 총 {\tt n}번이 된다.


이처럼 간단한 예에서는 {\tt for} 루프를 사용하는게 더 쉬울 수 도 있다.
하지만 우리가 이후에 보게 될 좀 더 복잡한 예제들은 {\tt for} 루프로
표현하기가 어렵다.  어려운 거라면 미리 시작해서 익숙해지는 것이 좋다.
\index{for loop}
\index{loop!for}


\section{재귀 함수의 스택 상태도}
%Stack diagrams for recursive functions
\label{recursive.stack}
\index{stack diagram}
\index{function frame}
\index{frame}

\ref{stackdiagram}절에서 함수 호출 중에 프로그램의 상태를 스택 상태도로
표현하였었다.  똑같은 그림을 재귀문을 이해하는데도 쓸 수 있다.

함수가 호출될 때마다 Python은 프레임을 생성하여 함수의 지역 변수와 매개
변수를 저장한다.  재귀문을 쓰게 되면 스택에 하나 이상의 프레임이
동시에 여러 개가 존재할 수 있다.

그림~\ref{fig.stack2}은 {\tt countdown}을 {\tt n = 3}을 인자로 하여
호출했을 때의 스택 상태도를 나타낸다.

\begin{figure}
\centerline
{\includegraphics[scale=0.8]{figs/stack2.pdf}}
\caption{스택 상태도.}
\label{fig.stack2}
\end{figure}

스택의 최상위 프레임에 \verb"__main__"가 있다.  프레임에 아무 것도 없는
이유는 \verb"__main__"가 어떤 변수도 생성하지 않았고 인자도 전달받지
못했기 때문이다.
\index{base case}
\index{recursion!base case}

네 개의 {\tt countdown} 프레임들은 매개 변수 {\tt n}이 모두 다르다.
{\tt n = 0}이 있는 스택 프레임을 {\bf 기준 케이스(base case)}라
부른다.  더 이상 재귀 호출을 하지 않기 때문에 그 이하로는 프레임이
없다.


연습삼아 \verb"s = 'Hello'"와 {\tt n=2}를 인자로 \verb"print_n"가
호출되었을 때의 스택 상태도를 그려보자.  그리고 함수 객체와 {\tt n}을
인자로 하는 \verb"do_n" 함수를 만들어서 그 함수를 {\tt n}번 호출해
보자.


\section{무한 재귀문}
%Infinite recursion
\index{infinite recursion}
\index{recursion!infinite}
\index{runtime error}
\index{error!runtime}
\index{traceback}

재귀문이 기준 케이스에 도달하지 못한다면 재귀문은 끝없이 호출되고
프로그램은 절대로 종료하지 않는다.  이것이 {\bf 무한 재귀문(infinite
  recursion)}이고 일반적으로 나쁘다.   무한 재귀문의 한 짧은
예를 살펴보자.


\begin{verbatim}
def recurse():
    recurse()
\end{verbatim}
%
대부분의 프로그래밍 환경에서 무한 재귀문을 갖는 프로그램이 무한히
동작하는 일은 없다.  프로그램이 최대 재귀 가능 횟수에 도달하면 Python은
오류 메시지를 보고 한다.
\index{exception!RuntimeError}
\index{RuntimeError}

\begin{verbatim}
  File "<stdin>", line 2, in recurse
  File "<stdin>", line 2, in recurse
  File "<stdin>", line 2, in recurse
                  .   
                  .
                  .
  File "<stdin>", line 2, in recurse
RuntimeError: Maximum recursion depth exceeded
\end{verbatim}
%
이 트레이스백은 이 전 장에서 살펴보았던 보다 좀 더 길다.  오류가
발생하면서 스택에 1,000개의 {\tt recurse} 프레임이 생겼다.

실수로 무한 재귀문을 만나게 되면 재귀 호출에 기준 케이스가 있는지
함수를 확인해야 한다.  기준 케이스가 있다면 그 부분에 도달 가능한지
확인해야 한다.



\section{키보드 입력}
%Keyboard input
\index{keyboard input}

지금까지 작성한 모든 프로그램은 사용자 입력이 없었다.  실행하면
똑같은 동작만 했다.

Python은 {\tt input}이라는 내장 함수를 제공한다.  이 함수는 잠시 멈춰서
사용자가 입력하기를 기다린다.  사용자가 {\sf Return} 또는 {\sf
  Enter}키를 누르면 프로그램은 계속 동작하고 \verb"input"은 사용자가
입력한 문자열을 리턴한다.  Python 2에서는 똑같은 일을 하는 함수를
\verb"raw_input"이라 부른다.

\index{Python 2}
\index{input function}
\index{function!input}

\begin{verbatim}
>>> text = input()
What are you waiting for?
>>> text
'What are you waiting for?'
\end{verbatim}
%
사용자로 부터 입력을 받기 전에 어떤 것을 입력할지 알려주는 것이 좋다.
\verb"input"는 인자로 프롬프트의 내용을 갖는다.
\index{prompt}

\begin{verbatim}
>>> name = input('당신은 누구죠?\n')
당신은 누구죠?
영국의 왕, 아서!
>>> name
'영국의 왕, 아서!'
\end{verbatim}
%
프롬프트의 마지막 부분의 \verb"\n"은 {\bf 새 줄(newline)}을 나타내는
특수 기호로 줄을 바꾼다. 그렇기 때문에 사용자 입력 위치가 프롬프트 아래에 있다. 
\index{newline}

정수를 입력하기를 바란다면 리턴 값을 {\tt int}으로 변환할 수 있다. 

\begin{verbatim}
>>> prompt = '짐을 메달고 있지 않은 제비의 속도는?\n'
>>> speed = input(prompt)
짐을 메달고 있지 않은 제비의 속도는?
42
>>> int(speed)
42
\end{verbatim}
%
하지만 사용자가 숫자 말고 다른 것을 입력하면 오류가 발생한다. 

\begin{verbatim}
>>> speed = input(prompt)
짐을 메달고 있지 않은 제비의 속도는?
그 제비가 아프리카 제비야 아니면 영국 제비야?
>>> int(speed)
ValueError: invalid literal for int() with base 10
\end{verbatim}
%
이런 류의 오류를 다루는 법을 나중에 보도록 하겠다. 
\index{ValueError}
\index{exception!ValueError}


\section{디버깅}
%Debugging
\label{whitespace}
\index{debugging}
\index{traceback}

문법 오류나 실행 중에 발생하는 오류에는 많은 정보가 담겨 있다.  때로는
벅차게 많을 수도 있다.  그 중에 중요한 정보들은 다음과 같다. 


\begin{itemize}

\item 어떤 종류의 오류인가

\item 어디서 발생했는가

\end{itemize}

문법 오류는 대체적으로 찾기가 쉽지만 몇 개는 눈에 잘 안 띄기도 한다.
사이띄기와 탭 문자 같은 공백과 관련한 오류들은 보이지 않기 때문에 쉽게
지나치곤 한다.  그래서 찾기가 쉽지 않다.
\index{whitespace}

\begin{verbatim}
>>> x = 5
>>>  y = 6
  File "<stdin>", line 1
    y = 6
    ^
IndentationError: unexpected indent
\end{verbatim}
%
이 예제에서의 문제는 두 번째 줄이 한 칸 들여쓰기되어 있다는 것이다.
그렇지만 오류 메시지는 {\tt y}를 가리키고 있기 때문에 오해하기 쉽다.
일반적으로 오류 메시지는 오류가 발견된 시점을 나타내기 때문에 실제
오류는 코드의 이전에서 이미 있었을 수도 있다.  심지어는 전 줄에 오류가
있었을 수도 있다.
\index{error!runtime}
\index{runtime error}

실행 중 발생 오류에서도 마찬가지다.  신호대잡음비를 데시벨로 계산한다고
해보자.  계산식은 $SNR_{db} = 10 \log_{10} (P_{signal} /
P_{noise})$이다.  Python에서는 이 식을 작성하면 다음과 같다.

\begin{verbatim}
import math
signal_power = 9
noise_power = 10
ratio = signal_power // noise_power
decibels = 10 * math.log10(ratio)
print(decibels)
\end{verbatim}
%
이 프로그램을 실행하면 예외처리가 된다. 
%
\index{exception!OverflowError}
\index{OverflowError}

\begin{verbatim}
Traceback (most recent call last):
  File "snr.py", line 5, in ?
    decibels = 10 * math.log10(ratio)
ValueError: math domain error
\end{verbatim}
%
이 오류 메시지는 5번 줄을 가리키고 있지만 그 줄에는 문제가 없다.  실제
오류를 찾으려면 {\tt ratio}의 값을 출력해보는게 좋다.  출력해보면 0
값을 리턴 받는다.  실제 문제는 내림 나눗셈을 하는 4번 줄에 있다.
소수점 단위의 나눗셈을 했어야 했다.
\index{floor division}
\index{division!floor}

오류 메시지를 읽는데 충분한 시간을 들여야 하겠지만, 그 하는 말을 모두
참이라 믿어서는 안된다.


\section{용어 해설}
%Glossary

\begin{description}

\item[내림 나눗셈(floor division):] {\tt //}로 표시되는 연산자로서 두
  수를 나눈 결과를 정수가 되도록 소수점을 버림
  \index{floor division} 
  \index{division!floor}

\item[나머지 연산자(modulus operator):] 퍼센트({\tt \%}) 기호를 갖는
  연산자로 두 정수를 나눈 나머지 값을 돌려줌
\index{modulus operator}
\index{operator!modulus}

\item[불 표현식(boolean expression):] 결과가 항상 참({\tt True})이거나
  거짓({\tt False})인 표현식
\index{boolean expression}
\index{expression!boolean}

\item[관계 연산자(relational operator):] 피연산자를 비교하는 연산자로서
  다음의 종류가 있음: {\tt ==}, {\tt !=}, {\tt >}, {\tt <}, {\tt >=},
  {\tt <=}.

\item[논리 연산자(logical operator):] 불 표현식들을 서로 연결하는
  연산자: {\tt and}, {\tt or}, {\tt not}이 있음

\item[조건문(conditional statement):] 어떤 조건에 따라 실행의 흐름을
  결정하는 문장
\index{conditional statement}
\index{statement!conditional}

\item[조건(condition):]  조건문에서 어떤 분기가 실행될지를 결정하는 불 표현식
\index{condition}

\item[복합문(compound statement):] 헤더와 내용으로 구성된 문장으로
  헤더는 콜론(:)으로 끝이 나나. 복합문의 내용은 헤더 위치에서
  상대적으로 들여쓰기함
\index{compound statement}

\item[분기(branch):] 조건문에서 선택적으로 실행되는 일련의 문장
\index{branch}

\item[연쇄 조건문(chained conditional):]  여러 개의 선택적 분기문으로 구성된 조건문
\index{chained conditional}
\index{conditional!chained}

\item[중첩 조건문(nested conditional):]  어떤 조건문의 분기 내있는 또 다른 조건문
\index{nested conditional}
\index{conditional!nested}

\item[리턴문(return statement):] 함수를 호출한 문장으로 즉시 리턴하는 문장

\item[재귀문(recursion):]  현재 실행 중인 함수를 호출하는 과정
\index{recursion}

\item[기준 케이스(base case):]  재귀문에서 재귀 호출을 하지 않는 조건 분기
\index{base case}

\item[무한 재귀문(infinite recursion):] 기준 케이스가 없거나 절대
  도달하지 못하는 재귀문.  무한 재귀문을 실행하면 실행 중 오류가
  발생하게 됨
\index{infinite recursion}

\end{description}

\section{연습 문제}
%Exercises

\begin{exercise}

{\tt time} 모듈은 {\tt time} 함수를 제공한다.  임의의 기준 시점에서
시작하는 ``에포크(the epoch)''를 기준으로 현재 그린위치 평균시를
리턴함. UNIX 시스템에서는 에포크는 1970 1월 1일이다.


\begin{verbatim}
>>> import time
>>> time.time()
1437746094.5735958
\end{verbatim}

현재 시간을 읽어서 시, 분, 초, 그리고 에포크로부터 현재까지 지난 날
수를 리턴하는 스크립트를 작성하라.

\end{exercise}


\begin{exercise}
\index{Fermat's Last Theorem}

페르마의 마지막 정리는 $n$이 2보다 큰 경우 $a$, $b$, $c$의 양의 정수에
대해 다음의 방정식은 해를 갖지 않는다는 정리이다.

\[ a^n + b^n = c^n \]
%

\begin{enumerate}

\item 네 개의 변수---{\tt a}, {\tt b}, {\tt c}, {\tt n}---를 받아 $n$이
  2보다 큰 경우에 대해 페르마의 마지막 정리가 참인지 검사하는
  \verb"check_fermat" 함수를 작성하라. 

\[a^n + b^n = c^n \]
%
위의 식이 참이라면 ``어머나, 페르마가 틀렸었네!'', 거짓이라면 ``역시
없구나.''를 출력하도록 만들어라.


\item 사용자가 {\tt a}, {\tt b}, {\tt c}, {\tt n} 값을 정하도록 하고,
  입력 받은 값을 정수로 변환하여 \verb"check_fermat"의 입력으로 쓰는
  함수를 만들어라.  입력 받은 값으로 페르마의 정리가 참인지
  확인해보자.

\end{enumerate}

\end{exercise}


\begin{exercise}
\index{triangle}

임의의 세 개의 막대기의 길이에 따라 삼각형 만들기 가능 여부가
결정된다.  예를 들어, 한 막대기가 12인치이고 다른 두 개가 각각
1인치라고 해보자.  작은 두 개의 막대기로는 중간에서 만날 수가 없다.
임의의 세 길이가 있을 때 삼각형 만들기가 가능한지 확인하는 간단한
테스트가 있다.

\begin{quotation}
  세 선분 중 하나의 선분이 나머지 두 선분의 합보다 길면 삼각형을 만들
  수 없다.  짧으면 삼각형을 만들 수 있다.  (두 선분의 길의의 합이 세
  번째 길이와 같으면 ``퇴화(degenerate)'' 삼각형이라고 한다.)
\end{quotation}

\begin{enumerate}

\item 세 개의 정수를 인자로하여 주어진 수로 삼각형 만들기가 가능한지
  여부를 판단하는 \verb"is_triangle" 함수를 작성하라.  가능하면
  ``Yes'' 불가능하면 ``No''를 출력하도록 하라.

\item 사용자로 부터 세 정수를 입력받아, 그 수를 정수로 변환하고,
  \verb"is_triangle"를 사용하여 주어진 정수로 삼각형을 만들 수 있는지
  판단하는 함수를 작성하라.

\end{enumerate}

\end{exercise}

\begin{exercise}
다음 프로그램의 결과가 무엇인가?
프로그램이 결과를 표시할 때의 스택 상태도를 그려보아라. 

\begin{verbatim}
def recurse(n, s):
    if n == 0:
        print(s)
    else:
        recurse(n-1, n+s)

recurse(3, 0)
\end{verbatim}

\begin{enumerate}

\item 이 함수를 {\tt recurse(-1, 0)}으로 호출하면 어떻게 될까?

\item 이 함수를 사용하기 위해 필요한 모든 정보를 담은 설명 문자열을 작성하라.

\end{enumerate}

\end{exercise}

다음의 문제들은 \ref{turtlechap}장에서 설명한 {\tt turtle} 모듈을 사용한다. 
\index{TurtleWorld}

\begin{exercise}
 
다음의 함수를 읽고 어떤 동작을 하는지 생각해보자(필요하면
\ref{turtlechap}장의 예제를 읽어보자).  생각한 데로 동작하는지
실행해보자.


\begin{verbatim}
def draw(t, length, n):
    if n == 0:
        return
    angle = 50
    t.fd(length*n)
    t.lt(angle)
    draw(t, length, n-1)
    t.rt(2*angle)
    draw(t, length, n-1)
    t.lt(angle)
    t.bk(length*n)
\end{verbatim}

\end{exercise}


\begin{figure}
\centerline
{\includegraphics[scale=0.8]{figs/koch.pdf}}
\caption{코흐 곡선.}
\label{fig.koch}
\end{figure}

\begin{exercise}
\index{Koch curve}

코흐 곡선(The Koch curve)는 그림 \ref{fig.koch}처럼 보이는 프랙탈
곡선이다. 길이 $x$를 사용하여 코흐 곡선을 그려라.  그리는 방법은 다음과
같다.


\begin{enumerate}

\item $x/3$의 길이로 코흐 곡선을 그린다.

\item 왼쪽으로 60도 회전한다. 

\item $x/3$의 길이로 코흐 곡선을 그린다.

\item 오른쪽으로 120도 회전한다. 

\item $x/3$의 길이로 코흐 곡선을 그린다.

\item 왼쪽으로 60도 회전한다. 

\item $x/3$의 길이로 코흐 곡선을 그린다.

\end{enumerate}

만약, $x$가 3보다 작으면 $x$만큼의 길이로 직선을 그리면 된다. 

\begin{enumerate}

\item 거북이 객체와 길이를 매개 변수로 하는 {\tt koch} 함수를
  작성하라.  이 함수로 주어진 길이로 거북이를 사용하여 코흐 곡선을
  그려보아라.

\item 세 개의 코흐 곡선으로 눈송이의 외각선을 그리는 {\tt snowflake}
  함수를 작성하라.

해답: \url{http://thinkpython2.com/code/koch.py}.

\item 코흐 곡선을 일반화하는 방법은 여러 개가
  있다. \url{http://en.wikipedia.org/wiki/Koch_snowflake}에 나와 있는
  것 중 마음에 드는 것을 골라 만들어 보자.

\end{enumerate}
\end{exercise}


\chapter{열매가 있는 함수}
%Fruitful functions
\label{fruitchap}

우리가 사용한 수학 함수와 같은 많은 Python 함수들은 리턴 값이 있지만,
우리가 작성한 함수들은 리턴 값이 없다.  대신 값을 출력하거나 거북이를
이동시키는 것과 같은 영향만 있다.  이 장에서는 열매가 있는 함수에 대해
배울 것이다.



\section{리턴 값}
%Return values
\index{return value}

함수 호출에는 리턴 값이 있는데, 보통은 이 값을 변수에 할당하거나 어떤
수식의 일부로 사용한다.


\begin{verbatim}
e = math.exp(1.0)
height = radius * math.sin(radians)
\end{verbatim}
%
지금까지 우리는 비어 있는 함수들을 작성했었다.  격식 없게는 리턴 값이
없었고, 좀 정확하게는 그 함수들의 리턴 값은 {\tt None}이 었다.

이 장에서는 (마침내) 열매가 있는 함수를 작성할 것이다.  첫 번째 예제는
전달 받은 반지름으로 원의 넓이를 리턴하는 {\tt area} 함수이다.

\begin{verbatim}
def area(radius):
    a = math.pi * radius**2
    return a
\end{verbatim}
%
우리가 봤던 열매가 있는 함수에서의 {\tt
  return}문은  수식이 포함되 있었다.  리턴문은 ``리턴 값으로
따라오는 수식을 사용하여 즉시 리턴하라''는 의미를 갖고 있다.  리턴할
수식이 복잡해도 괜찮기 때문에 이 예제의 함수를 좀 더 간결하게 작성 해
볼 수도 있다.
\index{return statement}
\index{statement!return}

\begin{verbatim}
def area(radius):
    return math.pi * radius**2
\end{verbatim}
%
그렇기는 하지만, {\bf 임시 변수}로 사용한 {\tt a}가 있어 좀 더 쉽게
디버깅할 수도 있다.
\index{temporary variable}
\index{variable!temporary}

때로는 조건문의 분기마다 리턴문이 있는 경우도 있다. 

\begin{verbatim}
def absolute_value(x):
    if x < 0:
        return -x
    else:
        return x
\end{verbatim}
%
각 {\tt return}문이 선택지마다 있기 때문에 그 중 어느 하나만 실행된다. 

리턴문이 실행되면 실행 중이던 함수는 다음 문장을 수행하지 않고 즉시
종료한다.  {\tt return}문 이후의 문장이나 실행의 흐름 상 도달하지 않는
코드를 {\bf 죽은 코드(dead code)}라 부른다.
\index{dead code}

열매가 있는 함수에서는 프로그램의 어떤 실행 흐름이더라도 {\tt
  return}문에 도달하도록 하는 것이 좋다.  예를 들어보자.

\begin{verbatim}
def absolute_value(x):
    if x < 0:
        return -x
    if x > 0:
        return x
\end{verbatim}
%
이 함수는 잘못되었다.  만약 {\tt x}가 0이면 두 선택지 모두 거짓이
된다.  그러면 {\tt return}문에 도달하지 않고 종료하게 된다.  이 경우의
리턴 값은 {\tt None}이지 0의 절대치가 아니다.
\index{None special value}
\index{special value!None}

\begin{verbatim}
>>> print(absolute_value(0))
None
\end{verbatim}
%
참고로, Python의 내부 함수 중에는 절대치를 계산하는 {\tt abs} 함수가 있다. 
\index{abs function}
\index{function!abs}

연습 문제로 {\tt x}와 {\tt y} 두 수를 받아 {\tt x > y}이면 {\tt 1}을
리턴하고 {\tt x == y}이면 {\tt 0}을 그리고 {\tt x < y}이면 {\tt -1}을
리턴하는 {\tt compare} 함수를 작성해보자.
\index{compare function}
\index{function!compare}


\section{점진적 개발}
%Incremental development
\label{incremental.development}
\index{development plan!incremental}

큰 함수를 작성하다 보면 디버깅에 더 많은 시간을 쓰는 때가 있다. 

엄청나게 복잡한 프로그램을 작성하기 위해 {\bf 점진적 개발(incremental
  development)} 방법론을 시도해 볼 것을 권한다.  점진적 개발 방법론의
목적은 아주 작은 크기의 코드를 추가하고 그 부분에 대한 디버깅을 하여
디버깅에 너무 오랜 시간을 쓰는 것을 피하도록 한다.
\index{testing!incremental development}
\index{Pythagorean theorem}

예를 들어, $(x_1, y_1)$과 $(x_2, y_2)$의 좌표를 갖는 두 점 사이의
거리를 잰다고 해보자.  피타고라스의 정리에 따라 거리는 다음과 같이
계산할 수 있다.

\begin{displaymath}
\mathrm{distance} = \sqrt{(x_2 - x_1)^2 + (y_2 - y_1)^2}
\end{displaymath}
%
첫 단계는 Python에서 {\tt distance} 함수를 정의하는 것이다.  입력
값(매개 변수)은 무엇인지 출력 값(리턴 값)은 무엇인지를 정해야 한다.

이 경우에는 입력 값은 네 개의 수로 이루어진 두 개의 점이다.  리턴 값은
부동소수점으로 표현되는 거리이다.

입력과 출력 값이 정해지면 함수의 구조를 잡을 수 있다. 

\begin{verbatim}
def distance(x1, y1, x2, y2):
    return 0.0
\end{verbatim}
%
당연하겠지만, 이 버전의 함수는 거리를 계산할 수 없다.  이 함수는 언제는
0만 리턴한다.  문법적으로는 정확하고 실행도 가능하다.  좀 더 복잡해지기
전에 검사가 가능하다는 말이다.

함수를 시험해 보려면 샘플 인자 값으로 호출하면 된다. 


\begin{verbatim}
>>> distance(1, 2, 4, 6)
0.0
\end{verbatim}
%
두 점 사이의 수평 거리가 3 차이나고 수직 거리가 4 차이 나도록 점들을
선택했다.  피타고라스의 수에 따라 직각 삼각형의 세 변의 길이의 비가
3:4:5 이기 때문에 결과는 5가 되야 한다.  정답을 알면 함수 검사가
수월해지는 법이다.
\index{testing!knowing the answer}

이제 이 함수가 문법적으로 정확하다는 것을 확인했으니 내용에 코드를
추가해도 된다.  다음으로 밟아야 할 순서는 $x_2 - x_1$과 $y_2 - y_1$의
계산으로 점 간의 차를 구하는 것이다.  다음 버전의 함수는 차를 계산하고
임시 변수에 저장한 후 화면에 표시하는 것이다.


\begin{verbatim}
def distance(x1, y1, x2, y2):
    dx = x2 - x1
    dy = y2 - y1
    print('dx는', dx)
    print('dy는', dy)
    return 0.0
\end{verbatim}
%
함수가 동작한다면, \verb"dx는 3" 그리고 \verb"dy는 4"를 출력할 것이다.
그리고 그렇게 출력되었다면 인자도 함수에 제대로 전달했고 첫 계산도
정확히 수행했다는 것을 알 수 있다.  만약 잘못 나왔다면 검사해야 할
코드가 몇 줄 되지 않는 것에 위안을 얻으면 된다.

그 다음 과정은 {\tt dx}와 {\tt dy}의 제곱의 합을 계산해야 한다. 

\begin{verbatim}
def distance(x1, y1, x2, y2):
    dx = x2 - x1
    dy = y2 - y1
    dsquared = dx**2 + dy**2
    print('dsquared is: ', dsquared)
    return 0.0
\end{verbatim}
%
마찬가지로, 변경한 코드를 실행해보고 결과가 맞는지 검사해봐야
한다(결과는 25다).  마지막으로 {\tt math.sqrt}를 써서 결과를 리턴하면
된다.
\index{sqrt}
\index{function!sqrt}

\begin{verbatim}
def distance(x1, y1, x2, y2):
    dx = x2 - x1
    dy = y2 - y1
    dsquared = dx**2 + dy**2
    result = math.sqrt(dsquared)
    return result
\end{verbatim}
%
지금까지 제대로 동작했다면, 다 끝났다.  확인하기 원한다면 리턴하기 전에
{\tt result}를 출력하면 된다. 

함수의 최종 버전은 실행했을 때 아무 것도 출력하지 않는다.  그저 값만
리턴할 뿐 이다.  {\tt print}문은 디버깅 때문에 포함한 것이기 때문에
함수가 동작한다면 제거해야 한다.  이런 류의 코드를 보고 {\bf
  발판(scaffolding)}이라 한다.  프로그램을 작성하는데 도움은 되지만
최종 결과물의 일부는 아니기 때문이다.
\index{scaffolding}

처음 시작할 때는 한 두 줄의 코드만 추가해야 한다.  좀 더 익숙해지면 좀
더 긴 길이의 코드를 작성하고 디버깅할 수 있게 된다.  초보이던
전문가이던, 점진적 개발 방법론은 디버깅 시간을 엄청나게 줄일 수 있다.

방법론의 핵심은 다음과 같다. 

\begin{enumerate}

\item 동작하는 프로그램으로 부터 시작하여 점진적인 변화를 만들어라.
  그래야 어떤 순간이라도 오류를 발견하면 어디에 원인이 있는지 쉽게 찾을
  수 있다.

\item 중간 값들을 변수에 저장하여 출력해보고 검사 할 수 있게 만들어라

\item 프로그램이 동작한다면 발판으로 사용한 코드를 삭제하고 여러 줄로
  작성한 문장들을 복합문으로 통합해라.  단, 통합했을 때 코드 읽기가
  너무 어려워지면 안된다.

\end{enumerate}

연습삼아 {\tt hypotenuse}라는 함수를 만들어 보자.  이 함수는 직각
삼각형의 빗변을 계산하는 함수로 다른 두 변의 길이를 인자로 받는다.
개발 과정의 각 단계를 거칠 때마다 결과를 표시하도록 해보자.
\index{hypotenuse}



\section{합성}
%Composition
\index{composition}
\index{function composition}

이제는 이미 알리라 생각이 되지만, 함수 내에서 다른 함수를 호출하는 것이
가능하다.  원의 중심 점과 원주 위의 한 점을 받아 원의 면적을 계산하는
함수로 예를 들어 보자.

{\tt xc}와 {\tt yc}에 원의 중심 점을 저장하고 원주 위의 한 점은 {\tt
  xp}와 {\tt yp}에 저장된다고 하자.  첫 번째 단계는 이 두 점을 사용하여
원의 반지름을 계산하는 것이다.이미 작성한 {\tt distance} 함수를
사용하자.


\begin{verbatim}
radius = distance(xc, yc, xp, yp)
\end{verbatim}
%
다음으로 반지름을 사용하여 원의 면적을 구하자.  이것도 이미 작성했었다.

\begin{verbatim}
result = area(radius)
\end{verbatim}
%
이 모든 단계들을 한 함수로 캡슐화 해보자.
\index{encapsulation}

\begin{verbatim}
def circle_area(xc, yc, xp, yp):
    radius = distance(xc, yc, xp, yp)
    result = area(radius)
    return result
\end{verbatim}
%
임시 변수 {\tt radius}와 {\tt result}는 개발과 디버깅하는 과정에서만
유용하기 때문에 프로그램이 제대로 동작하면 다음과 같이 간결하게 함수
호출을 합성할 수 있다.

\begin{verbatim}
def circle_area(xc, yc, xp, yp):
    return area(distance(xc, yc, xp, yp))
\end{verbatim}
%

\section{불 함수}
%Boolean functions
\label{boolean}

함수는 불 값을 리턴할 수 있다.  그렇게 하면 복잡한 검사문을 함수에 감출
수 있기 때문에 유용하다.
\index{boolean function}
예를 살펴 보자.

\begin{verbatim}
def is_divisible(x, y):
    if x % y == 0:
        return True
    else:
        return False
\end{verbatim}
%
불 함수의 이름은 참/거짓을 묻는 질문처럼 만드는 것이 일반적이다.
\verb"is_divisible"은 {\tt x}가 {\tt y}로 나누어 떨어지는지 {\tt True}
또는 {\tt False}로 리턴한다.

사용 예를 보자. 

\begin{verbatim}
>>> is_divisible(6, 4)
False
>>> is_divisible(6, 3)
True
\end{verbatim}
%
{\tt ==} 연산자의 결과는 참 또는 거짓이기때문에 그 결과를 바로
리턴하도록 함수를 간결하게 작성할 수 있다.


\begin{verbatim}
def is_divisible(x, y):
    return x % y == 0
\end{verbatim}
%
불 함수는 조건문에 많이 쓰인다. 
\index{conditional statement}
\index{statement!conditional}

\begin{verbatim}
if is_divisible(x, y):
    print('x is divisible by y')
\end{verbatim}
%
아래와 같이 쓰고 싶은 유혹이 들 수 있다. 

\begin{verbatim}
if is_divisible(x, y) == True:
    print('x is divisible by y')
\end{verbatim}
%
두 번 검사는 불필요하다.

연습삼아 \verb"is_between(x, y, z)" 함수를 작성해보자.  이 함수는 $x
\le y \le z$이면 {\tt True}를 아니면 {\tt False}를 리턴한다.


\section{또 다시 재귀문}
%More recursion
\label{more.recursion}
\index{recursion}
\index{Turing complete language}
\index{language!Turing complete}
\index{Turing, Alan}
\index{Turing Thesis}

Python의 일부만을 다뤘지만, 지금까지 다룬 내용이 프로그래밍 언어의
{\em 전부}라는 사실을 알았으면 좋겠다.  지금의 언어로 연산하고 싶은
어떤 것이든 다 표현할 수 있다.  누군가가 만든 프로그램도 지금까지 배운
것들만으로 다시 만들 수 있다(사실 마우스, 디스크같은 것들을 다루려면
다른 명령어들이 필요하겠지만 그게 다다).

이 주장을 증명하는 것은 쉽지 않은 일이긴 하지만 최초의 컴퓨터 과학자인
앨런 튜링(Alan Turing)이 증명했다(그가 수학자라고 주장하는 사람들이
있기는 하지만, 초창기의 대부분의 컴퓨터 과학자들인 수학자로 시작했다).
그의 이름을 따서 튜링 명제라 한다.  이에 관한 완전하고(정확한) 논고는
마이클 싶서(Michael Sipser)의 책 {\em Introduction to the Theory of
  Computation}을 읽어보기를 바란다.

지금까지 배운 도구들로 무엇을 할 수 있는지 보이기 위해 재귀적으로
정의된 수학 함수를 살펴보도록 하겠다.  정의하는 대상을 정의에서 다시
사용한다는 점에서 재귀적 정의는 순환 정의와 유사하다.  순환 정의는
그렇게 도움이 되지 않는다.


\begin{description}

\item[vorpal:] vorpal한 것을 나타내는 형용사
\index{vorpal}
\index{circular definition}
\index{definition!circular}

\end{description}

사전에서 이런 단어를 만나면 짜증이 날 것이다.  반면, $!$로 표기하는
계승(factorial) 함수는 다음과 같이 정의 된다.
%
\begin{eqnarray*}
&&  0! = 1 \\
&&  n! = n (n-1)!
\end{eqnarray*}
%
이 정의는 0의 계승은 1이고 어떤 값 $n$의 계승은 $n$ 곱하기 $n-1$의
계승으로 정의된다.

그래서, $3!$은 3 곱하기 $2!$이고, 다시 2곱하기 $1!$, 그리고 1 곱하기
$0!$이 된다.  정리하면 $3!$은 3 곱하기 2 곱하기 1 곱하기 1이고,
계산하면 6이다.
\index{factorial function}
\index{function!factorial}
\index{recursive definition}

어떤 것을 재귀문으로 정의하였다면 Pytho 프로그램을 써서 평가해면 된다.
먼저, 매개 변수를 정한다.  {\tt factorial}의 경우 당연히 정수를
전달해야 한다.

\begin{verbatim}
def factorial(n):
\end{verbatim}
%
인자가 0이라면 1을 리턴하면 된다.

\begin{verbatim}
def factorial(n):
    if n == 0:
        return 1
\end{verbatim}
%
그 외의 경우를 다루는 방법이 흥미로운데, 재귀 호출을 써서 $n-1$의
계승을 계산한 후 $n$으로 곱하면 된다.

\begin{verbatim}
def factorial(n):
    if n == 0:
        return 1
    else:
        recurse = factorial(n-1)
        result = n * recurse
        return result
\end{verbatim}
%
이 프로그램의 실행의 흐름은 \ref{recursion}장에서 본 {\tt countdown}의
실행 흐름과 비슷하다. {\tt factorial}의 인자로 3을 주고 시작하면 다음과
같이 동작한다.


3은 0이 아니기 때문에 두 번째 분기를 선택하여 {\tt n-1}의 계승을 계산한다. 

\begin{quote}
  2은 0이 아니기 때문에 두 번째 분기를 선택하여 {\tt n-1}의 계승을
  계산한다.



  \begin{quote}
    1은 0이 아니기 때문에 두 번째 분기를 선택하여 {\tt n-1}의 계승을
    계산한다.

    \begin{quote}
      0은 0이기 때문에 더 이상 재귀 호출을 하지 않고 첫 번째 분기를
      선택하여 1을 리턴한다.
    \end{quote}

    리턴 값 1을  $n$과 곱한 결과를 리턴한다.  이 때, $n$은 1이다. 
  \end{quote}

    리턴 값 1을  $n$과 곱한 결과를 리턴한다.  이 때, $n$은 2이다. 

\end{quote}

리턴 값 2을 $n$과 곱한 결과를 리턴한다.  이 때, $n$은 3이다.  계산 결과
6이 처음 재귀 호출 과정을 시작한 함수의 리턴 값이 된다.

\index{stack diagram}

그림~\ref{fig.stack3}이 함수의 실행 흐름에 따른 스택 상태도를 나타낸다. 

\begin{figure}
\centerline
{\includegraphics[scale=0.8]{figs/stack3.pdf}}
\caption{스택 상태도}
\label{fig.stack3}
\end{figure}

리턴 값이 스택 위로 전달되는 것을 볼 수 있다.  각 프레임에 나타난 리턴
값은 {\tt n}과 {\tt recurse}의 곱인 {\tt result}이다.
\index{function frame}
\index{frame}

마지막 프레임에는 {\tt recurse}와 {\tt result}가 없다.  이 변수들을
생성하는 분기가 실행되지 않았기 때문이다.



\section{믿음의 도약}
%Leap of faith
\index{recursion}
\index{leap of faith}

실행 흐름에 따라 프로그램을 읽는 것도 한 방법이기는 하지만, 코드의
길이에 압도될 수도 있다.  그 대안으로 ``믿음의 도약(leap of faith)''을
하는 방법을 소개하겠다.  함수 호출을 만나면 실행 흐름대로 그 함수의
내용을 다 읽는 대신, 그 함수가 제대로 동작하고 리턴 값도 제대로
전달한다고 가정 하자.

사실, 내장 함수은 이미 제대로 동작한다는 믿음을 갖고 사용했었다.  {\tt
  math.cos}나 {\tt math.exp}를 호출 했을 때 그 함수의 내용을 일일이
확인해보지 않았다. 그저 내장 함수들은 훌륭한 프로그래머가 작성했을
것이라 믿고 동작할 것이라 생각했다.

자신이 작성한 함수도 똑같은 믿음을 갖고 사용해도 된다.
\ref{boolean}절에서 작성한 \verb"is_divisible"을 떠올려 보자.  이
함수가 정확하다고 믿은 순간---코드 확인과 검사를 거친 후---그 함수의
내용을 다시 읽어보지 않고 사용했다.
\index{testing!leap of faith}

재귀 프로그램도 다르지 않다.  재귀 호출을 만나면 실행 흐름을 쫓는 대신
재귀 호출이 제대로 동작한다고 믿어보자(그리고 정확한 결과도 리턴한다고
해보자).  그리고 ``$n-1$의 계승을 구할 수 있다고 한다면 $n$의 계승도
계산할 수 있을까?''라고 물어야 한다.  $n$만 곱하면 되기 때문에 당연히
가능하다.

물론, 작성을 다 하지 않은 함수를 제대로 동작한다고 가정하는게 이상할 수
있지만 그렇기 때문에 믿음의 도약이라 부르는 것이다.



\section{또 하나의 예}
%One more example
\label{one.more.example}

\index{fibonacci function} \index{function!fibonacci} {\tt factorial}을
살펴보았는데, 가장 흔한 재귀적으로 정의한 수학 함수는 {\tt
  fibonacci}이다.  정의는 다음과 같다(참고:
\url{http://en.wikipedia.org/wiki/Fibonacci_number}).
%
\begin{eqnarray*}
&& \mathrm{fibonacci}(0) = 0 \\
&& \mathrm{fibonacci}(1) = 1 \\
&& \mathrm{fibonacci}(n) = \mathrm{fibonacci}(n-1) + \mathrm{fibonacci}(n-2)
\end{eqnarray*}
%
Python으로 해석하면 다음과 같다. 

\begin{verbatim}
def fibonacci(n):
    if n == 0:
        return 0
    elif  n == 1:
        return 1
    else:
        return fibonacci(n-1) + fibonacci(n-2)
\end{verbatim}
%
아무리 작은 $n$이라 할지라도 이 함수의 실행 흐름을 따라가려면 머리가
터질 것이다.  믿음의 도약을 한다고 생각하고 두개의 재귀 호출이 제대로
동작한다고 믿어보자.  그러면 결과들을 더했을 때 올바른 값을 얻는다는
것이 명확해질 것이다.
\index{flow of execution}


\section{데이터 형 검사}
%Checking types
\label{guardian}

{\tt factorial}을 호출 할 때 인자가 1.5면 어떻게 될까?
\index{type checking}
\index{error checking}
\index{factorial function}
\index{RuntimeError}

\begin{verbatim}
>>> factorial(1.5)
RuntimeError: Maximum recursion depth exceeded
\end{verbatim}
%
무한 재귀문 오류인 것처럼 보인다.  어째서 그럴가?  {\tt n == 0}이라는
기준 케이스도 있는데 말이다.  {\tt n}이 정수가 아니면 기준 케이스에
도달 못하고 영원히 재귀 호출을 하게 된다.
\index{infinite recursion}
\index{recursion!infinite}

첫 번째 재귀 호출에서 {\tt n}의 값은 0.5이다.  그 다음에는 -0.5가
된다.  그 이후에는 수는 작아지기만 하지 절대로 0은 안된다.

두 가지 선택이 있다.  {\tt factorial} 함수를 일반화해서 부동소수점 수도
계산할 수 있게 하거나 {\tt factorial}이 인자의 데이터 형을 검사할 수도
있다.  첫 번째는 감마 함수라고 부르고 이 책이 다룰 수 있는 범위 밖에
있다.  그러니 두 번째 것을 시도해보자.
\index{gamma function}

내부 함수 중 {\tt isinstance}을 사용해서 인자의 데이터 형을 검사할 수
있다.  그와 동시에 인자로 받은 값이 양수인지도 확인할 수 있다.
\index{isinstance function}
\index{function!isinstance}

\begin{verbatim}
def factorial(n):
    if not isinstance(n, int):
        print('정수에 대해서만 정의되어 있음.')
        return None
    elif n < 0:
        print('음의 정수에 대해서는 정의 안됨.')
        return None
    elif n == 0:
        return 1
    else:
        return n * factorial(n-1)
\end{verbatim}
%
첫 기준 케이스는 정수가 아닌 경우를 다루고 두 번째는 음의 정수를
다룬다.  두 경우다 프로그램은 오류 메시지를 출력 하고 무언가
잘못되었다는 것을 표시하기 위해 {\tt None}을 리턴한다.


\begin{verbatim}
>>> print(factorial('fred'))
정수에 대해서만 정의되어 있음.
None
>>> print(factorial(-2))
음의 양수에 대해서는 정의 안됨.
None
\end{verbatim}
% 
두 검사를 통과했다면, $n$은 양수의 정수이거나 0이라는 것을 알기 때문에
재귀문이 종료한다는 것을 증명할 수 있다.
\index{guardian pattern}
\index{pattern!guardian}

이 프로그램은 {\bf 가디언(guardian}, 보호자{\bf)}이라는 패턴을
사용하였다.  첫 두 조건문이 오류가 발생할만한 값들로부터 코드를
보호하는 가디언 역할을 한다.  가디언 패턴으로 코드의 정확성을 증명할 수
있다. 

\ref{raise}절에서는 오류 메시지를 출력하는 것보다 좀 더 유연한 대안인
엑셉션(exception, 예외) 처리에 대해 살펴 볼 것이다.


\section{디버깅}
%Debugging
\label{factdebug}
큰 프로그램을 작은 함수들로 나누어 작성하는 것은 디버깅을 할 수 있는
지점들을 자연스럽게 만드는 효과가 있다.  함수가 동작하지 않을 때 검사해
볼 세가지가 있다.  
\index{debugging}

\begin{itemize}

\item 함수가 전달 받는 인자에 문제가 있다. 사전 조건이 잘못되었다. 

\item 함수에 문제가 있다. 사후 조건이 잘못되었다. 

\item 리턴 값 또는 그 값을 사용하는데 문제가 있다. 

\end{itemize}

첫 가능성을 제거하려면 함수 시작부분에 {\tt print}문을 써서 매개 변수의
값(그리고 데이터 형)을 출력해 보면 된다.  또는 사전 조건이 정확한지
검사하는 코드를 명시적으로 작성하면 된다.
\index{precondition}
\index{postcondition}

매개 변수가 맞아 보이면 모든 {\tt return}문 전에 {\tt print}문을
삽입하여 리턴 값을 검사해보자.  함수 호출의 결과를 수기로도
검산해보자.  검산이 쉬운 값을 넣어 함수를
호출해보자(\ref{incremental.development}절을 참고하자).

함수가 제대로 동작하는 것처럼 보이면 호출하는 문장에서 리턴 값을
정확하게 사용하는지(또는 사용은 하는지) 확인해보자.
\index{flow of execution}


함수가 시작할 때 그리고 끝날 때 {\tt print}문을 넣어보면 실행 흐름을
시각화하는데 도움이 된다. 다음은 {\tt factorial}에 {\tt print}문을 넣은
버전이다.

\begin{verbatim}
def factorial(n):
    space = ' ' * (4 * n)
    print(space, 'factorial', n)
    if n == 0:
        print(space, 'returning 1')
        return 1
    else:
        recurse = factorial(n-1)
        result = n * recurse
        print(space, 'returning', result)
        return result
\end{verbatim}
%
{\tt space}는 공백 문자열로 결과의 들여쓰기 정도를 나타낸다. 다음은
{\tt factorial(4)}의 결과이다.

\begin{verbatim}
                 factorial 4
             factorial 3
         factorial 2
     factorial 1
 factorial 0
 returning 1
     returning 1
         returning 2
             returning 6
                 returning 24
\end{verbatim}
%
실행 흐름이 혼란스럽다면 이런 식으로 결과를 표시하는 것도 도움이 된다.
효과적인 발판을 만드는 것은 시간이 걸리지만 작은 발판이라할지라도
디버깅 시간을 크게 줄일 수 있다.


\section{용어 해설}
%Glossary

\begin{description}

\item[임시변수(temporary variable):] 복잡한 계산의 중간 값을 저장하는데
  사용하는 변수
\index{temporary variable}
\index{variable!temporary}

\item[죽은 코드(dead code):] 대체적으로 {\tt return}문 다음에 오는
  코드로 프로그램에서 실행되지 않는 부분
\index{dead code}

\item[점진적 개발(incremental development):] 한 번에 작은 크기의 코드를
  추가하고 검사하여 디버깅 시간을 줄이려는 프로그램 걔발 계획
\index{incremental development}

\item[발판(scaffolding):] 프로그램 개발 중에는 사용되나 최종 버전에는
  제외되는 코드
\index{scaffolding}

\item[가디언(보호자, guardian):] 오류를 발생시킬 수도 있는 상황을
  조건문으로 처리하거나 검사하는 프로그래밍 패턴

\index{guardian pattern}
\index{pattern!guardian}

\end{description}


\section{연습 문제}
%Exercises

\begin{exercise}

다음 프로그램의 스택 상태도를 그려라.  프로그램은 무엇을 표시하는가?
\index{stack diagram}

\begin{verbatim}
def b(z):
    prod = a(z, z)
    print(z, prod)
    return prod

def a(x, y):
    x = x + 1
    return x * y

def c(x, y, z):
    total = x + y + z
    square = b(total)**2
    return square

x = 1
y = x + 1
print(c(x, y+3, x+y))
\end{verbatim}

\end{exercise}


\begin{exercise}
\label{ackermann}

아커만 함수 $A(m, n)$은 다음과 같이 정의 되었다. 

\begin{eqnarray*}
A(m, n) = \begin{cases} 
              n+1 & \mbox{if } m = 0 \\ 
        A(m-1, 1) & \mbox{if } m > 0 \mbox{ and } n = 0 \\ 
A(m-1, A(m, n-1)) & \mbox{if } m > 0 \mbox{ and } n > 0.
\end{cases} 
\end{eqnarray*}
%
상세 설명은 \url{http://en.wikipedia.org/wiki/Ackermann_function}을
참고하라.  아커만 함수를 계산하는 {\tt ack} 함수를 작성하라.  {\tt
  ack(3, 4)}의 값을 구하고 125가 맞는지 확인하라.  {\tt m}이나 {\tt
  n}이 더 큰 값이면 어떻게 되는가? 해답은
\url{http://thinkpython2.com/code/ackermann.py}에 있다.
\index{Ackermann function}
\index{function!ack}

\end{exercise}


\begin{exercise}
\label{palindrome}
회문(pallindrome)은 ``noon''과 ``redivider''처럼 단어의 순방향 철자가
역방향 철자와 동일한 단어를 뜻한다.  재귀적으로 표현했을 때, 첫 글자와
마지막 글자가 같고 중간 부분이 회문이면 그 단어는 회문이다.
\index{palindrome}

다음의 함수들은 문자열을 인자로 받아 첫 글자, 마지막 글자, 그리고
중간의 글자들을 리턴한다.

\begin{verbatim}
def first(word):
    return word[0]

def last(word):
    return word[-1]

def middle(word):
    return word[1:-1]
\end{verbatim}
%
동작 방식은 \ref{strings}장에서 살펴볼 것이다. 

\begin{enumerate}

\item {\tt palindrome.py}라는 이름을 함수들을 저장하고 검사해보자.  두
  글자 길이의 단어를 {\tt middle} 함수에 인자로 전달하면 어떻게
  되는가?  한 글자이면 어떻게 되는가? 빈 문자열 \verb"''"을 인자로
  전달하면 어떻게 되는가?

\item 문자열을 인자로 받아 그 문자열이 회문이면 {\tt True}를 아니면
  {\tt False}를 리턴하는 \verb"is_palindrome" 함수를 작성하라.  참고로
  문자열의 길이를 확인하기 위해 내장 함수인 {\tt len}을 쓸 수 있다.

\end{enumerate}

해답: \url{http://thinkpython2.com/code/palindrome_soln.py}.

\end{exercise}

\begin{exercise}

  어떤 수 $a$는 $b$로 나눌 수 있고 $a/b$가 $b$의 거듭제곱이면 $a$는
  $b$의 거듭제곱이다. {\tt a}와 {\tt b}를 매개 변수로 받아 {\tt a}가
  {\tt b}의 거듭제곱이면 {\tt True}를 리턴하는 \verb"is_power" 함수를
  작성하라.  참고: 기준 케이스 선정이 중요하다.


\end{exercise}


\begin{exercise}
\index{greatest common divisor (GCD)}
\index{GCD (greatest common divisor)}

$a$와 $b$의 최대 공약수(the Greatest Common Divisor, GCD)는 나머지 없이
두 수를 나눌 수 있는 가장 큰 수이다.

최대 공약수를 찾는 한 방법은 $a$을 $b$로 나눴을 때의 나머지 $r$에 대해
$gcd(a, b) = gcd(b, r)$의 관계가 성립한다는 관찰을 사용하면 된다.  기준
케이스로 $gcd(a, 0) = a$를 사용하라.

매개 변수 {\tt a}와 {\tt b}를 사용하고 최대 공약수를 리턴하는
\verb"gcd" 함수를 작성하라.


출처: 애벨슨(Abelson)과 서스만(Sussman)의 {\em Structure and
  Interpretation of Computer Programs}의 예제를 참고하였다.

\end{exercise}


\chapter{반복}
%Iteration

이 장에서는 문장을 되풀이하여 실행하는 기능인 반복을 다룬다.
\ref{recursion}절에서 재귀문을 사용한 반복의 일종을 살펴봤었다.
\ref{repetition}절에서 {\tt for} 루프도 봤었다.  이 장에서는 {\tt
  while}문을 사용한 또 다른 반복문을 배울 것이다.  그 전에 변수 할당에
대해 좀 더 이야기하고 시작하겠다.


\section{재할당}
%Reassignment
\index{assignment}
\index{statement!assignment}
\index{reassignment}

같은 변수에 한 번 이상 값을 할당해도 괜찮다는 것을 이제 깨달았을
것이다.  기존의 변수에 새로운 값을 할당하면 그 변수는 새로운 값을
가리킨다(이전에 가리키던 값은 더 이상 가리키지 않는다).



\begin{verbatim}
>>> x = 5
>>> x
5
>>> x = 7
>>> x
7
\end{verbatim}
%
{\tt x}는 처음에 5를 그리고 두 번째 7이라는 값을 갖는다.

그림 \ref{fig.assign2}의 스택 상태도에 {\bf 재할당(reassignment)}를 나타냈다.  
\index{state diagram} \index{diagram!state}

현 시점에서 많이들 혼돈하는 것을 집고 넘어가자.  Python에서 등호({\tt
  =})를 사용하는데 이 기호를 수학 명제로서 {\tt a = b}를
해석하려고한다.  수학에서는 {\tt a}와 {\tt b}가 서로 동일하다는
주장이다.  하지만, 그렇게 해석하면 틀린 것이다. 
\index{equality and assignment}


첫째로 등가는 대칭적 관계를 나타내지만 할당은 대칭적이지 않다.  예를
들어보자.  수학에서는 $a = 7$이면 $7 = a$이다.  Python에서는 {\tt a =
  7 }이라는 문장은 괜찮지만 {\tt 7 = a}라는 문장은 틀린 문장이다.


수학에서는 어떤 식이 등가라면 그 주장은 항상 참이다.  $a = b$라고 하면
$a$는 항상 $b$와 동일하다.  Python에서 할당문은 두 변수가 서로 동일하게
만들지만 항상 그대로 유지되는 것은 아니다.


\begin{verbatim}
>>> a = 5
>>> b = a    # a와 b는 같음
>>> a = 3    # a와 b는 더 이상 같지 않음
>>> b
5
\end{verbatim}
%
세 번째 줄은 {\tt a}를 변경하지만 {\tt b}는 변경시키지 않는다. 그렇기
때문에 둘은 더 이상 같지 않다.

변수 재할당은 유용하지만 신중하게 사용해야 한다.  변수의 값이 자주
바뀌면 코드 읽기와 디버깅이 어려워진다.

\begin{figure}
\centerline
{\includegraphics[scale=0.8]{figs/assign2.pdf}}
\caption{스택 상태도}
\label{fig.assign2}
\end{figure}



\section{변수 갱신하기}
%Updating variables
\label{update}

\index{update}
\index{variable!updating}

흔한 할당 방법은 {\bf 갱신(update)}이다.  이런 종류는 변수의 새로운
값은 이전 변수의 값에 의존한다.

\begin{verbatim}
>>> x = x + 1
\end{verbatim}
%
이 문장은 ``현재 {\tt x}의 값에 1을 더하고 그 값으로 {\tt x}을 갱신한다. 

존재하지 않는 변수를 갱신하면 오류가 발생한다.  왜냐하면 Python은
오른쪽의 표현을 먼저 계산하고 {\tt x}에 값으로 할당하기 때문이다.


\begin{verbatim}
>>> x = x + 1
NameError: name 'x' is not defined
\end{verbatim}
%
변수의 값을 갱신하기 전에 변수를 먼저 간단한 할당문으로 {\bf
  초기화(initialize)}해야 한다.
\index{initialization (before update)}

\begin{verbatim}
>>> x = 0
>>> x = x + 1
\end{verbatim}
%
변수의 값에 1을 더하는 것을 {\bf 증가(increment)}, 1을 빼는 것을 {\bf
  감소(decrement)}라 부른다.
\index{increment}
\index{decrement}




\section{{\tt while}문}
%The {\tt while} statement
\index{statement!while}
\index{while loop}
\index{loop!while}
\index{iteration}

컴퓨터는 반복적인 작업을 자동하는데 주로 사용된다.  동일하거나 유사한
작업을 오류 없이 처리하는 것을 컴퓨터는 잘하지만 사람은 그렇지 못하다.
컴퓨터 프로그램에서 퇴풀이되는 작업을 {\bf 반복(iteration)}이라 한다.

우리는 재귀문을 사용하여 반복하는 {\tt countdown}과 \verb"print_n"
함수를 봤었다.  반복 작업이 너무나 흔하기 때문에 Python은 언어 차원에서
반복 기능을 제공하고 있다.  그 중 하나는 \ref{repetition}에서 봤던
{\tt for}문이다.  다시 한 번 살펴볼 것이다.

또 다른 것은 {\tt while}문이다.  {\tt countdown}을 {\tt while}문으로
바꿔서 다시 작성해보자.


\begin{verbatim}
def countdown(n):
    while n > 0:
        print(n)
        n = n - 1
    print('발사!')
\end{verbatim}
%
{\tt while}문이 마치 영어처럼 거의 자연스럽게 읽혀진다.  해석하면
``{\tt n}이 0보다 큰 동안에 {\tt n}의 값을 표시하고 그 값을
감소시킨다.  0이 되면 {\tt 발사!}라고 표시''가 된다.
\index{flow of execution}

좀 더 체계적으로 {\tt while}문의 실행 흐름을 살펴보자. 

\begin{enumerate}

\item 조건이 참인지 거짓인지 판단한다. 

\item 거짓이면 {\tt while}문을 끝내고 다음 문장을 실행한다. 

\item 조건이 참이면 {\tt while}의 내용을 실행하고 첫 단계로 간다. 

\end{enumerate}

이와 같은 흐름을 루프(loop, 반복)라 부른다. 세 번째 단계가 처음으로
되돌아 반복하기 때문이다.
\index{condition}
\index{loop}
\index{body}

루프의 내용은 하나 또는 그 이상의 변수를 바꿔야 한다.  그래야 조건이
언젠가는 거짓이 되서 루프가 끝이나기 때문이다.  그렇지 않으면 루프는
무한히 되풀이될 것이다.  이런 경우를 {\bf 무한 루프(infinite loop)}이라
부른다.  샴푸의 사용법을 읽어보면 컴퓨터 과학자들은 놀라지 않을 수가
없다.  ``비누거품, 린스, 반복''라고 써 있는데, 이는 무한 루프이기
때문이다.
\index{infinite loop}
\index{loop!infinite}

{\tt countdown}의 경우 루프가 끝이 있음을 증명할 수 있다.  만약 {\tt
  n}이 0이거나 음수이면 루프는 시작도 안된다.  그 외에는 0이 될 때까지
매번 루프를 돌 때마다 {\tt n}은 감소된다.

어떤 루프의 경우에는 판단하기가 쉽지않다.  에를 들어보자. 

\begin{verbatim}
def sequence(n):
    while n != 1:
        print(n)
        if n % 2 == 0:        # n은 짝수
            n = n / 2
        else:                 # n은 홀수
            n = n*3 + 1
\end{verbatim}
%
이 루프의 조건은 {\tt n != 1}이기 때문에 루프는 {\tt n}이 {\tt 1}이
되어 거짓이 될 때까지 계속된다.

프로그램은 루프를 돌 때마다 {\tt n}의 값을 표시하고 짝수 또는 홀수
검사를 한다.  짝수라면 {\tt n}를 2로 나누고, 홀수라면 {\tt n}은 {\tt
  n*3 +1}으로 대체된다.  {\tt sequence}에 인자를 3으로 전달했다고
해보자.  {\tt n}의 결과는 3, 10, 5, 16, 8, 4, 2, 1이 된다.

{\tt n}가 증가되기도 하고 감소하기도 하기 때문에 1에 반드시
도달한다거나 프로그램이 종료한다는 증명은 간단하지 않다.  {\tt n}이
특정 값인 경우라면 종료한다는 것을 증명할 수 있다.  시작 값이 2의
배수라면 {\tt n}은 매번 짝수가 되기 때문에 루프를 반복하다보면 1에
도달하게 된다.  16으로 시작한 경우가 그렇게 끝이 난다.
\index{Collatz conjecture}

좀 더 어려운 질문은 이 프로그램이 {\em 모든} 양수에 대해서
종료하는가이다.  아직까지는 그 어느 누구도 이것을 증명{\em 도}
반증하지{\em 도} 못했다!  (참고:
\url{http://en.wikipedia.org/wiki/Collatz_conjecture}.)

연습 문제로 \ref{recursion}절의 \verb"print_n" 함수를 반복문을 사용하여
재작성해보라.


\section{{\tt break}문}
\index{break statement}
\index{statement!break}

때로는 루프의 반정도를 실행하기 전까지는 루프를 종료할 때가 되었는지
모를 때가 있다.  그런 경우에 {\tt break}문을 사용하여 루프 밖으로 나올
수 있다.

사용자가 {\tt done}을 입력하기 전까지 계속 입력을 받는 예를 살펴보자.
다음과 같이 작성하면 된다.

\begin{verbatim}
while True:
    line = input('> ')
    if line == 'done':
        break
    print(line)

print('Done!')
\end{verbatim}
%
루프의 조건이 {\tt True}라고 되어 있기 때문에 항상 참이다.  이 루프는
{\tt break}문을 만나기 전까지 계속 반복된다.

되풀이 될 때마다 꺽쇠 표시로 사용자에게 입력을 요청한다.  사용자가
{\tt done}이라 쓰면 {\tt break}문이 루프를 끝낸다.  그 외에는 사용자가
무엇을 입력하든 그것을 화면에 표시하고 루프의 처음으로 되돌아 간다.
여기 실행 예시가 있다.


\begin{verbatim}
> not done
not done
> done
Done!
\end{verbatim}
%
{\tt while} 루프를 이렇게 사용하는 두 가지 이유가 있다.  첫 째는 조건이
루프 내 어디서든(루프 맨위에서 한 번이 아니라) 비교가 되기 때문이다.
두 번째는 종료 조건을 부정적으로(``이것이 발생하기 전까지 계속'')
표현하는 대신 긍정적으로(``이것이 발생하면 멈춤'') 표현할 수 있기
때문이다.


\section{제곱 근}
%Square roots
\label{squareroot}
\index{square root}

루프는 근사치를 구하고 반복적으로 그 값을 더 정확하게 만드는 수치해석
프로그램에서 주로 쓰인다.
\index{Newton's method}

예를 들어 보자.  제곱 근을 구하는 방법 중 하나는 뉴튼의 기법(Newton's
method)이다.  $a$의 제곱 근을 알고 싶다고 해보자.  어떤 추정치 $x$에서
시작하든 다음의 식으로 좀 더 정확한 근사치를 얻을 수 있다.

\[ y = \frac{x + a/x}{2} \]
%
$a$는 4 $x$는 3이라고 해보자.

\begin{verbatim}
>>> a = 4
>>> x = 3
>>> y = (x + a/x) / 2
>>> y
2.16666666667
\end{verbatim}
%
정답에 좀 더 가까운 결과를 얻었다($\sqrt{4} = 2$).  이 과정을 새로운
추정치로 반복한다면 좀 더 정답에 가까워진다.

\begin{verbatim}
>>> x = y
>>> y = (x + a/x) / 2
>>> y
2.00641025641
\end{verbatim}
%
몇 번 더 갱신하면 추정치는 거의 정답에 가까워진다. 
\index{update}

\begin{verbatim}
>>> x = y
>>> y = (x + a/x) / 2
>>> y
2.00001024003
>>> x = y
>>> y = (x + a/x) / 2
>>> y
2.00000000003
\end{verbatim}
%
일반적으로 이 과정을 몇 번이나 거쳐야 정답을 얻게 되는지 모른다.
그렇지만, 정답에 가까워지면 알게된다.  왜냐하면 추정치의 변화하지 않기
때문이다.


\begin{verbatim}
>>> x = y
>>> y = (x + a/x) / 2
>>> y
2.0
>>> x = y
>>> y = (x + a/x) / 2
>>> y
2.0
\end{verbatim}
%
{\tt y == x}이면 멈추면 된다.  다음은 초기 추정치 {\tt x}로부터 변화가
없을 때까지 정확도를 높이는 루프이다.

\begin{verbatim}
while True:
    print(x)
    y = (x + a/x) / 2
    if y == x:
        break
    x = y
\end{verbatim}
%
대부분의 {\tt a}는 동작을 하지만, {\tt float}형을 등호에 쓸 때는
조심해야 한다.  부동소수점 값은 근사치이기 때문이다.  $1/3$처럼
유리수인 경우나 $\sqrt{2}$처럼 무리수인 경우의 대부분은 {\tt
  float}형으로 그 값을 정확하게 표현될 수 없다는 것을 기억해야 한다.
\index{floating-point}
\index{epsilon}

{\tt float}형인 {\tt x}와 {\tt y}가 서로 정확히 일치하는지 등호로
검사하는 대신 내장 함수인 {\tt abs}를 사용하여 그 수의 절대치 또는
크기를 비교하는 것이 좋다.

\begin{verbatim}
    if abs(y-x) < epsilon:
        break
\end{verbatim}
%
\verb"epsilon"는 {\tt 0.0000001}와 같은 값으로 얼마나 작아야 같다고 할
수 있는지 판단하는데 도움이 된다.


\section{알고리즘}
%Algorithms
\index{algorithm}

뉴튼의 기법은 {\bf 알고리즘(algorithm)}의 예이다.  어떤 종류의 문제(이
경우에는 제곱 근의 계산)를 해결하기 위한 기계적인 과정이다.

알고리즘이 무엇인지 이해하기 위해서는 알고리즘이 아닌 무엇인가로부터
시작하는 것이 좋겠다.  한 자리 수 숫자로 곱셉하는 법을 배웠을 때
구구단을 외웠을 것이다.  사실상은 100개의 서로 다른 해답을 외웠을
뿐이다.  이런 종류의 지식은 알고리즘적 접근이 전혀 아니다.

당신이 만약 ``게으르다면'' 요령을 배울 필요가 있다.  예를 들어 어떤 수
$n$과 9의 곱을 구한다고 해보자.  일의 자리에 $n-1$을 쓰고 십의 자리에
$10-n$을 쓰면 된다.  이 요령은 한 자리 수 숫자와 9의 곱에 대한 일반
해이다.  이런 것이 바로 알고리즘이다!
\index{addition with carrying}
\index{carrying, addition with}
\index{subtraction!with borrowing}
\index{borrowing, subtraction with}

올림이 있는 덧셈이나 빌려서 뺄셈하기와 나눗셈하는 기술들이 모두
알고리즘이다.  알고리즘이 갖는 특성 중 하나는 지적 능력이 전혀 필요
없다는 것이다.  기계적으로 간단한 규칙에 따라 이전 단계에서 다음 단계를
따르면 되기 때문이다.

알고리즘의 실행은 지겨운 일이지만 설계하는 것은 흥미진진하고 지적으로
도전적이다.  뿐만아니라 컴퓨터 과학의 중심이기도 하다.

사람들이 자연스럽게 어려움 없이 또는 깊이 생각하지 않고 하는 것들
중에는 알고리즘으로 표현하기 매우 어려운 것들이 있다.  자연어 인식이
좋은 예이다.  우리는 하지만 아직까지 {\em 어떻게} 그렇게 할 수 있는지
누구도 설명하지 못했다.  최소한 알고리즘으로 풀어내지는 못했다.



\section{디버깅}
%Debugging
\label{bisectbug}

프로그램의 크기가 커지다보면 더 많은 시간을 디버깅에 쓰게 되는 것을
보게 된다.  더 많은 코드는 오류를 만들 확률이 많아지고 버그가 숨을 곳이
더 많아진다는 말이다.
\index{debugging!by bisection}
\index{bisection, debugging by}

디버깅 시간을 줄이는 방법 중 하나는 ``등분으로 디버깅하기''이다.  예를
들어 100 줄짜리 프로그램이 있다고 했을 때 한 줄씩 검사한다면 100번
검사해야 한다.

대신에, 문제를 반으로 쪼개보자.  프로그램의 중간 또는 그 근처에 있는
중간 과정의 값 중 검사할 수 있는 것을 확인해보자.  {\tt print}문(증명을
도와줄 어떤 것)을 추가한 후 프로그램을 실행하자.


중간 점검이 부정확하다면 전반부에 오류가 있다는 말이다.  정확하다면
문제는 후반부에 있다. 

이런 식으로 검사를 할 때마다 검사해야할 줄의 수를 반씩 줄여나가면
된다.  이론상으로는 이 과정을 6번(100번 보다 작다) 반복하면 검사해야 할
줄의 수가 하나 또는 두 줄 정도가 된다.

실제로는 ``프로그램의 중간''이 늘 명확하지도 않고 그 부분을 검사한다는
것이 가능하지도 않다.  정확한 중간 지점을 찾기 위해 라인 수를 세서
반으로 나눈다는 것도 말이 안된다.  대신에 오류가 있을만한 부분을
생각해보고 검사 문장을 쉽게 삽입할 수 있는 위치를 생각해보자.  그리고
검사 문장 삽입 위치 전과 그 후에 오류가 있을 확률이 서로 비슷한 지점에
검사 문장을 삽입하자.





\section{용어 해설}
%Glossary

\begin{description}

\item[재할당(reassignment):]  이미 존재하는 변수에 새로운 값을 할당하는 것.
\index{reassignment}

\item[갱신(update):] 변수의 새로운 값이 이전 값에 의존하는 할당문.
\index{update}

\item[초기화(initialization):] 갱신될 변수에 초기 값을 할당하는 것.
\index{initialization!variable}

\item[증가(increment):] 변수의 값을 증가 시키는 갱신(주로 1 증가).
\index{increment}

\item[감소(decrement):] 변수의 값을 감소 시키는 갱신.
\index{decrement}

\item[반복(iteration):] 재귀 함수 호출이나 루프를 사용하여 문장들을
  되풀이하여 실행하는 것
\index{iteration}

\item[무한 루프(infinite loop):] 종료 조건이 절대로 충족되지 않는 루프.
\index{infinite loop}

\item[알고리즘(algorithm):]  어떤 종류의 문제를 해결하는 일반적인 과정.
\index{algorithm}

\end{description}


\section{연습 문제}
%Exercises

\begin{exercise}
\index{algorithm!square root}

\ref{squareroot} 절의 루프문을 복사하여 \verb"mysqrt"라는 함수로
캡슐화하라.  이 때 {\tt a}를 매개 변수로하고 타당한 {\tt x}을
선정하자.  그리고 {\tt a}의 제곱 근의 추정치를 리턴하도록 하라.
\index{encapsulation}

검사를 위해 \verb"test_square_root"라는 함수를 만들고 다음과 같이
출력하도록 만들자.

\begin{verbatim}
a   mysqrt(a)     math.sqrt(a)  diff
-   ---------     ------------  ----
1.0 1.0           1.0           0.0
2.0 1.41421356237 1.41421356237 2.22044604925e-16
3.0 1.73205080757 1.73205080757 0.0
4.0 2.0           2.0           0.0
5.0 2.2360679775  2.2360679775  0.0
6.0 2.44948974278 2.44948974278 0.0
7.0 2.64575131106 2.64575131106 0.0
8.0 2.82842712475 2.82842712475 4.4408920985e-16
9.0 3.0           3.0           0.0
\end{verbatim}
%
첫 번재 열은 숫자 $a$이다.  두 번째 열은 \verb"mysqrt"으로 계산된 $a$의
제곱근이다.  세 번째 열은 {\tt math.sqrt}으로 계산된 제곱 근이다.  네
번째 열은 두 추정치의 차이의 절대치를 나타낸다.
\end{exercise}


\begin{exercise}
\index{eval function}
\index{function!eval}

내장 함수인 {\tt eval}은 문자열을 받아 Python 인터프리터를 사용하여
계산한다.  다음 예를 살펴보자.

\begin{verbatim}
>>> eval('1 + 2 * 3')
7
>>> import math
>>> eval('math.sqrt(5)')
2.2360679774997898
>>> eval('type(math.pi)')
<class 'float'>
\end{verbatim}
%
반복저으로 사용자의 입력을 기다리는 \verb"eval_loop" 함수를
작성해보자.  전달받은 값을 {\tt eval} 함수로 계산하고 그 결과를
출력하도록 만들자.

사용자가 \verb"'done'"을 입력하기 전까지 계속 동작하고 종료하면서
마지막 계산한 수식의 결과를 표시하도록 만들자.

\end{exercise}


\begin{exercise}
\index{Ramanujan, Srinivasa}

스리니바사 라마누잔(Srinivasa Ramanujan)은 무한 급수를 발견하였다.
무한 급수는 $1 / \pi$의 근사치를 구하는데 쓰인다.
\index{pi}

\[ \frac{1}{\pi} = \frac{2\sqrt{2}}{9801} 
\sum^\infty_{k=0} \frac{(4k)!(1103+26390k)}{(k!)^4 396^{4k}} \]

위의 공식을 사용하여 $\pi$의 추청지를 계산하여 리턴하는
\verb"estimate_pi" 함수를 작성해보자.  {\tt while} 루프를 사용하여
추정치의 마지막 항이 {\tt 1e-15}(Python에서 $10^{-15}$의 표기)보다 작을
때까지 반복하라.  얻은 결과는 {\tt math.pi}와 비교해 볼 수 있다.

해답: \url{http://thinkpython2.com/code/pi.py}.

\end{exercise}


\chapter{Strings}
\label{strings}

Strings are not like integers, floats, and booleans.  A string
is a {\bf sequence}, which means it is
an ordered collection of other values.  In this chapter you'll see
how to access the characters that make up a string, and you'll
learn about some of the methods strings provide.
\index{sequence}


\section{A string is a sequence}

\index{sequence}
\index{character}
\index{bracket operator}
\index{operator!bracket}
A string is a sequence of characters.  
You can access the characters one at a time with the
bracket operator:

\begin{verbatim}
>>> fruit = 'banana'
>>> letter = fruit[1]
\end{verbatim}
%
The second statement selects character number 1 from {\tt
fruit} and assigns it to {\tt letter}.  
\index{index}

The expression in brackets is called an {\bf index}.  
The index indicates which character in the sequence you
want (hence the name).

But you might not get what you expect:

\begin{verbatim}
>>> letter
'a'
\end{verbatim}
%
For most people, the first letter of \verb"'banana'" is {\tt b}, not
{\tt a}.  But for computer scientists, the index is an offset from the
beginning of the string, and the offset of the first letter is zero.

\begin{verbatim}
>>> letter = fruit[0]
>>> letter
'b'
\end{verbatim}
%
So {\tt b} is the 0th letter (``zero-eth'') of \verb"'banana'", {\tt
  a} is the 1th letter (``one-eth''), and {\tt n} is the 2th letter
(``two-eth'').  \index{index!starting at zero} \index{zero, index
  starting at}

As an index you can use an expression that contains variables and
operators:
\index{index}

\begin{verbatim}
>>> i = 1
>>> fruit[i]
'a'
>>> fruit[i+1]
'n'
\end{verbatim}
%

But the value of the index has to be an integer.  Otherwise you
get:
\index{exception!TypeError}
\index{TypeError}

\begin{verbatim}
>>> letter = fruit[1.5]
TypeError: string indices must be integers
\end{verbatim}
%

\section{{\tt len}}
\index{len function}
\index{function!len}

{\tt len} is a built-in function that returns the number of characters
in a string:

\begin{verbatim}
>>> fruit = 'banana'
>>> len(fruit)
6
\end{verbatim}
%
To get the last letter of a string, you might be tempted to try something
like this:
\index{exception!IndexError}
\index{IndexError}

\begin{verbatim}
>>> length = len(fruit)
>>> last = fruit[length]
IndexError: string index out of range
\end{verbatim}
%
The reason for the {\tt IndexError} is that there is no letter in {\tt
'banana'} with the index 6.  Since we started counting at zero, the
six letters are numbered 0 to 5.  To get the last character, you have
to subtract 1 from {\tt length}:

\begin{verbatim}
>>> last = fruit[length-1]
>>> last
'a'
\end{verbatim}
%
Or you can use negative indices, which count backward from
the end of the string.  The expression {\tt fruit[-1]} yields the last
letter, {\tt fruit[-2]} yields the second to last, and so on.
\index{index!negative}
\index{negative index}


\section{Traversal with a {\tt for} loop}
\label{for}
\index{traversal}
\index{loop!traversal}
\index{for loop}
\index{loop!for}
\index{statement!for}
\index{traversal}

A lot of computations involve processing a string one character at a
time.  Often they start at the beginning, select each character in
turn, do something to it, and continue until the end.  This pattern of
processing is called a {\bf traversal}.  One way to write a traversal
is with a {\tt while} loop:

\begin{verbatim}
index = 0
while index < len(fruit):
    letter = fruit[index]
    print(letter)
    index = index + 1
\end{verbatim}
%
This loop traverses the string and displays each letter on a line by
itself.  The loop condition is {\tt index < len(fruit)}, so
when {\tt index} is equal to the length of the string, the
condition is false, and the body of the loop doesn't run.  The
last character accessed is the one with the index {\tt len(fruit)-1},
which is the last character in the string.

As an exercise, write a function that takes a string as an argument
and displays the letters backward, one per line.

Another way to write a traversal is with a {\tt for} loop:

\begin{verbatim}
for letter in fruit:
    print(letter)
\end{verbatim}
%
Each time through the loop, the next character in the string is assigned
to the variable {\tt letter}.  The loop continues until no characters are
left.
\index{concatenation}
\index{abecedarian}
\index{McCloskey, Robert}

The following example shows how to use concatenation (string addition)
and a {\tt for} loop to generate an abecedarian series (that is, in
alphabetical order).  In Robert McCloskey's book {\em Make
Way for Ducklings}, the names of the ducklings are Jack, Kack, Lack,
Mack, Nack, Ouack, Pack, and Quack.  This loop outputs these names in
order:

\begin{verbatim}
prefixes = 'JKLMNOPQ'
suffix = 'ack'

for letter in prefixes:
    print(letter + suffix)
\end{verbatim}
%
The output is:

\begin{verbatim}
Jack
Kack
Lack
Mack
Nack
Oack
Pack
Qack
\end{verbatim}
%
Of course, that's not quite right because ``Ouack'' and ``Quack'' are
misspelled.  As an exercise, modify the program to fix this error.



\section{String slices}
\label{slice}
\index{slice operator} \index{operator!slice} \index{index!slice}
\index{string!slice} \index{slice!string}

A segment of a string is called a {\bf slice}.  Selecting a slice is
similar to selecting a character:

\begin{verbatim}
>>> s = 'Monty Python'
>>> s[0:5]
'Monty'
>>> s[6:12]
'Python'
\end{verbatim}
%
The operator {\tt [n:m]} returns the part of the string from the 
``n-eth'' character to the ``m-eth'' character, including the first but
excluding the last.  This behavior is counterintuitive, but it might
help to imagine the indices pointing {\em between} the
characters, as in Figure~\ref{fig.banana}.

\begin{figure}
\centerline
{\includegraphics[scale=0.8]{figs/banana.pdf}}
\caption{Slice indices.}
\label{fig.banana}
\end{figure}

If you omit the first index (before the colon), the slice starts at
the beginning of the string.  If you omit the second index, the slice
goes to the end of the string:

\begin{verbatim}
>>> fruit = 'banana'
>>> fruit[:3]
'ban'
>>> fruit[3:]
'ana'
\end{verbatim}
%
If the first index is greater than or equal to the second the result
is an {\bf empty string}, represented by two quotation marks:
\index{quotation mark}

\begin{verbatim}
>>> fruit = 'banana'
>>> fruit[3:3]
''
\end{verbatim}
%
An empty string contains no characters and has length 0, but other
than that, it is the same as any other string.

Continuing this example, what do you think 
{\tt fruit[:]} means?  Try it and see.
\index{copy!slice}
\index{slice!copy}



\section{Strings are immutable}
\index{mutability}
\index{immutability}
\index{string!immutable}

It is tempting to use the {\tt []} operator on the left side of an
assignment, with the intention of changing a character in a string.
For example:
\index{TypeError}
\index{exception!TypeError}

\begin{verbatim}
>>> greeting = 'Hello, world!'
>>> greeting[0] = 'J'
TypeError: 'str' object does not support item assignment
\end{verbatim}
%
The ``object'' in this case is the string and the ``item'' is
the character you tried to assign.  For now, an object is
the same thing as a value, but we will refine that definition
later (Section~\ref{equivalence}).  
\index{object}
\index{item}
\index{item assignment}
\index{assignment!item}
\index{immutability}

The reason for the error is that
strings are {\bf immutable}, which means you can't change an
existing string.  The best you can do is create a new string
that is a variation on the original:

\begin{verbatim}
>>> greeting = 'Hello, world!'
>>> new_greeting = 'J' + greeting[1:]
>>> new_greeting
'Jello, world!'
\end{verbatim}
%
This example concatenates a new first letter onto
a slice of {\tt greeting}.  It has no effect on
the original string.
\index{concatenation}


\section{Searching}
\label{find}

What does the following function do?
\index{find function}
\index{function!find}

\begin{verbatim}
def find(word, letter):
    index = 0
    while index < len(word):
        if word[index] == letter:
            return index
        index = index + 1
    return -1
\end{verbatim}
%
In a sense, {\tt find} is the inverse of the {\tt []} operator.
Instead of taking an index and extracting the corresponding character,
it takes a character and finds the index where that character
appears.  If the character is not found, the function returns {\tt
-1}.

This is the first example we have seen of a {\tt return} statement
inside a loop.  If {\tt word[index] == letter}, the function breaks
out of the loop and returns immediately.

If the character doesn't appear in the string, the program
exits the loop normally and  returns {\tt -1}.

This pattern of computation---traversing a sequence and returning
when we find what we are looking for---is called a {\bf search}.
\index{traversal}
\index{search pattern}
\index{pattern!search}

As an exercise, modify {\tt find} so that it has a
third parameter, the index in {\tt word} where it should start
looking.


\section{Looping and counting}
\label{counter}
\index{counter}
\index{counting and looping}
\index{looping and counting}
\index{looping!with strings}

The following program counts the number of times the letter {\tt a}
appears in a string:

\begin{verbatim}
word = 'banana'
count = 0
for letter in word:
    if letter == 'a':
        count = count + 1
print(count)
\end{verbatim}
%
This program demonstrates another pattern of computation called a {\bf
counter}.  The variable {\tt count} is initialized to 0 and then
incremented each time an {\tt a} is found.
When the loop exits, {\tt count}
contains the result---the total number of {\tt a}'s.

\index{encapsulation}
As an exercise, encapsulate this code in a function named {\tt
count}, and generalize it so that it accepts the string and the
letter as arguments.

Then rewrite the function so that instead of
traversing the string, it uses the three-parameter version of {\tt
find} from the previous section.


\section{String methods}
\label{optional}

Strings provide methods that perform a variety of useful operations.
A method is similar to a function---it takes arguments and
returns a value---but the syntax is different.  For example, the
method {\tt upper} takes a string and returns a new string with
all uppercase letters.
\index{method}
\index{string!method}

Instead of the function syntax {\tt upper(word)}, it uses
the method syntax {\tt word.upper()}.

\begin{verbatim}
>>> word = 'banana'
>>> new_word = word.upper()
>>> new_word
'BANANA'
\end{verbatim}
%
This form of dot notation specifies the name of the method, {\tt
upper}, and the name of the string to apply the method to, {\tt
word}.  The empty parentheses indicate that this method takes no
arguments.
\index{parentheses!empty}
\index{dot notation}

A method call is called an {\bf invocation}; in this case, we would
say that we are invoking {\tt upper} on {\tt word}.
\index{invocation}

As it turns out, there is a string method named {\tt find} that
is remarkably similar to the function we wrote:

\begin{verbatim}
>>> word = 'banana'
>>> index = word.find('a')
>>> index
1
\end{verbatim}
%
In this example, we invoke {\tt find} on {\tt word} and pass
the letter we are looking for as a parameter.

Actually, the {\tt find} method is more general than our function;
it can find substrings, not just characters:

\begin{verbatim}
>>> word.find('na')
2
\end{verbatim}
%
By default, {\tt find} starts at the beginning of the string, but
it can take a second argument, the index where it should start:
\index{optional argument}
\index{argument!optional}

\begin{verbatim}
>>> word.find('na', 3)
4
\end{verbatim}
%
This is an example of an {\bf optional argument};
{\tt find} can
also take a third argument, the index where it should stop:

\begin{verbatim}
>>> name = 'bob'
>>> name.find('b', 1, 2)
-1
\end{verbatim}
%
This search fails because {\tt b} does not
appear in the index range from {\tt 1} to {\tt 2}, not including {\tt
2}.  Searching up to, but not including, the second index makes
{\tt find} consistent with the slice operator.



\section{The {\tt in} operator}
\label{inboth}
\index{in operator}
\index{operator!in}
\index{boolean operator}
\index{operator!boolean}

The word {\tt in} is a boolean operator that takes two strings and
returns {\tt True} if the first appears as a substring in the second:

\begin{verbatim}
>>> 'a' in 'banana'
True
>>> 'seed' in 'banana'
False
\end{verbatim}
%
For example, the following function prints all the
letters from {\tt word1} that also appear in {\tt word2}:

\begin{verbatim}
def in_both(word1, word2):
    for letter in word1:
        if letter in word2:
            print(letter)
\end{verbatim}
%
With well-chosen variable names,
Python sometimes reads like English.  You could read
this loop, ``for (each) letter in (the first) word, if (the) letter 
(appears) in (the second) word, print (the) letter.''

Here's what you get if you compare apples and oranges:

\begin{verbatim}
>>> in_both('apples', 'oranges')
a
e
s
\end{verbatim}
%

\section{String comparison}
\index{string!comparison}
\index{comparison!string}

The relational operators work on strings.  To see if two strings are equal:

\begin{verbatim}
if word == 'banana':
    print('All right, bananas.')
\end{verbatim}
%
Other relational operations are useful for putting words in alphabetical
order:

\begin{verbatim}
if word < 'banana':
    print('Your word, ' + word + ', comes before banana.')
elif word > 'banana':
    print('Your word, ' + word + ', comes after banana.')
else:
    print('All right, bananas.')
\end{verbatim}
%
Python does not handle uppercase and lowercase letters the same way
people do.  All the uppercase letters come before all the
lowercase letters, so:

\begin{verbatim}
Your word, Pineapple, comes before banana.
\end{verbatim}
%
A common way to address this problem is to convert strings to a
standard format, such as all lowercase, before performing the
comparison.  Keep that in mind in case you have to defend yourself
against a man armed with a Pineapple.


\section{디버깅}
%Debugging
\index{debugging}
\index{traversal}

When you use indices to traverse the values in a sequence,
it is tricky to get the beginning and end of the traversal
right.  Here is a function that is supposed to compare two
words and return {\tt True} if one of the words is the reverse
of the other, but it contains two errors:

\begin{verbatim}
def is_reverse(word1, word2):
    if len(word1) != len(word2):
        return False
    
    i = 0
    j = len(word2)

    while j > 0:
        if word1[i] != word2[j]:
            return False
        i = i+1
        j = j-1

    return True
\end{verbatim}
%
The first {\tt if} statement checks whether the words are the
same length.  If not, we can return {\tt False} immediately.
Otherwise, for the rest of the function, we can assume that the words
are the same length.  This is an example of the guardian pattern
in Section~\ref{guardian}.
\index{guardian pattern}
\index{pattern!guardian}
\index{index}

{\tt i} and {\tt j} are indices: {\tt i} traverses {\tt word1}
forward while {\tt j} traverses {\tt word2} backward.  If we find
two letters that don't match, we can return {\tt False} immediately.
If we get through the whole loop and all the letters match, we
return {\tt True}.

If we test this function with the words ``pots'' and ``stop'', we
expect the return value {\tt True}, but we get an IndexError:
\index{IndexError}
\index{exception!IndexError}

\begin{verbatim}
>>> is_reverse('pots', 'stop')
...
  File "reverse.py", line 15, in is_reverse
    if word1[i] != word2[j]:
IndexError: string index out of range
\end{verbatim}
%
For debugging this kind of error, my first move is to
print the values of the indices immediately before the line
where the error appears.

\begin{verbatim}
    while j > 0:
        print(i, j)        # print here
        
        if word1[i] != word2[j]:
            return False
        i = i+1
        j = j-1
\end{verbatim}
%
Now when I run the program again, I get more information:

\begin{verbatim}
>>> is_reverse('pots', 'stop')
0 4
...
IndexError: string index out of range
\end{verbatim}
%
The first time through the loop, the value of {\tt j} is 4,
which is out of range for the string \verb"'pots'".
The index of the last character is 3, so the
initial value for {\tt j} should be {\tt len(word2)-1}.

If I fix that error and run the program again, I get:

\begin{verbatim}
>>> is_reverse('pots', 'stop')
0 3
1 2
2 1
True
\end{verbatim}
%
This time we get the right answer, but it looks like the loop only ran
three times, which is suspicious.  To get a better idea of what is
happening, it is useful to draw a state diagram.  During the first
iteration, the frame for \verb"is_reverse" is shown in
Figure~\ref{fig.state4}.  \index{state diagram} \index{diagram!state}

\begin{figure}
\centerline
{\includegraphics[scale=0.8]{figs/state4.pdf}}
\caption{State diagram.}
\label{fig.state4}
\end{figure}

I took some license by arranging the variables in the frame
and adding dotted lines to show that the values of {\tt i} and
{\tt j} indicate characters in {\tt word1} and {\tt word2}.

Starting with this diagram, run the program on paper, changing the
values of {\tt i} and {\tt j} during each iteration.  Find and fix the
second error in this function.
\label{isreverse}


\section{용어 해설}
%Glossary

\begin{description}

\item[object:] Something a variable can refer to.  For now,
you can use ``object'' and ``value'' interchangeably.
\index{object}

\item[sequence:] An ordered collection of
values where each value is identified by an integer index.
\index{sequence}

\item[item:] One of the values in a sequence.
\index{item}

\item[index:] An integer value used to select an item in
a sequence, such as a character in a string.  In Python
indices start from 0.
\index{index}

\item[slice:] A part of a string specified by a range of indices.
\index{slice}

\item[empty string:] A string with no characters and length 0, represented
by two quotation marks.
\index{empty string}

\item[immutable:] The property of a sequence whose items cannot
be changed.
\index{immutability}

\item[traverse:] To iterate through the items in a sequence,
performing a similar operation on each.
\index{traversal}

\item[search:] A pattern of traversal that stops
when it finds what it is looking for.
\index{search pattern}
\index{pattern!search}

\item[counter:] A variable used to count something, usually initialized
to zero and then incremented.
\index{counter}

\item[invocation:] A statement that calls a method.
\index{invocation}

\item[optional argument:] A function or method argument that is not
required.
\index{optional argument}
\index{argument!optional}

\end{description}


\section{연습 문제}
%Exercises

\begin{exercise}
\index{string method}
\index{method!string}

Read the documentation of the string methods at
\url{http://docs.python.org/3/library/stdtypes.html#string-methods}.
You might want to experiment with some of them to make sure you
understand how they work.  {\tt strip} and {\tt replace} are
particularly useful.

The documentation uses a syntax that might be confusing.
For example, in \verb"find(sub[, start[, end]])", the brackets
indicate optional arguments.  So {\tt sub} is required, but
{\tt start} is optional, and if you include {\tt start},
then {\tt end} is optional.
\index{optional argument}
\index{argument!optional}

\end{exercise}


\begin{exercise}
\index{count method}
\index{method!count}

There is a string method called {\tt count} that is similar
to the function in Section~\ref{counter}.  Read the documentation
of this method
and write an invocation that counts the number of {\tt a}'s
in \verb"'banana'".
\end{exercise}


\begin{exercise}
\index{step size}
\index{slice operator}
\index{operator!slice}

A string slice can take a third index that specifies the ``step
size''; that is, the number of spaces between successive characters.
A step size of 2 means every other character; 3 means every third,
etc.

\begin{verbatim}
>>> fruit = 'banana'
>>> fruit[0:5:2]
'bnn'
\end{verbatim}

A step size of -1 goes through the word backwards, so
the slice \verb"[::-1]" generates a reversed string.
\index{palindrome}

Use this idiom to write a one-line version of \verb"is_palindrome"
from Exercise~\ref{palindrome}.
\end{exercise}


\begin{exercise}

The following functions are all {\em intended} to check whether a
string contains any lowercase letters, but at least some of them are
wrong.  For each function, describe what the function actually does
(assuming that the parameter is a string).

\begin{verbatim}
def any_lowercase1(s):
    for c in s:
        if c.islower():
            return True
        else:
            return False

def any_lowercase2(s):
    for c in s:
        if 'c'.islower():
            return 'True'
        else:
            return 'False'

def any_lowercase3(s):
    for c in s:
        flag = c.islower()
    return flag

def any_lowercase4(s):
    flag = False
    for c in s:
        flag = flag or c.islower()
    return flag

def any_lowercase5(s):
    for c in s:
        if not c.islower():
            return False
    return True
\end{verbatim}

\end{exercise}


\begin{exercise}
\index{letter rotation}
\index{rotation, letter}

\label{exrotate}
A Caesar cypher is a weak form of encryption that involves ``rotating'' each
letter by a fixed number of places.  To rotate a letter means
to shift it through the alphabet, wrapping around to the beginning if
necessary, so 'A' rotated by 3 is 'D' and 'Z' rotated by 1 is 'A'.

To rotate a word, rotate each letter by the same amount.
For example, ``cheer'' rotated by 7 is ``jolly'' and ``melon'' rotated
by -10 is ``cubed''.  In the movie {\em 2001: A Space Odyssey}, the 
ship computer is called HAL, which is IBM rotated by -1.

%For example ``sleep''
%rotated by 9 is ``bunny'' and ``latex'' rotated by 7 is ``shale''.

Write a function called \verb"rotate_word"
that takes a string and an integer as parameters, and returns
a new string that contains the letters from the original string
rotated by the given amount.  

You might want to use the built-in function {\tt ord}, which converts
a character to a numeric code, and {\tt chr}, which converts numeric
codes to characters.  Letters of the alphabet are encoded in alphabetical
order, so for example:

\begin{verbatim}
>>> ord('c') - ord('a')
2
\end{verbatim}

Because \verb"'c'" is the two-eth letter of the alphabet.  But
beware: the numeric codes for upper case letters are different.

Potentially offensive jokes on the Internet are sometimes encoded in
ROT13, which is a Caesar cypher with rotation 13.  If you are not
easily offended, find and decode some of them.  Solution:
\url{http://thinkpython2.com/code/rotate.py}.

\end{exercise}


\chapter{Case study: word play}
\label{wordplay}

This chapter presents the second case study, which involves
solving word puzzles by searching for words that have certain
properties.  For example, we'll find the longest palindromes
in English and search for words whose letters appear in
alphabetical order.  And I will present another program development
plan: reduction to a previously solved problem.


\section{Reading word lists}
\label{wordlist}

For the exercises in this chapter we need a list of English words.
There are lots of word lists available on the Web, but the one most
suitable for our purpose is one of the word lists collected and
contributed to the public domain by Grady Ward as part of the Moby
lexicon project (see \url{http://wikipedia.org/wiki/Moby_Project}).  It
is a list of 113,809 official crosswords; that is, words that are
considered valid in crossword puzzles and other word games.  In the
Moby collection, the filename is {\tt 113809of.fic}; you can download
a copy, with the simpler name {\tt words.txt}, from
\url{http://thinkpython2.com/code/words.txt}.
\index{Moby Project}
\index{crosswords}

This file is in plain text, so you can open it with a text
editor, but you can also read it from Python.  The built-in
function {\tt open} takes the name of the file as a parameter
and returns a {\bf file object} you can use to read the file.
\index{open function}
\index{function!open}
\index{plain text}
\index{text!plain}
\index{object!file}
\index{file object}

\begin{verbatim}
>>> fin = open('words.txt')
\end{verbatim}
%
{\tt fin} is a common name for a file object used for input.  The file
object provides several methods for reading, including {\tt readline},
which reads characters from the file until it gets to a newline and
returns the result as a string: \index{readline method}
\index{method!readline}

\begin{verbatim}
>>> fin.readline()
'aa\r\n'
\end{verbatim}
%
The first word in this particular list is ``aa'', which is a kind of
lava.  The sequence \verb"\r\n" represents two whitespace characters,
a carriage return and a newline, that separate this word from the
next.

The file object keeps track of where it is in the file, so
if you call {\tt readline} again, you get the next word:

\begin{verbatim}
>>> fin.readline()
'aah\r\n'
\end{verbatim}
%
The next word is ``aah'', which is a perfectly legitimate
word, so stop looking at me like that.
Or, if it's the whitespace that's bothering you,
we can get rid of it with the string method {\tt strip}:
\index{strip method}
\index{method!strip}

\begin{verbatim}
>>> line = fin.readline()
>>> word = line.strip()
>>> word
'aahed'
\end{verbatim}
%
You can also use a file object as part of a {\tt for} loop.
This program reads {\tt words.txt} and prints each word, one
per line:
\index{open function}
\index{function!open}

\begin{verbatim}
fin = open('words.txt')
for line in fin:
    word = line.strip()
    print(word)
\end{verbatim}
%

\section{연습 문제}
%Exercises

There are solutions to these exercises in the next section.
You should at least attempt each one before you read the solutions.

\begin{exercise}
Write a program that reads {\tt words.txt} and prints only the
words with more than 20 characters (not counting whitespace).
\index{whitespace}

\end{exercise}

\begin{exercise}

In 1939 Ernest Vincent Wright published a 50,000 word novel called
{\em Gadsby} that does not contain the letter ``e''.  Since ``e'' is
the most common letter in English, that's not easy to do.

In fact, it is difficult to construct a solitary thought without using
that most common symbol.  It is slow going at first, but with caution
and hours of training you can gradually gain facility.

All right, I'll stop now.

Write a function called \verb"has_no_e" that returns {\tt True} if
the given word doesn't have the letter ``e'' in it.

Modify your program from the previous section to print only the words
that have no ``e'' and compute the percentage of the words in the list
that have no ``e''.
\index{lipogram}

\end{exercise}


\begin{exercise} 

Write a function named {\tt avoids}
that takes a word and a string of forbidden letters, and
that returns {\tt True} if the word doesn't use any of the forbidden
letters.

Modify your program to prompt the user to enter a string
of forbidden letters and then print the number of words that
don't contain any of them.
Can you find a combination of 5 forbidden letters that
excludes the smallest number of words?

\end{exercise}



\begin{exercise}

Write a function named \verb"uses_only" that takes a word and a
string of letters, and that returns {\tt True} if the word contains
only letters in the list.  Can you make a sentence using only the
letters {\tt acefhlo}?  Other than ``Hoe alfalfa?''

\end{exercise}


\begin{exercise} 

Write a function named \verb"uses_all" that takes a word and a
string of required letters, and that returns {\tt True} if the word
uses all the required letters at least once.  How many words are there
that use all the vowels {\tt aeiou}?  How about {\tt aeiouy}?

\end{exercise}


\begin{exercise}

Write a function called \verb"is_abecedarian" that returns
{\tt True} if the letters in a word appear in alphabetical order
(double letters are ok).  
How many abecedarian words are there?

\index{abecedarian}

\end{exercise}



\section{Search}
\label{search}
\index{search pattern}
\index{pattern!search}

All of the exercises in the previous section have something
in common; they can be solved with the search pattern we saw
in Section~\ref{find}.  The simplest example is:

\begin{verbatim}
def has_no_e(word):
    for letter in word:
        if letter == 'e':
            return False
    return True
\end{verbatim}
%
The {\tt for} loop traverses the characters in {\tt word}.  If we find
the letter ``e'', we can immediately return {\tt False}; otherwise we
have to go to the next letter.  If we exit the loop normally, that
means we didn't find an ``e'', so we return {\tt True}.
\index{traversal}

\index{in operator}
\index{operator!in}
You could write this function more concisely using the {\tt in}
operator, but I started with this version because it 
demonstrates the logic of the search pattern.

\index{generalization}
{\tt avoids} is a more general version of \verb"has_no_e" but it
has the same structure:

\begin{verbatim}
def avoids(word, forbidden):
    for letter in word:
        if letter in forbidden:
            return False
    return True
\end{verbatim}
%
We can return {\tt False} as soon as we find a forbidden letter;
if we get to the end of the loop, we return {\tt True}.

\verb"uses_only" is similar except that the sense of the condition
is reversed:

\begin{verbatim}
def uses_only(word, available):
    for letter in word: 
        if letter not in available:
            return False
    return True
\end{verbatim}
%
Instead of a list of forbidden letters, we have a list of available
letters.  If we find a letter in {\tt word} that is not in
{\tt available}, we can return {\tt False}.

\verb"uses_all" is similar except that we reverse the role
of the word and the string of letters:

\begin{verbatim}
def uses_all(word, required):
    for letter in required: 
        if letter not in word:
            return False
    return True
\end{verbatim}
%
Instead of traversing the letters in {\tt word}, the loop
traverses the required letters.  If any of the required letters
do not appear in the word, we can return {\tt False}.
\index{traversal}

If you were really thinking like a computer scientist, you would
have recognized that \verb"uses_all" was an instance of a
previously solved problem, and you would have written:

\begin{verbatim}
def uses_all(word, required):
    return uses_only(required, word)
\end{verbatim}
%
This is an example of a program development plan called {\bf
  reduction to a previously solved problem}, which means that you
recognize the problem you are working on as an instance of a solved
problem and apply an existing solution.  \index{reduction to a
  previously solved problem} \index{development plan!reduction}


\section{Looping with indices}
\index{looping!with indices}
\index{index!looping with}

I wrote the functions in the previous section with {\tt for}
loops because I only needed the characters in the strings; I didn't
have to do anything with the indices.

For \verb"is_abecedarian" we have to compare adjacent letters,
which is a little tricky with a {\tt for} loop:

\begin{verbatim}
def is_abecedarian(word):
    previous = word[0]
    for c in word:
        if c < previous:
            return False
        previous = c
    return True
\end{verbatim}

An alternative is to use recursion:

\begin{verbatim}
def is_abecedarian(word):
    if len(word) <= 1:
        return True
    if word[0] > word[1]:
        return False
    return is_abecedarian(word[1:])
\end{verbatim}

Another option is to use a {\tt while} loop:

\begin{verbatim}
def is_abecedarian(word):
    i = 0
    while i < len(word)-1:
        if word[i+1] < word[i]:
            return False
        i = i+1
    return True
\end{verbatim}
%
The loop starts at {\tt i=0} and ends when {\tt i=len(word)-1}.  Each
time through the loop, it compares the $i$th character (which you can
think of as the current character) to the $i+1$th character (which you
can think of as the next).

If the next character is less than (alphabetically before) the current
one, then we have discovered a break in the abecedarian trend, and
we return {\tt False}.

If we get to the end of the loop without finding a fault, then the
word passes the test.  To convince yourself that the loop ends
correctly, consider an example like \verb"'flossy'".  The
length of the word is 6, so
the last time the loop runs is when {\tt i} is 4, which is the
index of the second-to-last character.  On the last iteration,
it compares the second-to-last character to the last, which is
what we want.
\index{palindrome}

Here is a version of \verb"is_palindrome" (see
Exercise~\ref{palindrome}) that uses two indices; one starts at the
beginning and goes up; the other starts at the end and goes down.

\begin{verbatim}
def is_palindrome(word):
    i = 0
    j = len(word)-1

    while i<j:
        if word[i] != word[j]:
            return False
        i = i+1
        j = j-1

    return True
\end{verbatim}

Or we could reduce to a previously solved
problem and write:
\index{reduction to a previously solved problem}
\index{development plan!reduction}

\begin{verbatim}
def is_palindrome(word):
    return is_reverse(word, word)
\end{verbatim}
%
Using \verb"is_reverse" from Section~\ref{isreverse}.


\section{디버깅}
%Debugging
\index{debugging}
\index{testing!is hard}
\index{program testing}

Testing programs is hard.  The functions in this chapter are
relatively easy to test because you can check the results by hand.
Even so, it is somewhere between difficult and impossible to choose a
set of words that test for all possible errors.

Taking \verb"has_no_e" as an example, there are two obvious
cases to check: words that have an `e' should return {\tt False}, and
words that don't should return {\tt True}.  You should have no
trouble coming up with one of each.

Within each case, there are some less obvious subcases.  Among the
words that have an ``e'', you should test words with an ``e'' at the
beginning, the end, and somewhere in the middle.  You should test long
words, short words, and very short words, like the empty string.  The
empty string is an example of a {\bf special case}, which is one of
the non-obvious cases where errors often lurk.
\index{special case}

In addition to the test cases you generate, you can also test
your program with a word list like {\tt words.txt}.  By scanning
the output, you might be able to catch errors, but be careful:
you might catch one kind of error (words that should not be
included, but are) and not another (words that should be included,
but aren't).

In general, testing can help you find bugs, but it is not easy to
generate a good set of test cases, and even if you do, you can't
be sure your program is correct.
According to a legendary computer scientist:
\index{testing!and absence of bugs}

\begin{quote}
Program testing can be used to show the presence of bugs, but never to
show their absence!

--- Edsger W. Dijkstra
\end{quote}
\index{Dijkstra, Edsger}


\section{용어 해설}
%Glossary

\begin{description}

\item[file object:] A value that represents an open file.
\index{file object}
\index{object!file}

\item[reduction to a previously solved problem:] A way of solving a
  problem by expressing it as an instance of a previously solved
  problem.  \index{reduction to a previously solved problem}
  \index{development plan!reduction}

\item[special case:] A test case that is atypical or non-obvious
(and less likely to be handled correctly).
\index{special case}

\end{description}


\section{연습 문제}
%Exercises

\begin{exercise}
\index{Car Talk}
\index{Puzzler}
\index{double letters}

This question is based on a Puzzler that was broadcast on the radio
program {\em Car Talk} 
(\url{http://www.cartalk.com/content/puzzlers}):

\begin{quote}
Give me a word with three consecutive double letters. I'll give you a
couple of words that almost qualify, but don't. For example, the word
committee, c-o-m-m-i-t-t-e-e. It would be great except for the `i' that
sneaks in there. Or Mississippi: M-i-s-s-i-s-s-i-p-p-i. If you could
take out those i's it would work. But there is a word that has three
consecutive pairs of letters and to the best of my knowledge this may
be the only word. Of course there are probably 500 more but I can only
think of one. What is the word?
\end{quote}

Write a program to find it.
Solution: \url{http://thinkpython2.com/code/cartalk1.py}.

\end{exercise}


\begin{exercise}
Here's another {\em Car Talk}
Puzzler (\url{http://www.cartalk.com/content/puzzlers}):
\index{Car Talk}
\index{Puzzler}
\index{odometer}
\index{palindrome}

\begin{quote}
``I was driving on the highway the other day and I happened to
notice my odometer. Like most odometers, it shows six digits,
in whole miles only. So, if my car had 300,000
miles, for example, I'd see 3-0-0-0-0-0.

``Now, what I saw that day was very interesting. I noticed that the
last 4 digits were palindromic; that is, they read the same forward as
backward. For example, 5-4-4-5 is a palindrome, so my odometer
could have read 3-1-5-4-4-5.

``One mile later, the last 5 numbers were palindromic. For example, it
could have read 3-6-5-4-5-6.  One mile after that, the middle 4 out of
6 numbers were palindromic.  And you ready for this? One mile later,
all 6 were palindromic!

``The question is, what was on the odometer when I first looked?''
\end{quote}

Write a Python program that tests all the six-digit numbers and prints
any numbers that satisfy these requirements.  
Solution: \url{http://thinkpython2.com/code/cartalk2.py}.

\end{exercise}


\begin{exercise}
Here's another {\em Car Talk} Puzzler you can solve with a
search (\url{http://www.cartalk.com/content/puzzlers}):
\index{Car Talk}
\index{Puzzler}
\index{palindrome}

\begin{quote}
``Recently I had a visit with my mom and we realized that
the two digits that make up my age when reversed resulted in her
age. For example, if she's 73, I'm 37. We wondered how often this has
happened over the years but we got sidetracked with other topics and
we never came up with an answer.

``When I got home I figured out that the digits of our ages have been
reversible six times so far. I also figured out that if we're lucky it
would happen again in a few years, and if we're really lucky it would
happen one more time after that. In other words, it would have
happened 8 times over all. So the question is, how old am I now?''

\end{quote}

Write a Python program that searches for solutions to this Puzzler.
Hint: you might find the string method {\tt zfill} useful.

Solution: \url{http://thinkpython2.com/code/cartalk3.py}.

\end{exercise}



\chapter{Lists}

This chapter presents one of Python's most useful built-in types, lists.
You will also learn more about objects and what can happen when you have
more than one name for the same object.


\section{A list is a sequence}
\label{sequence}

Like a string, a {\bf list} is a sequence of values.  In a string, the
values are characters; in a list, they can be any type.  The values in
a list are called {\bf elements} or sometimes {\bf items}.
\index{list}
\index{type!list}
\index{element}
\index{sequence}
\index{item}

There are several ways to create a new list; the simplest is to
enclose the elements in square brackets (\verb"[" and \verb"]"):

\begin{verbatim}
[10, 20, 30, 40]
['crunchy frog', 'ram bladder', 'lark vomit']
\end{verbatim}
%
The first example is a list of four integers.  The second is a list of
three strings.  The elements of a list don't have to be the same type.
The following list contains a string, a float, an integer, and
(lo!) another list:

\begin{verbatim}
['spam', 2.0, 5, [10, 20]]
\end{verbatim}
%
A list within another list is {\bf nested}.
\index{nested list}
\index{list!nested}

A list that contains no elements is
called an empty list; you can create one with empty
brackets, \verb"[]".
\index{empty list}
\index{list!empty}

As you might expect, you can assign list values to variables:

\begin{verbatim}
>>> cheeses = ['Cheddar', 'Edam', 'Gouda']
>>> numbers = [42, 123]
>>> empty = []
>>> print(cheeses, numbers, empty)
['Cheddar', 'Edam', 'Gouda'] [42, 123] []
\end{verbatim}
%
\index{assignment}


\section{Lists are mutable}
\label{mutable}
\index{list!element}
\index{access}
\index{index}
\index{bracket operator}
\index{operator!bracket}

The syntax for accessing the elements of a list is the same as for
accessing the characters of a string---the bracket operator.  The
expression inside the brackets specifies the index.  Remember that the
indices start at 0:

\begin{verbatim}
>>> cheeses[0]
'Cheddar'
\end{verbatim}
%
Unlike strings, lists are mutable.  When the bracket operator appears
on the left side of an assignment, it identifies the element of the
list that will be assigned.
\index{mutability}

\begin{verbatim}
>>> numbers = [42, 123]
>>> numbers[1] = 5
>>> numbers
[42, 5]
\end{verbatim}
%
The one-eth element of {\tt numbers}, which
used to be 123, is now 5.
\index{index!starting at zero}
\index{zero, index starting at}

Figure~\ref{fig.liststate} shows 
the state diagram for {\tt
cheeses}, {\tt numbers} and {\tt empty}:
\index{state diagram}
\index{diagram!state}

\begin{figure}
\centerline
{\includegraphics[scale=0.8]{figs/liststate.pdf}}
\caption{State diagram.}
\label{fig.liststate}
\end{figure}

Lists are represented by boxes with the word ``list'' outside
and the elements of the list inside.  {\tt cheeses} refers to
a list with three elements indexed 0, 1 and 2.
{\tt numbers} contains two elements; the diagram shows that the
value of the second element has been reassigned from 123 to 5.
{\tt empty} refers to a list with no elements.
\index{item assignment}
\index{assignment!item}
\index{reassignment}

List indices work the same way as string indices:

\begin{itemize}

\item Any integer expression can be used as an index.

\item If you try to read or write an element that does not exist, you
get an {\tt IndexError}.
\index{exception!IndexError}
\index{IndexError}

\item If an index has a negative value, it counts backward from the
end of the list.

\end{itemize}
\index{list!index}

\index{list!membership}
\index{membership!list}
\index{in operator}
\index{operator!in}

The {\tt in} operator also works on lists.

\begin{verbatim}
>>> cheeses = ['Cheddar', 'Edam', 'Gouda']
>>> 'Edam' in cheeses
True
>>> 'Brie' in cheeses
False
\end{verbatim}


\section{Traversing a list}
\index{list!traversal}
\index{traversal!list}
\index{for loop}
\index{loop!for}
\index{statement!for}

The most common way to traverse the elements of a list is
with a {\tt for} loop.  The syntax is the same as for strings:

\begin{verbatim}
for cheese in cheeses:
    print(cheese)
\end{verbatim}
%
This works well if you only need to read the elements of the
list.  But if you want to write or update the elements, you
need the indices.  A common way to do that is to combine
the built-in functions {\tt range} and {\tt len}:
\index{looping!with indices}
\index{index!looping with}

\begin{verbatim}
for i in range(len(numbers)):
    numbers[i] = numbers[i] * 2
\end{verbatim}
%
This loop traverses the list and updates each element.  {\tt len}
returns the number of elements in the list.  {\tt range} returns
a list of indices from 0 to $n-1$, where $n$ is the length of
the list.  Each time through the loop {\tt i} gets the index
of the next element.  The assignment statement in the body uses
{\tt i} to read the old value of the element and to assign the
new value.
\index{item update}
\index{update!item}

A {\tt for} loop over an empty list never runs the body:

\begin{verbatim}
for x in []:
    print('This never happens.')
\end{verbatim}
%
Although a list can contain another list, the nested
list still counts as a single element.  The length of this list is
four:
\index{nested list}
\index{list!nested}

\begin{verbatim}
['spam', 1, ['Brie', 'Roquefort', 'Pol le Veq'], [1, 2, 3]]
\end{verbatim}



\section{List operations}
\index{list!operation}

The {\tt +} operator concatenates lists:
\index{concatenation!list}
\index{list!concatenation}

\begin{verbatim}
>>> a = [1, 2, 3]
>>> b = [4, 5, 6]
>>> c = a + b
>>> c
[1, 2, 3, 4, 5, 6]
\end{verbatim}
%
The {\tt *} operator repeats a list a given number of times:
\index{repetition!list}
\index{list!repetition}

\begin{verbatim}
>>> [0] * 4
[0, 0, 0, 0]
>>> [1, 2, 3] * 3
[1, 2, 3, 1, 2, 3, 1, 2, 3]
\end{verbatim}
%
The first example repeats {\tt [0]} four times.  The second example
repeats the list {\tt [1, 2, 3]} three times.


\section{List slices}
\index{slice operator}
\index{operator!slice}
\index{index!slice}
\index{list!slice}
\index{slice!list}

The slice operator also works on lists:

\begin{verbatim}
>>> t = ['a', 'b', 'c', 'd', 'e', 'f']
>>> t[1:3]
['b', 'c']
>>> t[:4]
['a', 'b', 'c', 'd']
>>> t[3:]
['d', 'e', 'f']
\end{verbatim}
%
If you omit the first index, the slice starts at the beginning.
If you omit the second, the slice goes to the end.  So if you
omit both, the slice is a copy of the whole list.
\index{list!copy}
\index{slice!copy}
\index{copy!slice}

\begin{verbatim}
>>> t[:]
['a', 'b', 'c', 'd', 'e', 'f']
\end{verbatim}
%
Since lists are mutable, it is often useful to make a copy
before performing operations that modify lists.
\index{mutability}

A slice operator on the left side of an assignment
can update multiple elements:
\index{slice!update}
\index{update!slice}

\begin{verbatim}
>>> t = ['a', 'b', 'c', 'd', 'e', 'f']
>>> t[1:3] = ['x', 'y']
>>> t
['a', 'x', 'y', 'd', 'e', 'f']
\end{verbatim}
%

% You can add elements to a list by squeezing them into an empty
% slice:

% % \begin{verbatim}
% >>> t = ['a', 'd', 'e', 'f']
% >>> t[1:1] = ['b', 'c']
% >>> print t
% ['a', 'b', 'c', 'd', 'e', 'f']
% \end{verbatim}
% \afterverb
%
% And you can remove elements from a list by assigning the empty list to
% them:

% % \begin{verbatim}
% >>> t = ['a', 'b', 'c', 'd', 'e', 'f']
% >>> t[1:3] = []
% >>> print t
% ['a', 'd', 'e', 'f']
% \end{verbatim}
% \afterverb
%
% But both of those operations can be expressed more clearly
% with list methods.


\section{List methods}
\index{list!method}
\index{method, list}

Python provides methods that operate on lists.  For example,
{\tt append} adds a new element to the end of a list:
\index{append method}
\index{method!append}

\begin{verbatim}
>>> t = ['a', 'b', 'c']
>>> t.append('d')
>>> t
['a', 'b', 'c', 'd']
\end{verbatim}
%
{\tt extend} takes a list as an argument and appends all of
the elements:
\index{extend method}
\index{method!extend}

\begin{verbatim}
>>> t1 = ['a', 'b', 'c']
>>> t2 = ['d', 'e']
>>> t1.extend(t2)
>>> t1
['a', 'b', 'c', 'd', 'e']
\end{verbatim}
%
This example leaves {\tt t2} unmodified.

{\tt sort} arranges the elements of the list from low to high:
\index{sort method}
\index{method!sort}

\begin{verbatim}
>>> t = ['d', 'c', 'e', 'b', 'a']
>>> t.sort()
>>> t
['a', 'b', 'c', 'd', 'e']
\end{verbatim}
%
Most list methods are void; they modify the list and return {\tt None}.
If you accidentally write {\tt t = t.sort()}, you will be disappointed
with the result.
\index{void method}
\index{method!void}
\index{None special value}
\index{special value!None}


\section{Map, filter and reduce}
\label{filter}

To add up all the numbers in a list, you can use a loop like this:

% see add.py

\begin{verbatim}
def add_all(t):
    total = 0
    for x in t:
        total += x
    return total
\end{verbatim}
%
{\tt total} is initialized to 0.  Each time through the loop,
{\tt x} gets one element from the list.  The {\tt +=} operator
provides a short way to update a variable.  This 
{\bf augmented assignment statement},
\index{update operator}
\index{operator!update}
\index{assignment!augmented}
\index{augmented assignment}

\begin{verbatim}
    total += x
\end{verbatim}
%
is equivalent to

\begin{verbatim}
    total = total + x
\end{verbatim}
%
As the loop runs, {\tt total} accumulates the sum of the
elements; a variable used this way is sometimes called an
{\bf accumulator}.
\index{accumulator!sum}

Adding up the elements of a list is such a common operation
that Python provides it as a built-in function, {\tt sum}:

\begin{verbatim}
>>> t = [1, 2, 3]
>>> sum(t)
6
\end{verbatim}
%
An operation like this that combines a sequence of elements into
a single value is sometimes called {\bf reduce}.
\index{reduce pattern}
\index{pattern!reduce}
\index{traversal}

Sometimes you want to traverse one list while building
another.  For example, the following function takes a list of strings
and returns a new list that contains capitalized strings:

\begin{verbatim}
def capitalize_all(t):
    res = []
    for s in t:
        res.append(s.capitalize())
    return res
\end{verbatim}
%
{\tt res} is initialized with an empty list; each time through
the loop, we append the next element.  So {\tt res} is another
kind of accumulator.
\index{accumulator!list}

An operation like \verb"capitalize_all" is sometimes called a {\bf
map} because it ``maps'' a function (in this case the method {\tt
capitalize}) onto each of the elements in a sequence.
\index{map pattern}
\index{pattern!map}
\index{filter pattern}
\index{pattern!filter}

Another common operation is to select some of the elements from
a list and return a sublist.  For example, the following
function takes a list of strings and returns a list that contains
only the uppercase strings:

\begin{verbatim}
def only_upper(t):
    res = []
    for s in t:
        if s.isupper():
            res.append(s)
    return res
\end{verbatim}
%
{\tt isupper} is a string method that returns {\tt True} if
the string contains only upper case letters.

An operation like \verb"only_upper" is called a {\bf filter} because
it selects some of the elements and filters out the others.

Most common list operations can be expressed as a combination
of map, filter and reduce.


\section{Deleting elements}
\index{element deletion}
\index{deletion, element of list}

There are several ways to delete elements from a list.  If you
know the index of the element you want, you can use
{\tt pop}:
\index{pop method}
\index{method!pop}

\begin{verbatim}
>>> t = ['a', 'b', 'c']
>>> x = t.pop(1)
>>> t
['a', 'c']
>>> x
'b'
\end{verbatim}
%
{\tt pop} modifies the list and returns the element that was removed.
If you don't provide an index, it deletes and returns the
last element.

If you don't need the removed value, you can use the {\tt del}
operator:
\index{del operator}
\index{operator!del}

\begin{verbatim}
>>> t = ['a', 'b', 'c']
>>> del t[1]
>>> t
['a', 'c']
\end{verbatim}
%
If you know the element you want to remove (but not the index), you
can use {\tt remove}:
\index{remove method}
\index{method!remove}

\begin{verbatim}
>>> t = ['a', 'b', 'c']
>>> t.remove('b')
>>> t
['a', 'c']
\end{verbatim}
%
The return value from {\tt remove} is {\tt None}.
\index{None special value}
\index{special value!None}

To remove more than one element, you can use {\tt del} with
a slice index:

\begin{verbatim}
>>> t = ['a', 'b', 'c', 'd', 'e', 'f']
>>> del t[1:5]
>>> t
['a', 'f']
\end{verbatim}
%
As usual, the slice selects all the elements up to but not
including the second index.



\section{Lists and strings}
\index{list}
\index{string}
\index{sequence}

A string is a sequence of characters and a list is a sequence
of values, but a list of characters is not the same as a
string.  To convert from a string to a list of characters,
you can use {\tt list}:
\index{list!function}
\index{function!list}

\begin{verbatim}
>>> s = 'spam'
>>> t = list(s)
>>> t
['s', 'p', 'a', 'm']
\end{verbatim}
%
Because {\tt list} is the name of a built-in function, you should
avoid using it as a variable name.  I also avoid {\tt l} because
it looks too much like {\tt 1}.  So that's why I use {\tt t}.

The {\tt list} function breaks a string into individual letters.  If
you want to break a string into words, you can use the {\tt split}
method:
\index{split method}
\index{method!split}

\begin{verbatim}
>>> s = 'pining for the fjords'
>>> t = s.split()
>>> t
['pining', 'for', 'the', 'fjords']
\end{verbatim}
%
An optional argument called a {\bf delimiter} specifies which
characters to use as word boundaries.
The following example
uses a hyphen as a delimiter:
\index{optional argument}
\index{argument!optional}
\index{delimiter}

\begin{verbatim}
>>> s = 'spam-spam-spam'
>>> delimiter = '-'
>>> t = s.split(delimiter)
>>> t
['spam', 'spam', 'spam']
\end{verbatim}
%
{\tt join} is the inverse of {\tt split}.  It
takes a list of strings and
concatenates the elements.  {\tt join} is a string method,
so you have to invoke it on the delimiter and pass the
list as a parameter:
\index{join method}
\index{method!join}
\index{concatenation}

\begin{verbatim}
>>> t = ['pining', 'for', 'the', 'fjords']
>>> delimiter = ' '
>>> s = delimiter.join(t)
>>> s
'pining for the fjords'
\end{verbatim}
%
In this case the delimiter is a space character, so
{\tt join} puts a space between words.  To concatenate
strings without spaces, you can use the empty string,
\verb"''", as a delimiter. 
\index{empty string}
\index{string!empty}


\section{Objects and values}
\label{equivalence}
\index{object}
\index{value}

If we run these assignment statements:

\begin{verbatim}
a = 'banana'
b = 'banana'
\end{verbatim}
%
We know that {\tt a} and {\tt b} both refer to a
string, but we don't
know whether they refer to the {\em same} string.
There are two possible states, shown in Figure~\ref{fig.list1}.
\index{aliasing}

\begin{figure}
\centerline
{\includegraphics[scale=0.8]{figs/list1.pdf}}
\caption{State diagram.}
\label{fig.list1}
\end{figure}

In one case, {\tt a} and {\tt b} refer to two different objects that
have the same value.  In the second case, they refer to the same
object.
\index{is operator}
\index{operator!is}

To check whether two variables refer to the same object, you can
use the {\tt is} operator.

\begin{verbatim}
>>> a = 'banana'
>>> b = 'banana'
>>> a is b
True
\end{verbatim}
%
In this example, Python only created one string object, and both {\tt
  a} and {\tt b} refer to it.  But when you create two lists, you get
two objects:

\begin{verbatim}
>>> a = [1, 2, 3]
>>> b = [1, 2, 3]
>>> a is b
False
\end{verbatim}
%
So the state diagram looks like Figure~\ref{fig.list2}.
\index{state diagram}
\index{diagram!state}

\begin{figure}
\centerline
{\includegraphics[scale=0.8]{figs/list2.pdf}}
\caption{State diagram.}
\label{fig.list2}
\end{figure}

In this case we would say that the two lists are {\bf equivalent},
because they have the same elements, but not {\bf identical}, because
they are not the same object.  If two objects are identical, they are
also equivalent, but if they are equivalent, they are not necessarily
identical.
\index{equivalence}
\index{identity}

Until now, we have been using ``object'' and ``value''
interchangeably, but it is more precise to say that an object has a
value.  If you evaluate {\tt [1, 2, 3]}, you get a list
object whose value is a sequence of integers.  If another
list has the same elements, we say it has the same value, but
it is not the same object.
\index{object}
\index{value}


\section{Aliasing}
\index{aliasing}
\index{reference!aliasing}

If {\tt a} refers to an object and you assign {\tt b = a},
then both variables refer to the same object:

\begin{verbatim}
>>> a = [1, 2, 3]
>>> b = a
>>> b is a
True
\end{verbatim}
%
The state diagram looks like Figure~\ref{fig.list3}.
\index{state diagram}
\index{diagram!state}

\begin{figure}
\centerline
{\includegraphics[scale=0.8]{figs/list3.pdf}}
\caption{State diagram.}
\label{fig.list3}
\end{figure}

The association of a variable with an object is called a {\bf
reference}.  In this example, there are two references to the same
object.
\index{reference}

An object with more than one reference has more
than one name, so we say that the object is {\bf aliased}.
\index{mutability}

If the aliased object is mutable, changes made with one alias affect
the other:

\begin{verbatim}
>>> b[0] = 42
>>> a
[42, 2, 3]
\end{verbatim}
%
Although this behavior can be useful, it is error-prone.  In general,
it is safer to avoid aliasing when you are working with mutable
objects.
\index{immutability}

For immutable objects like strings, aliasing is not as much of a
problem.  In this example:

\begin{verbatim}
a = 'banana'
b = 'banana'
\end{verbatim}
%
It almost never makes a difference whether {\tt a} and {\tt b} refer
to the same string or not.


\section{List arguments}
\label{list.arguments}
\index{list!as argument}
\index{argument}
\index{argument!list}
\index{reference}
\index{parameter}

When you pass a list to a function, the function gets a reference to
the list.  If the function modifies the list, the caller sees
the change.  For example, \verb"delete_head" removes the first element
from a list:

\begin{verbatim}
def delete_head(t):
    del t[0]
\end{verbatim}
%
Here's how it is used:

\begin{verbatim}
>>> letters = ['a', 'b', 'c']
>>> delete_head(letters)
>>> letters
['b', 'c']
\end{verbatim}
%
The parameter {\tt t} and the variable {\tt letters} are
aliases for the same object.  The stack diagram looks like
Figure~\ref{fig.stack5}.
\index{stack diagram}
\index{diagram!stack}

\begin{figure}
\centerline
{\includegraphics[scale=0.8]{figs/stack5.pdf}}
\caption{Stack diagram.}
\label{fig.stack5}
\end{figure}

Since the list is shared by two frames, I drew
it between them.

It is important to distinguish between operations that
modify lists and operations that create new lists.  For
example, the {\tt append} method modifies a list, but the
{\tt +} operator creates a new list.
\index{append method}
\index{method!append}
\index{list!concatenation}
\index{concatenation!list}

Here's an example using {\tt append}:
%
\begin{verbatim}
>>> t1 = [1, 2]
>>> t2 = t1.append(3)
>>> t1
[1, 2, 3]
>>> t2
None
\end{verbatim}
%
The return value from {\tt append} is {\tt None}.

Here's an example using the {\tt +} operator:
%
\begin{verbatim}
>>> t3 = t1 + [4]
>>> t1
[1, 2, 3]
>>> t3
[1, 2, 3, 4]
\end{verbatim}
%
The result of the operator is a new list, and the original list is
unchanged.

This difference is important when you write functions that
are supposed to modify lists.  For example, this function
{\em does not} delete the head of a list:
%
\begin{verbatim}
def bad_delete_head(t):
    t = t[1:]              # WRONG!
\end{verbatim}
%
The slice operator creates a new list and the assignment
makes {\tt t} refer to it, but that doesn't affect the caller.
\index{slice operator}
\index{operator!slice}
%
\begin{verbatim}
>>> t4 = [1, 2, 3]
>>> bad_delete_head(t4)
>>> t4
[1, 2, 3]
\end{verbatim}
%
At the beginning of \verb"bad_delete_head", {\tt t} and {\tt t4}
refer to the same list.  At the end, {\tt t} refers to a new list,
but {\tt t4} still refers to the original, unmodified list.

An alternative is to write a function that creates and
returns a new list.  For
example, {\tt tail} returns all but the first
element of a list:

\begin{verbatim}
def tail(t):
    return t[1:]
\end{verbatim}
%
This function leaves the original list unmodified.
Here's how it is used:

\begin{verbatim}
>>> letters = ['a', 'b', 'c']
>>> rest = tail(letters)
>>> rest
['b', 'c']
\end{verbatim}



\section{디버깅}
%Debugging
\index{debugging}

Careless use of lists (and other mutable objects)
can lead to long hours of debugging.  Here are some common
pitfalls and ways to avoid them:

\begin{enumerate}

\item Most list methods modify the argument and
  return {\tt None}.  This is the opposite of the string methods,
  which return a new string and leave the original alone.

If you are used to writing string code like this:

\begin{verbatim}
word = word.strip()
\end{verbatim}

It is tempting to write list code like this:

\begin{verbatim}
t = t.sort()           # WRONG!
\end{verbatim}
\index{sort method}
\index{method!sort}

Because {\tt sort} returns {\tt None}, the
next operation you perform with {\tt t} is likely to fail.

Before using list methods and operators, you should read the
documentation carefully and then test them in interactive mode.

\item Pick an idiom and stick with it.

Part of the problem with lists is that there are too many
ways to do things.  For example, to remove an element from
a list, you can use {\tt pop}, {\tt remove}, {\tt del},
or even a slice assignment.

To add an element, you can use the {\tt append} method or
the {\tt +} operator.  Assuming that {\tt t} is a list and
{\tt x} is a list element, these are correct: 

\begin{verbatim}
t.append(x)
t = t + [x]
t += [x]
\end{verbatim}

And these are wrong:

\begin{verbatim}
t.append([x])          # WRONG!
t = t.append(x)        # WRONG!
t + [x]                # WRONG!
t = t + x              # WRONG!
\end{verbatim}

Try out each of these examples in interactive mode to make sure
you understand what they do.  Notice that only the last
one causes a runtime error; the other three are legal, but they
do the wrong thing.


\item Make copies to avoid aliasing.
\index{aliasing!copying to avoid}
\index{copy!to avoid aliasing}

If you want to use a method like {\tt sort} that modifies
the argument, but you need to keep the original list as
well, you can make a copy.

\begin{verbatim}
>>> t = [3, 1, 2]
>>> t2 = t[:]
>>> t2.sort()
>>> t
[3, 1, 2]
>>> t2
[1, 2, 3]
\end{verbatim}

In this example you could also use the built-in function {\tt sorted},
which returns a new, sorted list and leaves the original alone.
\index{sorted!function}
\index{function!sorted}

\begin{verbatim}
>>> t2 = sorted(t)
>>> t
[3, 1, 2]
>>> t2
[1, 2, 3]
\end{verbatim}

\end{enumerate}



\section{용어 해설}
%Glossary

\begin{description}

\item[list:] A sequence of values.
\index{list}

\item[element:] One of the values in a list (or other sequence),
also called items.
\index{element}

\item[nested list:] A list that is an element of another list.
\index{nested list}

\item[accumulator:] A variable used in a loop to add up or
accumulate a result.
\index{accumulator}

\item[augmented assignment:] A statement that updates the value
of a variable using an operator like \verb"+=".
\index{assignment!augmented}
\index{augmented assignment}
\index{traversal}

\item[reduce:] A processing pattern that traverses a sequence 
and accumulates the elements into a single result.
\index{reduce pattern}
\index{pattern!reduce}

\item[map:] A processing pattern that traverses a sequence and
performs an operation on each element.
\index{map pattern}
\index{pattern!map}

\item[filter:] A processing pattern that traverses a list and
selects the elements that satisfy some criterion.
\index{filter pattern}
\index{pattern!filter}

\item[object:] Something a variable can refer to.  An object
has a type and a value.
\index{object}

\item[equivalent:] Having the same value.
\index{equivalent}

\item[identical:] Being the same object (which implies equivalence).
\index{identical}

\item[reference:] The association between a variable and its value.
\index{reference}

\item[aliasing:] A circumstance where two or more variables refer to the same
object.
\index{aliasing}

\item[delimiter:] A character or string used to indicate where a
string should be split.
\index{delimiter}

\end{description}


\section{연습 문제}
%Exercises

You can download solutions to these exercises from
\url{http://thinkpython2.com/code/list_exercises.py}.

\begin{exercise}

Write a function called \verb"nested_sum" that takes a list of lists
of integers and adds up the elements from all of the nested lists.
For example:

\begin{verbatim}
>>> t = [[1, 2], [3], [4, 5, 6]]
>>> nested_sum(t)
21
\end{verbatim}

\end{exercise}

\begin{exercise}
\label{cumulative}
\index{cumulative sum}

Write a function called {\tt cumsum} that takes a list of numbers and
returns the cumulative sum; that is, a new list where the $i$th
element is the sum of the first $i+1$ elements from the original list.
For example:

\begin{verbatim}
>>> t = [1, 2, 3]
>>> cumsum(t)
[1, 3, 6]
\end{verbatim}

\end{exercise}

\begin{exercise}

Write a function called \verb"middle" that takes a list and
returns a new list that contains all but the first and last
elements.  For example:

\begin{verbatim}
>>> t = [1, 2, 3, 4]
>>> middle(t)
[2, 3]
\end{verbatim}

\end{exercise}

\begin{exercise}

Write a function called \verb"chop" that takes a list, modifies it
by removing the first and last elements, and returns {\tt None}.
For example:

\begin{verbatim}
>>> t = [1, 2, 3, 4]
>>> chop(t)
>>> t
[2, 3]
\end{verbatim}

\end{exercise}


\begin{exercise}
Write a function called \verb"is_sorted" that takes a list as a
parameter and returns {\tt True} if the list is sorted in ascending
order and {\tt False} otherwise.  For example:

\begin{verbatim}
>>> is_sorted([1, 2, 2])
True
>>> is_sorted(['b', 'a'])
False
\end{verbatim}

\end{exercise}


\begin{exercise}
\label{anagram}
\index{anagram}

Two words are anagrams if you can rearrange the letters from one
to spell the other.  Write a function called \verb"is_anagram"
that takes two strings and returns {\tt True} if they are anagrams.
\end{exercise}



\begin{exercise}
\label{duplicate}
\index{duplicate}
\index{uniqueness}

Write a function called \verb"has_duplicates" that takes
a list and returns {\tt True} if there is any element that
appears more than once.  It should not modify the original
list.

\end{exercise}


\begin{exercise}

This exercise pertains to the so-called Birthday Paradox, which you
can read about at \url{http://en.wikipedia.org/wiki/Birthday_paradox}.
\index{birthday paradox}

If there are 23 students in your class, what are the chances
that two of you have the same birthday?  You can estimate this
probability by generating random samples of 23 birthdays
and checking for matches.  Hint: you can generate random birthdays
with the {\tt randint} function in the {\tt random} module.
\index{random module}
\index{module!random}
\index{randint function}
\index{function!randint}

You can download my
solution from \url{http://thinkpython2.com/code/birthday.py}.

\end{exercise}



\begin{exercise}
\index{append method}
\index{method append}
\index{list!concatenation}
\index{concatenation!list}

Write a function that reads the file {\tt words.txt} and builds
a list with one element per word.  Write two versions of
this function, one using the {\tt append} method and the
other using the idiom {\tt t = t + [x]}.  Which one takes
longer to run?  Why?

Solution: \url{http://thinkpython2.com/code/wordlist.py}.
\index{time module}
\index{module!time}

\end{exercise}


\begin{exercise}
\label{wordlist1}
\label{bisection}
\index{membership!bisection search}
\index{bisection search}
\index{search, bisection}
\index{membership!binary search}
\index{binary search}
\index{search, binary}

To check whether a word is in the word list, you could use
the {\tt in} operator, but it would be slow because it searches
through the words in order.

Because the words are in alphabetical order, we can speed things up
with a bisection search (also known as binary search), which is
similar to what you do when you look a word up in the dictionary.  You
start in the middle and check to see whether the word you are looking
for comes before the word in the middle of the list.  If so, you
search the first half of the list the same way.  Otherwise you search
the second half.

Either way, you cut the remaining search space in half.  If the
word list has 113,809 words, it will take about 17 steps to
find the word or conclude that it's not there.

Write a function called \verb"in_bisect" that takes a sorted list
and a target value and returns the index of the value
in the list if it's there, or {\tt None} if it's not.
\index{bisect module}
\index{module!bisect}

Or you could read the documentation of the {\tt bisect} module
and use that!  Solution: \url{http://thinkpython2.com/code/inlist.py}.

\end{exercise}

\begin{exercise}
\index{reverse word pair}

Two words are a ``reverse pair'' if each is the reverse of the
other.  Write a program that finds all the reverse pairs in the
word list.  Solution: \url{http://thinkpython2.com/code/reverse_pair.py}.

\end{exercise}

\begin{exercise}
\index{interlocking words}

Two words ``interlock'' if taking alternating letters from each forms
a new word.  For example, ``shoe'' and ``cold''
interlock to form ``schooled''.
Solution: \url{http://thinkpython2.com/code/interlock.py}.
Credit: This exercise is inspired by an example at \url{http://puzzlers.org}.

\begin{enumerate}

\item Write a program that finds all pairs of words that interlock.
  Hint: don't enumerate all pairs!

\item Can you find any words that are three-way interlocked; that is,
  every third letter forms a word, starting from the first, second or
  third?

\end{enumerate}
\end{exercise}


\chapter{Dictionaries}

This chapter presents another built-in type called a dictionary.
Dictionaries are one of Python's best features; they are the
building blocks of many efficient and elegant algorithms.


\section{A dictionary is a mapping}

\index{dictionary}
\index{dictionary}
\index{type!dict}
\index{key}
\index{key-value pair}
\index{index}
A {\bf dictionary} is like a list, but more general.  In a list,
the indices have to be integers; in a dictionary they can
be (almost) any type.

A dictionary contains a collection of indices, which are called {\bf
  keys}, and a collection of values.  Each key is associated with a
single value.  The association of a key and a value is called a {\bf
  key-value pair} or sometimes an {\bf item}.  \index{item}

In mathematical language, a dictionary represents a {\bf mapping}
from keys to values, so you can also say that each key
``maps to'' a value.
As an example, we'll build a dictionary that maps from English
to Spanish words, so the keys and the values are all strings.

The function {\tt dict} creates a new dictionary with no items.
Because {\tt dict} is the name of a built-in function, you
should avoid using it as a variable name.
\index{dict function}
\index{function!dict}

\begin{verbatim}
>>> eng2sp = dict()
>>> eng2sp
{}
\end{verbatim}

The squiggly-brackets, \verb"{}", represent an empty dictionary.
To add items to the dictionary, you can use square brackets:
\index{squiggly bracket}
\index{bracket!squiggly}

\begin{verbatim}
>>> eng2sp['one'] = 'uno'
\end{verbatim}
%
This line creates an item that maps from the key
\verb"'one'" to the value \verb"'uno'".  If we print the
dictionary again, we see a key-value pair with a colon
between the key and value:

\begin{verbatim}
>>> eng2sp
{'one': 'uno'}
\end{verbatim}
%
This output format is also an input format.  For example,
you can create a new dictionary with three items:

\begin{verbatim}
>>> eng2sp = {'one': 'uno', 'two': 'dos', 'three': 'tres'}
\end{verbatim}
%
But if you print {\tt eng2sp}, you might be surprised:

\begin{verbatim}
>>> eng2sp
{'one': 'uno', 'three': 'tres', 'two': 'dos'}
\end{verbatim}
%
The order of the key-value pairs might not be the same.  If
you type the same example on your computer, you might get a
different result.  In general, the order of items in
a dictionary is unpredictable.

But that's not a problem because
the elements of a dictionary are never indexed with integer indices.
Instead, you use the keys to look up the corresponding values:

\begin{verbatim}
>>> eng2sp['two']
'dos'
\end{verbatim}
%
The key \verb"'two'" always maps to the value \verb"'dos'" so the order
of the items doesn't matter.

If the key isn't in the dictionary, you get an exception:
\index{exception!KeyError}
\index{KeyError}

\begin{verbatim}
>>> eng2sp['four']
KeyError: 'four'
\end{verbatim}
%
The {\tt len} function works on dictionaries; it returns the
number of key-value pairs:
\index{len function}
\index{function!len}

\begin{verbatim}
>>> len(eng2sp)
3
\end{verbatim}
%
The {\tt in} operator works on dictionaries, too; it tells you whether
something appears as a {\em key} in the dictionary (appearing
as a value is not good enough).
\index{membership!dictionary}
\index{in operator}
\index{operator!in}

\begin{verbatim}
>>> 'one' in eng2sp
True
>>> 'uno' in eng2sp
False
\end{verbatim}
%
To see whether something appears as a value in a dictionary, you
can use the method {\tt values}, which returns a collection of
values, and then use the {\tt in} operator:
\index{values method}
\index{method!values}

\begin{verbatim}
>>> vals = eng2sp.values()
>>> 'uno' in vals
True
\end{verbatim}
%
The {\tt in} operator uses different algorithms for lists and
dictionaries.  For lists, it searches the elements of the list in
order, as in Section~\ref{find}.  As the list gets longer, the search
time gets longer in direct proportion.

For dictionaries, Python uses an
algorithm called a {\bf hashtable} that has a remarkable property: the
{\tt in} operator takes about the same amount of time no matter how
many items are in the dictionary.  I explain how that's possible
in Section~\ref{hashtable}, but the explanation might not make
sense until you've read a few more chapters.


\section{Dictionary as a collection of counters}
\label{histogram}
\index{counter}

Suppose you are given a string and you want to count how many
times each letter appears.  There are several ways you could do it:

\begin{enumerate}

\item You could create 26 variables, one for each letter of the
alphabet.  Then you could traverse the string and, for each
character, increment the corresponding counter, probably using
a chained conditional.

\item You could create a list with 26 elements.  Then you could
convert each character to a number (using the built-in function
{\tt ord}), use the number as an index into the list, and increment
the appropriate counter.

\item You could create a dictionary with characters as keys
and counters as the corresponding values.  The first time you
see a character, you would add an item to the dictionary.  After
that you would increment the value of an existing item.

\end{enumerate}

Each of these options performs the same computation, but each
of them implements that computation in a different way.
\index{implementation}

An {\bf implementation} is a way of performing a computation;
some implementations are better than others.  For example,
an advantage of the dictionary implementation is that we don't
have to know ahead of time which letters appear in the string
and we only have to make room for the letters that do appear.

Here is what the code might look like:

\begin{verbatim}
def histogram(s):
    d = dict()
    for c in s:
        if c not in d:
            d[c] = 1
        else:
            d[c] += 1
    return d
\end{verbatim}
%
The name of the function is {\tt histogram}, which is a statistical
term for a collection of counters (or frequencies).
\index{histogram}
\index{frequency}
\index{traversal}

The first line of the
function creates an empty dictionary.  The {\tt for} loop traverses
the string.  Each time through the loop, if the character {\tt c} is
not in the dictionary, we create a new item with key {\tt c} and the
initial value 1 (since we have seen this letter once).  If {\tt c} is
already in the dictionary we increment {\tt d[c]}.
\index{histogram}

Here's how it works:

\begin{verbatim}
>>> h = histogram('brontosaurus')
>>> h
{'a': 1, 'b': 1, 'o': 2, 'n': 1, 's': 2, 'r': 2, 'u': 2, 't': 1}
\end{verbatim}
%
The histogram indicates that the letters \verb"'a'" and \verb"'b'"
appear once; \verb"'o'" appears twice, and so on.


\index{get method}
\index{method!get}
Dictionaries have a method called {\tt get} that takes a key
and a default value.  If the key appears in the dictionary,
{\tt get} returns the corresponding value; otherwise it returns
the default value.  For example:

\begin{verbatim}
>>> h = histogram('a')
>>> h
{'a': 1}
>>> h.get('a', 0)
1
>>> h.get('b', 0)
0
\end{verbatim}
%
As an exercise, use {\tt get} to write {\tt histogram} more concisely.  You
should be able to eliminate the {\tt if} statement.


\section{Looping and dictionaries}
\index{dictionary!looping with}
\index{looping!with dictionaries}
\index{traversal}

If you use a dictionary in a {\tt for} statement, it traverses
the keys of the dictionary.  For example, \verb"print_hist"
prints each key and the corresponding value:

\begin{verbatim}
def print_hist(h):
    for c in h:
        print(c, h[c])
\end{verbatim}
%
Here's what the output looks like:

\begin{verbatim}
>>> h = histogram('parrot')
>>> print_hist(h)
a 1
p 1
r 2
t 1
o 1
\end{verbatim}
%
Again, the keys are in no particular order.  To traverse the keys
in sorted order, you can use the built-in function {\tt sorted}:
\index{sorted!function}
\index{function!sorted}

\begin{verbatim}
>>> for key in sorted(h):
...     print(key, h[key])
a 1
o 1
p 1
r 2
t 1
\end{verbatim}

%TODO: get this on Atlas


\section{Reverse lookup}
\label{raise}
\index{dictionary!lookup}
\index{dictionary!reverse lookup}
\index{lookup, dictionary}
\index{reverse lookup, dictionary}

Given a dictionary {\tt d} and a key {\tt k}, it is easy to
find the corresponding value {\tt v = d[k]}.  This operation
is called a {\bf lookup}.

But what if you have {\tt v} and you want to find {\tt k}?
You have two problems: first, there might be more than one
key that maps to the value {\tt v}.  Depending on the application,
you might be able to pick one, or you might have to make
a list that contains all of them.  Second, there is no
simple syntax to do a {\bf reverse lookup}; you have to search.

Here is a function that takes a value and returns the first
key that maps to that value:

\begin{verbatim}
def reverse_lookup(d, v):
    for k in d:
        if d[k] == v:
            return k
    raise LookupError()
\end{verbatim}
%
This function is yet another example of the search pattern, but it
uses a feature we haven't seen before, {\tt raise}.  The 
{\bf raise statement} causes an exception; in this case it causes a
{\tt LookupError}, which is a built-in exception used to indicate
that a lookup operation failed.
\index{search}
\index{pattern!search} \index{raise statement} \index{statement!raise}
\index{exception!LookupError} \index{LookupError}

If we get to the end of the loop, that means {\tt v}
doesn't appear in the dictionary as a value, so we raise an
exception.

Here is an example of a successful reverse lookup:

\begin{verbatim}
>>> h = histogram('parrot')
>>> key = reverse_lookup(h, 2)
>>> key
'r'
\end{verbatim}
%
And an unsuccessful one:

\begin{verbatim}
>>> key = reverse_lookup(h, 3)
Traceback (most recent call last):
  File "<stdin>", line 1, in <module>
  File "<stdin>", line 5, in reverse_lookup
LookupError
\end{verbatim}
%
The effect when you raise an exception is the same as when
Python raises one: it prints a traceback and an error message.
\index{traceback}
\index{optional argument}
\index{argument!optional}

The {\tt raise} statement can take a detailed error message as an
optional argument.  For example:

\begin{verbatim}
>>> raise LookupError('value does not appear in the dictionary')
Traceback (most recent call last):
  File "<stdin>", line 1, in ?
LookupError: value does not appear in the dictionary
\end{verbatim}
%
A reverse lookup is much slower than a forward lookup; if you
have to do it often, or if the dictionary gets big, the performance
of your program will suffer.


\section{Dictionaries and lists}
\label{invert}

Lists can appear as values in a dictionary.  For example, if you
are given a dictionary that maps from letters to frequencies, you
might want to invert it; that is, create a dictionary that maps
from frequencies to letters.  Since there might be several letters
with the same frequency, each value in the inverted dictionary
should be a list of letters.
\index{invert dictionary}
\index{dictionary!invert}

Here is a function that inverts a dictionary:

\begin{verbatim}
def invert_dict(d):
    inverse = dict()
    for key in d:
        val = d[key]
        if val not in inverse:
            inverse[val] = [key]
        else:
            inverse[val].append(key)
    return inverse
\end{verbatim}
%
Each time through the loop, {\tt key} gets a key from {\tt d} and 
{\tt val} gets the corresponding value.  If {\tt val} is not in {\tt
  inverse}, that means we haven't seen it before, so we create a new
item and initialize it with a {\bf singleton} (a list that contains a
single element).  Otherwise we have seen this value before, so we
append the corresponding key to the list.  \index{singleton}

Here is an example:

\begin{verbatim}
>>> hist = histogram('parrot')
>>> hist
{'a': 1, 'p': 1, 'r': 2, 't': 1, 'o': 1}
>>> inverse = invert_dict(hist)
>>> inverse
{1: ['a', 'p', 't', 'o'], 2: ['r']}
\end{verbatim}

\begin{figure}
\centerline
{\includegraphics[scale=0.8]{figs/dict1.pdf}}
\caption{State diagram.}
\label{fig.dict1}
\end{figure}

Figure~\ref{fig.dict1} is a state diagram showing {\tt hist} and {\tt inverse}.
A dictionary is represented as a box with the type {\tt dict} above it
and the key-value pairs inside.  If the values are integers, floats or
strings, I draw them inside the box, but I usually draw lists
outside the box, just to keep the diagram simple.
\index{state diagram}
\index{diagram!state}

Lists can be values in a dictionary, as this example shows, but they
cannot be keys.  Here's what happens if you try:
\index{TypeError}
\index{exception!TypeError}


\begin{verbatim}
>>> t = [1, 2, 3]
>>> d = dict()
>>> d[t] = 'oops'
Traceback (most recent call last):
  File "<stdin>", line 1, in ?
TypeError: list objects are unhashable
\end{verbatim}
%
I mentioned earlier that a dictionary is implemented using
a hashtable and that means that the keys have to be {\bf hashable}.
\index{hash function}
\index{hashable}

A {\bf hash} is a function that takes a value (of any kind)
and returns an integer.  Dictionaries use these integers,
called hash values, to store and look up key-value pairs.
\index{immutability}

This system works fine if the keys are immutable.  But if the
keys are mutable, like lists, bad things happen.  For example,
when you create a key-value pair, Python hashes the key and 
stores it in the corresponding location.  If you modify the
key and then hash it again, it would go to a different location.
In that case you might have two entries for the same key,
or you might not be able to find a key.  Either way, the
dictionary wouldn't work correctly.

That's why keys have to be hashable, and why mutable types like
lists aren't.  The simplest way to get around this limitation is to
use tuples, which we will see in the next chapter.

Since dictionaries are mutable, they can't be used as keys,
but they {\em can} be used as values.


\section{Memos}
\label{memoize}

If you played with the {\tt fibonacci} function from
Section~\ref{one.more.example}, you might have noticed that the bigger
the argument you provide, the longer the function takes to run.
Furthermore, the run time increases quickly.
\index{fibonacci function}
\index{function!fibonacci}

To understand why, consider Figure~\ref{fig.fibonacci}, which shows
the {\bf call graph} for {\tt fibonacci} with {\tt n=4}:

\begin{figure}
\centerline
{\includegraphics[scale=0.7]{figs/fibonacci.pdf}}
\caption{Call graph.}
\label{fig.fibonacci}
\end{figure}

A call graph shows a set of function frames, with lines connecting each
frame to the frames of the functions it calls.  At the top of the
graph, {\tt fibonacci} with {\tt n=4} calls {\tt fibonacci} with {\tt
n=3} and {\tt n=2}.  In turn, {\tt fibonacci} with {\tt n=3} calls
{\tt fibonacci} with {\tt n=2} and {\tt n=1}.  And so on.
\index{function frame}
\index{frame}
\index{call graph}

Count how many times {\tt fibonacci(0)} and {\tt fibonacci(1)} are
called.  This is an inefficient solution to the problem, and it gets
worse as the argument gets bigger.
\index{memo}

One solution is to keep track of values that have already been
computed by storing them in a dictionary.  A previously computed value
that is stored for later use is called a {\bf memo}.  Here is a
``memoized'' version of {\tt fibonacci}:

\begin{verbatim}
known = {0:0, 1:1}

def fibonacci(n):
    if n in known:
        return known[n]

    res = fibonacci(n-1) + fibonacci(n-2)
    known[n] = res
    return res
\end{verbatim}
%
{\tt known} is a dictionary that keeps track of the Fibonacci
numbers we already know.  It starts with
two items: 0 maps to 0 and 1 maps to 1.

Whenever {\tt fibonacci} is called, it checks {\tt known}.
If the result is already there, it can return
immediately.  Otherwise it has to 
compute the new value, add it to the dictionary, and return it.

If you run this version of {\tt fibonacci} and compare it with
the original, you will find that it is much faster.



\section{Global variables}
\index{global variable}
\index{variable!global}

In the previous example, {\tt known} is created outside the function,
so it belongs to the special frame called \verb"__main__".
Variables in \verb"__main__" are sometimes called {\bf global}
because they can be accessed from any function.  Unlike local
variables, which disappear when their function ends, global variables
persist from one function call to the next.
\index{flag}
\index{main}

It is common to use global variables for {\bf flags}; that is, 
boolean variables that indicate (``flag'') whether a condition
is true.  For example, some programs use
a flag named {\tt verbose} to control the level of detail in the
output:

\begin{verbatim}
verbose = True

def example1():
    if verbose:
        print('Running example1')
\end{verbatim}
%
If you try to reassign a global variable, you might be surprised.
The following example is supposed to keep track of whether the
function has been called:
\index{reassignment}

\begin{verbatim}
been_called = False

def example2():
    been_called = True         # WRONG
\end{verbatim}
%
But if you run it you will see that the value of \verb"been_called"
doesn't change.  The problem is that {\tt example2} creates a new local
variable named \verb"been_called".  The local variable goes away when
the function ends, and has no effect on the global variable.
\index{global statement}
\index{statement!global}
\index{declaration}

To reassign a global variable inside a function you have to
{\bf declare} the global variable before you use it:

\begin{verbatim}
been_called = False

def example2():
    global been_called 
    been_called = True
\end{verbatim}
%
The {\bf global statement} tells the interpreter
something like, ``In this function, when I say \verb"been_called", I
mean the global variable; don't create a local one.''
\index{update!global variable}
\index{global variable!update}

Here's an example that tries to update a global variable:

\begin{verbatim}
count = 0

def example3():
    count = count + 1          # WRONG
\end{verbatim}
%
If you run it you get:
\index{UnboundLocalError}
\index{exception!UnboundLocalError}

\begin{verbatim}
UnboundLocalError: local variable 'count' referenced before assignment
\end{verbatim}
%
Python assumes that {\tt count} is local, and under that assumption
you are reading it before writing it.  The solution, again,
is to declare {\tt count} global.
\index{counter}

\begin{verbatim}
def example3():
    global count
    count += 1
\end{verbatim}
%
If a global variable refers to a mutable value, you can modify
the value without declaring the variable:
\index{mutability}

\begin{verbatim}
known = {0:0, 1:1}

def example4():
    known[2] = 1
\end{verbatim}
%
So you can add, remove and replace elements of a global list or
dictionary, but if you want to reassign the variable, you
have to declare it:

\begin{verbatim}
def example5():
    global known
    known = dict()
\end{verbatim}
%
Global variables can be useful, but if you have a lot of them,
and you modify them frequently, they can make programs
hard to debug.


\section{디버깅}
%Debugging
\index{debugging}

As you work with bigger datasets it can become unwieldy to
debug by printing and checking the output by hand.  Here are some
suggestions for debugging large datasets:

\begin{description}

\item[Scale down the input:] If possible, reduce the size of the
dataset.  For example if the program reads a text file, start with
just the first 10 lines, or with the smallest example you can find.
You can either edit the files themselves, or (better) modify the
program so it reads only the first {\tt n} lines.

If there is an error, you can reduce {\tt n} to the smallest
value that manifests the error, and then increase it gradually
as you find and correct errors.

\item[Check summaries and types:] Instead of printing and checking the
entire dataset, consider printing summaries of the data: for example,
the number of items in a dictionary or the total of a list of numbers.

A common cause of runtime errors is a value that is not the right
type.  For debugging this kind of error, it is often enough to print
the type of a value.

\item[Write self-checks:]  Sometimes you can write code to check
for errors automatically.  For example, if you are computing the
average of a list of numbers, you could check that the result is
not greater than the largest element in the list or less than
the smallest.  This is called a ``sanity check'' because it detects
results that are ``insane''.
\index{sanity check}
\index{consistency check}

Another kind of check compares the results of two different
computations to see if they are consistent.  This is called a
``consistency check''.

\item[Format the output:] Formatting debugging output
can make it easier to spot an error.  We saw an example in
Section~\ref{factdebug}.  Another tool you might find useful is the {\tt pprint} module, which provides
a {\tt pprint} function that displays built-in types in
a more human-readable format ({\tt pprint} stands for
``pretty print'').
\index{pretty print}
\index{pprint module}
\index{module!pprint}

\end{description}

Again, time you spend building scaffolding can reduce
the time you spend debugging.
\index{scaffolding}


\section{용어 해설}
%Glossary

\begin{description}

\item[mapping:] A relationship in which each element of one set
corresponds to an element of another set.
\index{mapping}

\item[dictionary:] A mapping from keys to their
corresponding values.
\index{dictionary}

\item[key-value pair:] The representation of the mapping from
a key to a value.
\index{key-value pair}

\item[item:] In a dictionary, another name for a key-value
  pair.
\index{item!dictionary}

\item[key:] An object that appears in a dictionary as the
first part of a key-value pair.
\index{key}

\item[value:] An object that appears in a dictionary as the
second part of a key-value pair.  This is more specific than
our previous use of the word ``value''.
\index{value}

\item[implementation:] A way of performing a computation.
\index{implementation}

\item[hashtable:] The algorithm used to implement Python
dictionaries.
\index{hashtable}

\item[hash function:] A function used by a hashtable to compute the
location for a key.
\index{hash function}

\item[hashable:] A type that has a hash function.  Immutable
types like integers,
floats and strings are hashable; mutable types like lists and
dictionaries are not.
\index{hashable}

\item[lookup:] A dictionary operation that takes a key and finds
the corresponding value.
\index{lookup}

\item[reverse lookup:] A dictionary operation that takes a value and finds
one or more keys that map to it.
\index{reverse lookup}

\item[raise statement:]  A statement that (deliberately) raises an exception.
\index{raise statement}
\index{statement!raise}

\item[singleton:] A list (or other sequence) with a single element.
\index{singleton}

\item[call graph:] A diagram that shows every frame created during
the execution of a program, with an arrow from each caller to
each callee. 
\index{call graph}
\index{diagram!call graph}

\item[memo:] A computed value stored to avoid unnecessary future 
computation.
\index{memo}

\item[global variable:]  A variable defined outside a function.  Global
variables can be accessed from any function.
\index{global variable}

\item[global statement:]  A statement that declares a variable name
global.
\index{global statement}
\index{statement!global}

\item[flag:] A boolean variable used to indicate whether a condition
is true.
\index{flag}

\item[declaration:] A statement like {\tt global} that tells the
interpreter something about a variable.
\index{declaration}

\end{description}


\section{연습 문제}
%Exercises

\begin{exercise}
\label{wordlist2}
\index{set membership}
\index{membership!set}

Write a function that reads the words in {\tt words.txt} and
stores them as keys in a dictionary.  It doesn't matter what the
values are.  Then you can use the {\tt in} operator
as a fast way to check whether a string is in
the dictionary.

If you did Exercise~\ref{wordlist1}, you can compare the speed
of this implementation with the list {\tt in} operator and the
bisection search.

\end{exercise}


\begin{exercise}
\label{setdefault}

Read the documentation of the dictionary method {\tt setdefault}
and use it to write a more concise version of \verb"invert_dict".
Solution: \url{http://thinkpython2.com/code/invert_dict.py}.
\index{setdefault method}
\index{method!setdefault}

\end{exercise}


\begin{exercise}
Memoize the Ackermann function from Exercise~\ref{ackermann} and see if
memoization makes it possible to evaluate the function with bigger
arguments.  Hint: no.
Solution: \url{http://thinkpython2.com/code/ackermann_memo.py}.
\index{Ackermann function}
\index{function!ack}

\end{exercise}



\begin{exercise}
\index{duplicate}

If you did Exercise~\ref{duplicate}, you already have
a function named \verb"has_duplicates" that takes a list
as a parameter and returns {\tt True} if there is any object
that appears more than once in the list.

Use a dictionary to write a faster, simpler version of
\verb"has_duplicates". 
Solution: \url{http://thinkpython2.com/code/has_duplicates.py}.

\end{exercise}


\begin{exercise}
\label{exrotatepairs}
\index{letter rotation}
\index{rotation!letters}

Two words are ``rotate pairs'' if you can rotate one of them
and get the other (see \verb"rotate_word" in Exercise~\ref{exrotate}).

Write a program that reads a wordlist and finds all the rotate
pairs.  Solution: \url{http://thinkpython2.com/code/rotate_pairs.py}.

\end{exercise}


\begin{exercise}
\index{Car Talk}
\index{Puzzler}

Here's another Puzzler from {\em Car Talk} 
(\url{http://www.cartalk.com/content/puzzlers}):

\begin{quote}
This was sent in by a fellow named Dan O'Leary. He came upon a common
one-syllable, five-letter word recently that has the following unique
property. When you remove the first letter, the remaining letters form
a homophone of the original word, that is a word that sounds exactly
the same. Replace the first letter, that is, put it back and remove
the second letter and the result is yet another homophone of the
original word. And the question is, what's the word?

Now I'm going to give you an example that doesn't work. Let's look at
the five-letter word, `wrack.' W-R-A-C-K, you know like to `wrack with
pain.' If I remove the first letter, I am left with a four-letter
word, 'R-A-C-K.' As in, `Holy cow, did you see the rack on that buck!
It must have been a nine-pointer!' It's a perfect homophone. If you
put the `w' back, and remove the `r,' instead, you're left with the
word, `wack,' which is a real word, it's just not a homophone of the
other two words.

But there is, however, at least one word that Dan and we know of,
which will yield two homophones if you remove either of the first two
letters to make two, new four-letter words. The question is, what's
the word?
\end{quote}
\index{homophone}
\index{reducible word}
\index{word, reducible}

You can use the dictionary from Exercise~\ref{wordlist2} to check
whether a string is in the word list.

To check whether two words are homophones, you can use the CMU
Pronouncing Dictionary.  You can download it from
\url{http://www.speech.cs.cmu.edu/cgi-bin/cmudict} or from
\url{http://thinkpython2.com/code/c06d} and you can also download
\url{http://thinkpython2.com/code/pronounce.py}, which provides a function
named \verb"read_dictionary" that reads the pronouncing dictionary and
returns a Python dictionary that maps from each word to a string that
describes its primary pronunciation.

Write a program that lists all the words that solve the Puzzler.
Solution: \url{http://thinkpython2.com/code/homophone.py}.

\end{exercise}



\chapter{Tuples}
\label{tuplechap}

This chapter presents one more built-in type, the tuple, and then
shows how lists, dictionaries, and tuples work together.
I also present a useful feature for variable-length argument lists,
the gather and scatter operators.

One note: there is no consensus on how to pronounce ``tuple''.
Some people say ``tuh-ple'', which rhymes with ``supple''.  But
in the context of programming, most people say ``too-ple'', which
rhymes with ``quadruple''.


\section{Tuples are immutable}
\index{tuple}
\index{type!tuple}
\index{sequence}

A tuple is a sequence of values.  The values can be any type, and
they are indexed by integers, so in that respect tuples are a lot
like lists.  The important difference is that tuples are immutable.
\index{mutability}
\index{immutability}

Syntactically, a tuple is a comma-separated list of values:

\begin{verbatim}
>>> t = 'a', 'b', 'c', 'd', 'e'
\end{verbatim}
%
Although it is not necessary, it is common to enclose tuples in
parentheses:
\index{parentheses!tuples in}

\begin{verbatim}
>>> t = ('a', 'b', 'c', 'd', 'e')
\end{verbatim}
%
To create a tuple with a single element, you have to include a final
comma:
\index{singleton}
\index{tuple!singleton}

\begin{verbatim}
>>> t1 = 'a',
>>> type(t1)
<class 'tuple'>
\end{verbatim}
%
A value in parentheses is not a tuple:

\begin{verbatim}
>>> t2 = ('a')
>>> type(t2)
<class 'str'>
\end{verbatim}
%
Another way to create a tuple is the built-in function {\tt tuple}.
With no argument, it creates an empty tuple:
\index{tuple function}
\index{function!tuple}

\begin{verbatim}
>>> t = tuple()
>>> t
()
\end{verbatim}
%
If the argument is a sequence (string, list or tuple), the result
is a tuple with the elements of the sequence:

\begin{verbatim}
>>> t = tuple('lupins')
>>> t
('l', 'u', 'p', 'i', 'n', 's')
\end{verbatim}
%
Because {\tt tuple} is the name of a built-in function, you should
avoid using it as a variable name.

Most list operators also work on tuples.  The bracket operator
indexes an element:
\index{bracket operator}
\index{operator!bracket}

\begin{verbatim}
>>> t = ('a', 'b', 'c', 'd', 'e')
>>> t[0]
'a'
\end{verbatim}
%
And the slice operator selects a range of elements.
\index{slice operator}
\index{operator!slice}
\index{tuple!slice}
\index{slice!tuple}

\begin{verbatim}
>>> t[1:3]
('b', 'c')
\end{verbatim}
%
But if you try to modify one of the elements of the tuple, you get
an error:
\index{exception!TypeError}
\index{TypeError}
\index{item assignment}
\index{assignment!item}

\begin{verbatim}
>>> t[0] = 'A'
TypeError: object doesn't support item assignment
\end{verbatim}
%
Because tuples are immutable, you can't modify the elements.  But you
can replace one tuple with another:

\begin{verbatim}
>>> t = ('A',) + t[1:]
>>> t
('A', 'b', 'c', 'd', 'e')
\end{verbatim}
%
This statement makes a new tuple and then makes {\tt t} refer to it.

The relational operators work with tuples and other sequences;
Python starts by comparing the first element from each
sequence.  If they are equal, it goes on to the next elements,
and so on, until it finds elements that differ.  Subsequent
elements are not considered (even if they are really big).
\index{comparison!tuple}
\index{tuple!comparison}

\begin{verbatim}
>>> (0, 1, 2) < (0, 3, 4)
True
>>> (0, 1, 2000000) < (0, 3, 4)
True
\end{verbatim}



\section{Tuple assignment}
\label{tuple.assignment}
\index{tuple!assignment}
\index{assignment!tuple}
\index{swap pattern}
\index{pattern!swap}

It is often useful to swap the values of two variables.
With conventional assignments, you have to use a temporary
variable.  For example, to swap {\tt a} and {\tt b}:

\begin{verbatim}
>>> temp = a
>>> a = b
>>> b = temp
\end{verbatim}
%
This solution is cumbersome; {\bf tuple assignment} is more elegant:

\begin{verbatim}
>>> a, b = b, a
\end{verbatim}
%
The left side is a tuple of variables; the right side is a tuple of
expressions.  Each value is assigned to its respective variable.  
All the expressions on the right side are evaluated before any
of the assignments.

The number of variables on the left and the number of
values on the right have to be the same:
\index{exception!ValueError}
\index{ValueError}

\begin{verbatim}
>>> a, b = 1, 2, 3
ValueError: too many values to unpack
\end{verbatim}
%
More generally, the right side can be any kind of sequence
(string, list or tuple).  For example, to split an email address
into a user name and a domain, you could write:
\index{split method}
\index{method!split}
\index{email address}

\begin{verbatim}
>>> addr = 'monty@python.org'
>>> uname, domain = addr.split('@')
\end{verbatim}
%
The return value from {\tt split} is a list with two elements;
the first element is assigned to {\tt uname}, the second to
{\tt domain}.

\begin{verbatim}
>>> uname
'monty'
>>> domain
'python.org'
\end{verbatim}
%

\section{Tuples as return values}
\index{tuple}
\index{value!tuple}
\index{return value!tuple}
\index{function, tuple as return value}

Strictly speaking, a function can only return one value, but
if the value is a tuple, the effect is the same as returning
multiple values.  For example, if you want to divide two integers
and compute the quotient and remainder, it is inefficient to
compute {\tt x/y} and then {\tt x\%y}.  It is better to compute
them both at the same time.
\index{divmod}

The built-in function {\tt divmod} takes two arguments and
returns a tuple of two values, the quotient and remainder.
You can store the result as a tuple:

\begin{verbatim}
>>> t = divmod(7, 3)
>>> t
(2, 1)
\end{verbatim}
%
Or use tuple assignment to store the elements separately:
\index{tuple assignment}
\index{assignment!tuple}

\begin{verbatim}
>>> quot, rem = divmod(7, 3)
>>> quot
2
>>> rem
1
\end{verbatim}
%
Here is an example of a function that returns a tuple:

\begin{verbatim}
def min_max(t):
    return min(t), max(t)
\end{verbatim}
%
{\tt max} and {\tt min} are built-in functions that find
the largest and smallest elements of a sequence.  \verb"min_max"
computes both and returns a tuple of two values.
\index{max function}
\index{function!max}
\index{min function}
\index{function!min}


\section{Variable-length argument tuples}
\label{gather}
\index{variable-length argument tuple}
\index{argument!variable-length tuple}
\index{gather}
\index{parameter!gather}
\index{argument!gather}

Functions can take a variable number of arguments.  A parameter
name that begins with {\tt *} {\bf gathers} arguments into
a tuple.  For example, {\tt printall}
takes any number of arguments and prints them:

\begin{verbatim}
def printall(*args):
    print(args)
\end{verbatim}
%
The gather parameter can have any name you like, but {\tt args} is
conventional.  Here's how the function works:

\begin{verbatim}
>>> printall(1, 2.0, '3')
(1, 2.0, '3')
\end{verbatim}
%
The complement of gather is {\bf scatter}.  If you have a
sequence of values and you want to pass it to a function
as multiple arguments, you can use the {\tt *} operator.
For example, {\tt divmod} takes exactly two arguments; it
doesn't work with a tuple:
\index{scatter}
\index{argument scatter}
\index{TypeError}
\index{exception!TypeError}

\begin{verbatim}
>>> t = (7, 3)
>>> divmod(t)
TypeError: divmod expected 2 arguments, got 1
\end{verbatim}
%
But if you scatter the tuple, it works:

\begin{verbatim}
>>> divmod(*t)
(2, 1)
\end{verbatim}
%
Many of the built-in functions use
variable-length argument tuples.  For example, {\tt max}
and {\tt min} can take any number of arguments:
\index{max function}
\index{function!max}
\index{min function}
\index{function!min}

\begin{verbatim}
>>> max(1, 2, 3)
3
\end{verbatim}
%
But {\tt sum} does not.
\index{sum function}
\index{function!sum}

\begin{verbatim}
>>> sum(1, 2, 3)
TypeError: sum expected at most 2 arguments, got 3
\end{verbatim}
%
As an exercise, write a function called {\tt sumall} that takes any number
of arguments and returns their sum.


\section{Lists and tuples}
\index{zip function}
\index{function!zip}

{\tt zip} is a built-in function that takes two or more sequences and
returns a list of tuples where each tuple contains one
element from each sequence.  The name of the function refers to
a zipper, which joins and interleaves two rows of teeth.

This example zips a string and a list:

\begin{verbatim}
>>> s = 'abc'
>>> t = [0, 1, 2]
>>> zip(s, t)
<zip object at 0x7f7d0a9e7c48>
\end{verbatim}
%
The result is a {\bf zip object} that knows how to iterate through
the pairs.  The most common use of {\tt zip} is in a {\tt for} loop:

\begin{verbatim}
>>> for pair in zip(s, t):
...     print(pair)
...
('a', 0)
('b', 1)
('c', 2)
\end{verbatim}
%
A zip object is a kind of {\bf iterator}, which is any object
that iterates through a sequence.  Iterators are similar to lists in some
ways, but unlike lists, you can't use an index to select an element from
an iterator.
\index{iterator}

If you want to use list operators and methods, you can
use a zip object to make a list:

\begin{verbatim}
>>> list(zip(s, t))
[('a', 0), ('b', 1), ('c', 2)]
\end{verbatim}
%
The result is a list of tuples; in this example, each tuple contains
a character from the string and the corresponding element from
the list.
\index{list!of tuples}

If the sequences are not the same length, the result has the
length of the shorter one.

\begin{verbatim}
>>> list(zip('Anne', 'Elk'))
[('A', 'E'), ('n', 'l'), ('n', 'k')]
\end{verbatim}
%
You can use tuple assignment in a {\tt for} loop to traverse a list of
tuples:
\index{traversal}
\index{tuple assignment}
\index{assignment!tuple}

\begin{verbatim}
t = [('a', 0), ('b', 1), ('c', 2)]
for letter, number in t:
    print(number, letter)
\end{verbatim}
%
Each time through the loop, Python selects the next tuple in
the list and assigns the elements to {\tt letter} and 
{\tt number}.  The output of this loop is:
\index{loop}

\begin{verbatim}
0 a
1 b
2 c
\end{verbatim}
%
If you combine {\tt zip}, {\tt for} and tuple assignment, you get a
useful idiom for traversing two (or more) sequences at the same
time.  For example, \verb"has_match" takes two sequences, {\tt t1} and
{\tt t2}, and returns {\tt True} if there is an index {\tt i}
such that {\tt t1[i] == t2[i]}:
\index{for loop}

\begin{verbatim}
def has_match(t1, t2):
    for x, y in zip(t1, t2):
        if x == y:
            return True
    return False
\end{verbatim}
%
If you need to traverse the elements of a sequence and their
indices, you can use the built-in function {\tt enumerate}:
\index{traversal}
\index{enumerate function}
\index{function!enumerate}

\begin{verbatim}
for index, element in enumerate('abc'):
    print(index, element)
\end{verbatim}
%
The result from {\tt enumerate} is an enumerate object, which
iterates a sequence of pairs; each pair contains an index (starting
from 0) and an element from the given sequence.
In this example, the output is

\begin{verbatim}
0 a
1 b
2 c
\end{verbatim}
%
Again.
\index{iterator}
\index{object!enumerate}
\index{enumerate object}


\section{Dictionaries and tuples}
\label{dictuple}
\index{dictionary}
\index{items method}
\index{method!items}
\index{key-value pair}

Dictionaries have a method called {\tt items} that returns a sequence of
tuples, where each tuple is a key-value pair.

\begin{verbatim}
>>> d = {'a':0, 'b':1, 'c':2}
>>> t = d.items()
>>> t
dict_items([('c', 2), ('a', 0), ('b', 1)])
\end{verbatim}
%
The result is a \verb"dict_items" object, which is an iterator that
iterates the key-value pairs.  You can use it in a {\tt for} loop
like this:
\index{iterator}

\begin{verbatim}
>>> for key, value in d.items():
...     print(key, value)
...
c 2
a 0
b 1
\end{verbatim}
%
As you should expect from a dictionary, the items are in no
particular order.

Going in the other direction, you can use a list of tuples to
initialize a new dictionary: \index{dictionary!initialize}

\begin{verbatim}
>>> t = [('a', 0), ('c', 2), ('b', 1)]
>>> d = dict(t)
>>> d
{'a': 0, 'c': 2, 'b': 1}
\end{verbatim}

Combining {\tt dict} with {\tt zip} yields a concise way
to create a dictionary:
\index{zip function!use with dict}

\begin{verbatim}
>>> d = dict(zip('abc', range(3)))
>>> d
{'a': 0, 'c': 2, 'b': 1}
\end{verbatim}
%
The dictionary method {\tt update} also takes a list of tuples
and adds them, as key-value pairs, to an existing dictionary.
\index{update method}
\index{method!update}
\index{traverse!dictionary}
\index{dictionary!traversal}

It is common to use tuples as keys in dictionaries (primarily because
you can't use lists).  For example, a telephone directory might map
from last-name, first-name pairs to telephone numbers.  Assuming
that we have defined {\tt last}, {\tt first} and {\tt number}, we
could write:
\index{tuple!as key in dictionary}
\index{hashable}

\begin{verbatim}
directory[last, first] = number
\end{verbatim}
%
The expression in brackets is a tuple.  We could use tuple
assignment to traverse this dictionary.
\index{tuple!in brackets}

\begin{verbatim}
for last, first in directory:
    print(first, last, directory[last,first])
\end{verbatim}
%
This loop traverses the keys in {\tt directory}, which are tuples.  It
assigns the elements of each tuple to {\tt last} and {\tt first}, then
prints the name and corresponding telephone number.

There are two ways to represent tuples in a state diagram.  The more
detailed version shows the indices and elements just as they appear in
a list.  For example, the tuple \verb"('Cleese', 'John')" would appear
as in Figure~\ref{fig.tuple1}.
\index{state diagram}
\index{diagram!state}

\begin{figure}
\centerline
{\includegraphics[scale=0.8]{figs/tuple1.pdf}}
\caption{State diagram.}
\label{fig.tuple1}
\end{figure}

But in a larger diagram you might want to leave out the
details.  For example, a diagram of the telephone directory might
appear as in Figure~\ref{fig.dict2}.

\begin{figure}
\centerline
{\includegraphics[scale=0.8]{figs/dict2.pdf}}
\caption{State diagram.}
\label{fig.dict2}
\end{figure}

Here the tuples are shown using Python syntax as a graphical
shorthand.  The telephone number in the diagram is the complaints line
for the BBC, so please don't call it.


\section{Sequences of sequences}
\index{sequence}

I have focused on lists of tuples, but almost all of the examples in
this chapter also work with lists of lists, tuples of tuples, and
tuples of lists.  To avoid enumerating the possible combinations, it
is sometimes easier to talk about sequences of sequences.

In many contexts, the different kinds of sequences (strings, lists and
tuples) can be used interchangeably.  So how should you choose one
over the others?
\index{string}
\index{list}
\index{tuple}
\index{mutability}
\index{immutability}

To start with the obvious, strings are more limited than other
sequences because the elements have to be characters.  They are
also immutable.  If you need the ability to change the characters
in a string (as opposed to creating a new string), you might
want to use a list of characters instead.

Lists are more common than tuples, mostly because they are mutable.
But there are a few cases where you might prefer tuples:

\begin{enumerate}

\item In some contexts, like a {\tt return} statement, it is
syntactically simpler to create a tuple than a list.

\item If you want to use a sequence as a dictionary key, you
have to use an immutable type like a tuple or string.

\item If you are passing a sequence as an argument to a function,
using tuples reduces the potential for unexpected behavior
due to aliasing.

\end{enumerate}

Because tuples are immutable, they don't provide methods like {\tt
  sort} and {\tt reverse}, which modify existing lists.  But Python
provides the built-in function {\tt sorted}, which takes any sequence
and returns a new list with the same elements in sorted order, and
{\tt reversed}, which takes a sequence and returns an iterator that
traverses the list in reverse order.
\index{sorted function}
\index{function!sorted} \index{reversed function}
\index{function!reversed}
\index{iterator}


\section{디버깅}
%Debugging
\index{debugging}
\index{data structure}
\index{shape error}
\index{error!shape}

Lists, dictionaries and tuples are examples of {\bf data
  structures}; in this chapter we are starting to see compound data
structures, like lists of tuples, or dictionaries that contain tuples
as keys and lists as values.  Compound data structures are useful, but
they are prone to what I call {\bf shape errors}; that is, errors
caused when a data structure has the wrong type, size, or structure.
For example, if you are expecting a list with one integer and I
give you a plain old integer (not in a list), it won't work.
\index{structshape module}
\index{module!structshape}

To help debug these kinds of errors, I have written a module
called {\tt structshape} that provides a function, also called
{\tt structshape}, that takes any kind of data structure as
an argument and returns a string that summarizes its shape.
You can download it from \url{http://thinkpython2.com/code/structshape.py}

Here's the result for a simple list:

\begin{verbatim}
>>> from structshape import structshape
>>> t = [1, 2, 3]
>>> structshape(t)
'list of 3 int'
\end{verbatim}
%
A fancier program might write ``list of 3 int{\em s}'', but it
was easier not to deal with plurals.  Here's a list of lists:

\begin{verbatim}
>>> t2 = [[1,2], [3,4], [5,6]]
>>> structshape(t2)
'list of 3 list of 2 int'
\end{verbatim}
%
If the elements of the list are not the same type,
{\tt structshape} groups them, in order, by type:

\begin{verbatim}
>>> t3 = [1, 2, 3, 4.0, '5', '6', [7], [8], 9]
>>> structshape(t3)
'list of (3 int, float, 2 str, 2 list of int, int)'
\end{verbatim}
%
Here's a list of tuples:

\begin{verbatim}
>>> s = 'abc'
>>> lt = list(zip(t, s))
>>> structshape(lt)
'list of 3 tuple of (int, str)'
\end{verbatim}
%
And here's a dictionary with 3 items that map integers to strings.

\begin{verbatim}
>>> d = dict(lt) 
>>> structshape(d)
'dict of 3 int->str'
\end{verbatim}
%
If you are having trouble keeping track of your data structures,
{\tt structshape} can help.


\section{용어 해설}
%Glossary

\begin{description}

\item[tuple:] An immutable sequence of elements.
\index{tuple}

\item[tuple assignment:] An assignment with a sequence on the
right side and a tuple of variables on the left.  The right
side is evaluated and then its elements are assigned to the
variables on the left.
\index{tuple assignment}
\index{assignment!tuple}

\item[gather:] The operation of assembling a variable-length
argument tuple.
\index{gather}

\item[scatter:] The operation of treating a sequence as a list of
arguments.
\index{scatter}

\item[zip object:] The result of calling a built-in function {\tt zip};
an object that iterates through a sequence of tuples.
\index{zip object}
\index{object!zip}

\item[iterator:] An object that can iterate through a sequence, but
which does not provide list operators and methods.
\index{iterator}

\item[data structure:] A collection of related values, often
organized in lists, dictionaries, tuples, etc.
\index{data structure}

\item[shape error:] An error caused because a value has the
wrong shape; that is, the wrong type or size.
\index{shape}

\end{description}


\section{연습 문제}
%Exercises

\begin{exercise}

Write a function called \verb"most_frequent" that takes a string and
prints the letters in decreasing order of frequency.  Find text
samples from several different languages and see how letter frequency
varies between languages.  Compare your results with the tables at
\url{http://en.wikipedia.org/wiki/Letter_frequencies}.  Solution:
\url{http://thinkpython2.com/code/most_frequent.py}.  \index{letter
  frequency} \index{frequency!letter}

\end{exercise}


\begin{exercise}
\label{anagrams}
\index{anagram set}
\index{set!anagram}

More anagrams!

\begin{enumerate}

\item Write a program
that reads a word list from a file (see Section~\ref{wordlist}) and
prints all the sets of words that are anagrams.

Here is an example of what the output might look like:

\begin{verbatim}
['deltas', 'desalt', 'lasted', 'salted', 'slated', 'staled']
['retainers', 'ternaries']
['generating', 'greatening']
['resmelts', 'smelters', 'termless']
\end{verbatim}
%
Hint: you might want to build a dictionary that maps from a
collection of letters to a list of words that can be spelled with those
letters.  The question is, how can you represent the collection of
letters in a way that can be used as a key?

\item Modify the previous program so that it prints the longest list
of anagrams first, followed by the second longest, and so on.
\index{Scrabble}
\index{bingo}

\item In Scrabble a ``bingo'' is when you play all seven tiles in
your rack, along with a letter on the board, to form an eight-letter
word.  What collection of 8 letters forms the most possible bingos?
Hint: there are seven.

% (7, ['angriest', 'astringe', 'ganister', 'gantries', 'granites',
% 'ingrates', 'rangiest'])

Solution: \url{http://thinkpython2.com/code/anagram_sets.py}.

\end{enumerate}
\end{exercise}

\begin{exercise}
\index{metathesis}

Two words form a ``metathesis pair'' if you can transform one into the
other by swapping two letters; for example, ``converse'' and
``conserve''.  Write a program that finds all of the metathesis pairs
in the dictionary.  Hint: don't test all pairs of words, and don't
test all possible swaps.  Solution:
\url{http://thinkpython2.com/code/metathesis.py}.  Credit: This
exercise is inspired by an example at \url{http://puzzlers.org}.

\end{exercise}


\begin{exercise}
\index{Car Talk}
\index{Puzzler}

Here's another Car Talk Puzzler
(\url{http://www.cartalk.com/content/puzzlers}):

\begin{quote}
What is the longest English word, that remains a valid English word,
as you remove its letters one at a time?

Now, letters can be removed from either end, or the middle, but you
can't rearrange any of the letters. Every time you drop a letter, you
wind up with another English word. If you do that, you're eventually
going to wind up with one letter and that too is going to be an
English word---one that's found in the dictionary. I want to know
what's the longest word and how many letters does it
have?

I'm going to give you a little modest example: Sprite. Ok? You start
off with sprite, you take a letter off, one from the interior of the
word, take the r away, and we're left with the word spite, then we
take the e off the end, we're left with spit, we take the s off, we're
left with pit, it, and I.
\end{quote}
\index{reducible word}
\index{word, reducible}

Write a program to find all words that can be reduced in this way,
and then find the longest one.

This exercise is a little more challenging than most, so here are
some suggestions:

\begin{enumerate}

\item You might want to write a function that takes a word and
  computes a list of all the words that can be formed by removing one
  letter.  These are the ``children'' of the word.
\index{recursive definition}
\index{definition!recursive}

\item Recursively, a word is reducible if any of its children
are reducible.  As a base case, you can consider the empty
string reducible.

\item The wordlist I provided, {\tt words.txt}, doesn't
contain single letter words.  So you might want to add
``I'', ``a'', and the empty string.

\item To improve the performance of your program, you might want
to memoize the words that are known to be reducible.

\end{enumerate}

Solution: \url{http://thinkpython2.com/code/reducible.py}.

\end{exercise}




%\begin{exercise}
%\url{http://en.wikipedia.org/wiki/Word_Ladder}
%\end{exercise}




\chapter{Case study: data structure selection}

At this point you have learned about Python's core data structures,
and you have seen some of the algorithms that use them.
If you would like to know more about algorithms, this might be a good
time to read Chapter~\ref{algorithms}.
But you don't have to read it before you go on; you can read
it whenever you are interested.

This chapter presents a case study with exercises that let
you think about choosing data structures and practice using them.


\section{Word frequency analysis}
\label{analysis}

As usual, you should at least attempt the exercises
before you read my solutions.

\begin{exercise}

Write a program that reads a file, breaks each line into
words, strips whitespace and punctuation from the words, and
converts them to lowercase.
\index{string module}
\index{module!string}

Hint: The {\tt string} module provides a string named {\tt whitespace},
which contains space, tab, newline, etc., and {\tt
  punctuation} which contains the punctuation characters.  Let's see
if we can make Python swear:

\begin{verbatim}
>>> import string
>>> string.punctuation
'!"#$%&'()*+,-./:;<=>?@[\]^_`{|}~'
\end{verbatim}
%
Also, you might consider using the string methods {\tt strip},
{\tt replace} and {\tt translate}.
\index{strip method}
\index{method!strip}
\index{replace method}
\index{method!replace}
\index{translate method}
\index{method!translate}

\end{exercise}


\begin{exercise}
\index{Project Gutenberg}

Go to Project Gutenberg (\url{http://gutenberg.org}) and download 
your favorite out-of-copyright book in plain text format.
\index{plain text}
\index{text!plain}

Modify your program from the previous exercise to read the book
you downloaded, skip over the header information at the beginning
of the file, and process the rest of the words as before.

Then modify the program to count the total number of words in
the book, and the number of times each word is used.
\index{word frequency}
\index{frequency!word}

Print the number of different words used in the book.  Compare
different books by different authors, written in different eras.
Which author uses the most extensive vocabulary?
\end{exercise}


\begin{exercise}

Modify the program from the previous exercise to print the
20 most frequently used words in the book.

\end{exercise}


\begin{exercise}

Modify the previous program to read a word list (see
Section~\ref{wordlist}) and then print all the words in the book that
are not in the word list.  How many of them are typos?  How many of
them are common words that {\em should} be in the word list, and how
many of them are really obscure?

\end{exercise}


\section{Random numbers}
\index{random number}
\index{number, random}
\index{deterministic}
\index{pseudorandom}

Given the same inputs, most computer programs generate the same
outputs every time, so they are said to be {\bf deterministic}.
Determinism is usually a good thing, since we expect the same
calculation to yield the same result.  For some applications, though,
we want the computer to be unpredictable.  Games are an obvious
example, but there are more.

Making a program truly nondeterministic turns out to be difficult,
but there are ways to make it at least seem nondeterministic.  One of
them is to use algorithms that generate {\bf pseudorandom} numbers.
Pseudorandom numbers are not truly random because they are generated
by a deterministic computation, but just by looking at the numbers it
is all but impossible to distinguish them from random.
\index{random module}
\index{module!random}

The {\tt random} module provides functions that generate
pseudorandom numbers (which I will simply call ``random'' from
here on).
\index{random function}
\index{function!random}

The function {\tt random} returns a random float
between 0.0 and 1.0 (including 0.0 but not 1.0).  Each time you
call {\tt random}, you get the next number in a long series.  To see a
sample, run this loop:

\begin{verbatim}
import random

for i in range(10):
    x = random.random()
    print(x)
\end{verbatim}
%
The function {\tt randint} takes parameters {\tt low} and
{\tt high} and returns an integer between {\tt low} and
{\tt high} (including both).
\index{randint function}
\index{function!randint}

\begin{verbatim}
>>> random.randint(5, 10)
5
>>> random.randint(5, 10)
9
\end{verbatim}
%
To choose an element from a sequence at random, you can use
{\tt choice}:
\index{choice function}
\index{function!choice}

\begin{verbatim}
>>> t = [1, 2, 3]
>>> random.choice(t)
2
>>> random.choice(t)
3
\end{verbatim}
%
The {\tt random} module also provides functions to generate
random values from continuous distributions including
Gaussian, exponential, gamma, and a few more.

\begin{exercise}
\index{histogram!random choice}

Write a function named \verb"choose_from_hist" that takes
a histogram as defined in Section~\ref{histogram} and returns a 
random value from the histogram, chosen with probability
in proportion to frequency.  For example, for this histogram:

\begin{verbatim}
>>> t = ['a', 'a', 'b']
>>> hist = histogram(t)
>>> hist
{'a': 2, 'b': 1}
\end{verbatim}
%
your function should return \verb"'a'" with probability $2/3$ and \verb"'b'"
with probability $1/3$.
\end{exercise}


\section{Word histogram}

You should attempt the previous exercises before you go on.
You can download my solution from
 \url{http://thinkpython2.com/code/analyze_book1.py}.  You will
also need \url{http://thinkpython2.com/code/emma.txt}.

Here is a program that reads a file and builds a histogram of the
words in the file:
\index{histogram!word frequencies}

\begin{verbatim}
import string

def process_file(filename):
    hist = dict()
    fp = open(filename)
    for line in fp:
        process_line(line, hist)
    return hist

def process_line(line, hist):
    line = line.replace('-', ' ')
    
    for word in line.split():
        word = word.strip(string.punctuation + string.whitespace)
        word = word.lower()
        hist[word] = hist.get(word, 0) + 1

hist = process_file('emma.txt')
\end{verbatim}
%
This program reads {\tt emma.txt}, which contains the text of {\em
  Emma} by Jane Austen.
\index{Austin, Jane}

\verb"process_file" loops through the lines of the file,
passing them one at a time to \verb"process_line".  The histogram
{\tt hist} is being used as an accumulator.
\index{accumulator!histogram}
\index{traversal}

\verb"process_line" uses the string method {\tt replace} to replace
hyphens with spaces before using {\tt split} to break the line into a
list of strings.  It traverses the list of words and uses {\tt strip}
and {\tt lower} to remove punctuation and convert to lower case.  (It
is a shorthand to say that strings are ``converted''; remember that
strings are immutable, so methods like {\tt strip} and {\tt lower}
return new strings.)

Finally, \verb"process_line" updates the histogram by creating a new
item or incrementing an existing one.
\index{update!histogram}

To count the total number of words in the file, we can add up
the frequencies in the histogram:

\begin{verbatim}
def total_words(hist):
    return sum(hist.values())
\end{verbatim}
%
The number of different words is just the number of items in
the dictionary:

\begin{verbatim}
def different_words(hist):
    return len(hist)
\end{verbatim}
%
Here is some code to print the results:

\begin{verbatim}
print('Total number of words:', total_words(hist))
print('Number of different words:', different_words(hist))
\end{verbatim}
%
And the results:

\begin{verbatim}
Total number of words: 161080
Number of different words: 7214
\end{verbatim}
%

\section{Most common words}

To find the most common words, we can make a list of tuples,
where each tuple contains a word and its frequency,
and sort it.

The following function takes a histogram and returns a list of
word-frequency tuples:

\begin{verbatim}
def most_common(hist):
    t = []
    for key, value in hist.items():
        t.append((value, key))

    t.sort(reverse=True)
    return t
\end{verbatim}

In each tuple, the frequency appears first, so the resulting list is
sorted by frequency.  Here is a loop that prints the ten most common
words:

\begin{verbatim}
t = most_common(hist)
print('The most common words are:')
for freq, word in t[:10]:
    print(word, freq, sep='\t')
\end{verbatim}
%
I use the keyword argument {\tt sep} to tell {\tt print} to use a tab
character as a ``separator'', rather than a space, so the second
column is lined up.  Here are the results from {\em Emma}:

\begin{verbatim}
The most common words are:
to      5242
the     5205
and     4897
of      4295
i       3191
a       3130
it      2529
her     2483
was     2400
she     2364
\end{verbatim}
%
This code can be simplified using the {\tt key} parameter of
the {\tt sort} function.  If you are curious, you can read about it
at \url{https://wiki.python.org/moin/HowTo/Sorting}.


\section{Optional parameters}
\index{optional parameter}
\index{parameter!optional}

We have seen built-in functions and methods that take optional
arguments.  It is possible to write programmer-defined functions
with optional arguments, too.  For example, here is a function that
prints the most common words in a histogram
\index{programmer-defined function}
\index{function!programmer defined}

\begin{verbatim}
def print_most_common(hist, num=10):
    t = most_common(hist)
    print('The most common words are:')
    for freq, word in t[:num]:
        print(word, freq, sep='\t')
\end{verbatim}

The first parameter is required; the second is optional.
The {\bf default value} of {\tt num} is 10.
\index{default value}
\index{value!default}

If you only provide one argument:

\begin{verbatim}
print_most_common(hist)
\end{verbatim}

{\tt num} gets the default value.  If you provide two arguments:

\begin{verbatim}
print_most_common(hist, 20)
\end{verbatim}

{\tt num} gets the value of the argument instead.  In other
words, the optional argument {\bf overrides} the default value.
\index{override}

If a function has both required and optional parameters, all
the required parameters have to come first, followed by the
optional ones.


\section{Dictionary subtraction}
\label{dictsub}
\index{dictionary!subtraction}
\index{subtraction!dictionary}

Finding the words from the book that are not in the word list
from {\tt words.txt} is a problem you might recognize as set
subtraction; that is, we want to find all the words from one
set (the words in the book) that are not in the other (the
words in the list).

{\tt subtract} takes dictionaries {\tt d1} and {\tt d2} and returns a
new dictionary that contains all the keys from {\tt d1} that are not
in {\tt d2}.  Since we don't really care about the values, we
set them all to None.

\begin{verbatim}
def subtract(d1, d2):
    res = dict()
    for key in d1:
        if key not in d2:
            res[key] = None
    return res
\end{verbatim}
%
To find the words in the book that are not in {\tt words.txt},
we can use \verb"process_file" to build a histogram for
{\tt words.txt}, and then subtract:

\begin{verbatim}
words = process_file('words.txt')
diff = subtract(hist, words)

print("Words in the book that aren't in the word list:")
for word in diff:
    print(word, end=' ')
\end{verbatim}
%
Here are some of the results from {\em Emma}:

\begin{verbatim}
Words in the book that aren't in the word list:
rencontre jane's blanche woodhouses disingenuousness 
friend's venice apartment ...
\end{verbatim}
%
Some of these words are names and possessives.  Others, like
``rencontre'', are no longer in common use.  But a few are common
words that should really be in the list!

\begin{exercise}
\index{set}
\index{type!set}

Python provides a data structure called {\tt set} that provides many
common set operations.  You can read about them in Section~\ref{sets},
or read the documentation at
\url{http://docs.python.org/3/library/stdtypes.html#types-set}.

Write a program that uses set subtraction to find words in the book
that are not in the word list.  Solution:
\url{http://thinkpython2.com/code/analyze_book2.py}.

\end{exercise}


\section{Random words}
\label{randomwords}
\index{histogram!random choice}

To choose a random word from the histogram, the simplest algorithm
is to build a list with multiple copies of each word, according
to the observed frequency, and then choose from the list:

\begin{verbatim}
def random_word(h):
    t = []
    for word, freq in h.items():
        t.extend([word] * freq)

    return random.choice(t)
\end{verbatim}
%
The expression {\tt [word] * freq} creates a list with {\tt freq}
copies of the string {\tt word}.  The {\tt extend}
method is similar to {\tt append} except that the argument is
a sequence.

This algorithm works, but it is not very efficient; each time you
choose a random word, it rebuilds the list, which is as big as
the original book.  An obvious improvement is to build the list
once and then make multiple selections, but the list is still big.

An alternative is:

\begin{enumerate}

\item Use {\tt keys} to get a list of the words in the book.

\item Build a list that contains the cumulative sum of the word
  frequencies (see Exercise~\ref{cumulative}).  The last item
  in this list is the total number of words in the book, $n$.
  
\item Choose a random number from 1 to $n$.  Use a bisection search
  (See Exercise~\ref{bisection}) to find the index where the random
  number would be inserted in the cumulative sum.

\item Use the index to find the corresponding word in the word list.

\end{enumerate}

\begin{exercise}
\label{randhist}
\index{algorithm}

Write a program that uses this algorithm to choose a random word from
the book.  Solution:
\url{http://thinkpython2.com/code/analyze_book3.py}.

\end{exercise}



\section{Markov analysis}
\label{markov}
\index{Markov analysis}

If you choose words from the book at random, you can get a
sense of the vocabulary, but you probably won't get a sentence:

\begin{verbatim}
this the small regard harriet which knightley's it most things
\end{verbatim}
%
A series of random words seldom makes sense because there
is no relationship between successive words.  For example, in
a real sentence you would expect an article like ``the'' to
be followed by an adjective or a noun, and probably not a verb
or adverb.

One way to measure these kinds of relationships is Markov
analysis, which
characterizes, for a given sequence of words, the probability of the
words that might come next.  For example, the song {\em Eric, the Half a
  Bee} begins:

\begin{quote}
Half a bee, philosophically, \\
Must, ipso facto, half not be. \\
But half the bee has got to be \\
Vis a vis, its entity. D'you see? \\
\\
But can a bee be said to be \\
Or not to be an entire bee \\
When half the bee is not a bee \\
Due to some ancient injury? \\
\end{quote}
%
In this text,
the phrase ``half the'' is always followed by the word ``bee'',
but the phrase ``the bee'' might be followed by either
``has'' or ``is''.
\index{prefix}
\index{suffix}
\index{mapping}

The result of Markov analysis is a mapping from each prefix
(like ``half the'' and ``the bee'') to all possible suffixes
(like ``has'' and ``is'').
\index{random text}
\index{text!random}

Given this mapping, you can generate a random text by
starting with any prefix and choosing at random from the
possible suffixes.  Next, you can combine the end of the
prefix and the new suffix to form the next prefix, and repeat.

For example, if you start with the prefix ``Half a'', then the
next word has to be ``bee'', because the prefix only appears
once in the text.  The next prefix is ``a bee'', so the
next suffix might be ``philosophically'', ``be'' or ``due''.

In this example the length of the prefix is always two, but
you can do Markov analysis with any prefix length.

\begin{exercise}

Markov analysis:

\begin{enumerate}

\item Write a program to read a text from a file and perform Markov
analysis.  The result should be a dictionary that maps from
prefixes to a collection of possible suffixes.  The collection
might be a list, tuple, or dictionary; it is up to you to make
an appropriate choice.  You can test your program with prefix
length two, but you should write the program in a way that makes
it easy to try other lengths.

\item Add a function to the previous program to generate random text
based on the Markov analysis.  Here is an example from {\em Emma}
with prefix length 2:

\begin{quote}
He was very clever, be it sweetness or be angry, ashamed or only
amused, at such a stroke. She had never thought of Hannah till you
were never meant for me?" "I cannot make speeches, Emma:" he soon cut
it all himself.
\end{quote}

For this example, I left the punctuation attached to the words.
The result is almost syntactically correct, but not quite.
Semantically, it almost makes sense, but not quite.

What happens if you increase the prefix length?  Does the random
text make more sense?

\item Once your program is working, you might want to try a mash-up:
if you combine text from two or more books, the random
text you generate will blend the vocabulary and phrases from
the sources in interesting ways.
\index{mash-up}

\end{enumerate}

Credit: This case study is based on an example from Kernighan and
Pike, {\em The Practice of Programming}, Addison-Wesley, 1999.

\end{exercise}

You should attempt this exercise before you go on; then you can can
download my solution from \url{http://thinkpython2.com/code/markov.py}.
You will also need \url{http://thinkpython2.com/code/emma.txt}.


\section{Data structures}
\index{data structure}

Using Markov analysis to generate random text is fun, but there is
also a point to this exercise: data structure selection.  In your
solution to the previous exercises, you had to choose:

\begin{itemize}

\item How to represent the prefixes.

\item How to represent the collection of possible suffixes.

\item How to represent the mapping from each prefix to
the collection of possible suffixes.

\end{itemize}

The last one is easy: a dictionary is the obvious choice
for a mapping from keys to corresponding values.

For the prefixes, the most obvious options are string,
list of strings, or tuple of strings.

For the suffixes,
one option is a list; another is a histogram (dictionary).
\index{implementation}

How should you choose?  The first step is to think about
the operations you will need to implement for each data structure.
For the prefixes, we need to be able to remove words from
the beginning and add to the end.  For example, if the current
prefix is ``Half a'', and the next word is ``bee'', you need
to be able to form the next prefix, ``a bee''.
\index{tuple!as key in dictionary}

Your first choice might be a list, since it is easy to add
and remove elements, but we also need to be able to use the
prefixes as keys in a dictionary, so that rules out lists.
With tuples, you can't append or remove, but you can use
the addition operator to form a new tuple:

\begin{verbatim}
def shift(prefix, word):
    return prefix[1:] + (word,)
\end{verbatim}
%
{\tt shift} takes a tuple of words, {\tt prefix}, and a string, 
{\tt word}, and forms a new tuple that has all the words
in {\tt prefix} except the first, and {\tt word} added to
the end.

For the collection of suffixes, the operations we need to
perform include adding a new suffix (or increasing the frequency
of an existing one), and choosing a random suffix.

Adding a new suffix is equally easy for the list implementation
or the histogram.  Choosing a random element from a list
is easy; choosing from a histogram is harder to do
efficiently (see Exercise~\ref{randhist}).

So far we have been talking mostly about ease of implementation,
but there are other factors to consider in choosing data structures.
One is run time.  Sometimes there is a theoretical reason to expect
one data structure to be faster than other; for example, I mentioned
that the {\tt in} operator is faster for dictionaries than for lists,
at least when the number of elements is large.

But often you don't know ahead of time which implementation will
be faster.  One option is to implement both of them and see which
is better.  This approach is called {\bf benchmarking}.  A practical
alternative is to choose the data structure that is
easiest to implement, and then see if it is fast enough for the
intended application.  If so, there is no need to go on.  If not,
there are tools, like the {\tt profile} module, that can identify
the places in a program that take the most time.
\index{benchmarking}
\index{profile module}
\index{module!profile}

The other factor to consider is storage space.  For example, using a
histogram for the collection of suffixes might take less space because
you only have to store each word once, no matter how many times it
appears in the text.  In some cases, saving space can also make your
program run faster, and in the extreme, your program might not run at
all if you run out of memory.  But for many applications, space is a
secondary consideration after run time.

One final thought: in this discussion, I have implied that
we should use one data structure for both analysis and generation.  But
since these are separate phases, it would also be possible to use one
structure for analysis and then convert to another structure for
generation.  This would be a net win if the time saved during
generation exceeded the time spent in conversion.


\section{디버깅}
%Debugging
\index{debugging}

When you are debugging a program, and especially if you are
working on a hard bug, there are five things to try:

\begin{description}

\item[Reading:] Examine your code, read it back to yourself, and
check that it says what you meant to say.

\item[Running:] Experiment by making changes and running different
versions.  Often if you display the right thing at the right place
in the program, the problem becomes obvious, but sometimes you have to
build scaffolding.

\item[Ruminating:] Take some time to think!  What kind of error
is it: syntax, runtime, or semantic?  What information can you get from
the error messages, or from the output of the program?  What kind of
error could cause the problem you're seeing?  What did you change
last, before the problem appeared?

\item[Rubberducking:] If you explain the problem to someone else, you
  sometimes find the answer before you finish asking the question.
  Often you don't need the other person; you could just talk to a rubber
  duck.  And that's the origin of the well-known strategy called {\bf
    rubber duck debugging}.  I am not making this up; see
  \url{https://en.wikipedia.org/wiki/Rubber_duck_debugging}.

\item[Retreating:] At some point, the best thing to do is back
off, undoing recent changes, until you get back to a program that
works and that you understand.  Then you can start rebuilding.

\end{description}

Beginning programmers sometimes get stuck on one of these activities
and forget the others.  Each activity comes with its own failure
mode.
\index{typographical error}

For example, reading your code might help if the problem is a
typographical error, but not if the problem is a conceptual
misunderstanding.  If you don't understand what your program does, you
can read it 100 times and never see the error, because the error is in
your head.
\index{experimental debugging}

Running experiments can help, especially if you run small, simple
tests.  But if you run experiments without thinking or reading your
code, you might fall into a pattern I call ``random walk programming'',
which is the process of making random changes until the program
does the right thing.  Needless to say, random walk programming
can take a long time.
\index{random walk programming}
\index{development plan!random walk programming}

You have to take time to think.  Debugging is like an
experimental science.  You should have at least one hypothesis about
what the problem is.  If there are two or more possibilities, try to
think of a test that would eliminate one of them.

But even the best debugging techniques will fail if there are too many
errors, or if the code you are trying to fix is too big and
complicated.  Sometimes the best option is to retreat, simplifying the
program until you get to something that works and that you
understand.

Beginning programmers are often reluctant to retreat because
they can't stand to delete a line of code (even if it's wrong).
If it makes you feel better, copy your program into another file
before you start stripping it down.  Then you can copy the pieces
back one at a time.

Finding a hard bug requires reading, running, ruminating, and
sometimes retreating.  If you get stuck on one of these activities,
try the others.


\section{용어 해설}
%Glossary

\begin{description}

\item[deterministic:] Pertaining to a program that does the same
thing each time it runs, given the same inputs.
\index{deterministic}

\item[pseudorandom:] Pertaining to a sequence of numbers that appears
to be random, but is generated by a deterministic program.
\index{pseudorandom}

\item[default value:] The value given to an optional parameter if no
argument is provided.
\index{default value}

\item[override:] To replace a default value with an argument.
\index{override}

\item[benchmarking:] The process of choosing between data structures
by implementing alternatives and testing them on a sample of the
possible inputs.  
\index{benchmarking}

\item[rubber duck debugging:] Debugging by explaining your problem
to an inanimate object such as a rubber duck.  Articulating the
problem can help you solve it, even if the rubber duck doesn't know
Python. 
\index{rubber duck debugging}
\index{debugging!rubber duck}

\end{description}


\section{연습 문제}
%Exercises

\begin{exercise}
\index{word frequency}
\index{frequency!word}
\index{Zipf's law}

The ``rank'' of a word is its position in a list of words
sorted by frequency: the most common word has rank 1, the
second most common has rank 2, etc.

Zipf's law describes a relationship between the ranks and frequencies
of words in natural languages
(\url{http://en.wikipedia.org/wiki/Zipf's_law}).  Specifically, it
predicts that the frequency, $f$, of the word with rank $r$ is:

\[ f = c r^{-s} \]
%
where $s$ and $c$ are parameters that depend on the language and the
text.  If you take the logarithm of both sides of this equation, you
get:
\index{logarithm}

\[ \log f = \log c - s \log r \]
%
So if you plot log $f$ versus log $r$, you should get
a straight line with slope $-s$ and intercept log $c$.

Write a program that reads a text from a file, counts
word frequencies, and prints one line
for each word, in descending order of frequency, with
log $f$ and log $r$.  Use the graphing program of your
choice to plot the results and check whether they form
a straight line.  Can you estimate the value of $s$?

Solution: \url{http://thinkpython2.com/code/zipf.py}.
To run my solution, you need the plotting module {\tt matplotlib}.
If you installed Anaconda, you already have {\tt matplotlib};
otherwise you might have to install it.
\index{matplotlib}

\end{exercise}



\chapter{Files}

This chapter introduces the idea of ``persistent'' programs that
keep data in permanent storage, and shows how to use different
kinds of permanent storage, like files and databases.


\section{Persistence}
\index{file}
\index{type!file}
\index{persistence}

Most of the programs we have seen so far are transient in the
sense that they run for a short time and produce some output,
but when they end, their data disappears.  If you run the program
again, it starts with a clean slate.

Other programs are {\bf persistent}: they run for a long time
(or all the time); they keep at least some of their data
in permanent storage (a hard drive, for example); and
if they shut down and restart, they pick up where they left off.

Examples of persistent programs are operating systems, which
run pretty much whenever a computer is on, and web servers,
which run all the time, waiting for requests to come in on
the network.

One of the simplest ways for programs to maintain their data
is by reading and writing text files.  We have already seen
programs that read text files; in this chapter we will see programs
that write them.

An alternative is to store the state of the program in a database.
In this chapter I will present a simple database and a module,
{\tt pickle}, that makes it easy to store program data.
\index{pickle module}
\index{module!pickle}


\section{Reading and writing}
\index{file!reading and writing}

A text file is a sequence of characters stored on a permanent
medium like a hard drive, flash memory, or CD-ROM.  We saw how
to open and read a file in Section~\ref{wordlist}.
\index{open function}
\index{function!open}

To write a file, you have to open it with mode \verb"'w'" as a second
parameter:

\begin{verbatim}
>>> fout = open('output.txt', 'w')
\end{verbatim}
%
If the file already exists, opening it in write mode clears out
the old data and starts fresh, so be careful!
If the file doesn't exist, a new one is created.

{\tt open} returns a file object that provides methods for working
with the file.
The {\tt write} method puts data into the file.

\begin{verbatim}
>>> line1 = "This here's the wattle,\n"
>>> fout.write(line1)
24
\end{verbatim}
%
The return value is the number of characters that were written.
The file object keeps track of where it is, so if
you call {\tt write} again, it adds the new data to the end of
the file.

\begin{verbatim}
>>> line2 = "the emblem of our land.\n"
>>> fout.write(line2)
24
\end{verbatim}
%
When you are done writing, you should close the file.

\begin{verbatim}
>>> fout.close()
\end{verbatim}
%
\index{close method}
\index{method!close}
%
If you don't close the file, it gets closed for you when the
program ends.


\section{Format operator}
\index{format operator}
\index{operator!format}

The argument of {\tt write} has to be a string, so if we want
to put other values in a file, we have to convert them to
strings.  The easiest way to do that is with {\tt str}:

\begin{verbatim}
>>> x = 52
>>> fout.write(str(x))
\end{verbatim}
%
An alternative is to use the {\bf format operator}, {\tt \%}.  When
applied to integers, {\tt \%} is the modulus operator.  But
when the first operand is a string, {\tt \%} is the format operator.
\index{format string}

The first operand is the {\bf format string}, which contains
one or more {\bf format sequences}, which
specify how
the second operand is formatted.  The result is a string.
\index{format sequence}

For example, the format sequence \verb"'%d'" means that
the second operand should be formatted as a decimal
integer:

\begin{verbatim}
>>> camels = 42
>>> '%d' % camels
'42'
\end{verbatim}
%
The result is the string \verb"'42'", which is not to be confused
with the integer value {\tt 42}.

A format sequence can appear anywhere in the string,
so you can embed a value in a sentence:

\begin{verbatim}
>>> 'I have spotted %d camels.' % camels
'I have spotted 42 camels.'
\end{verbatim}
%
If there is more than one format sequence in the string,
the second argument has to be a tuple.  Each format sequence is
matched with an element of the tuple, in order.

The following example uses \verb"'%d'" to format an integer,
\verb"'%g'" to format a floating-point number, and
\verb"'%s'" to format a string:

\begin{verbatim}
>>> 'In %d years I have spotted %g %s.' % (3, 0.1, 'camels')
'In 3 years I have spotted 0.1 camels.'
\end{verbatim}
%
The number of elements in the tuple has to match the number
of format sequences in the string.  Also, the types of the
elements have to match the format sequences:
\index{exception!TypeError}
\index{TypeError}

\begin{verbatim}
>>> '%d %d %d' % (1, 2)
TypeError: not enough arguments for format string
>>> '%d' % 'dollars'
TypeError: %d format: a number is required, not str
\end{verbatim}
%
In the first example, there aren't enough elements; in the
second, the element is the wrong type.

For more information on the format operator, see
\url{https://docs.python.org/3/library/stdtypes.html#printf-style-string-formatting}.  A more powerful alternative is the string
format method, which you can read about at
\url{https://docs.python.org/3/library/stdtypes.html#str.format}.

% You can specify the number of digits as part of the format sequence.
% For example, the sequence \verb"'%8.2f'"
% formats a floating-point number to be 8 characters long, with
% 2 digits after the decimal point:

% % \begin{verbatim}
% >>> '%8.2f' % 3.14159
% '    3.14'
% \end{verbatim}
% \afterverb
% %
% The result takes up eight spaces with two
% digits after the decimal point.  


\section{Filenames and paths}
\label{paths}
\index{filename}
\index{path}
\index{directory}
\index{folder}

Files are organized into {\bf directories} (also called ``folders'').
Every running program has a ``current directory'', which is the
default directory for most operations.  
For example, when you open a file for reading, Python looks for it in the
current directory.
\index{os module}
\index{module!os}

The {\tt os} module provides functions for working with files and
directories (``os'' stands for ``operating system'').  {\tt os.getcwd}
returns the name of the current directory:
\index{getcwd function}
\index{function!getcwd}

\begin{verbatim}
>>> import os
>>> cwd = os.getcwd()
>>> cwd
'/home/dinsdale'
\end{verbatim}
%
{\tt cwd} stands for ``current working directory''.  The result in
this example is {\tt /home/dinsdale}, which is the home directory of a
user named {\tt dinsdale}.
\index{working directory}
\index{directory!working}

A string like \verb"'/home/dinsdale'" that identifies a file or
directory is called a {\bf path}.

A simple filename, like {\tt memo.txt} is also considered a path,
but it is a {\bf relative path} because it relates to the current
directory.  If the current directory is {\tt /home/dinsdale}, the
filename {\tt memo.txt} would refer to {\tt /home/dinsdale/memo.txt}.
\index{relative path} \index{path!relative}
\index{absolute path} \index{path!absolute}

A path that begins with {\tt /} does not depend on the current
directory; it is called an {\bf absolute path}.  To find the absolute
path to a file, you can use {\tt os.path.abspath}:

\begin{verbatim}
>>> os.path.abspath('memo.txt')
'/home/dinsdale/memo.txt'
\end{verbatim}
%
{\tt os.path} provides other functions for working with filenames
and paths.  For example,
{\tt os.path.exists} checks
whether a file or directory exists:
\index{exists function}
\index{function!exists}

\begin{verbatim}
>>> os.path.exists('memo.txt')
True
\end{verbatim}
%
If it exists, {\tt os.path.isdir} checks whether it's a directory:

\begin{verbatim}
>>> os.path.isdir('memo.txt')
False
>>> os.path.isdir('/home/dinsdale')
True
\end{verbatim}
%
Similarly, {\tt os.path.isfile} checks whether it's a file.

{\tt os.listdir} returns a list of the files (and other directories)
in the given directory:

\begin{verbatim}
>>> os.listdir(cwd)
['music', 'photos', 'memo.txt']
\end{verbatim}
%
To demonstrate these functions, the following example
``walks'' through a directory, prints
the names of all the files, and calls itself recursively on
all the directories.
\index{walk, directory}
\index{directory!walk}

\begin{verbatim}
def walk(dirname):
    for name in os.listdir(dirname):
        path = os.path.join(dirname, name)

        if os.path.isfile(path):
            print(path)
        else:
            walk(path)
\end{verbatim}
%
{\tt os.path.join} takes a directory and a file name and joins
them into a complete path.  

The {\tt os} module provides a function called {\tt walk} that is
similar to this one but more versatile.  As an exercise, read the
documentation and use it to print the names of the files in a given
directory and its subdirectories.  You can download my solution from
\url{http://thinkpython2.com/code/walk.py}.


\section{Catching exceptions}
\label{catch}

A lot of things can go wrong when you try to read and write
files.  If you try to open a file that doesn't exist, you get an
{\tt IOError}:
\index{open function}
\index{function!open}
\index{exception!IOError}
\index{IOError}

\begin{verbatim}
>>> fin = open('bad_file')
IOError: [Errno 2] No such file or directory: 'bad_file'
\end{verbatim}
%
If you don't have permission to access a file:
\index{file!permission}
\index{permission, file}

\begin{verbatim}
>>> fout = open('/etc/passwd', 'w')
PermissionError: [Errno 13] Permission denied: '/etc/passwd'
\end{verbatim}
%
And if you try to open a directory for reading, you get

\begin{verbatim}
>>> fin = open('/home')
IsADirectoryError: [Errno 21] Is a directory: '/home'
\end{verbatim}
%
To avoid these errors, you could use functions like {\tt os.path.exists}
and {\tt os.path.isfile}, but it would take a lot of time and code
to check all the possibilities (if ``{\tt Errno 21}'' is any
indication, there are at least 21 things that can go wrong).
\index{exception, catching}
\index{try statement}
\index{statement!try}

It is better to go ahead and try---and deal with problems if they
happen---which is exactly what the {\tt try} statement does.  The
syntax is similar to an {\tt if...else} statement:

\begin{verbatim}
try:    
    fin = open('bad_file')
except:
    print('Something went wrong.')
\end{verbatim}
%
Python starts by executing the {\tt try} clause.  If all goes
well, it skips the {\tt except} clause and proceeds.  If an
exception occurs, it jumps out of the {\tt try} clause and
runs the {\tt except} clause.

Handling an exception with a {\tt try} statement is called {\bf
catching} an exception.  In this example, the {\tt except} clause
prints an error message that is not very helpful.  In general,
catching an exception gives you a chance to fix the problem, or try
again, or at least end the program gracefully.


\section{Databases}
\index{database}

A {\bf database} is a file that is organized for storing data.  Many
databases are organized like a dictionary in the sense that they map
from keys to values.  The biggest difference between a database and a
dictionary is that the database is on disk (or other permanent
storage), so it persists after the program ends.  \index{dbm
  module} \index{module!dbm}

The module {\tt dbm} provides an interface for creating
and updating database files.
As an example, I'll create a database
that contains captions for image files.
\index{open function}
\index{function!open}

Opening a database is similar to opening other files:

\begin{verbatim}
>>> import dbm
>>> db = dbm.open('captions', 'c')
\end{verbatim}
%
The mode \verb"'c'" means that the database should be created if
it doesn't already exist.  The result is a database object
that can be used (for most operations) like a dictionary.
\index{database object}
\index{object!database}

When you create a new item, {\tt dbm} updates the database file.
\index{update!database}

\begin{verbatim}
>>> db['cleese.png'] = 'Photo of John Cleese.'
\end{verbatim}
%
When you access one of the items, {\tt dbm} reads the file:

\begin{verbatim}
>>> db['cleese.png']
b'Photo of John Cleese.'
\end{verbatim}
%
The result is a {\bf bytes object}, which is why it begins with {\tt
  b}.  A bytes object is similar to a string in many ways.  When you
get farther into Python, the difference becomes important, but for now
we can ignore it.
\index{bytes object}
\index{object!bytes}

If you make another assignment to an existing key, {\tt dbm} replaces
the old value:

\begin{verbatim}
>>> db['cleese.png'] = 'Photo of John Cleese doing a silly walk.'
>>> db['cleese.png']
b'Photo of John Cleese doing a silly walk.'
\end{verbatim}
%

Some dictionary methods, like {\tt keys} and {\tt items}, don't
work with database objects.  But iteration with a {\tt for}
loop works:
\index{dictionary methods!dbm module}

\begin{verbatim}
for key in db:
    print(key, db[key])
\end{verbatim}
%
As with other files, you should close the database when you are
done:

\begin{verbatim}
>>> db.close()
\end{verbatim}
%
\index{close method}
\index{method!close}


\section{Pickling}
\index{pickling}

A limitation of {\tt dbm} is that the keys and values have to be
strings or bytes.  If you try to use any other type, you get an error.
\index{pickle module} \index{module!pickle}

The {\tt pickle} module can help.  It translates
almost any type of object into a string suitable for storage in a
database, and then translates strings back into objects.

{\tt pickle.dumps} takes an object as a parameter and returns
a string representation ({\tt dumps} is short for ``dump string''):

\begin{verbatim}
>>> import pickle
>>> t = [1, 2, 3]
>>> pickle.dumps(t)
b'\x80\x03]q\x00(K\x01K\x02K\x03e.'
\end{verbatim}
%
The format isn't obvious to human readers; it is meant to be
easy for {\tt pickle} to interpret.  {\tt pickle.loads}
(``load string'') reconstitutes the object:

\begin{verbatim}
>>> t1 = [1, 2, 3]
>>> s = pickle.dumps(t1)
>>> t2 = pickle.loads(s)
>>> t2
[1, 2, 3]
\end{verbatim}
%
Although the new object has the same value as the old, it is
not (in general) the same object:

\begin{verbatim}
>>> t1 == t2
True
>>> t1 is t2
False
\end{verbatim}
%
In other words, pickling and then unpickling has the same effect
as copying the object.

You can use {\tt pickle} to store non-strings in a database.
In fact, this combination is so common that it has been
encapsulated in a module called {\tt shelve}.  
\index{shelve module}
\index{module!shelve}


\section{Pipes}
\index{shell}
\index{pipe}

Most operating systems provide a command-line interface,
also known as a {\bf shell}.  Shells usually provide commands
to navigate the file system and launch applications.  For
example, in Unix you can change directories with {\tt cd},
display the contents of a directory with {\tt ls}, and launch
a web browser by typing (for example) {\tt firefox}.
\index{ls (Unix command)}
\index{Unix command!ls}

Any program that you can launch from the shell can also be
launched from Python using a {\bf pipe object}, which
represents a running program.

For example, the Unix command {\tt ls -l} normally displays the
contents of the current directory in long format.  You can
launch {\tt ls} with {\tt os.popen}\footnote{{\tt popen} is deprecated
now, which means we are supposed to stop using it and start using
the {\tt subprocess} module.  But for simple cases, I find
{\tt subprocess} more complicated than necessary.  So I am going
to keep using {\tt popen} until they take it away.}:
\index{popen function}
\index{function!popen}

\begin{verbatim}
>>> cmd = 'ls -l'
>>> fp = os.popen(cmd)
\end{verbatim}
%
The argument is a string that contains a shell command.  The
return value is an object that behaves like an open
file.  You can read the output from the {\tt ls} process one
line at a time with {\tt readline} or get the whole thing at
once with {\tt read}:
\index{readline method}
\index{method!readline}
\index{read method}
\index{method!read}

\begin{verbatim}
>>> res = fp.read()
\end{verbatim}
%
When you are done, you close the pipe like a file:
\index{close method}
\index{method!close}

\begin{verbatim}
>>> stat = fp.close()
>>> print(stat)
None
\end{verbatim}
%
The return value is the final status of the {\tt ls} process;
{\tt None} means that it ended normally (with no errors).

For example, most Unix systems provide a command called {\tt md5sum}
that reads the contents of a file and computes a ``checksum''.
You can read about MD5 at \url{http://en.wikipedia.org/wiki/Md5}.  This
command provides an efficient way to check whether two files
have the same contents.  The probability that different contents
yield the same checksum is very small (that is, unlikely to happen
before the universe collapses).
\index{md5}
\index{checksum}

You can use a pipe to run {\tt md5sum} from Python and get the result:

\begin{verbatim}
>>> filename = 'book.tex'
>>> cmd = 'md5sum ' + filename
>>> fp = os.popen(cmd)
>>> res = fp.read()
>>> stat = fp.close()
>>> print(res)
1e0033f0ed0656636de0d75144ba32e0  book.tex
>>> print(stat)
None
\end{verbatim}



\section{Writing modules}
\label{modules}
\index{module, writing}
\index{word count}

Any file that contains Python code can be imported as a module.
For example, suppose you have a file named {\tt wc.py} with the following
code:

\begin{verbatim}
def linecount(filename):
    count = 0
    for line in open(filename):
        count += 1
    return count

print(linecount('wc.py'))
\end{verbatim}
%
If you run this program, it reads itself and prints the number
of lines in the file, which is 7.
You can also import it like this:

\begin{verbatim}
>>> import wc
7
\end{verbatim}
%
Now you have a module object {\tt wc}:
\index{module object}
\index{object!module}

\begin{verbatim}
>>> wc
<module 'wc' from 'wc.py'>
\end{verbatim}
%
The module object provides \verb"linecount":

\begin{verbatim}
>>> wc.linecount('wc.py')
7
\end{verbatim}
%
So that's how you write modules in Python.

The only problem with this example is that when you import
the module it runs the test code at the bottom.  Normally
when you import a module, it defines new functions but it
doesn't run them.
\index{import statement}
\index{statement!import}

Programs that will be imported as modules often
use the following idiom:

\begin{verbatim}
if __name__ == '__main__':
    print(linecount('wc.py'))
\end{verbatim}
%
\verb"__name__" is a built-in variable that is set when the
program starts.  If the program is running as a script,
\verb"__name__" has the value \verb"'__main__'"; in that
case, the test code runs.  Otherwise,
if the module is being imported, the test code is skipped.

\index{name built-in variable}
\index{main}

As an exercise, type this example into a file named {\tt wc.py} and run
it as a script.  Then run the Python interpreter and
{\tt import wc}.  What is the value of \verb"__name__"
when the module is being imported?

Warning: If you import a module that has already been imported,
Python does nothing.  It does not re-read the file, even if it has
changed.
\index{module!reload}
\index{reload function}
\index{function!reload}

If you want to reload a module, you can use the built-in function 
{\tt reload}, but it can be tricky, so the safest thing to do is
restart the interpreter and then import the module again.


\section{디버깅}
%Debugging
\index{debugging}
\index{whitespace}

When you are reading and writing files, you might run into problems
with whitespace.  These errors can be hard to debug because spaces,
tabs and newlines are normally invisible:

\begin{verbatim}
>>> s = '1 2\t 3\n 4'
>>> print(s)
1 2	 3
 4
\end{verbatim}
\index{repr function}
\index{function!repr}
\index{string representation}

The built-in function {\tt repr} can help.  It takes any object as an
argument and returns a string representation of the object.  For
strings, it represents whitespace
characters with backslash sequences:

\begin{verbatim}
>>> print(repr(s))
'1 2\t 3\n 4'
\end{verbatim}

This can be helpful for debugging.

One other problem you might run into is that different systems
use different characters to indicate the end of a line.  Some
systems use a newline, represented \verb"\n".  Others use
a return character, represented \verb"\r".  Some use both.
If you move files between different systems, these inconsistencies
can cause problems.
\index{end of line character}

For most systems, there are applications to convert from one
format to another.  You can find them (and read more about this
issue) at \url{http://en.wikipedia.org/wiki/Newline}.  Or, of course, you
could write one yourself.


\section{용어 해설}
%Glossary

\begin{description}

\item[persistent:] Pertaining to a program that runs indefinitely
and keeps at least some of its data in permanent storage.
\index{persistence}

\item[format operator:] An operator, {\tt \%}, that takes a format
string and a tuple and generates a string that includes
the elements of the tuple formatted as specified by the format string.
\index{format operator}
\index{operator!format}

\item[format string:] A string, used with the format operator, that
contains format sequences.  
\index{format string}

\item[format sequence:] A sequence of characters in a format string,
like {\tt \%d}, that specifies how a value should be formatted.
\index{format sequence}

\item[text file:] A sequence of characters stored in permanent
storage like a hard drive.
\index{text file}

\item[directory:] A named collection of files, also called a folder.
\index{directory}

\item[path:] A string that identifies a file.
\index{path}

\item[relative path:] A path that starts from the current directory.
\index{relative path}

\item[absolute path:] A path that starts from the topmost directory
in the file system.
\index{absolute path}

\item[catch:] To prevent an exception from terminating
a program using the {\tt try}
and {\tt except} statements.
\index{catch}

\item[database:] A file whose contents are organized like a dictionary
with keys that correspond to values.
\index{database}

\item[bytes object:] An object similar to a string.
\index{bytes object}
\index{object!bytes}

\item[shell:] A program that allows users to type commands and then
executes them by starting other programs.
\index{shell}

\item[pipe object:] An object that represents a running program, allowing
a Python program to run commands and read the results.
\index{pipe object}
\index{object!pipe}

\end{description}


\section{연습 문제}
%Exercises

\begin{exercise}

Write a function called {\tt sed} that takes as arguments a pattern string,
a replacement string, and two filenames; it should read the first file
and write the contents into the second file (creating it if
necessary).  If the pattern string appears anywhere in the file, it
should be replaced with the replacement string.

If an error occurs while opening, reading, writing or closing files,
your program should catch the exception, print an error message, and
exit.  Solution: \url{http://thinkpython2.com/code/sed.py}.

\end{exercise}


\begin{exercise}
\index{anagram set}
\index{set!anagram}

If you download my solution to Exercise~\ref{anagrams} from
\url{http://thinkpython2.com/code/anagram_sets.py}, you'll see that it creates
a dictionary that maps from a sorted string of letters to the list of
words that can be spelled with those letters.  For example,
\verb"'opst'" maps to the list
\verb"['opts', 'post', 'pots', 'spot', 'stop', 'tops']".

Write a module that imports \verb"anagram_sets" and provides
two new functions: \verb"store_anagrams" should store the
anagram dictionary in a ``shelf''; \verb"read_anagrams" should
look up a word and return a list of its anagrams.
Solution: \url{http://thinkpython2.com/code/anagram_db.py}.

\end{exercise}


\begin{exercise}
\label{checksum}
\index{MP3}

In a large collection of MP3 files, there may be more than one
copy of the same song, stored in different directories or with
different file names.  The goal of this exercise is to search for
duplicates.

\begin{enumerate}

\item Write a program that searches a directory and all of its
subdirectories, recursively, and returns a list of complete paths
for all files with a given suffix (like {\tt .mp3}).
Hint: {\tt os.path} provides several useful functions for
manipulating file and path names.
\index{duplicate}
\index{MD5 algorithm}
\index{algorithm!MD5}
\index{checksum}

\item To recognize duplicates, you can use {\tt md5sum}
to compute a ``checksum'' for each files.  If two files have
the same checksum, they probably have the same contents.
\index{md5sum}

\item To double-check, you can use the Unix command {\tt diff}.
\index{diff}

\end{enumerate}

Solution: \url{http://thinkpython2.com/code/find_duplicates.py}.

\end{exercise}



\chapter{Classes and objects}
\label{clobjects}

At this point you know how to use
functions to organize code and 
built-in types to organize data.  The next step is to learn
``object-oriented programming'', which uses programmer-defined types
to organize both code and data.  Object-oriented programming is
a big topic; it will take a few chapters to get there.
\index{object-oriented programming}

Code examples from this chapter are available from
\url{http://thinkpython2.com/code/Point1.py}; solutions
to the exercises are available from
\url{http://thinkpython2.com/code/Point1_soln.py}.


\section{Programmer-defined types}
\label{point}
\index{programmer-defined type}
\index{type!programmer-defined}

We have used many of Python's built-in types; now we are going
to define a new type.  As an example, we will create a type
called {\tt Point} that represents a point in two-dimensional
space.
\index{point, mathematical}

In mathematical notation, points are often written in
parentheses with a comma separating the coordinates. For example,
$(0,0)$ represents the origin, and $(x,y)$ represents the
point $x$ units to the right and $y$ units up from the origin.

There are several ways we might represent points in Python:

\begin{itemize}

\item We could store the coordinates separately in two
variables, {\tt x} and {\tt y}.

\item We could store the coordinates as elements in a list
or tuple.

\item We could create a new type to represent points as
objects.

\end{itemize}
\index{representation}

Creating a new type
is more complicated than the other options, but
it has advantages that will be apparent soon.

A programmer-defined type is also called a {\bf class}.
A class definition looks like this:
\index{class}
\index{object!class}
\index{class definition}
\index{definition!class}

\begin{verbatim}
class Point:
    """Represents a point in 2-D space."""
\end{verbatim}
%
The header indicates that the new class is called {\tt Point}.
The body is a docstring that explains what the class is for.
You can define variables and methods inside a class definition,
but we will get back to that later.
\index{Point class}
\index{class!Point}
\index{docstring}

Defining a class named {\tt Point} creates a {\bf class object}.

\begin{verbatim}
>>> Point
<class '__main__.Point'>
\end{verbatim}
%
Because {\tt Point} is defined at the top level, its ``full
name'' is \verb"__main__.Point".
\index{object!class}
\index{class object}

The class object is like a factory for creating objects.  To create a
Point, you call {\tt Point} as if it were a function.

\begin{verbatim}
>>> blank = Point()
>>> blank
<__main__.Point object at 0xb7e9d3ac>
\end{verbatim}
%
The return value is a reference to a Point object, which we
assign to {\tt blank}.  

Creating a new object is called
{\bf instantiation}, and the object is an {\bf instance} of
the class.
\index{instance}
\index{instantiation}

When you print an instance, Python tells you what class it
belongs to and where it is stored in memory (the prefix
{\tt 0x} means that the following number is in hexadecimal).
\index{hexadecimal}

Every object is an instance of some class, so ``object'' and
``instance'' are interchangeable.  But in this chapter I use
``instance'' to indicate that I am talking about a programmer-defined
type.


\section{Attributes}
\label{attributes}
\index{instance attribute}
\index{attribute!instance}
\index{dot notation}

You can assign values to an instance using dot notation:

\begin{verbatim}
>>> blank.x = 3.0
>>> blank.y = 4.0
\end{verbatim}
%
This syntax is similar to the syntax for selecting a variable from a
module, such as {\tt math.pi} or {\tt string.whitespace}.  In this case,
though, we are assigning values to named elements of an object.
These elements are called {\bf attributes}.

As a noun, ``AT-trib-ute'' is pronounced with emphasis on the first
syllable, as opposed to ``a-TRIB-ute'', which is a verb.

The following diagram shows the result of these assignments.
A state diagram that shows an object and its attributes is
called an {\bf object diagram}; see Figure~\ref{fig.point}.
\index{state diagram}
\index{diagram!state}
\index{object diagram}
\index{diagram!object}

\begin{figure}
\centerline
{\includegraphics[scale=0.8]{figs/point.pdf}}
\caption{Object diagram.}
\label{fig.point}
\end{figure}

The variable {\tt blank} refers to a Point object, which
contains two attributes.  Each attribute refers to a
floating-point number.

You can read the value of an attribute using the same syntax:

\begin{verbatim}
>>> blank.y
4.0
>>> x = blank.x
>>> x
3.0
\end{verbatim}
%
The expression {\tt blank.x} means, ``Go to the object {\tt blank}
refers to and get the value of {\tt x}.''  In the example, we assign that
value to a variable named {\tt x}.  There is no conflict between
the variable {\tt x} and the attribute {\tt x}.

You can use dot notation as part of any expression.  For example:

\begin{verbatim}
>>> '(%g, %g)' % (blank.x, blank.y)
'(3.0, 4.0)'
>>> distance = math.sqrt(blank.x**2 + blank.y**2)
>>> distance
5.0
\end{verbatim}
%
You can pass an instance as an argument in the usual way.
For example:
\index{instance!as argument}

\begin{verbatim}
def print_point(p):
    print('(%g, %g)' % (p.x, p.y))
\end{verbatim}
%
\verb"print_point" takes a point as an argument and displays it in
mathematical notation.  To invoke it, you can pass {\tt blank} as
an argument:

\begin{verbatim}
>>> print_point(blank)
(3.0, 4.0)
\end{verbatim}
%
Inside the function, {\tt p} is an alias for {\tt blank}, so if
the function modifies {\tt p}, {\tt blank} changes.
\index{aliasing}

As an exercise, write a function called \verb"distance_between_points"
that takes two Points as arguments and returns the distance between
them.


\section{Rectangles}
\label{rectangles}

Sometimes it is obvious what the attributes of an object should be,
but other times you have to make decisions.  For example, imagine you
are designing a class to represent rectangles.  What attributes would
you use to specify the location and size of a rectangle?  You can
ignore angle; to keep things simple, assume that the rectangle is
either vertical or horizontal.
\index{representation}

There are at least two possibilities: 

\begin{itemize}

\item You could specify one corner of the rectangle
(or the center), the width, and the height.

\item You could specify two opposing corners.

\end{itemize}

At this point it is hard to say whether either is better than
the other, so we'll implement the first one, just as an example.
\index{Rectangle class}
\index{class!Rectangle}

Here is the class definition:

\begin{verbatim}
class Rectangle:
    """Represents a rectangle. 

    attributes: width, height, corner.
    """
\end{verbatim}
%
The docstring lists the attributes:  {\tt width} and
{\tt height} are numbers; {\tt corner} is a Point object that
specifies the lower-left corner.

To represent a rectangle, you have to instantiate a Rectangle
object and assign values to the attributes:

\begin{verbatim}
box = Rectangle()
box.width = 100.0
box.height = 200.0
box.corner = Point()
box.corner.x = 0.0
box.corner.y = 0.0
\end{verbatim}
%
The expression {\tt box.corner.x} means,
``Go to the object {\tt box} refers to and select the attribute named
{\tt corner}; then go to that object and select the attribute named
{\tt x}.''

\begin{figure}
\centerline
{\includegraphics[scale=0.8]{figs/rectangle.pdf}}
\caption{Object diagram.}
\label{fig.rectangle}
\end{figure}


Figure~\ref{fig.rectangle} shows the state of this object.
An object that is an attribute of another object is {\bf embedded}.
\index{state diagram}
\index{diagram!state}
\index{object diagram}
\index{diagram!object}
\index{embedded object}
\index{object!embedded}


\section{Instances as return values}
\index{instance!as return value}
\index{return value}

Functions can return instances.  For example, \verb"find_center"
takes a {\tt Rectangle} as an argument and returns a {\tt Point}
that contains the coordinates of the center of the {\tt Rectangle}:

\begin{verbatim}
def find_center(rect):
    p = Point()
    p.x = rect.corner.x + rect.width/2
    p.y = rect.corner.y + rect.height/2
    return p
\end{verbatim}
%
Here is an example that passes {\tt box} as an argument and assigns
the resulting Point to {\tt center}:

\begin{verbatim}
>>> center = find_center(box)
>>> print_point(center)
(50, 100)
\end{verbatim}
%

\section{Objects are mutable}
\index{object!mutable}
\index{mutability}

You can change the state of an object by making an assignment to one of
its attributes.  For example, to change the size of a rectangle
without changing its position, you can modify the values of {\tt
width} and {\tt height}:

\begin{verbatim}
box.width = box.width + 50
box.height = box.height + 100
\end{verbatim}
%
You can also write functions that modify objects.  For example,
\verb"grow_rectangle" takes a Rectangle object and two numbers,
{\tt dwidth} and {\tt dheight}, and adds the numbers to the
width and height of the rectangle:

\begin{verbatim}
def grow_rectangle(rect, dwidth, dheight):
    rect.width += dwidth
    rect.height += dheight
\end{verbatim}
%
Here is an example that demonstrates the effect:

\begin{verbatim}
>>> box.width, box.height
(150.0, 300.0)
>>> grow_rectangle(box, 50, 100)
>>> box.width, box.height
(200.0, 400.0)
\end{verbatim}
%
Inside the function, {\tt rect} is an
alias for {\tt box}, so when the function modifies {\tt rect}, 
{\tt box} changes.

As an exercise, write a function named \verb"move_rectangle" that takes
a Rectangle and two numbers named {\tt dx} and {\tt dy}.  It
should change the location of the rectangle by adding {\tt dx}
to the {\tt x} coordinate of {\tt corner} and adding {\tt dy}
to the {\tt y} coordinate of {\tt corner}.


\section{Copying}
\label{copying}
\index{aliasing}

Aliasing can make a program difficult to read because changes
in one place might have unexpected effects in another place.
It is hard to keep track of all the variables that might refer
to a given object.
\index{copying objects}
\index{object!copying}
\index{copy module}
\index{module!copy}

Copying an object is often an alternative to aliasing.
The {\tt copy} module contains a function called {\tt copy} that
can duplicate any object:

\begin{verbatim}
>>> p1 = Point()
>>> p1.x = 3.0
>>> p1.y = 4.0

>>> import copy
>>> p2 = copy.copy(p1)
\end{verbatim}
%
{\tt p1} and {\tt p2} contain the same data, but they are
not the same Point.

\begin{verbatim}
>>> print_point(p1)
(3, 4)
>>> print_point(p2)
(3, 4)
>>> p1 is p2
False
>>> p1 == p2
False
\end{verbatim}
%
The {\tt is} operator indicates that {\tt p1} and {\tt p2} are not the
same object, which is what we expected.  But you might have expected
{\tt ==} to yield {\tt True} because these points contain the same
data.  In that case, you will be disappointed to learn that for
instances, the default behavior of the {\tt ==} operator is the same
as the {\tt is} operator; it checks object identity, not object
equivalence.  That's because for programmer-defined types, Python doesn't
know what should be considered equivalent.  At least, not yet.
\index{is operator}
\index{operator!is}
\index{identity}
\index{equivalence}

If you use {\tt copy.copy} to duplicate a Rectangle, you will find
that it copies the Rectangle object but not the embedded Point.
\index{embedded object!copying}

\begin{verbatim}
>>> box2 = copy.copy(box)
>>> box2 is box
False
>>> box2.corner is box.corner
True
\end{verbatim}

\begin{figure}
\centerline
{\includegraphics[scale=0.8]{figs/rectangle2.pdf}}
\caption{Object diagram.}
\label{fig.rectangle2}
\end{figure}

Figure~\ref{fig.rectangle2} shows what the object diagram looks like.
\index{state diagram}
\index{diagram!state}
\index{object diagram}
\index{diagram!object}
This operation is called a {\bf shallow copy} because it copies the
object and any references it contains, but not the embedded objects.
\index{shallow copy}
\index{copy!shallow}

For most applications, this is not what you want.  In this example,
invoking \verb"grow_rectangle" on one of the Rectangles would not
affect the other, but invoking \verb"move_rectangle" on either would
affect both!  This behavior is confusing and error-prone.
\index{deep copy}
\index{copy!deep}

Fortunately, the {\tt copy} module provides a method named {\tt
deepcopy} that copies not only the object but also 
the objects it refers to, and the objects {\em they} refer to,
and so on.
You will not be surprised to learn that this operation is
called a {\bf deep copy}.
\index{deepcopy function}
\index{function!deepcopy}

\begin{verbatim}
>>> box3 = copy.deepcopy(box)
>>> box3 is box
False
>>> box3.corner is box.corner
False
\end{verbatim}
%
{\tt box3} and {\tt box} are completely separate objects.

As an exercise, write a version of \verb"move_rectangle" that creates and
returns a new Rectangle instead of modifying the old one.


\section{디버깅}
%Debugging
\label{hasattr}
\index{debugging}

When you start working with objects, you are likely to encounter
some new exceptions.  If you try to access an attribute
that doesn't exist, you get an {\tt AttributeError}:
\index{exception!AttributeError}
\index{AttributeError}

\begin{verbatim}
>>> p = Point()
>>> p.x = 3
>>> p.y = 4
>>> p.z
AttributeError: Point instance has no attribute 'z'
\end{verbatim}
%
If you are not sure what type an object is, you can ask:
\index{type function}
\index{function!type}

\begin{verbatim}
>>> type(p)
<class '__main__.Point'>
\end{verbatim}
%
You can also use {\tt isinstance} to check whether an object
is an instance of a class:
\index{isinstance function}
\index{function!isinstance}

\begin{verbatim}
>>> isinstance(p, Point)
True
\end{verbatim}
%
If you are not sure whether an object has a particular attribute,
you can use the built-in function {\tt hasattr}:
\index{hasattr function}
\index{function!hasattr}

\begin{verbatim}
>>> hasattr(p, 'x')
True
>>> hasattr(p, 'z')
False
\end{verbatim}
%
The first argument can be any object; the second argument is a {\em
string} that contains the name of the attribute.
\index{attribute}

You can also use a {\tt try} statement to see if the object has the
attributes you need:
\index{try statement}
\index{statement!try}

\begin{verbatim}
try:
    x = p.x
except AttributeError:
    x = 0
\end{verbatim}

This approach can make it easier to write functions that work with
different types; more on that topic is
coming up in Section~\ref{polymorphism}.


\section{용어 해설}
%Glossary

\begin{description}

\item[class:] A programmer-defined type.  A class definition creates a new
class object.
\index{class}
\index{programmer-defined type}
\index{type!programmer-defined}

\item[class object:] An object that contains information about a
programmer-defined type.  The class object can be used to create instances
of the type.
\index{class object}
\index{object!class}

\item[instance:] An object that belongs to a class.
\index{instance}

\item[instantiate:] To create a new object.
\index{instantiate}

\item[attribute:] One of the named values associated with an object.
\index{attribute!instance}
\index{instance attribute}

\item[embedded object:] An object that is stored as an attribute
of another object.
\index{embedded object}
\index{object!embedded}

\item[shallow copy:] To copy the contents of an object, including
any references to embedded objects;
implemented by the {\tt copy} function in the {\tt copy} module.
\index{shallow copy}

\item[deep copy:] To copy the contents of an object as well as any
embedded objects, and any objects embedded in them, and so on;
implemented by the {\tt deepcopy} function in the {\tt copy} module.
\index{deep copy}

\item[object diagram:] A diagram that shows objects, their
attributes, and the values of the attributes.
\index{object diagram}
\index{diagram!object}

\end{description}


\section{연습 문제}
%Exercises

\begin{exercise}

Write a definition for a class named {\tt Circle} with attributes
{\tt center} and {\tt radius}, where {\tt center} is a Point object
and radius is a number.

Instantiate a Circle object that represents a circle with its center
at $(150, 100)$ and radius 75.

Write a function named \verb"point_in_circle" that takes a Circle and
a Point and returns True if the Point lies in or on the boundary of
the circle.

Write a function named \verb"rect_in_circle" that takes a Circle and a
Rectangle and returns True if the Rectangle lies entirely in or on the boundary
of the circle.

Write a function named \verb"rect_circle_overlap" that takes a Circle
and a Rectangle and returns True if any of the corners of the Rectangle fall
inside the circle.  Or as a more challenging version, return True if
any part of the Rectangle falls inside the circle.

Solution: \url{http://thinkpython2.com/code/Circle.py}.

\end{exercise}


\begin{exercise}

Write a function called \verb"draw_rect" that takes a Turtle object
and a Rectangle and uses the Turtle to draw the Rectangle.  See
Chapter~\ref{turtlechap} for examples using Turtle objects.

Write a function called \verb"draw_circle" that takes a Turtle and
a Circle and draws the Circle.

Solution: \url{http://thinkpython2.com/code/draw.py}.

\end{exercise}



\chapter{Classes and functions}
\label{time}

Now that we know how to create new types, the next
step is to write functions that take programmer-defined objects
as parameters and return them as results.  In this chapter I
also present ``functional programming style'' and two new
program development plans.

Code examples from this chapter are available from
\url{http://thinkpython2.com/code/Time1.py}.
Solutions to the exercises are at
\url{http://thinkpython2.com/code/Time1_soln.py}.


\section{Time}
\label{isafter}

As another example of a programmer-defined type, we'll define a class
called {\tt Time} that records the time of day.  The class definition
looks like this: \index{programmer-defined type}
\index{type!programmer-defined} \index{Time class} \index{class!Time}

\begin{verbatim}
class Time:
    """Represents the time of day.
       
    attributes: hour, minute, second
    """
\end{verbatim}
%
We can create a new {\tt Time} object and assign
attributes for hours, minutes, and seconds:

\begin{verbatim}
time = Time()
time.hour = 11
time.minute = 59
time.second = 30
\end{verbatim}
%
The state diagram for the {\tt Time} object looks like Figure~\ref{fig.time}.
\index{state diagram}
\index{diagram!state}
\index{object diagram}
\index{diagram!object}

As an exercise, write a function called \verb"print_time" that takes a 
Time object and prints it in the form {\tt hour:minute:second}.
Hint: the format sequence \verb"'%.2d'" prints an integer using
at least two digits, including a leading zero if necessary.

Write a boolean function called \verb"is_after" that
takes two Time objects, {\tt t1} and {\tt t2}, and
returns {\tt True} if {\tt t1} follows {\tt t2} chronologically and
{\tt False} otherwise.  Challenge: don't use an {\tt if} statement.

\begin{figure}
\centerline
{\includegraphics[scale=0.8]{figs/time.pdf}}
\caption{Object diagram.}
\label{fig.time}
\end{figure}


\section{Pure functions}
\index{prototype and patch}
\index{development plan!prototype and patch}

In the next few sections, we'll write two functions that add time
values.  They demonstrate two kinds of functions: pure functions and
modifiers.  They also demonstrate a development plan I'll call {\bf
  prototype and patch}, which is a way of tackling a complex problem
by starting with a simple prototype and incrementally dealing with the
complications.

Here is a simple prototype of \verb"add_time":

\begin{verbatim}
def add_time(t1, t2):
    sum = Time()
    sum.hour = t1.hour + t2.hour
    sum.minute = t1.minute + t2.minute
    sum.second = t1.second + t2.second
    return sum
\end{verbatim}
%
The function creates a new {\tt Time} object, initializes its
attributes, and returns a reference to the new object.  This is called
a {\bf pure function} because it does not modify any of the objects
passed to it as arguments and it has no effect,
like displaying a value or getting user input, 
other than returning a value.
\index{pure function}
\index{function type!pure}

To test this function, I'll create two Time objects: {\tt start}
contains the start time of a movie, like {\em Monty Python and the
Holy Grail}, and {\tt duration} contains the run time of the movie,
which is one hour 35 minutes.
\index{Monty Python and the Holy Grail}

\verb"add_time" figures out when the movie will be done.

\begin{verbatim}
>>> start = Time()
>>> start.hour = 9
>>> start.minute = 45
>>> start.second =  0

>>> duration = Time()
>>> duration.hour = 1
>>> duration.minute = 35
>>> duration.second = 0

>>> done = add_time(start, duration)
>>> print_time(done)
10:80:00
\end{verbatim}
%
The result, {\tt 10:80:00} might not be what you were hoping
for.  The problem is that this function does not deal with cases where the
number of seconds or minutes adds up to more than sixty.  When that
happens, we have to ``carry'' the extra seconds into the minute column
or the extra minutes into the hour column.
\index{carrying, addition with}

Here's an improved version:

\begin{verbatim}
def add_time(t1, t2):
    sum = Time()
    sum.hour = t1.hour + t2.hour
    sum.minute = t1.minute + t2.minute
    sum.second = t1.second + t2.second

    if sum.second >= 60:
        sum.second -= 60
        sum.minute += 1

    if sum.minute >= 60:
        sum.minute -= 60
        sum.hour += 1

    return sum
\end{verbatim}
%
Although this function is correct, it is starting to get big.
We will see a shorter alternative later.


\section{Modifiers}
\label{increment}
\index{modifier}
\index{function type!modifier}

Sometimes it is useful for a function to modify the objects it gets as
parameters.  In that case, the changes are visible to the caller.
Functions that work this way are called {\bf modifiers}.
\index{increment}

{\tt increment}, which adds a given number of seconds to a {\tt Time}
object, can be written naturally as a
modifier.  Here is a rough draft:

\begin{verbatim}
def increment(time, seconds):
    time.second += seconds

    if time.second >= 60:
        time.second -= 60
        time.minute += 1

    if time.minute >= 60:
        time.minute -= 60
        time.hour += 1
\end{verbatim}
%
The first line performs the basic operation; the remainder deals
with the special cases we saw before.
\index{special case}

Is this function correct?  What happens if {\tt seconds}
is much greater than sixty?  

In that case, it is not enough to carry once; we have to keep doing it
until {\tt time.second} is less than sixty.  One solution is to
replace the {\tt if} statements with {\tt while} statements.  That
would make the function correct, but not very efficient.  As an
exercise, write a correct version of {\tt increment} that doesn't
contain any loops.

Anything that can be done with modifiers can also be done with pure
functions.  In fact, some programming languages only allow pure
functions.  There is some evidence that programs that use pure
functions are faster to develop and less error-prone than programs
that use modifiers.  But modifiers are convenient at times,
and functional programs tend to be less efficient.

In general, I recommend that you write pure functions whenever it is
reasonable and resort to modifiers only if there is a compelling
advantage.  This approach might be called a {\bf functional
programming style}.
\index{functional programming style}

As an exercise, write a ``pure'' version of {\tt increment} that
creates and returns a new Time object rather than modifying the
parameter.


\section{Prototyping versus planning}
\label{prototype}
\index{prototype and patch}
\index{development plan!prototype and patch}
\index{planned development}
\index{development plan!designed}

The development plan I am demonstrating is called ``prototype and
patch''.  For each function, I wrote a prototype that performed the
basic calculation and then tested it, patching errors along the
way.

This approach can be effective, especially if you don't yet have a
deep understanding of the problem.  But incremental corrections can
generate code that is unnecessarily complicated---since it deals with
many special cases---and unreliable---since it is hard to know if you
have found all the errors.

An alternative is {\bf designed development}, in which high-level
insight into the problem can make the programming much easier.  In
this case, the insight is that a Time object is really a three-digit
number in base 60 (see \url{http://en.wikipedia.org/wiki/Sexagesimal}.)!  The
{\tt second} attribute is the ``ones column'', the {\tt minute}
attribute is the ``sixties column'', and the {\tt hour} attribute is
the ``thirty-six hundreds column''.
\index{sexagesimal}

When we wrote \verb"add_time" and {\tt increment}, we were effectively
doing addition in base 60, which is why we had to carry from one
column to the next.
\index{carrying, addition with}

This observation suggests another approach to the whole problem---we
can convert Time objects to integers and take advantage of the fact
that the computer knows how to do integer arithmetic.  

Here is a function that converts Times to integers:

\begin{verbatim}
def time_to_int(time):
    minutes = time.hour * 60 + time.minute
    seconds = minutes * 60 + time.second
    return seconds
\end{verbatim}
%
And here is a function that converts an integer to a Time
(recall that {\tt divmod} divides the first argument by the second
and returns the quotient and remainder as a tuple).
\index{divmod}

\begin{verbatim}
def int_to_time(seconds):
    time = Time()
    minutes, time.second = divmod(seconds, 60)
    time.hour, time.minute = divmod(minutes, 60)
    return time
\end{verbatim}
%
You might have to think a bit, and run some tests, to convince
yourself that these functions are correct.  One way to test them is to
check that \verb"time_to_int(int_to_time(x)) == x" for many values of
{\tt x}.  This is an example of a consistency check.
\index{consistency check}

Once you are convinced they are correct, you can use them to 
rewrite \verb"add_time":

\begin{verbatim}
def add_time(t1, t2):
    seconds = time_to_int(t1) + time_to_int(t2)
    return int_to_time(seconds)
\end{verbatim}
%
This version is shorter than the original, and easier to verify.  As
an exercise, rewrite {\tt increment} using \verb"time_to_int" and
\verb"int_to_time".

In some ways, converting from base 60 to base 10 and back is harder
than just dealing with times.  Base conversion is more abstract; our
intuition for dealing with time values is better.

But if we have the insight to treat times as base 60 numbers and make
the investment of writing the conversion functions (\verb"time_to_int"
and \verb"int_to_time"), we get a program that is shorter, easier to
read and debug, and more reliable.

It is also easier to add features later.  For example, imagine
subtracting two Times to find the duration between them.  The
naive approach would be to implement subtraction with borrowing.
Using the conversion functions would be easier and more likely to be
correct.
\index{subtraction with borrowing}
\index{borrowing, subtraction with}
\index{generalization}

Ironically, sometimes making a problem harder (or more general) makes it
easier (because there are fewer special cases and fewer opportunities
for error).


\section{디버깅}
%Debugging
\index{debugging}

A Time object is well-formed if the values of {\tt minute} and {\tt
second} are between 0 and 60 (including 0 but not 60) and if 
{\tt hour} is positive.  {\tt hour} and {\tt minute} should be
integral values, but we might allow {\tt second} to have a
fraction part.
\index{invariant}

Requirements like these are called {\bf invariants} because
they should always be true.  To put it a different way, if they
are not true, something has gone wrong.

Writing code to check invariants can help detect errors
and find their causes.  For example, you might have a function
like \verb"valid_time" that takes a Time object and returns
{\tt False} if it violates an invariant:

\begin{verbatim}
def valid_time(time):
    if time.hour < 0 or time.minute < 0 or time.second < 0:
        return False
    if time.minute >= 60 or time.second >= 60:
        return False
    return True
\end{verbatim}
%
At the beginning of each function you could check the
arguments to make sure they are valid:
\index{raise statement}
\index{statement!raise}

\begin{verbatim}
def add_time(t1, t2):
    if not valid_time(t1) or not valid_time(t2):
        raise ValueError('invalid Time object in add_time')
    seconds = time_to_int(t1) + time_to_int(t2)
    return int_to_time(seconds)
\end{verbatim}
%
Or you could use an {\bf assert statement}, which checks a given invariant
and raises an exception if it fails:
\index{assert statement}
\index{statement!assert}

\begin{verbatim}
def add_time(t1, t2):
    assert valid_time(t1) and valid_time(t2)
    seconds = time_to_int(t1) + time_to_int(t2)
    return int_to_time(seconds)
\end{verbatim}
%
{\tt assert} statements are useful because they distinguish
code that deals with normal conditions from code
that checks for errors.


\section{용어 해설}
%Glossary

\begin{description}

\item[prototype and patch:] A development plan that involves
writing a rough draft of a program, testing, and correcting errors as
they are found.
\index{prototype and patch}

\item[designed development:] A development plan that involves
high-level insight into the problem and more planning than incremental
development or prototype development.
\index{designed development}

\item[pure function:] A function that does not modify any of the objects it
receives as arguments.  Most pure functions are fruitful.
\index{pure function}

\item[modifier:] A function that changes one or more of the objects it
  receives as arguments.  Most modifiers are void; that is, they
  return {\tt None}.  \index{modifier}

\item[functional programming style:] A style of program design in which the
majority of functions are pure.
\index{functional programming style}

\item[invariant:] A condition that should always be true during the
execution of a program.
\index{invariant}

\item[assert statement:] A statement that check a condition and raises
an exception if it fails.
\index{assert statement}
\index{statement!assert}

\end{description}


\section{연습 문제}
%Exercises

Code examples from this chapter are available from
\url{http://thinkpython2.com/code/Time1.py}; solutions to the
exercises are available from \url{http://thinkpython2.com/code/Time1_soln.py}.

\begin{exercise}

Write a function called \verb"mul_time" that takes a Time object
and a number and returns a new Time object that contains
the product of the original Time and the number.

Then use \verb"mul_time" to write a function that takes a Time
object that represents the finishing time in a race, and a number
that represents the distance, and returns a Time object that represents
the average pace (time per mile).
\index{running pace}

\end{exercise}


\begin{exercise}
\index{datetime module}
\index{module!datetime}

The {\tt datetime} module provides {\tt time} objects
that are similar to the Time objects in this chapter, but
they provide a rich set of methods and operators.  Read the
documentation at \url{http://docs.python.org/3/library/datetime.html}.

\begin{enumerate}

\item Use the {\tt datetime} module to write a program that gets the
  current date and prints the day of the week.

\item Write a program that takes a birthday as input and prints the
  user's age and the number of days, hours, minutes and seconds until
  their next birthday.
\index{birthday}

\item For two people born on different days, there is a day when one
  is twice as old as the other.  That's their Double Day.  Write a
  program that takes two birthdays and computes their Double Day.

\item For a little more challenge, write the more general version that
  computes the day when one person is $n$ times older than the other.
\index{Double Day}

\end{enumerate}

Solution: \url{http://thinkpython2.com/code/double.py}

\end{exercise}


\chapter{Classes and methods}

Although we are using some of Python's object-oriented features,
the programs from the last two chapters are not really
object-oriented because they don't represent the relationships
between programmer-defined types and the functions that operate
on them.  The next step is to transform those functions into
methods that make the relationships explicit.

Code examples from this chapter are available from
\url{http://thinkpython2.com/code/Time2.py}, and solutions
to the exercises are in \url{http://thinkpython2.com/code/Point2_soln.py}.


\section{Object-oriented features}
\index{object-oriented programming}

Python is an {\bf object-oriented programming language}, which means
that it provides features that support object-oriented
programming, which has these defining characteristics:

\begin{itemize}

\item Programs include class and method definitions.

\item Most of the computation is expressed in terms of operations on
  objects.

\item Objects often represent things
in the real world, and methods often
correspond to the ways things in the real world interact.

\end{itemize}

For example, the {\tt Time} class defined in Chapter~\ref{time}
corresponds to the way people record the time of day, and the
functions we defined correspond to the kinds of things people do with
times.  Similarly, the {\tt Point} and {\tt Rectangle} classes
in Chapter~\ref{clobjects}
correspond to the mathematical concepts of a point and a rectangle.

So far, we have not taken advantage of the features Python provides to
support object-oriented programming.  These
features are not strictly necessary; most of them provide
alternative syntax for things we have already done.  But in many cases,
the alternative is more concise and more accurately conveys the
structure of the program.

For example, in {\tt Time1.py} there is no obvious
connection between the class definition and the function definitions
that follow.  With some examination, it is apparent that every function
takes at least one {\tt Time} object as an argument.
\index{method}
\index{function}

This observation is the motivation for {\bf methods}; a method is
a function that is associated with a particular class.
We have seen methods for strings, lists, dictionaries and tuples.
In this chapter, we will define methods for programmer-defined types.
\index{syntax}
\index{semantics}
\index{programmer-defined type}
\index{type!programmer-defined}

Methods are semantically the same as functions, but there are
two syntactic differences:

\begin{itemize}

\item Methods are defined inside a class definition in order
to make the relationship between the class and the method explicit.

\item The syntax for invoking a method is different from the
syntax for calling a function.

\end{itemize}

In the next few sections, we will take the functions from the previous
two chapters and transform them into methods.  This transformation is
purely mechanical; you can do it by following a sequence of
steps.  If you are comfortable converting from one form to another,
you will be able to choose the best form for whatever you are doing.


\section{Printing objects}
\index{object!printing}

In Chapter~\ref{time}, we defined a class named
{\tt Time} and in Section~\ref{isafter}, you 
wrote a function named \verb"print_time":

\begin{verbatim}
class Time:
    """Represents the time of day."""

def print_time(time):
    print('%.2d:%.2d:%.2d' % (time.hour, time.minute, time.second))
\end{verbatim}
%
To call this function, you have to pass a {\tt Time} object as an
argument:

\begin{verbatim}
>>> start = Time()
>>> start.hour = 9
>>> start.minute = 45
>>> start.second = 00
>>> print_time(start)
09:45:00
\end{verbatim}
%
To make \verb"print_time" a method, all we have to do is
move the function definition inside the class definition.  Notice
the change in indentation.
\index{indentation}

\begin{verbatim}
class Time:
    def print_time(time):
        print('%.2d:%.2d:%.2d' % (time.hour, time.minute, time.second))
\end{verbatim}
%
Now there are two ways to call \verb"print_time".  The first
(and less common) way is to use function syntax:
\index{function syntax}
\index{dot notation}

\begin{verbatim}
>>> Time.print_time(start)
09:45:00
\end{verbatim}
%
In this use of dot notation, {\tt Time} is the name of the class,
and \verb"print_time" is the name of the method.  {\tt start} is
passed as a parameter.

The second (and more concise) way is to use method syntax:
\index{method syntax}

\begin{verbatim}
>>> start.print_time()
09:45:00
\end{verbatim}
%
In this use of dot notation, \verb"print_time" is the name of the
method (again), and {\tt start} is the object the method is
invoked on, which is called the {\bf subject}.  Just as the
subject of a sentence is what the sentence is about, the subject
of a method invocation is what the method is about.
\index{subject}

Inside the method, the subject is assigned to the first
parameter, so in this case {\tt start} is assigned
to {\tt time}.
\index{self (parameter name)}
\index{parameter!self}

By convention, the first parameter of a method is
called {\tt self}, so it would be more common to write
\verb"print_time" like this:

\begin{verbatim}
class Time:
    def print_time(self):
        print('%.2d:%.2d:%.2d' % (self.hour, self.minute, self.second))
\end{verbatim}
%
The reason for this convention is an implicit metaphor:
\index{metaphor, method invocation}

\begin{itemize}

\item The syntax for a function call, \verb"print_time(start)",
  suggests that the function is the active agent.  It says something
  like, ``Hey \verb"print_time"!  Here's an object for you to print.''

\item In object-oriented programming, the objects are the active
  agents.  A method invocation like \verb"start.print_time()" says
  ``Hey {\tt start}!  Please print yourself.''

\end{itemize}

This change in perspective might be more polite, but it is not obvious
that it is useful.  In the examples we have seen so far, it may not
be.  But sometimes shifting responsibility from the functions onto the
objects makes it possible to write more versatile functions (or
methods), and makes it easier to maintain and reuse code.

As an exercise, rewrite \verb"time_to_int" (from
Section~\ref{prototype}) as a method.  You might be tempted to
rewrite \verb"int_to_time" as a method, too, but that doesn't
really make sense because there would be no object to invoke
it on.


\section{Another example}
\index{increment}

Here's a version of {\tt increment} (from Section~\ref{increment})
rewritten as a method:

\begin{verbatim}
# inside class Time:

    def increment(self, seconds):
        seconds += self.time_to_int()
        return int_to_time(seconds)
\end{verbatim}
%
This version assumes that \verb"time_to_int" is written
as a method.  Also, note that
it is a pure function, not a modifier.

Here's how you would invoke {\tt increment}:

\begin{verbatim}
>>> start.print_time()
09:45:00
>>> end = start.increment(1337)
>>> end.print_time()
10:07:17
\end{verbatim}
%
The subject, {\tt start}, gets assigned to the first parameter,
{\tt self}.  The argument, {\tt 1337}, gets assigned to the
second parameter, {\tt seconds}.

This mechanism can be confusing, especially if you make an error.
For example, if you invoke {\tt increment} with two arguments, you
get:
\index{exception!TypeError}
\index{TypeError}

\begin{verbatim}
>>> end = start.increment(1337, 460)
TypeError: increment() takes 2 positional arguments but 3 were given
\end{verbatim}
%
The error message is initially confusing, because there are
only two arguments in parentheses.  But the subject is also
considered an argument, so all together that's three.

By the way, a {\bf positional argument} is an argument that
doesn't have a parameter name; that is, it is not a keyword
argument.  In this function call:
\index{positional argument}
\index{argument!positional}

\begin{verbatim}
sketch(parrot, cage, dead=True)
\end{verbatim}

{\tt parrot} and {\tt cage} are positional, and {\tt dead} is
a keyword argument.


\section{A more complicated example}

Rewriting \verb"is_after" (from Section~\ref{isafter}) is slightly
more complicated because it takes two Time objects as parameters.  In
this case it is conventional to name the first parameter {\tt self}
and the second parameter {\tt other}: \index{other (parameter name)}
\index{parameter!other}

\begin{verbatim}
# inside class Time:

    def is_after(self, other):
        return self.time_to_int() > other.time_to_int()
\end{verbatim}
%
To use this method, you have to invoke it on one object and pass
the other as an argument:

\begin{verbatim}
>>> end.is_after(start)
True
\end{verbatim}
%
One nice thing about this syntax is that it almost reads
like English: ``end is after start?''


\section{The init method}
\index{init method}
\index{method!init}

The init method (short for ``initialization'') is
a special method that gets invoked when an object is instantiated.  
Its full name is \verb"__init__" (two underscore characters,
followed by {\tt init}, and then two more underscores).  An
init method for the {\tt Time} class might look like this:

\begin{verbatim}
# inside class Time:

    def __init__(self, hour=0, minute=0, second=0):
        self.hour = hour
        self.minute = minute
        self.second = second
\end{verbatim}
%
It is common for the parameters of \verb"__init__"
to have the same names as the attributes.  The statement

\begin{verbatim}
        self.hour = hour
\end{verbatim}
%
stores the value of the parameter {\tt hour} as an attribute
of {\tt self}.
\index{optional parameter}
\index{parameter!optional}
\index{default value}
\index{override}

The parameters are optional, so if you call {\tt Time} with
no arguments, you get the default values.

\begin{verbatim}
>>> time = Time()
>>> time.print_time()
00:00:00
\end{verbatim}
%
If you provide one argument, it overrides {\tt hour}:

\begin{verbatim}
>>> time = Time (9)
>>> time.print_time()
09:00:00
\end{verbatim}
%
If you provide two arguments, they override {\tt hour} and
{\tt minute}.

\begin{verbatim}
>>> time = Time(9, 45)
>>> time.print_time()
09:45:00
\end{verbatim}
%
And if you provide three arguments, they override all three
default values.

As an exercise, write an init method for the {\tt Point} class that takes
{\tt x} and {\tt y} as optional parameters and assigns
them to the corresponding attributes.
\index{Point class}
\index{class!Point}


\section{The {\tt \_\_str\_\_} method}
\index{str method@\_\_str\_\_ method}
\index{method!\_\_str\_\_}

\verb"__str__" is a special method, like \verb"__init__",
that is supposed to return a string representation of an object.
\index{string representation}

For example, here is a {\tt str} method for Time objects:

\begin{verbatim}
# inside class Time:

    def __str__(self):
        return '%.2d:%.2d:%.2d' % (self.hour, self.minute, self.second)
\end{verbatim}
%
When you {\tt print} an object, Python invokes the {\tt str} method:
\index{print statement}
\index{statement!print}

\begin{verbatim}
>>> time = Time(9, 45)
>>> print(time)
09:45:00
\end{verbatim}
%
When I write a new class, I almost always start by writing 
\verb"__init__", which makes it easier to instantiate objects, and 
\verb"__str__", which is useful for debugging.

As an exercise, write a {\tt str} method for the {\tt Point} class.
Create a Point object and print it.


\section{Operator overloading}
\label{operator.overloading}

By defining other special methods, you can specify the behavior
of operators on programmer-defined types.  For example, if you define
a method named \verb"__add__" for the {\tt Time} class, you can use the
{\tt +} operator on Time objects.
\index{programmer-defined type}
\index{type!programmer-defined}

Here is what the definition might look like:
\index{add method}
\index{method!add}

\begin{verbatim}
# inside class Time:

    def __add__(self, other):
        seconds = self.time_to_int() + other.time_to_int()
        return int_to_time(seconds)
\end{verbatim}
%
And here is how you could use it:

\begin{verbatim}
>>> start = Time(9, 45)
>>> duration = Time(1, 35)
>>> print(start + duration)
11:20:00
\end{verbatim}
%
When you apply the {\tt +} operator to Time objects, Python invokes
\verb"__add__".  When you print the result, Python invokes 
\verb"__str__".  So there is a lot happening behind the scenes!
\index{operator overloading}

Changing the behavior of an operator so that it works with
programmer-defined types is called {\bf operator overloading}.  For every
operator in Python there is a corresponding special method, like 
\verb"__add__".  For more details, see
\url{http://docs.python.org/3/reference/datamodel.html#specialnames}.

As an exercise, write an {\tt add} method for the Point class.  


\section{Type-based dispatch}

In the previous section we added two Time objects, but you
also might want to add an integer to a Time object.  The
following is a version of \verb"__add__"
that checks the type of {\tt other} and invokes either
\verb"add_time" or {\tt increment}:

\begin{verbatim}
# inside class Time:

    def __add__(self, other):
        if isinstance(other, Time):
            return self.add_time(other)
        else:
            return self.increment(other)

    def add_time(self, other):
        seconds = self.time_to_int() + other.time_to_int()
        return int_to_time(seconds)

    def increment(self, seconds):
        seconds += self.time_to_int()
        return int_to_time(seconds)
\end{verbatim}
%
The built-in function {\tt isinstance} takes a value and a
class object, and returns {\tt True} if the value is an instance
of the class.
\index{isinstance function}
\index{function!isinstance}

If {\tt other} is a Time object, \verb"__add__" invokes
\verb"add_time".  Otherwise it assumes that the parameter
is a number and invokes {\tt increment}.  This operation is
called a {\bf type-based dispatch} because it dispatches the
computation to different methods based on the type of the
arguments.
\index{type-based dispatch}
\index{dispatch, type-based}

Here are examples that use the {\tt +} operator with different
types:

\begin{verbatim}
>>> start = Time(9, 45)
>>> duration = Time(1, 35)
>>> print(start + duration)
11:20:00
>>> print(start + 1337)
10:07:17
\end{verbatim}
%
Unfortunately, this implementation of addition is not commutative.
If the integer is the first operand, you get
\index{commutativity}

\begin{verbatim}
>>> print(1337 + start)
TypeError: unsupported operand type(s) for +: 'int' and 'instance'
\end{verbatim}
%
The problem is, instead of asking the Time object to add an integer,
Python is asking an integer to add a Time object, and it doesn't know
how.  But there is a clever solution for this problem: the
special method \verb"__radd__", which stands for ``right-side add''.
This method is invoked when a Time object appears on the right side of
the {\tt +} operator.  Here's the definition:
\index{radd method}
\index{method!radd}

\begin{verbatim}
# inside class Time:

    def __radd__(self, other):
        return self.__add__(other)
\end{verbatim}
%
And here's how it's used:

\begin{verbatim}
>>> print(1337 + start)
10:07:17
\end{verbatim}
%

As an exercise, write an {\tt add} method for Points that works with
either a Point object or a tuple:

\begin{itemize}

\item If the second operand is a Point, the method should return a new
Point whose $x$ coordinate is the sum of the $x$ coordinates of the
operands, and likewise for the $y$ coordinates.

\item If the second operand is a tuple, the method should add the
first element of the tuple to the $x$ coordinate and the second
element to the $y$ coordinate, and return a new Point with the result. 

\end{itemize}




\section{Polymorphism}
\label{polymorphism}

Type-based dispatch is useful when it is necessary, but (fortunately)
it is not always necessary.  Often you can avoid it by writing functions
that work correctly for arguments with different types.
\index{type-based dispatch}
\index{dispatch!type-based}

Many of the functions we wrote for strings also
work for other sequence types.
For example, in Section~\ref{histogram}
we used {\tt histogram} to count the number of times each letter
appears in a word.

\begin{verbatim}
def histogram(s):
    d = dict()
    for c in s:
        if c not in d:
            d[c] = 1
        else:
            d[c] = d[c]+1
    return d
\end{verbatim}
%
This function also works for lists, tuples, and even dictionaries,
as long as the elements of {\tt s} are hashable, so they can be used
as keys in {\tt d}.

\begin{verbatim}
>>> t = ['spam', 'egg', 'spam', 'spam', 'bacon', 'spam']
>>> histogram(t)
{'bacon': 1, 'egg': 1, 'spam': 4}
\end{verbatim}
%
Functions that work with several types are called {\bf polymorphic}.
Polymorphism can facilitate code reuse.  For example, the built-in
function {\tt sum}, which adds the elements of a sequence, works
as long as the elements of the sequence support addition.
\index{polymorphism}

Since Time objects provide an {\tt add} method, they work
with {\tt sum}:

\begin{verbatim}
>>> t1 = Time(7, 43)
>>> t2 = Time(7, 41)
>>> t3 = Time(7, 37)
>>> total = sum([t1, t2, t3])
>>> print(total)
23:01:00
\end{verbatim}
%
In general, if all of the operations inside a function 
work with a given type, the function works with that type.

The best kind of polymorphism is the unintentional kind, where
you discover that a function you already wrote can be
applied to a type you never planned for.


\section{디버깅}
%Debugging
\index{debugging}

It is legal to add attributes to objects at any point in the execution
of a program, but if you have objects with the same type that don't
have the same attributes, it is easy to make mistakes.
It is considered a good idea to
initialize all of an object's attributes in the init method.
\index{init method}
\index{attribute!initializing}

If you are not sure whether an object has a particular attribute, you
can use the built-in function {\tt hasattr} (see Section~\ref{hasattr}).
\index{hasattr function}
\index{function!hasattr}
\index{dict attribute@\_\_dict\_\_ attribute}
\index{attribute!\_\_dict\_\_}

Another way to access attributes is the built-in function {\tt vars},
which takes an object and returns a dictionary that maps from
attribute names (as strings) to their values:

\begin{verbatim}
>>> p = Point(3, 4)
>>> vars(p)
{'y': 4, 'x': 3}
\end{verbatim}
%
For purposes of debugging, you might find it useful to keep this
function handy:

\begin{verbatim}
def print_attributes(obj):
    for attr in vars(obj):
        print(attr, getattr(obj, attr))
\end{verbatim}
%
\verb"print_attributes" traverses the dictionary
and prints each attribute name and its corresponding value.
\index{traversal!dictionary}
\index{dictionary!traversal}

The built-in function {\tt getattr} takes an object and an attribute
name (as a string) and returns the attribute's value.
\index{getattr function}
\index{function!getattr}


\section{Interface and implementation}

One of the goals of object-oriented design is to make software more
maintainable, which means that you can keep the program working when
other parts of the system change, and modify the program to meet new
requirements.
\index{interface}
\index{implementation}
\index{maintainable}
\index{object-oriented design}

A design principle that helps achieve that goal is to keep
interfaces separate from implementations.  For objects, that means
that the methods a class provides should not depend on how the
attributes are represented.
\index{attribute}

For example, in this chapter we developed a class that represents
a time of day.  Methods provided by this class include
\verb"time_to_int", \verb"is_after", and \verb"add_time".

We could implement those methods in several ways.  The details of the
implementation depend on how we represent time.  In this chapter, the
attributes of a {\tt Time} object are {\tt hour}, {\tt minute}, and
{\tt second}.

As an alternative, we could replace these attributes with
a single integer representing the number of seconds
since midnight.  This implementation would make some methods,
like \verb"is_after", easier to write, but it makes other methods
harder.

After you deploy a new class, you might discover a better
implementation.  If other parts of the program are using your
class, it might be time-consuming and error-prone to change the
interface.  

But if you designed the interface carefully, you can
change the implementation without changing the interface, which
means that other parts of the program don't have to change.


\section{용어 해설}
%Glossary

\begin{description}

\item[object-oriented language:] A language that provides features,
  such as programmer-defined types and methods, that facilitate
  object-oriented programming.
\index{object-oriented language}

\item[object-oriented programming:] A style of programming in which
data and the operations that manipulate it are organized into classes
and methods.
\index{object-oriented programming}

\item[method:] A function that is defined inside a class definition and
is invoked on instances of that class.
\index{method}

\item[subject:] The object a method is invoked on.
\index{subject}

\item[positional argument:]  An argument that does not include
a parameter name, so it is not a keyword argument.
\index{positional argument}
\index{argument!positional}

\item[operator overloading:] Changing the behavior of an operator like
{\tt +} so it works with a programmer-defined type.
\index{overloading}
\index{operator!overloading}

\item[type-based dispatch:] A programming pattern that checks the type
of an operand and invokes different functions for different types.
\index{type-based dispatch}

\item[polymorphic:] Pertaining to a function that can work with more
  than one type.  
\index{polymorphism}

\item[information hiding:] The principle that the interface provided 
by an object should not depend on its implementation, in particular
the representation of its attributes.
\index{information hiding}

\end{description}


\section{연습 문제}
%Exercises

\begin{exercise}

Download the code from this chapter from
\url{http://thinkpython2.com/code/Time2.py}.  Change the attributes of
    {\tt Time} to be a single integer representing seconds since
    midnight.  Then modify the methods (and the function
    \verb"int_to_time") to work with the new implementation.  You
    should not have to modify the test code in {\tt main}.  When you
    are done, the output should be the same as before.  Solution:
    \url{http://thinkpython2.com/code/Time2_soln.py}.

\end{exercise}


\begin{exercise}
\label{kangaroo}
\index{default value!avoiding mutable}
\index{mutable object, as default value}
\index{worst bug}
\index{bug!worst}
\index{Kangaroo class}
\index{class!Kangaroo}

This exercise is a cautionary tale about one of the most
common, and difficult to find, errors in Python.
Write a definition for a class named {\tt Kangaroo} with the following
methods:

\begin{enumerate}

\item An \verb"__init__" method that initializes an attribute named 
\verb"pouch_contents" to an empty list.

\item A method named \verb"put_in_pouch" that takes an object
of any type and adds it to \verb"pouch_contents".

\item A \verb"__str__" method that returns a string representation
of the Kangaroo object and the contents of the pouch.

\end{enumerate}
%
Test your code 
by creating two {\tt Kangaroo} objects, assigning them to variables
named {\tt kanga} and {\tt roo}, and then adding {\tt roo} to the
contents of {\tt kanga}'s pouch.

Download \url{http://thinkpython2.com/code/BadKangaroo.py}.  It contains
a solution to the previous problem with one big, nasty bug.
Find and fix the bug.

If you get stuck, you can download
\url{http://thinkpython2.com/code/GoodKangaroo.py}, which explains the
problem and demonstrates a solution.
\index{aliasing}
\index{embedded object}
\index{object!embedded}

\end{exercise}



\chapter{Inheritance}

The language feature most often associated with object-oriented
programming is {\bf inheritance}.  Inheritance is the ability to
define a new class that is a modified version of an existing class.
In this chapter I demonstrate inheritance using classes that represent
playing cards, decks of cards, and poker hands.
\index{deck} 
\index{card, playing} 
\index{poker}

If you don't play
poker, you can read about it at
\url{http://en.wikipedia.org/wiki/Poker}, but you don't have to; I'll
tell you what you need to know for the exercises.

Code examples from
this chapter are available from
\url{http://thinkpython2.com/code/Card.py}.


\section{Card objects}

There are fifty-two cards in a deck, each of which belongs to one of
four suits and one of thirteen ranks.  The suits are Spades, Hearts,
Diamonds, and Clubs (in descending order in bridge).  The ranks are
Ace, 2, 3, 4, 5, 6, 7, 8, 9, 10, Jack, Queen, and King.  Depending on
the game that you are playing, an Ace may be higher than King
or lower than 2.
\index{rank}
\index{suit}

If we want to define a new object to represent a playing card, it is
obvious what the attributes should be: {\tt rank} and
{\tt suit}.  It is not as obvious what type the attributes
should be.  One possibility is to use strings containing words like
\verb"'Spade'" for suits and \verb"'Queen'" for ranks.  One problem with
this implementation is that it would not be easy to compare cards to
see which had a higher rank or suit.
\index{encode}
\index{encrypt}
\index{map to}
\index{representation}

An alternative is to use integers to {\bf encode} the ranks and suits.
In this context, ``encode'' means that we are going to define a mapping
between numbers and suits, or between numbers and ranks.  This
kind of encoding is not meant to be a secret (that
would be ``encryption'').

\newcommand{\mymapsto}{$\mapsto$}

For example, this table shows the suits and the corresponding integer
codes:

\begin{tabular}{l c l}
Spades & \mymapsto & 3 \\
Hearts & \mymapsto & 2 \\
Diamonds & \mymapsto & 1 \\
Clubs & \mymapsto & 0
\end{tabular}

This code makes it easy to compare cards; because higher suits map to
higher numbers, we can compare suits by comparing their codes.

The mapping for ranks is fairly obvious; each of the numerical ranks
maps to the corresponding integer, and for face cards:

\begin{tabular}{l c l}
Jack & \mymapsto & 11 \\
Queen & \mymapsto & 12 \\
King & \mymapsto & 13 \\
\end{tabular}

I am using the \mymapsto~symbol to make it clear that these mappings
are not part of the Python program.  They are part of the program
design, but they don't appear explicitly in the code.
\index{Card class}
\index{class!Card}

The class definition for {\tt Card} looks like this:

\begin{verbatim}
class Card:
    """Represents a standard playing card."""

    def __init__(self, suit=0, rank=2):
        self.suit = suit
        self.rank = rank
\end{verbatim}
%
As usual, the init method takes an optional
parameter for each attribute.  The default card is
the 2 of Clubs.
\index{init method}
\index{method!init}

To create a Card, you call {\tt Card} with the
suit and rank of the card you want.

\begin{verbatim}
queen_of_diamonds = Card(1, 12)
\end{verbatim}
%


\section{Class attributes}
\label{class.attribute}
\index{class attribute}
\index{attribute!class}

In order to print Card objects in a way that people can easily
read, we need a mapping from the integer codes to the corresponding
ranks and suits.  A natural way to
do that is with lists of strings.  We assign these lists to {\bf class
attributes}:

\begin{verbatim}
# inside class Card:

    suit_names = ['Clubs', 'Diamonds', 'Hearts', 'Spades']
    rank_names = [None, 'Ace', '2', '3', '4', '5', '6', '7', 
              '8', '9', '10', 'Jack', 'Queen', 'King']

    def __str__(self):
        return '%s of %s' % (Card.rank_names[self.rank],
                             Card.suit_names[self.suit])
\end{verbatim}
%
Variables like \verb"suit_names" and \verb"rank_names", which are
defined inside a class but outside of any method, are called
class attributes because they are associated with the class object 
{\tt Card}.
\index{instance attribute}
\index{attribute!instance}

This term distinguishes them from variables like {\tt suit} and {\tt
  rank}, which are called {\bf instance attributes} because they are
associated with a particular instance.
\index{dot notation}

Both kinds of attribute are accessed using dot notation.  For
example, in \verb"__str__", {\tt self} is a Card object,
and {\tt self.rank} is its rank.  Similarly, {\tt Card}
is a class object, and \verb"Card.rank_names" is a
list of strings associated with the class.

Every card has its own {\tt suit} and {\tt rank}, but there
is only one copy of \verb"suit_names" and \verb"rank_names".

Putting it all together, the expression
\verb"Card.rank_names[self.rank]" means ``use the attribute {\tt rank}
from the object {\tt self} as an index into the list \verb"rank_names"
from the class {\tt Card}, and select the appropriate string.''

The first element of \verb"rank_names" is {\tt None} because there
is no card with rank zero.  By including {\tt None} as a place-keeper,
we get a mapping with the nice property that the index 2 maps to the
string \verb"'2'", and so on.  To avoid this tweak, we could have
used a dictionary instead of a list.

With the methods we have so far, we can create and print cards:

\begin{verbatim}
>>> card1 = Card(2, 11)
>>> print(card1)
Jack of Hearts
\end{verbatim}

\begin{figure}
\centerline
{\includegraphics[scale=0.8]{figs/card1.pdf}}
\caption{Object diagram.}
\label{fig.card1}
\end{figure}

Figure~\ref{fig.card1} is a diagram of the {\tt Card} class object and
one Card instance.  {\tt Card} is a class object; its type is {\tt
  type}.  {\tt card1} is an instance of {\tt Card}, so its type is
{\tt Card}.  To save space, I didn't draw the contents of
\verb"suit_names" and \verb"rank_names".  \index{state diagram}
\index{diagram!state} \index{object diagram} \index{diagram!object}


\section{Comparing cards}
\label{comparecard}
\index{operator!relational}
\index{relational operator}

For built-in types, there are relational operators
({\tt <}, {\tt >}, {\tt ==}, etc.)
that compare
values and determine when one is greater than, less than, or equal to
another.  For programmer-defined types, we can override the behavior of
the built-in operators by providing a method named
\verb"__lt__", which stands for ``less than''.
\index{programmer-defined type}
\index{type!programmer-defined}

\verb"__lt__" takes two parameters, {\tt self} and {\tt other},
and returns {\tt True} if {\tt self} is strictly less than {\tt other}.
\index{override}
\index{operator overloading}

The correct ordering for cards is not obvious.
For example, which
is better, the 3 of Clubs or the 2 of Diamonds?  One has a higher
rank, but the other has a higher suit.  In order to compare
cards, you have to decide whether rank or suit is more important.

The answer might depend on what game you are playing, but to keep
things simple, we'll make the arbitrary choice that suit is more
important, so all of the Spades outrank all of the Diamonds,
and so on.
\index{cmp method@\_\_cmp\_\_ method}
\index{method!\_\_cmp\_\_}

With that decided, we can write \verb"__lt__":

\begin{verbatim}
# inside class Card:

    def __lt__(self, other):
        # check the suits
        if self.suit < other.suit: return True
        if self.suit > other.suit: return False

        # suits are the same... check ranks
        return self.rank < other.rank
\end{verbatim}
%
You can write this more concisely using tuple comparison:
\index{tuple!comparison}
\index{comparison!tuple}

\begin{verbatim}
# inside class Card:

    def __lt__(self, other):
        t1 = self.suit, self.rank
        t2 = other.suit, other.rank
        return t1 < t2
\end{verbatim}
%
As an exercise, write an \verb"__lt__" method for Time objects.  You
can use tuple comparison, but you also might consider 
comparing integers.


\section{Decks}
\index{list!of objects}
\index{deck, playing cards}

Now that we have Cards, the next step is to define Decks.  Since a
deck is made up of cards, it is natural for each Deck to contain a
list of cards as an attribute.
\index{init method}
\index{method!init}

The following is a class definition for {\tt Deck}.  The
init method creates the attribute {\tt cards} and generates
the standard set of fifty-two cards:
\index{composition}
\index{loop!nested}
\index{Deck class}
\index{class!Deck}

\begin{verbatim}
class Deck:

    def __init__(self):
        self.cards = []
        for suit in range(4):
            for rank in range(1, 14):
                card = Card(suit, rank)
                self.cards.append(card)
\end{verbatim}
%
The easiest way to populate the deck is with a nested loop.  The outer
loop enumerates the suits from 0 to 3.  The inner loop enumerates the
ranks from 1 to 13.  Each iteration
creates a new Card with the current suit and rank,
and appends it to {\tt self.cards}.
\index{append method}
\index{method!append}


\section{Printing the deck}
\label{printdeck}
\index{str method@\_\_str\_\_ method}
\index{method!\_\_str\_\_}

Here is a \verb"__str__" method for {\tt Deck}:

\begin{verbatim}
#inside class Deck:

    def __str__(self):
        res = []
        for card in self.cards:
            res.append(str(card))
        return '\n'.join(res)
\end{verbatim}
%
This method demonstrates an efficient way to accumulate a large
string: building a list of strings and then using the string method
{\tt join}.  The built-in function {\tt str} invokes the
\verb"__str__" method on each card and returns the string
representation.  \index{accumulator!string} \index{string!accumulator}
\index{join method} \index{method!join} \index{newline}

Since we invoke {\tt join} on a newline character, the cards
are separated by newlines.  Here's what the result looks like:

\begin{verbatim}
>>> deck = Deck()
>>> print(deck)
Ace of Clubs
2 of Clubs
3 of Clubs
...
10 of Spades
Jack of Spades
Queen of Spades
King of Spades
\end{verbatim}
%
Even though the result appears on 52 lines, it is
one long string that contains newlines.


\section{Add, remove, shuffle and sort}

To deal cards, we would like a method that
removes a card from the deck and returns it.
The list method {\tt pop} provides a convenient way to do that:
\index{pop method}
\index{method!pop}

\begin{verbatim}
#inside class Deck:

    def pop_card(self):
        return self.cards.pop()
\end{verbatim}
%
Since {\tt pop} removes the {\em last} card in the list, we are
dealing from the bottom of the deck.
\index{append method}
\index{method!append}

To add a card, we can use the list method {\tt append}:

\begin{verbatim}
#inside class Deck:

    def add_card(self, card):
        self.cards.append(card)
\end{verbatim}
%
A method like this that uses another method without doing
much work is sometimes called a {\bf veneer}.  The metaphor
comes from woodworking, where a veneer is a thin
layer of good quality wood glued to the surface of a cheaper piece of
wood to improve the appearance.
\index{veneer}

In this case \verb"add_card" is a ``thin'' method that expresses
a list operation in terms appropriate for decks.  It
improves the appearance, or interface, of the
implementation.

As another example, we can write a Deck method named {\tt shuffle}
using the function {\tt shuffle} from the {\tt random} module:
\index{random module}
\index{module!random}
\index{shuffle function}
\index{function!shuffle}

\begin{verbatim}
# inside class Deck:
            
    def shuffle(self):
        random.shuffle(self.cards)
\end{verbatim}
%
Don't forget to import {\tt random}.

As an exercise, write a Deck method named {\tt sort} that uses the
list method {\tt sort} to sort the cards in a {\tt Deck}.  {\tt sort}
uses the \verb"__lt__" method we defined to determine the order.
\index{sort method} \index{method!sort}



\section{Inheritance}
\index{inheritance}
\index{object-oriented programming}

Inheritance is the ability to define a new class that is a modified
version of an existing class.  As an example, let's say we want a
class to represent a ``hand'', that is, the cards held by one player.
A hand is similar to a deck: both are made up of a collection of
cards, and both require operations like adding and removing cards.

A hand is also different from a deck; there are operations we want for
hands that don't make sense for a deck.  For example, in poker we
might compare two hands to see which one wins.  In bridge, we might
compute a score for a hand in order to make a bid.

This relationship between classes---similar, but different---lends
itself to inheritance.
To define a new class that inherits from an existing class,
you put the name of the existing class in parentheses:
\index{parentheses!parent class in}
\index{parent class}
\index{class!parent}
\index{Hand class}
\index{class!Hand}

\begin{verbatim}
class Hand(Deck):
    """Represents a hand of playing cards."""
\end{verbatim}
%
This definition indicates that {\tt Hand} inherits from {\tt Deck};
that means we can use methods like \verb"pop_card" and \verb"add_card"
for Hands as well as Decks.

When a new class inherits from an existing one, the existing
one is called the {\bf parent} and the new class is
called the {\bf child}.
\index{parent class}
\index{child class}
\index{class!child}

In this example, {\tt Hand} inherits \verb"__init__" from {\tt Deck},
but it doesn't really do what we want: instead of populating the hand
with 52 new cards, the init method for Hands should initialize {\tt
  cards} with an empty list.  \index{override} \index{init method}
\index{method!init}

If we provide an init method in the {\tt Hand} class, it overrides the
one in the {\tt Deck} class:

\begin{verbatim}
# inside class Hand:

    def __init__(self, label=''):
        self.cards = []
        self.label = label
\end{verbatim}
%
When you create a Hand, Python invokes this init method, not the
one in {\tt Deck}.

\begin{verbatim}
>>> hand = Hand('new hand')
>>> hand.cards
[]
>>> hand.label
'new hand'
\end{verbatim}
%
The other methods are inherited from {\tt Deck}, so we can use
\verb"pop_card" and \verb"add_card" to deal a card:

\begin{verbatim}
>>> deck = Deck()
>>> card = deck.pop_card()
>>> hand.add_card(card)
>>> print(hand)
King of Spades
\end{verbatim}
%
A natural next step is to encapsulate this code in a method
called \verb"move_cards":
\index{encapsulation}

\begin{verbatim}
#inside class Deck:

    def move_cards(self, hand, num):
        for i in range(num):
            hand.add_card(self.pop_card())
\end{verbatim}
%
\verb"move_cards" takes two arguments, a Hand object and the number of
cards to deal.  It modifies both {\tt self} and {\tt hand}, and
returns {\tt None}.

In some games, cards are moved from one hand to another,
or from a hand back to the deck.  You can use \verb"move_cards"
for any of these operations: {\tt self} can be either a Deck
or a Hand, and {\tt hand}, despite the name, can also be a {\tt Deck}.

Inheritance is a useful feature.  Some programs that would be
repetitive without inheritance can be written more elegantly
with it.  Inheritance can facilitate code reuse, since you can
customize the behavior of parent classes without having to modify
them.  In some cases, the inheritance structure reflects the natural
structure of the problem, which makes the design easier to
understand.

On the other hand, inheritance can make programs difficult to read.
When a method is invoked, it is sometimes not clear where to find its
definition.  The relevant code may be spread across several modules.
Also, many of the things that can be done using inheritance can be
done as well or better without it.


\section{Class diagrams}
\label{class.diagram}

So far we have seen stack diagrams, which show the state of
a program, and object diagrams, which show the attributes
of an object and their values.  These diagrams represent a snapshot
in the execution of a program, so they change as the program
runs.

They are also highly detailed; for some purposes, too
detailed.  A class diagram is a more abstract representation
of the structure of a program.  Instead of showing individual
objects, it shows classes and the relationships between them.

There are several kinds of relationship between classes:

\begin{itemize}

\item Objects in one class might contain references to objects
in another class.  For example, each Rectangle contains a reference
to a Point, and each Deck contains references to many Cards.
This kind of relationship is called {\bf HAS-A}, as in, ``a Rectangle
has a Point.''

\item One class might inherit from another.  This relationship
is called {\bf IS-A}, as in, ``a Hand is a kind of a Deck.''

\item One class might depend on another in the sense that objects
in one class take objects in the second class as parameters, or
use objects in the second class as part of a computation.  This
kind of relationship is called a {\bf dependency}.

\end{itemize}
\index{IS-A relationship}
\index{HAS-A relationship}
\index{class diagram}
\index{diagram!class}

A {\bf class diagram} is a graphical representation of these
relationships.  For example, Figure~\ref{fig.class1} shows the
relationships between {\tt Card}, {\tt Deck} and {\tt Hand}.

\begin{figure}
\centerline
{\includegraphics[scale=0.8]{figs/class1.pdf}}
\caption{Class diagram.}
\label{fig.class1}
\end{figure}

The arrow with a hollow triangle head represents an IS-A
relationship; in this case it indicates that Hand inherits
from Deck.

The standard arrow head represents a HAS-A
relationship; in this case a Deck has references to Card
objects.
\index{multiplicity (in class diagram)}

The star ({\tt *}) near the arrow head is a 
{\bf multiplicity}; it indicates how many Cards a Deck has.
A multiplicity can be a simple number, like {\tt 52}, a range,
like {\tt 5..7} or a star, which indicates that a Deck can
have any number of Cards.

There are no dependencies in this diagram.  They would normally
be shown with a dashed arrow.  Or if there are a lot of
dependencies, they are sometimes omitted.

A more detailed diagram might show that a Deck actually
contains a {\em list} of Cards, but built-in types
like list and dict are usually not included in class diagrams.


\section{디버깅}
%Debugging
\index{debugging}

Inheritance can make debugging difficult because when you invoke a
method on an object, it might be hard to figure out which method will
be invoked.  
\index{inheritance}

Suppose you are writing a function that works with Hand objects.
You would like it to work with all kinds of Hands, like
PokerHands, BridgeHands, etc.  If you invoke a method like
{\tt shuffle}, you might get the one defined in {\tt Deck},
but if any of the subclasses override this method, you'll
get that version instead.  This behavior is usually a good
thing, but it can be confusing.

Any time you are unsure about the flow of execution through your
program, the simplest solution is to add print statements at the
beginning of the relevant methods.  If {\tt Deck.shuffle} prints a
message that says something like {\tt Running Deck.shuffle}, then as
the program runs it traces the flow of execution.
\index{flow of execution}

As an alternative, you could use this function, which takes an
object and a method name (as a string) and returns the class that
provides the definition of the method:

\begin{verbatim}
def find_defining_class(obj, meth_name):
    for ty in type(obj).mro():
        if meth_name in ty.__dict__:
            return ty
\end{verbatim}
%
Here's an example:

\begin{verbatim}
>>> hand = Hand()
>>> find_defining_class(hand, 'shuffle')
<class 'Card.Deck'>
\end{verbatim}
%
So the {\tt shuffle} method for this Hand is the one in {\tt Deck}.
\index{mro method}
\index{method!mro}
\index{method resolution order}

\verb"find_defining_class" uses the {\tt mro} method to get the list
of class objects (types) that will be searched for methods.  ``MRO''
stands for ``method resolution order'', which is the sequence of
classes Python searches to ``resolve'' a method name.

Here's a design suggestion: when you override a method,
the interface of the new method should be the same as the old.  It
should take the same parameters, return the same type, and obey the
same preconditions and postconditions.  If you follow this rule, you
will find that any function designed to work with an instance of a
parent class, like a Deck, will also work with instances of child
classes like a Hand and PokerHand.
\index{override}
\index{interface}
\index{precondition}
\index{postcondition}

If you violate this rule, which is called the ``Liskov substitution
principle'', your code will collapse like (sorry) a house of cards.
\index{Liskov substitution principle}


\section{Data encapsulation}

The previous chapters demonstrate a development plan we might call
``object-oriented design''.  We identified objects we needed---like
{\tt Point}, {\tt Rectangle} and {\tt Time}---and defined classes to
represent them.  In each case there is an obvious correspondence
between the object and some entity in the real world (or at least a
mathematical world).  
\index{development plan!data encapsulation}

But sometimes it is less obvious what objects you need
and how they should interact.  In that case you need a different
development plan.  In the same way that we discovered function
interfaces by encapsulation and generalization, we can discover
class interfaces by {\bf data encapsulation}.
\index{data encapsulation}

Markov analysis, from Section~\ref{markov}, provides a good example.
If you download my code from \url{http://thinkpython2.com/code/markov.py},
you'll see that it uses two global variables---\verb"suffix_map" and
\verb"prefix"---that are read and written from several functions.

\begin{verbatim}
suffix_map = {}        
prefix = ()            
\end{verbatim}

Because these variables are global, we can only run one analysis at a
time.  If we read two texts, their prefixes and suffixes would be
added to the same data structures (which makes for some interesting
generated text).

To run multiple analyses, and keep them separate, we can encapsulate
the state of each analysis in an object.
Here's what that looks like:

\begin{verbatim}
class Markov:

    def __init__(self):
        self.suffix_map = {}
        self.prefix = ()    
\end{verbatim}

Next, we transform the functions into methods.  For example,
here's \verb"process_word":

\begin{verbatim}
    def process_word(self, word, order=2):
        if len(self.prefix) < order:
            self.prefix += (word,)
            return

        try:
            self.suffix_map[self.prefix].append(word)
        except KeyError:
            # if there is no entry for this prefix, make one
            self.suffix_map[self.prefix] = [word]

        self.prefix = shift(self.prefix, word)        
\end{verbatim}

Transforming a program like this---changing the design without
changing the behavior---is another example of refactoring
(see Section~\ref{refactoring}).
\index{refactoring}

This example suggests a development plan for designing objects and
methods:

\begin{enumerate}

\item Start by writing functions that read and write global
variables (when necessary).

\item Once you get the program working, look for associations
between global variables and the functions that use them.

\item Encapsulate related variables as attributes of an object.

\item Transform the associated functions into methods of the new
class.

\end{enumerate}

As an exercise, download my Markov code from
\url{http://thinkpython2.com/code/markov.py}, and follow the steps
described above to encapsulate the global variables as attributes of a
new class called {\tt Markov}.  Solution:
\url{http://thinkpython2.com/code/Markov.py} (note the capital M).


\section{용어 해설}
%Glossary

\begin{description}

\item[encode:]  To represent one set of values using another
set of values by constructing a mapping between them.
\index{encode}

\item[class attribute:] An attribute associated with a class
object.  Class attributes are defined inside
a class definition but outside any method.
\index{class attribute}
\index{attribute!class}

\item[instance attribute:] An attribute associated with an
instance of a class.
\index{instance attribute}
\index{attribute!instance}

\item[veneer:] A method or function that provides a different
interface to another function without doing much computation.
\index{veneer}

\item[inheritance:] The ability to define a new class that is a
modified version of a previously defined class.
\index{inheritance}

\item[parent class:] The class from which a child class inherits.
\index{parent class}

\item[child class:] A new class created by inheriting from an
existing class; also called a ``subclass''.
\index{child class}
\index{class!child}

\item[IS-A relationship:] A relationship between a child class
and its parent class.
\index{IS-A relationship}

\item[HAS-A relationship:] A relationship between two classes
where instances of one class contain references to instances of
the other.
\index{HAS-A relationship}

\item[dependency:] A relationship between two classes
where instances of one class use instances of the other class,
but do not store them as attributes.
\index{HAS-A relationship}

\item[class diagram:] A diagram that shows the classes in a program
and the relationships between them.
\index{class diagram}
\index{diagram!class}

\item[multiplicity:] A notation in a class diagram that shows, for
a HAS-A relationship, how many references there are to instances
of another class.
\index{multiplicity (in class diagram)}

\item[data encapsulation:]  A program development plan that
involves a prototype using global variables and a final version
that makes the global variables into instance attributes. 
\index{data encapsulation}
\index{development plan!data encapsulation}

\end{description}


\section{연습 문제}
%Exercises

\begin{exercise}
For the following program, draw a UML class diagram that shows
these classes and the relationships among them.

\begin{verbatim}
class PingPongParent:
    pass

class Ping(PingPongParent):
    def __init__(self, pong):
        self.pong = pong


class Pong(PingPongParent):
    def __init__(self, pings=None):
        if pings is None:
            self.pings = []
        else:
            self.pings = pings

    def add_ping(self, ping):
        self.pings.append(ping)

pong = Pong()
ping = Ping(pong)
pong.add_ping(ping)
\end{verbatim}


\end{exercise}



\begin{exercise}
Write a Deck method called \verb"deal_hands" that
takes two parameters, the number of hands and the number of cards per
hand.  It should create the appropriate number of Hand objects, deal
the appropriate number of cards per hand, and return a list of Hands.
\end{exercise}


\begin{exercise}
\label{poker}

The following are the possible hands in poker, in increasing order
of value and decreasing order of probability:
\index{poker}

\begin{description}

\item[pair:] two cards with the same rank
\vspace{-0.05in}

\item[two pair:] two pairs of cards with the same rank
\vspace{-0.05in}

\item[three of a kind:] three cards with the same rank
\vspace{-0.05in}

\item[straight:] five cards with ranks in sequence (aces can
be high or low, so {\tt Ace-2-3-4-5} is a straight and so is {\tt
10-Jack-Queen-King-Ace}, but {\tt Queen-King-Ace-2-3} is not.)
\vspace{-0.05in}

\item[flush:] five cards with the same suit
\vspace{-0.05in}

\item[full house:] three cards with one rank, two cards with another
\vspace{-0.05in}

\item[four of a kind:] four cards with the same rank
\vspace{-0.05in}

\item[straight flush:] five cards in sequence (as defined above) and
with the same suit
\vspace{-0.05in}

\end{description}
%
The goal of these exercises is to estimate
the probability of drawing these various hands.

\begin{enumerate}

\item Download the following files from \url{http://thinkpython2.com/code}:

\begin{description}

\item[{\tt Card.py}]: A complete version of the {\tt Card},
{\tt Deck} and {\tt Hand} classes in this chapter.

\item[{\tt PokerHand.py}]: An incomplete implementation of a class
that represents a poker hand, and some code that tests it.

\end{description}
%
\item If you run {\tt PokerHand.py}, it deals seven 7-card poker hands
and checks to see if any of them contains a flush.  Read this
code carefully before you go on.

\item Add methods to {\tt PokerHand.py} named \verb"has_pair",
\verb"has_twopair", etc. that return True or False according to
whether or not the hand meets the relevant criteria.  Your code should
work correctly for ``hands'' that contain any number of cards
(although 5 and 7 are the most common sizes).

\item Write a method named {\tt classify} that figures out
the highest-value classification for a hand and sets the
{\tt label} attribute accordingly.  For example, a 7-card hand
might contain a flush and a pair; it should be labeled ``flush''.

\item When you are convinced that your classification methods are
working, the next step is to estimate the probabilities of the various
hands.  Write a function in {\tt PokerHand.py} that shuffles a deck of
cards, divides it into hands, classifies the hands, and counts the
number of times various classifications appear.

\item Print a table of the classifications and their probabilities.
Run your program with larger and larger numbers of hands until the
output values converge to a reasonable degree of accuracy.  Compare
your results to the values at \url{http://en.wikipedia.org/wiki/Hand_rankings}.

\end{enumerate}

Solution: \url{http://thinkpython2.com/code/PokerHandSoln.py}.
\end{exercise}


\chapter{The Goodies}

One of my goals for this book has been to teach you as little Python
as possible.  When there were two ways to do something, I picked 
one and avoided mentioning the other.  Or sometimes I put the second
one into an exercise.

Now I want to go back for some of the good bits that got left behind.
Python provides a number of features that are not really necessary---you
can write good code without them---but with them you can sometimes
write code that's more concise, readable or efficient, and sometimes
all three.

% TODO: add the with statement

\section{Conditional expressions}

We saw conditional statements in Section~\ref{conditional.execution}.
Conditional statements are often used to choose one of two values;
for example:
\index{conditional expression}
\index{expression!conditional}

\begin{verbatim}
if x > 0:
    y = math.log(x)
else:
    y = float('nan')
\end{verbatim}

This statement checks whether {\tt x} is positive.  If so, it computes
{\tt math.log}.  If not, {\tt math.log} would raise a ValueError.  To
avoid stopping the program, we generate a ``NaN'', which is a special
floating-point value that represents ``Not a Number''.
\index{NaN}
\index{floating-point}

We can write this statement more concisely using a {\bf conditional
expression}:

\begin{verbatim}
y = math.log(x) if x > 0 else float('nan')
\end{verbatim}

You can almost read this line like English: ``{\tt y} gets log-{\tt x}
if {\tt x} is greater than 0; otherwise it gets NaN''.

Recursive functions can sometimes be rewritten using conditional
expressions.  For example, here is a recursive version of {\tt factorial}:
\index{factorial}
\index{function!factorial}

\begin{verbatim}
def factorial(n):
    if n == 0:
        return 1
    else:
        return n * factorial(n-1)
\end{verbatim}

We can rewrite it like this:

\begin{verbatim}
def factorial(n):
    return 1 if n == 0 else n * factorial(n-1)
\end{verbatim}

Another use of conditional expressions is handling optional
arguments.  For example, here is the init method from
{\tt GoodKangaroo} (see Exercise~\ref{kangaroo}):
\index{optional argument}
\index{argument!optional}

\begin{verbatim}
    def __init__(self, name, contents=None):
        self.name = name
        if contents == None:
            contents = []
        self.pouch_contents = contents
\end{verbatim}

We can rewrite this one like this:

\begin{verbatim}
    def __init__(self, name, contents=None):
        self.name = name
        self.pouch_contents = [] if contents == None else contents 
\end{verbatim}

In general, you can replace a conditional statement with a conditional
expression if both branches contain simple expressions that are
either returned or assigned to the same variable.
\index{conditional statement}
\index{statement!conditional}



\section{List comprehensions}

In Section~\ref{filter} we saw the map and filter patterns.  For
example, this function takes a list of strings, maps the string method
{\tt capitalize} to the elements, and returns a new list of strings:

\begin{verbatim}
def capitalize_all(t):
    res = []
    for s in t:
        res.append(s.capitalize())
    return res
\end{verbatim}

We can write this more concisely using a {\bf list comprehension}:
\index{list comprehension}

\begin{verbatim}
def capitalize_all(t):
    return [s.capitalize() for s in t]
\end{verbatim}

The bracket operators indicate that we are constructing a new
list.  The expression inside the brackets specifies the elements
of the list, and the {\tt for} clause indicates what sequence
we are traversing.
\index{list}
\index{for loop}

The syntax of a list comprehension is a little awkward because
the loop variable, {\tt s} in this example, appears in the expression
before we get to the definition.
\index{loop variable}

List comprehensions can also be used for filtering.  For example,
this function selects only the elements of {\tt t} that are
upper case, and returns a new list:
\index{filter pattern}
\index{pattern!filter}

\begin{verbatim}
def only_upper(t):
    res = []
    for s in t:
        if s.isupper():
            res.append(s)
    return res
\end{verbatim}

We can rewrite it using a list comprehension

\begin{verbatim}
def only_upper(t):
    return [s for s in t if s.isupper()]
\end{verbatim}

List comprehensions are concise and easy to read, at least for simple
expressions.  And they are usually faster than the equivalent for
loops, sometimes much faster.  So if you are mad at me for not
mentioning them earlier, I understand.

But, in my defense, list comprehensions are harder to debug because
you can't put a print statement inside the loop.  I suggest that you
use them only if the computation is simple enough that you are likely
to get it right the first time.  And for beginners that means never.
\index{debugging}



\section{Generator expressions}

{\bf Generator expressions} are similar to list comprehensions, but
with parentheses instead of square brackets:
\index{generator expression}
\index{expression!generator}

\begin{verbatim}
>>> g = (x**2 for x in range(5))
>>> g
<generator object <genexpr> at 0x7f4c45a786c0>
\end{verbatim}
%
The result is a generator object that knows how to iterate through
a sequence of values.  But unlike a list comprehension, it does not
compute the values all at once; it waits to be asked.  The built-in
function {\tt next} gets the next value from the generator:
\index{generator object}
\index{object!generator}

\begin{verbatim}
>>> next(g)
0
>>> next(g)
1
\end{verbatim}
%
When you get to the end of the sequence, {\tt next} raises a 
StopIteration exception.  You can also use a {\tt for} loop to iterate
through the values:
\index{StopIteration}
\index{exception!StopIteration}

\begin{verbatim}
>>> for val in g:
...     print(val)
4
9
16
\end{verbatim}
%
The generator object keeps track of where it is in the sequence,
so the {\tt for} loop picks up where {\tt next} left off.  Once the
generator is exhausted, it continues to raise {\tt StopException}:

\begin{verbatim}
>>> next(g)
StopIteration
\end{verbatim}

Generator expressions are often used with functions like {\tt sum},
{\tt max}, and {\tt min}:
\index{sum}
\index{function!sum}

\begin{verbatim}
>>> sum(x**2 for x in range(5))
30
\end{verbatim}


\section{{\tt any} and {\tt all}}

Python provides a built-in function, {\tt any}, that takes a sequence
of boolean values and returns {\tt True} if any of the values are {\tt
  True}.  It works on lists:
\index{any}
\index{built-in function!any}

\begin{verbatim}
>>> any([False, False, True])
True
\end{verbatim}
%
But it is often used with generator expressions:
\index{generator expression}
\index{expression!generator}

\begin{verbatim}
>>> any(letter == 't' for letter in 'monty')
True
\end{verbatim}
%
That example isn't very useful because it does the same thing
as the {\tt in} operator.  But we could use {\tt any} to rewrite
some of the search functions we wrote in Section~\ref{search}.  For
example, we could write {\tt avoids} like this:
\index{search pattern}
\index{pattern!search}

\begin{verbatim}
def avoids(word, forbidden):
    return not any(letter in forbidden for letter in word)
\end{verbatim}
%
The function almost reads like English, ``{\tt word} avoids
{\tt forbidden} if there are not any forbidden letters in {\tt word}.''

Using {\tt any} with a generator expression is efficient because
it stops immediately if it finds a {\tt True} value,
so it doesn't have to evaluate the whole sequence.

Python provides another built-in function, {\tt all}, that returns
{\tt True} if every element of the sequence is {\tt True}.  As
an exercise, use {\tt all} to re-write \verb"uses_all" from
Section~\ref{search}.
\index{all}
\index{built-in function!any}


\section{Sets}
\label{sets}

In Section~\ref{dictsub} I use dictionaries to find the words
that appear in a document but not in a word list.  The function
I wrote takes {\tt d1}, which contains the words from the document
as keys, and {\tt d2}, which contains the list of words.  It
returns a dictionary that contains the keys from {\tt d1} that
are not in {\tt d2}.

\begin{verbatim}
def subtract(d1, d2):
    res = dict()
    for key in d1:
        if key not in d2:
            res[key] = None
    return res
\end{verbatim}
%
In all of these dictionaries, the values are {\tt None} because
we never use them.  As a result, we waste some storage space.
\index{dictionary subtraction}

Python provides another built-in type, called a {\tt set}, that
behaves like a collection of dictionary keys with no values.  Adding
elements to a set is fast; so is checking membership.  And sets
provide methods and operators to compute common set operations.
\index{set}
\index{object!set}

For example, set subtraction is available as a method called
{\tt difference} or as an operator, {\tt -}.  So we can rewrite
{\tt subtract} like this:
\index{set subtraction}

\begin{verbatim}
def subtract(d1, d2):
    return set(d1) - set(d2)
\end{verbatim}
%
The result is a set instead of a dictionary, but for operations like
iteration, the behavior is the same.

Some of the exercises in this book can be done concisely and
efficiently with sets.  For example, here is a solution to
\verb"has_duplicates", from
Exercise~\ref{duplicate}, that uses a dictionary:

\begin{verbatim}
def has_duplicates(t):
    d = {}
    for x in t:
        if x in d:
            return True
        d[x] = True
    return False
\end{verbatim}

When an element appears for the first time, it is added to the
dictionary.  If the same element appears again, the function returns
{\tt True}.

Using sets, we can write the same function like this:

\begin{verbatim}
def has_duplicates(t):
    return len(set(t)) < len(t)
\end{verbatim}
%
An element can only appear in a set once, so if an element in {\tt t}
appears more than once, the set will be smaller than {\tt t}.  If there
are no duplicates, the set will be the same size as {\tt t}.
\index{duplicate}

We can also use sets to do some of the exercises in
Chapter~\ref{wordplay}.  For example, here's a version of
\verb"uses_only" with a loop:

\begin{verbatim}
def uses_only(word, available):
    for letter in word: 
        if letter not in available:
            return False
    return True
\end{verbatim}
%
\verb"uses_only" checks whether all letters in {\tt word} are
in {\tt available}.  We can rewrite it like this:

\begin{verbatim}
def uses_only(word, available):
    return set(word) <= set(available)
\end{verbatim}
%
The \verb"<=" operator checks whether one set is a subset or another,
including the possibility that they are equal, which is true if all
the letters in {\tt word} appear in {\tt available}.
\index{subset}

As an exercise, rewrite \verb"avoids" using sets.


\section{Counters}

A Counter is like a set, except that if an element appears more
than once, the Counter keeps track of how many times it appears.
If you are familiar with the mathematical idea of a {\bf multiset},
a Counter is a natural way to represent a multiset.
\index{Counter}
\index{object!Counter}
\index{multiset}

Counter is defined in a standard module called {\tt collections},
so you have to import it.  You can initialize a Counter with a string,
list, or anything else that supports iteration:
\index{collections}
\index{module!collections}

\begin{verbatim}
>>> from collections import Counter
>>> count = Counter('parrot')
>>> count
Counter({'r': 2, 't': 1, 'o': 1, 'p': 1, 'a': 1})
\end{verbatim}

Counters behave like dictionaries in many ways; they map from each
key to the number of times it appears.  As in dictionaries,
the keys have to be hashable.

Unlike dictionaries, Counters don't raise an exception if you access
an element that doesn't appear.  Instead, they return 0:

\begin{verbatim}
>>> count['d']
0
\end{verbatim}

We can use Counters to rewrite \verb"is_anagram" from
Exercise~\ref{anagram}:

\begin{verbatim}
def is_anagram(word1, word2):
    return Counter(word1) == Counter(word2)
\end{verbatim}

If two words are anagrams, they contain the same letters with the same
counts, so their Counters are equivalent.

Counters provide methods and operators to perform set-like operations,
including addition, subtraction, union and intersection.  And
they provide an often-useful method, \verb"most_common", which
returns a list of value-frequency pairs, sorted from most common to
least:

\begin{verbatim}
>>> count = Counter('parrot')
>>> for val, freq in count.most_common(3):
...     print(val, freq)
r 2
p 1
a 1
\end{verbatim}


\section{defaultdict}

The {\tt collections} module also provides {\tt defaultdict}, which is
like a dictionary except that if you access a key that doesn't exist,
it can generate a new value on the fly.
\index{defaultdict}
\index{object!defaultdict}
\index{collections}
\index{module!collections}

When you create a defaultdict, you provide a function that's used to
create new values.  A function used to create objects is sometimes
called a {\bf factory}.  The built-in functions that create lists, sets,
and other types can be used as factories:
\index{factory function}

\begin{verbatim}
>>> from collections import defaultdict
>>> d = defaultdict(list)
\end{verbatim}

Notice that the argument is {\tt list}, which is a class object,
not {\tt list()}, which is a new list.  The function you provide
doesn't get called unless you access a key that doesn't exist.

\begin{verbatim}
>>> t = d['new key']
>>> t
[]
\end{verbatim}

The new list, which we're calling {\tt t}, is also added to the
dictionary.  So if we modify {\tt t}, the change appears in {\tt d}:

\begin{verbatim}
>>> t.append('new value')
>>> d
defaultdict(<class 'list'>, {'new key': ['new value']})
\end{verbatim}

If you are making a dictionary of lists, you can often write simpler
code using {\tt defaultdict}.  In my solution to
Exercise~\ref{anagrams}, which you can get from
\url{http://thinkpython2.com/code/anagram_sets.py}, I make a
dictionary that maps from a sorted string of letters to the list of
words that can be spelled with those letters.  For example, {\tt
  'opst'} maps to the list {\tt ['opts', 'post', 'pots', 'spot',
    'stop', 'tops']}.

Here's the original code:

\begin{verbatim}
def all_anagrams(filename):
    d = {}
    for line in open(filename):
        word = line.strip().lower()
        t = signature(word)
        if t not in d:
            d[t] = [word]
        else:
            d[t].append(word)
    return d
\end{verbatim}

This can be simplified using {\tt setdefault}, which you might
have used in Exercise~\ref{setdefault}:
\index{setdefault}

\begin{verbatim}
def all_anagrams(filename):
    d = {}
    for line in open(filename):
        word = line.strip().lower()
        t = signature(word)
        d.setdefault(t, []).append(word)
    return d
\end{verbatim}

This solution has the drawback that it makes a new list
every time, regardless of whether it is needed.  For lists,
that's no big deal, but if the factory
function is complicated, it might be.
\index{factory function}

We can avoid this problem and 
simplify the code using a {\tt defaultdict}:

\begin{verbatim}
def all_anagrams(filename):
    d = defaultdict(list)
    for line in open(filename):
        word = line.strip().lower()
        t = signature(word)
        d[t].append(word)
    return d
\end{verbatim}

My solution to Exercise~\ref{poker}, which you can download from
\url{http://thinkpython2.com/code/PokerHandSoln.py},
uses {\tt setdefault} in the function
\verb"has_straightflush".  This solution has the drawback
of creating a {\tt Hand} object every time through the loop, whether
it is needed or not.  As an exercise, rewrite it using
a defaultdict.


\section{Named tuples}

Many simple objects are basically collections of related values.
For example, the Point object defined in Chapter~\ref{clobjects} contains
two numbers, {\tt x} and {\tt y}.  When you define a class like
this, you usually start with an init method and a str method:

\begin{verbatim}
class Point:

    def __init__(self, x=0, y=0):
        self.x = x
        self.y = y

    def __str__(self):
        return '(%g, %g)' % (self.x, self.y)
\end{verbatim}

This is a lot of code to convey a small amount of information.
Python provides a more concise way to say the same thing:

\begin{verbatim}
from collections import namedtuple
Point = namedtuple('Point', ['x', 'y'])
\end{verbatim}

The first argument is the name of the class you want to create.
The second is a list of the attributes Point objects should have,
as strings.  The return value from {\tt namedtuple} is a class object:
\index{namedtuple}
\index{object!namedtuple}
\index{collections}
\index{module!collections}

\begin{verbatim}
>>> Point
<class '__main__.Point'>
\end{verbatim}

{\tt Point} automatically provides methods like \verb"__init__" and
\verb"__str__" so you don't have to write them.
\index{class object}
\index{object!class}

To create a Point object, you use the Point class as a function:

\begin{verbatim}
>>> p = Point(1, 2)
>>> p
Point(x=1, y=2)
\end{verbatim}

The init method assigns the arguments to attributes using the names
you provided.  The str method prints a representation of the Point
object and its attributes.

You can access the elements of the named tuple by name:

\begin{verbatim}
>>> p.x, p.y
(1, 2)
\end{verbatim}

But you can also treat a named tuple as a tuple:

\begin{verbatim}
>>> p[0], p[1]
(1, 2)

>>> x, y = p
>>> x, y
(1, 2)
\end{verbatim}

Named tuples provide a quick way to define simple classes.
The drawback is that simple classes don't always stay simple.
You might decide later that you want to add methods to a named tuple.
In that case, you could define a new class that inherits from
the named tuple:
\index{inheritance}

\begin{verbatim}
class Pointier(Point):
    # add more methods here
\end{verbatim}

Or you could switch to a conventional class definition.


\section{Gathering keyword args}

In Section~\ref{gather}, we saw how to write a function that
gathers its arguments into a tuple:
\index{gather}

\begin{verbatim}
def printall(*args):
    print(args)
\end{verbatim}
%
You can call this function with any number of positional arguments
(that is, arguments that don't have keywords):
\index{positional argument}
\index{argument!positional}

\begin{verbatim}
>>> printall(1, 2.0, '3')
(1, 2.0, '3')
\end{verbatim}
%
But the {\tt *} operator doesn't gather keyword arguments:
\index{keyword argument}
\index{argument!keyword}

\begin{verbatim}
>>> printall(1, 2.0, third='3')
TypeError: printall() got an unexpected keyword argument 'third'
\end{verbatim}
%
To gather keyword arguments, you can use the {\tt **} operator:

\begin{verbatim}
def printall(*args, **kwargs):
    print(args, kwargs)
\end{verbatim}
%
You can call the keyword gathering parameter anything you want, but
{\tt kwargs} is a common choice.  The result is a dictionary that maps
keywords to values:

\begin{verbatim}
>>> printall(1, 2.0, third='3')
(1, 2.0) {'third': '3'}
\end{verbatim}
%
If you have a dictionary of keywords and values, you can use the
scatter operator, {\tt **} to call a function:
\index{scatter}

\begin{verbatim}
>>> d = dict(x=1, y=2)
>>> Point(**d)
Point(x=1, y=2)
\end{verbatim}
%
Without the scatter operator, the function would treat {\tt d} as
a single positional argument, so it would assign {\tt d} to
{\tt x} and complain because there's nothing to assign to {\tt y}:

\begin{verbatim}
>>> d = dict(x=1, y=2)
>>> Point(d)
Traceback (most recent call last):
  File "<stdin>", line 1, in <module>
TypeError: __new__() missing 1 required positional argument: 'y'
\end{verbatim}
%
When you are working with functions that have a large number of
parameters, it is often useful to create and pass around dictionaries
that specify frequently used options.


\section{용어 해설}
%Glossary

\begin{description}

\item[conditional expression:] An expression that has one of two
values, depending on a condition.
\index{conditional expression}
\index{expression!conditional}

\item[list comprehension:] An expression with a {\tt for} loop in square
brackets that yields a new list.
\index{list comprehension}

\item[generator expression:] An expression with a {\tt for} loop in parentheses
that yields a generator object.  
\index{generator expression}
\index{expression!generator}

\item[multiset:] A mathematical entity that represents a mapping
between the elements of a set and the number of times they appear.

\item[factory:] A function, usually passed as a parameter, used to
create objects. 
\index{factory}

\end{description}




\section{연습 문제}
%Exercises

\begin{exercise}

The following is a function computes the binomial
coefficient recursively.

\begin{verbatim}
def binomial_coeff(n, k):
    """Compute the binomial coefficient "n choose k".

    n: number of trials
    k: number of successes

    returns: int
    """
    if k == 0:
        return 1
    if n == 0:
        return 0

    res = binomial_coeff(n-1, k) + binomial_coeff(n-1, k-1)
    return res
\end{verbatim}

Rewrite the body of the function using nested conditional
expressions.

One note: this function is not very efficient because it ends up computing
the same values over and over.  You could make it more efficient by
memoizing (see Section~\ref{memoize}).  But you will find that it's harder to
memoize if you write it using conditional expressions.

\end{exercise}



\appendix

\chapter{Debugging}
\index{debugging}

When you are debugging, you should distinguish among different
kinds of errors in order to track them down more quickly:

\begin{itemize}

\item Syntax errors are discovered by the interpreter when it is
  translating the source code into byte code.  They indicate
  that there is something wrong with the structure of the program.
  Example: Omitting the colon at the end of a {\tt def} statement
  generates the somewhat redundant message {\tt SyntaxError: invalid
    syntax}.
\index{syntax error}
\index{error!syntax}

\item Runtime errors are produced by the interpreter if something goes
  wrong while the program is running.  Most runtime error messages
  include information about where the error occurred and what
  functions were executing.  Example: An infinite recursion eventually
  causes the runtime error ``maximum recursion depth exceeded''.
\index{runtime error}
\index{error!runtime}
\index{exception}

\item Semantic errors are problems with a program that runs without
  producing error messages but doesn't do the right thing.  Example:
  An expression may not be evaluated in the order you expect, yielding
  an incorrect result.
\index{semantic error}
\index{error!semantic}

\end{itemize}

The first step in debugging is to figure out which kind of
error you are dealing with.  Although the following sections are
organized by error type, some techniques are
applicable in more than one situation.


\section{Syntax errors}
\index{error message}

Syntax errors are usually easy to fix once you figure out what they
are.  Unfortunately, the error messages are often not helpful.
The most common messages are {\tt SyntaxError: invalid syntax} and
{\tt SyntaxError: invalid token}, neither of which is very informative.

On the other hand, the message does tell you where in the program the
problem occurred.  Actually, it tells you where Python
noticed a problem, which is not necessarily where the error
is.  Sometimes the error is prior to the location of the error
message, often on the preceding line.
\index{incremental development}
\index{development plan!incremental}

If you are building the program incrementally, you should have
a good idea about where the error is.  It will be in the last
line you added.

If you are copying code from a book, start by comparing
your code to the book's code very carefully.  Check every character.
At the same time, remember that the book might be wrong, so
if you see something that looks like a syntax error, it might be.

Here are some ways to avoid the most common syntax errors:
\index{syntax}

\begin{enumerate}

\item Make sure you are not using a Python keyword for a variable name.
\index{keyword}

\item Check that you have a colon at the end of the header of every
compound statement, including {\tt for}, {\tt while},
{\tt if}, and {\tt def} statements.
\index{header}
\index{colon}

\item Make sure that any strings in the code have matching
quotation marks.  Make sure that all quotation marks are
``straight quotes'', not ``curly quotes''.
\index{quotation mark}

\item If you have multiline strings with triple quotes (single or double), make
sure you have terminated the string properly.  An unterminated string
may cause an {\tt invalid token} error at the end of your program,
or it may treat the following part of the program as a string until it
comes to the next string.  In the second case, it might not produce an error
message at all!
\index{multiline string}
\index{string!multiline}

\item An unclosed opening operator---\verb+(+, \verb+{+, or
  \verb+[+---makes Python continue with the next line as part of the
  current statement.  Generally, an error occurs almost immediately in
  the next line.

\item Check for the classic {\tt =} instead of {\tt ==} inside
a conditional.
\index{conditional}

\item Check the indentation to make sure it lines up the way it
is supposed to.  Python can handle space and tabs, but if you mix
them it can cause problems.  The best way to avoid this problem
is to use a text editor that knows about Python and generates
consistent indentation.
\index{indentation}
\index{whitespace}

\item If you have non-ASCII characters in the code (including strings
and comments), that might cause a problem, although Python 3 usually
handles non-ASCII characters.  Be careful if you paste in text from
a web page or other source.

\end{enumerate}

If nothing works, move on to the next section...


\subsection{I keep making changes and it makes no difference.}

If the interpreter says there is an error and you don't see it, that
might be because you and the interpreter are not looking at the same
code.  Check your programming environment to make sure that the
program you are editing is the one Python is trying to run.

If you are not sure, try putting an obvious and deliberate syntax
error at the beginning of the program.  Now run it again.  If the
interpreter doesn't find the new error, you are not running the
new code.

There are a few likely culprits:

\begin{itemize}

\item You edited the file and forgot to save the changes before
running it again.  Some programming environments do this
for you, but some don't.

\item You changed the name of the file, but you are still running
the old name.

\item Something in your development environment is configured
incorrectly.

\item If you are writing a module and using {\tt import},
make sure you don't give your module the same name as one
of the standard Python modules.

\item If you are using {\tt import} to read a module, remember
that you have to restart the interpreter or use {\tt reload}
to read a modified file.  If you import the module again, it
doesn't do anything.
\index{module!reload}
\index{reload function}
\index{function!reload}

\end{itemize}

If you get stuck and you can't figure out what is going on, one
approach is to start again with a new program like ``Hello, World!'',
and make sure you can get a known program to run.  Then gradually add
the pieces of the original program to the new one.


\section{Runtime errors}

Once your program is syntactically correct,
Python can read it and at least start running it.  What could
possibly go wrong?


\subsection{My program does absolutely nothing.}

This problem is most common when your file consists of functions and
classes but does not actually invoke a function to start execution.
This may be intentional if you only plan to import this module to
supply classes and functions.

If it is not intentional, make sure there is a function call
in the program, and make sure the flow of execution reaches
it (see ``Flow of Execution'' below).


\subsection{My program hangs.}
\index{infinite loop}
\index{infinite recursion}
\index{hanging}

If a program stops and seems to be doing nothing, it is ``hanging''.
Often that means that it is caught in an infinite loop or infinite
recursion.

\begin{itemize}

\item If there is a particular loop that you suspect is the
problem, add a {\tt print} statement immediately before the loop that says
``entering the loop'' and another immediately after that says
``exiting the loop''.

Run the program.  If you get the first message and not the second,
you've got an infinite loop.  Go to the ``Infinite Loop'' section
below.

\item Most of the time, an infinite recursion will cause the program
to run for a while and then produce a ``RuntimeError: Maximum
recursion depth exceeded'' error.  If that happens, go to the
``Infinite Recursion'' section below.

If you are not getting this error but you suspect there is a problem
with a recursive method or function, you can still use the techniques
in the ``Infinite Recursion'' section.

\item If neither of those steps works, start testing other
loops and other recursive functions and methods.

\item If that doesn't work, then it is possible that
you don't understand the flow of execution in your program.
Go to the ``Flow of Execution'' section below.

\end{itemize}


\subsubsection{Infinite Loop}
\index{infinite loop}
\index{loop!infinite}
\index{condition}
\index{loop!condition}

If you think you have an infinite loop and you think you know
what loop is causing the problem, add a {\tt print} statement at
the end of the loop that prints the values of the variables in
the condition and the value of the condition.

For example:

\begin{verbatim}
while x > 0 and y < 0 :
    # do something to x
    # do something to y

    print('x: ', x)
    print('y: ', y)
    print("condition: ", (x > 0 and y < 0))
\end{verbatim}
%
Now when you run the program, you will see three lines of output
for each time through the loop.  The last time through the
loop, the condition should be {\tt False}.  If the loop keeps
going, you will be able to see the values of {\tt x} and {\tt y},
and you might figure out why they are not being updated correctly.


\subsubsection{Infinite Recursion}
\index{infinite recursion}
\index{recursion!infinite}

Most of the time, infinite recursion causes the program to run
for a while and then produce a {\tt Maximum recursion depth exceeded}
error.

If you suspect that a function is causing an infinite
recursion, make sure that there is a base case.
There should be some condition that causes the
function to return without making a recursive invocation.
If not, you need to rethink the algorithm and identify a base
case.

If there is a base case but the program doesn't seem to be reaching
it, add a {\tt print} statement at the beginning of the function
that prints the parameters.  Now when you run the program, you will see
a few lines of output every time the function is invoked,
and you will see the parameter values.  If the parameters are not moving
toward the base case, you will get some ideas about why not.


\subsubsection{Flow of Execution}
\index{flow of execution}

If you are not sure how the flow of execution is moving through
your program, add {\tt print} statements to the beginning of each
function with a message like ``entering function {\tt foo}'', where
{\tt foo} is the name of the function.

Now when you run the program, it will print a trace of each
function as it is invoked.


\subsection{When I run the program I get an exception.}
\index{exception}
\index{runtime error}

If something goes wrong during runtime, Python
prints a message that includes the name of the
exception, the line of the program where the problem occurred,
and a traceback.
\index{traceback}

The traceback identifies the function that is currently running, and
then the function that called it, and then the function that called
{\em that}, and so on.  In other words, it traces the sequence of
function calls that got you to where you are, including the line
number in your file where each call occurred.

The first step is to examine the place in the program where
the error occurred and see if you can figure out what happened.
These are some of the most common runtime errors:

\begin{description}

\item[NameError:]  You are trying to use a variable that doesn't
exist in the current environment.  Check if the name
is spelled right, or at least consistently.
And remember that local variables are local; you
cannot refer to them from outside the function where they are defined.
\index{NameError}
\index{exception!NameError}

\item[TypeError:] There are several possible causes:
\index{TypeError}
\index{exception!TypeError}

\begin{itemize}

\item  You are trying to use a value improperly.  Example: indexing
a string, list, or tuple with something other than an integer.
\index{index}

\item There is a mismatch between the items in a format string and
the items passed for conversion.  This can happen if either the number
of items does not match or an invalid conversion is called for.
\index{format operator}
\index{operator!format}

\item You are passing the wrong number of arguments to a function.
For methods, look at the method definition and
check that the first parameter is {\tt self}.  Then look at the
method invocation; make sure you are invoking the method on an
object with the right type and providing the other arguments
correctly.

\end{itemize}

\item[KeyError:]  You are trying to access an element of a dictionary
using a key that the dictionary does not contain.  If the keys
are strings, remember that capitalization matters.
\index{KeyError}
\index{exception!KeyError}
\index{dictionary}

\item[AttributeError:] You are trying to access an attribute or method
  that does not exist.  Check the spelling!  You can use the built-in
  function {\tt vars} to list the attributes that do exist.
\index{dir function}
\index{function!dir}

If an AttributeError indicates that an object has {\tt NoneType},
that means that it is {\tt None}.  So the problem is not the
attribute name, but the object.

The reason the object is none might be that you forgot
to return a value from a function; if you get to the end of
a function without hitting a {\tt return} statement, it returns
{\tt None}.  Another common cause is using the result from
a list method, like {\tt sort}, that returns {\tt None}.
\index{AttributeError}
\index{exception!AttributeError}

\item[IndexError:] The index you are using
to access a list, string, or tuple is greater than
its length minus one.  Immediately before the site of the error,
add a {\tt print} statement to display
the value of the index and the length of the array.
Is the array the right size?  Is the index the right value?
\index{IndexError}
\index{exception!IndexError}

\end{description}

The Python debugger ({\tt pdb}) is useful for tracking down
exceptions because it allows you to examine the state of the
program immediately before the error.  You can read
about {\tt pdb} at \url{https://docs.python.org/3/library/pdb.html}.
\index{debugger (pdb)}
\index{pdb (Python debugger)}


\subsection{I added so many {\tt print} statements I get inundated with
output.}
\index{print statement}
\index{statement!print}

One of the problems with using {\tt print} statements for debugging
is that you can end up buried in output.  There are two ways
to proceed: simplify the output or simplify the program.

To simplify the output, you can remove or comment out {\tt print}
statements that aren't helping, or combine them, or format
the output so it is easier to understand.

To simplify the program, there are several things you can do.  First,
scale down the problem the program is working on.  For example, if you
are searching a list, search a {\em small} list.  If the program takes
input from the user, give it the simplest input that causes the
problem.
\index{dead code}

Second, clean up the program.  Remove dead code and reorganize the
program to make it as easy to read as possible.  For example, if you
suspect that the problem is in a deeply nested part of the program,
try rewriting that part with simpler structure.  If you suspect a
large function, try splitting it into smaller functions and testing them
separately.
\index{testing!minimal test case}
\index{test case, minimal}

Often the process of finding the minimal test case leads you to the
bug.  If you find that a program works in one situation but not in
another, that gives you a clue about what is going on.

Similarly, rewriting a piece of code can help you find subtle
bugs.  If you make a change that you think shouldn't affect the
program, and it does, that can tip you off.


\section{Semantic errors}

In some ways, semantic errors are the hardest to debug,
because the interpreter provides no information
about what is wrong.  Only you know what the program is supposed to
do.
\index{semantic error}
\index{error!semantic}

The first step is to make a connection between the program
text and the behavior you are seeing.  You need a hypothesis
about what the program is actually doing.  One of the things
that makes that hard is that computers run so fast.

You will often wish that you could slow the program down to human
speed, and with some debuggers you can.  But the time it takes to
insert a few well-placed {\tt print} statements is often short compared to
setting up the debugger, inserting and removing breakpoints, and
``stepping'' the program to where the error is occurring.


\subsection{My program doesn't work.}

You should ask yourself these questions:

\begin{itemize}

\item Is there something the program was supposed to do but
which doesn't seem to be happening?  Find the section of the code
that performs that function and make sure it is executing when
you think it should.

\item Is something happening that shouldn't?  Find code in
your program that performs that function and see if it is
executing when it shouldn't.

\item Is a section of code producing an effect that is not
what you expected?  Make sure that you understand the code in
question, especially if it involves functions or methods in
other Python modules.  Read the documentation for the functions you call.
Try them out by writing simple test cases and checking the results.

\end{itemize}

In order to program, you need a mental model of how
programs work.  If you write a program that doesn't do what you expect,
often the problem is not in the program; it's in your mental
model.
\index{model, mental}
\index{mental model}

The best way to correct your mental model is to break the program
into its components (usually the functions and methods) and test
each component independently.  Once you find the discrepancy
between your model and reality, you can solve the problem.

Of course, you should be building and testing components as you
develop the program.  If you encounter a problem,
there should be only a small amount of new code
that is not known to be correct.


\subsection{I've got a big hairy expression and it doesn't
do what I expect.}
\index{expression!big and hairy}
\index{big, hairy expression}

Writing complex expressions is fine as long as they are readable,
but they can be hard to debug.  It is often a good idea to
break a complex expression into a series of assignments to
temporary variables.

For example:

\begin{verbatim}
self.hands[i].addCard(self.hands[self.findNeighbor(i)].popCard())
\end{verbatim}
%
This can be rewritten as:

\begin{verbatim}
neighbor = self.findNeighbor(i)
pickedCard = self.hands[neighbor].popCard()
self.hands[i].addCard(pickedCard)
\end{verbatim}
%
The explicit version is easier to read because the variable
names provide additional documentation, and it is easier to debug
because you can check the types of the intermediate variables
and display their values.
\index{temporary variable}
\index{variable!temporary}

Another problem that can occur with big expressions is
that the order of evaluation may not be what you expect.
For example, if you are translating the expression
$\frac{x}{2 \pi}$ into Python, you might write:

\begin{verbatim}
y = x / 2 * math.pi
\end{verbatim}
%
That is not correct because multiplication and division have
the same precedence and are evaluated from left to right.
So this expression computes $x \pi / 2$.
\index{order of operations}
\index{precedence}

A good way to debug expressions is to add parentheses to make
the order of evaluation explicit:

\begin{verbatim}
 y = x / (2 * math.pi)
\end{verbatim}
%
Whenever you are not sure of the order of evaluation, use
parentheses.  Not only will the program be correct (in the sense
of doing what you intended), it will also be more readable for
other people who haven't memorized the order of operations.


\subsection{I've got a function that doesn't return what I
expect.}
\index{return statement}
\index{statement!return}

If you have a {\tt return} statement with a complex expression,
you don't have a chance to print the result before
returning.  Again, you can use a temporary variable.  For
example, instead of:

\begin{verbatim}
return self.hands[i].removeMatches()
\end{verbatim}
%
you could write:

\begin{verbatim}
count = self.hands[i].removeMatches()
return count
\end{verbatim}
%
Now you have the opportunity to display the value of
{\tt count} before returning.


\subsection{I'm really, really stuck and I need help.}

First, try getting away from the computer for a few minutes.
Computers emit waves that affect the brain, causing these
symptoms:

\begin{itemize}

\item Frustration and rage.
\index{frustration}
\index{rage}
\index{debugging!emotional response}
\index{emotional debugging}

\item Superstitious beliefs (``the computer hates me'') and
magical thinking (``the program only works when I wear my
hat backward'').
\index{debugging!superstition}
\index{superstitious debugging}

\item Random walk programming (the attempt to program by writing
every possible program and choosing the one that does the right
thing).
\index{random walk programming}
\index{development plan!random walk programming}

\end{itemize}

If you find yourself suffering from any of these symptoms, get
up and go for a walk.  When you are calm, think about the program.
What is it doing?  What are some possible causes of that
behavior?  When was the last time you had a working program,
and what did you do next?

Sometimes it just takes time to find a bug.  I often find bugs
when I am away from the computer and let my mind wander.  Some
of the best places to find bugs are trains, showers, and in bed,
just before you fall asleep.


\subsection{No, I really need help.}

It happens.  Even the best programmers occasionally get stuck.
Sometimes you work on a program so long that you can't see the
error.  You need a fresh pair of eyes.

Before you bring someone else in, make sure you are prepared.
Your program should be as simple
as possible, and you should be working on the smallest input
that causes the error.  You should have {\tt print} statements in the
appropriate places (and the output they produce should be
comprehensible).  You should understand the problem well enough
to describe it concisely.

When you bring someone in to help, be sure to give
them the information they need:

\begin{itemize}

\item If there is an error message, what is it
and what part of the program does it indicate?

\item What was the last thing you did before this error occurred?
What were the last lines of code that you wrote, or what is
the new test case that fails?

\item What have you tried so far, and what have you learned?

\end{itemize}

When you find the bug, take a second to think about what you
could have done to find it faster.  Next time you see something
similar, you will be able to find the bug more quickly.

Remember, the goal is not just to make the program
work.  The goal is to learn how to make the program work.


\chapter{Analysis of Algorithms}
\label{algorithms}

\begin{quote}
This appendix is an edited excerpt from {\it Think Complexity}, by
Allen B. Downey, also published by O'Reilly Media (2012).  When you
are done with this book, you might want to move on to that one.
\end{quote}

{\bf Analysis of algorithms} is a branch of computer science that
studies the performance of algorithms, especially their run time and
space requirements.  See
\url{http://en.wikipedia.org/wiki/Analysis_of_algorithms}.
\index{algorithm} \index{analysis of algorithms}

The practical goal of algorithm analysis is to predict the performance
of different algorithms in order to guide design decisions.

During the 2008 United States Presidential Campaign, candidate
Barack Obama was asked to perform an impromptu analysis when
he visited Google.  Chief executive Eric Schmidt jokingly asked him
for ``the most efficient way to sort a million 32-bit integers.''
Obama had apparently been tipped off, because he quickly
replied, ``I think the bubble sort would be the wrong way to go.''
See \url{http://www.youtube.com/watch?v=k4RRi_ntQc8}.
\index{Obama, Barack}
\index{Schmidt, Eric}
\index{bubble sort}

This is true: bubble sort is conceptually simple but slow for
large datasets.  The answer Schmidt was probably looking for is
``radix sort'' (\url{http://en.wikipedia.org/wiki/Radix_sort})\footnote{
But if you get a question like this in an interview, I think
a better answer is, ``The fastest way to sort a million integers
is to use whatever sort function is provided by the language
I'm using.  Its performance is good enough for the vast majority
of applications, but if it turned out that my application was too
slow, I would use a profiler to see where the time was being
spent.  If it looked like a faster sort algorithm would have
a significant effect on performance, then I would look
around for a good implementation of radix sort.''}.
\index{radix sort}

The goal of algorithm analysis is to make meaningful
comparisons between algorithms, but there are some problems:
\index{comparing algorithms}

\begin{itemize}

\item The relative performance of the algorithms might
depend on characteristics of the hardware, so one algorithm
might be faster on Machine A, another on Machine B.
The general solution to this problem is to specify a
{\bf machine model} and analyze the number of steps, or
operations, an algorithm requires under a given model.
\index{machine model}

\item Relative performance might depend on the details of
the dataset.  For example, some sorting
algorithms run faster if the data are already partially sorted;
other algorithms run slower in this case.
A common way to avoid this problem is to analyze the
{\bf worst case} scenario.  It is sometimes useful to
analyze average case performance, but that's usually harder,
and it might not be obvious what set of cases to average over.
\index{worst case}
\index{average case}

\item Relative performance also depends on the size of the
problem.  A sorting algorithm that is fast for small lists
might be slow for long lists.
The usual solution to this problem is to express run time
(or number of operations) as a function of problem size,
and group functions into categories depending on how quickly
they grow as problem size increases.

\end{itemize}

The good thing about this kind of comparison is that it lends
itself to simple classification of algorithms.  For example,
if I know that the run time of Algorithm A tends to be
proportional to the size of the input, $n$, and Algorithm B
tends to be proportional to $n^2$, then I
expect A to be faster than B, at least for large values of $n$.

This kind of analysis comes with some caveats, but we'll get
to that later.


\section{Order of growth}

Suppose you have analyzed two algorithms and expressed
their run times in terms of the size of the input:
Algorithm A takes $100n+1$ steps to solve a problem with
size $n$; Algorithm B takes $n^2 + n + 1$ steps.
\index{order of growth}

The following table shows the run time of these algorithms
for different problem sizes:

\begin{tabular}{|r|r|r|}
\hline
Input     &   Run time of     & Run time of \\
size      &   Algorithm A     & Algorithm B \\
\hline
10        &   1 001           & 111         \\
100       &   10 001          & 10 101         \\
1 000     &   100 001         & 1 001 001         \\
10 000    &   1 000 001       & $> 10^{10}$         \\
\hline
\end{tabular}

At $n=10$, Algorithm A looks pretty bad; it takes almost 10 times
longer than Algorithm B.  But for $n=100$ they are about the same, and
for larger values A is much better.

The fundamental reason is that for large values of $n$, any function
that contains an $n^2$ term will grow faster than a function whose
leading term is $n$.  The {\bf leading term} is the term with the
highest exponent.
\index{leading term}
\index{exponent}

For Algorithm A, the leading term has a large coefficient, 100, which
is why B does better than A for small $n$.  But regardless of the
coefficients, there will always be some value of $n$ where
$a n^2 > b n$, for any values of $a$ and $b$.
\index{leading coefficient}

The same argument applies to the non-leading terms.  Even if the run
time of Algorithm A were $n+1000000$, it would still be better than
Algorithm B for sufficiently large $n$.

In general, we expect an algorithm with a smaller leading term to be a
better algorithm for large problems, but for smaller problems, there
may be a {\bf crossover point} where another algorithm is better.  The
location of the crossover point depends on the details of the
algorithms, the inputs, and the hardware, so it is usually ignored for
purposes of algorithmic analysis.  But that doesn't mean you can forget
about it.
\index{crossover point}

If two algorithms have the same leading order term, it is hard to say
which is better; again, the answer depends on the details.  So for
algorithmic analysis, functions with the same leading term
are considered equivalent, even if they have different coefficients.

An {\bf order of growth} is a set of functions whose growth
behavior is considered equivalent.  For example, $2n$, $100n$ and $n+1$ 
belong to the same order of growth, which is written $O(n)$ in
{\bf Big-Oh notation} and often called {\bf linear} because every function
in the set grows linearly with $n$.
\index{big-oh notation}
\index{linear growth}

All functions with the leading term $n^2$ belong to $O(n^2)$; they are
called {\bf quadratic}.
\index{quadratic growth}

The following table shows some of the orders of growth that
appear most commonly in algorithmic analysis,
in increasing order of badness.
\index{badness}

\begin{tabular}{|r|r|r|}
\hline
Order of     &   Name      \\
growth       &               \\
\hline
$O(1)$             & constant \\
$O(\log_b n)$      & logarithmic (for any $b$) \\
$O(n)$             & linear \\
$O(n \log_b n)$    & linearithmic \\
$O(n^2)$           & quadratic     \\
$O(n^3)$           & cubic     \\
$O(c^n)$           & exponential (for any $c$)    \\
\hline
\end{tabular}

For the logarithmic terms, the base of the logarithm doesn't matter;
changing bases is the equivalent of multiplying by a constant, which
doesn't change the order of growth.  Similarly, all exponential
functions belong to the same order of growth regardless of the base of
the exponent.
Exponential functions grow very quickly, so exponential algorithms are
only useful for small problems.
\index{logarithmic growth}
\index{exponential growth}


\begin{exercise}

Read the Wikipedia page on Big-Oh notation at
\url{http://en.wikipedia.org/wiki/Big_O_notation} and
answer the following questions:

\begin{enumerate}
\item What is the order of growth of $n^3 + n^2$?
What about $1000000 n^3 + n^2$?
What about $n^3 + 1000000 n^2$?

\item What is the order of growth of $(n^2 + n) \cdot (n + 1)$?  Before
  you start multiplying, remember that you only need the leading term.

\item If $f$ is in $O(g)$, for some unspecified function $g$, what can
  we say about $af+b$?

\item If $f_1$ and $f_2$ are in $O(g)$, what can we say about $f_1 + f_2$?

\item If  $f_1$ is in $O(g)$
and $f_2$ is in $O(h)$,
what can we say about  $f_1 + f_2$?

\item If  $f_1$ is in $O(g)$ and $f_2$ is $O(h)$,
what can we say about  $f_1 \cdot f_2$?
\end{enumerate}

\end{exercise}

Programmers who care about performance often find this kind of
analysis hard to swallow.  They have a point: sometimes the
coefficients and the non-leading terms make a real difference.
Sometimes the details of the hardware, the programming language, and
the characteristics of the input make a big difference.  And for small
problems asymptotic behavior is irrelevant.

But if you keep those caveats in mind, algorithmic analysis is a
useful tool.  At least for large problems, the ``better'' algorithm
is usually better, and sometimes it is {\em much} better.  The
difference between two algorithms with the same order of growth is
usually a constant factor, but the difference between a good algorithm
and a bad algorithm is unbounded!


\section{Analysis of basic Python operations}

In Python, most arithmetic operations are constant time;
multiplication usually takes longer than addition and subtraction, and
division takes even longer, but these run times don't depend on the
magnitude of the operands.  Very large integers are an exception; in
that case the run time increases with the number of digits.
\index{analysis of primitives}

Indexing operations---reading or writing elements in a sequence
or dictionary---are also constant time, regardless of the size
of the data structure.
\index{indexing}

A {\tt for} loop that traverses a sequence or dictionary is
usually linear, as long as all of the operations in the body
of the loop are constant time.  For example, adding up the
elements of a list is linear:

\begin{verbatim}
    total = 0
    for x in t:
        total += x
\end{verbatim}

The built-in function {\tt sum} is also linear because it does
the same thing, but it tends to be faster because it is a more
efficient implementation; in the language of algorithmic analysis,
it has a smaller leading coefficient.

As a rule of thumb, if the body of a loop is in $O(n^a)$ then
the whole loop is in $O(n^{a+1})$.  The exception is if you can
show that the loop exits after a constant number of iterations.
If a loop runs $k$ times regardless of $n$, then
the loop is in $O(n^a)$, even for large $k$.

Multiplying by $k$ doesn't change the order of growth, but neither
does dividing.  So if the body of a loop is in $O(n^a)$ and it runs
$n/k$ times, the loop is in $O(n^{a+1})$, even for large $k$.

Most string and tuple operations are linear, except indexing and {\tt
  len}, which are constant time.  The built-in functions {\tt min} and
{\tt max} are linear.  The run-time of a slice operation is
proportional to the length of the output, but independent of the size
of the input.
\index{string methods}
\index{tuple methods}

String concatenation is linear; the run time depends on the sum
of the lengths of the operands.
\index{string concatenation}

All string methods are linear, but if the lengths of
the strings are bounded by a constant---for example, operations on single
characters---they are considered constant time.
The string method {\tt join} is linear; the run time depends on
the total length of the strings.
\index{join@{\tt join}}

Most list methods are linear, but there are some exceptions:
\index{list methods}

\begin{itemize}

\item Adding an element to the end of a list is constant time on
average; when it runs out of room it occasionally gets copied
to a bigger location, but the total time for $n$ operations
is $O(n)$, so the average time for each
operation is $O(1)$.

\item Removing an element from the end of a list is constant time.

\item Sorting is $O(n \log n)$.
\index{sorting}

\end{itemize}

Most dictionary operations and methods are constant time, but
there are some exceptions:
\index{dictionary methods}

\begin{itemize}

\item The run time of {\tt update} is
  proportional to the size of the dictionary passed as a parameter,
  not the dictionary being updated.

\item {\tt keys}, {\tt values} and {\tt items} are constant time because 
  they return iterators.  But
  if you loop through the iterators, the loop will be linear.
\index{iterator}

\end{itemize}

The performance of dictionaries is one of the minor miracles of
computer science.  We will see how they work in
Section~\ref{hashtable}.


\begin{exercise}

Read the Wikipedia page on sorting algorithms at
\url{http://en.wikipedia.org/wiki/Sorting_algorithm} and answer
the following questions:
\index{sorting}

\begin{enumerate}

\item What is a ``comparison sort?'' What is the best worst-case order
  of growth for a comparison sort?  What is the best worst-case order
  of growth for any sort algorithm?
\index{comparison sort}

\item What is the order of growth of bubble sort, and why does Barack
  Obama think it is ``the wrong way to go?''

\item What is the order of growth of radix sort?  What preconditions
  do we need to use it?

\item What is a stable sort and why might it matter in practice?
\index{stable sort}

\item What is the worst sorting algorithm (that has a name)?

\item What sort algorithm does the C library use?  What sort algorithm
  does Python use?  Are these algorithms stable?  You might have to
  Google around to find these answers.

\item Many of the non-comparison sorts are linear, so why does does
  Python use an $O(n \log n)$ comparison sort?

\end{enumerate}

\end{exercise}


\section{Analysis of search algorithms}

A {\bf search} is an algorithm that takes a collection and a target
item and determines whether the target is in the collection, often
returning the index of the target.
\index{search}

The simplest search algorithm is a ``linear search'', which traverses
the items of the collection in order, stopping if it finds the target.
In the worst case it has to traverse the entire collection, so the run
time is linear.
\index{linear search}

The {\tt in} operator for sequences uses a linear search; so do string
methods like {\tt find} and {\tt count}.
\index{in@{\tt in} operator}

If the elements of the sequence are in order, you can use a {\bf
  bisection search}, which is $O(\log n)$.  Bisection search is
similar to the algorithm you might use to look a word up in a
dictionary (a paper dictionary, not the data structure).  Instead of
starting at the beginning and checking each item in order, you start
with the item in the middle and check whether the word you are looking
for comes before or after.  If it comes before, then you search the
first half of the sequence.  Otherwise you search the second half.
Either way, you cut the number of remaining items in half.
\index{bisection search}

If the sequence has 1,000,000 items, it will take about 20 steps to
find the word or conclude that it's not there.  So that's about 50,000
times faster than a linear search.

Bisection search can be much faster than linear search, but
it requires the sequence to be in order, which might require
extra work.

There is another data structure, called a {\bf hashtable} that
is even faster---it can do a search in constant time---and it
doesn't require the items to be sorted.  Python dictionaries
are implemented using hashtables, which is why most dictionary
operations, including the {\tt in} operator, are constant time.


\section{Hashtables}
\label{hashtable}

To explain how hashtables work and why their performance is so
good, I start with a simple implementation of a map and
gradually improve it until it's a hashtable.
\index{hashtable}

I use Python to demonstrate these implementations, but in real
life you wouldn't write code like this in Python; you would just use a
dictionary!  So for the rest of this chapter, you have to imagine that
dictionaries don't exist and you want to implement a data structure
that maps from keys to values.  The operations you have to
implement are:

\begin{description}

\item[{\tt add(k, v)}:] Add a new item that maps from key {\tt k}
to value {\tt v}.  With a Python dictionary, {\tt d}, this operation
is written {\tt d[k] = v}.

\item[{\tt get(k)}:] Look up and return the value that corresponds
to key {\tt k}.  With a Python dictionary, {\tt d}, this operation
is written {\tt d[k]} or {\tt d.get(k)}.

\end{description}

For now, I assume that each key only appears once.
The simplest implementation of this interface uses a list of
tuples, where each tuple is a key-value pair.
\index{LinearMap@{\tt LinearMap}}

\begin{verbatim}
class LinearMap:

    def __init__(self):
        self.items = []

    def add(self, k, v):
        self.items.append((k, v))

    def get(self, k):
        for key, val in self.items:
            if key == k:
                return val
        raise KeyError
\end{verbatim}

{\tt add} appends a key-value tuple to the list of items, which
takes constant time.

{\tt get} uses a {\tt for} loop to search the list:
if it finds the target key it returns the corresponding value;
otherwise it raises a {\tt KeyError}.
So {\tt get} is linear.
\index{KeyError@{\tt KeyError}}

An alternative is to keep the list sorted by key.  Then {\tt get}
could use a bisection search, which is $O(\log n)$.  But inserting a
new item in the middle of a list is linear, so this might not be the
best option.  There are other data structures that can implement {\tt
  add} and {\tt get} in log time, but that's still not as good as
constant time, so let's move on.
\index{red-black tree}

One way to improve {\tt LinearMap} is to break the list of key-value
pairs into smaller lists.  Here's an implementation called
{\tt BetterMap}, which is a list of 100 LinearMaps.  As we'll see
in a second, the order of growth for {\tt get} is still linear,
but {\tt BetterMap} is a step on the path toward hashtables:
\index{BetterMap@{\tt BetterMap}}

\begin{verbatim}
class BetterMap:

    def __init__(self, n=100):
        self.maps = []
        for i in range(n):
            self.maps.append(LinearMap())

    def find_map(self, k):
        index = hash(k) % len(self.maps)
        return self.maps[index]

    def add(self, k, v):
        m = self.find_map(k)
        m.add(k, v)

    def get(self, k):
        m = self.find_map(k)
        return m.get(k)
\end{verbatim}

\verb"__init__" makes a list of {\tt n} {\tt LinearMap}s.

\verb"find_map" is used by
{\tt add} and {\tt get}
to figure out which map to put the
new item in, or which map to search.

\verb"find_map" uses the built-in function {\tt hash}, which takes
almost any Python object and returns an integer.  A limitation of this
implementation is that it only works with hashable keys.  Mutable
types like lists and dictionaries are unhashable.
\index{hash function}

Hashable objects that are considered equivalent return the same hash
value, but the converse is not necessarily true: two objects with
different values can return the same hash value.

\verb"find_map" uses the modulus operator to wrap the hash values
into the range from 0 to {\tt len(self.maps)}, so the result is a legal
index into the list.  Of course, this means that many different
hash values will wrap onto the same index.  But if the hash function
spreads things out pretty evenly (which is what hash functions
are designed to do), then we expect $n/100$ items per LinearMap.

Since the run time of {\tt LinearMap.get} is proportional to the
number of items, we expect BetterMap to be about 100 times faster
than LinearMap.  The order of growth is still linear, but the
leading coefficient is smaller.  That's nice, but still not
as good as a hashtable.

Here (finally) is the crucial idea that makes hashtables fast: if you
can keep the maximum length of the LinearMaps bounded, {\tt
  LinearMap.get} is constant time.  All you have to do is keep track
of the number of items and when the number of
items per LinearMap exceeds a threshold, resize the hashtable by
adding more LinearMaps.
\index{bounded}

Here is an implementation of a hashtable:
\index{HashMap}

\begin{verbatim}
class HashMap:

    def __init__(self):
        self.maps = BetterMap(2)
        self.num = 0

    def get(self, k):
        return self.maps.get(k)

    def add(self, k, v):
        if self.num == len(self.maps.maps):
            self.resize()

        self.maps.add(k, v)
        self.num += 1

    def resize(self):
        new_maps = BetterMap(self.num * 2)

        for m in self.maps.maps:
            for k, v in m.items:
                new_maps.add(k, v)

        self.maps = new_maps
\end{verbatim}

Each {\tt HashMap} contains a {\tt BetterMap}; \verb"__init__" starts
with just 2 LinearMaps and initializes {\tt num}, which keeps track of
the number of items.

{\tt get} just dispatches to {\tt BetterMap}.  The real work happens
in {\tt add}, which checks the number of items and the size of the
{\tt BetterMap}: if they are equal, the average number of items per
LinearMap is 1, so it calls {\tt resize}.

{\tt resize} make a new {\tt BetterMap}, twice as big as the previous
one, and then ``rehashes'' the items from the old map to the new.

Rehashing is necessary because changing the number of LinearMaps
changes the denominator of the modulus operator in
\verb"find_map".  That means that some objects that used
to hash into the same LinearMap will get split up (which is
what we wanted, right?).
\index{rehashing}

Rehashing is linear, so
{\tt resize} is linear, which might seem bad, since I promised
that {\tt add} would be constant time.  But remember that
we don't have to resize every time, so {\tt add} is usually
constant time and only occasionally linear.  The total amount
of work to run {\tt add} $n$ times is proportional to $n$,
so the average time of each {\tt add} is constant time!
\index{constant time}

To see how this works, think about starting with an empty
HashTable and adding a sequence of items.  We start with 2 LinearMaps,
so the first 2 adds are fast (no resizing required).  Let's
say that they take one unit of work each.  The next add
requires a resize, so we have to rehash the first two
items (let's call that 2 more units of work) and then
add the third item (one more unit).  Adding the next item
costs 1 unit, so the total so far is
6 units of work for 4 items.

The next {\tt add} costs 5 units, but the next three
are only one unit each, so the total is 14 units for the
first 8 adds.

The next {\tt add} costs 9 units, but then we can add 7 more
before the next resize, so the total is 30 units for the
first 16 adds.

After 32 adds, the total cost is 62 units, and I hope you are starting
to see a pattern.  After $n$ adds, where $n$ is a power of two, the
total cost is $2n-2$ units, so the average work per add is
a little less than 2 units.  When $n$ is a power of two, that's
the best case; for other values of $n$ the average work is a little
higher, but that's not important.  The important thing is that it
is $O(1)$.
\index{average cost}

Figure~\ref{fig.hash} shows how this works graphically.  Each
block represents a unit of work.  The columns show the total
work for each add in order from left to right: the first two
{\tt adds} cost 1 units, the third costs 3 units, etc.

\begin{figure}
\centerline{\includegraphics[width=5.5in]{figs/towers.pdf}}
\caption{The cost of a hashtable add.\label{fig.hash}}
\end{figure}

The extra work of rehashing appears as a sequence of increasingly
tall towers with increasing space between them.  Now if you knock
over the towers, spreading the cost of resizing over all
adds, you can see graphically that the total cost after $n$
adds is $2n - 2$.

An important feature of this algorithm is that when we resize the
HashTable it grows geometrically; that is, we multiply the size by a
constant.  If you increase the size
arithmetically---adding a fixed number each time---the average time
per {\tt add} is linear.
\index{geometric resizing}

You can download my implementation of HashMap from
\url{http://thinkpython2.com/code/Map.py}, but remember that there
is no reason to use it; if you want a map, just use a Python dictionary.

\section{용어 해설}
%Glossary

\begin{description}

\item[analysis of algorithms:] A way to compare algorithms in terms of
their run time and/or space requirements.
\index{analysis of algorithms}

\item[machine model:] A simplified representation of a computer used
to describe algorithms.
\index{machine model}

\item[worst case:] The input that makes a given algorithm run slowest (or
require the most space.
\index{worst case}

\item[leading term:] In a polynomial, the term with the highest exponent.
\index{leading term}

\item[crossover point:] The problem size where two algorithms require
the same run time or space. 
\index{crossover point}

\item[order of growth:] A set of functions that all grow in a way
considered equivalent for purposes of analysis of algorithms. 
For example, all functions that grow linearly belong to the same
order of growth.
\index{order of growth}

\item[Big-Oh notation:] Notation for representing an order of growth;
for example, $O(n)$ represents the set of functions that grow
linearly. 
\index{Big-Oh notation}

\item[linear:] An algorithm whose run time is proportional to
problem size, at least for large problem sizes.
\index{linear}

\item[quadratic:] An algorithm whose run time is proportional to
$n^2$, where $n$ is a measure of problem size.
\index{quadratic}

\item[search:] The problem of locating an element of a collection
(like a list or dictionary) or determining that it is not present.
\index{search}

\item[hashtable:] A data structure that represents a collection of
key-value pairs and performs search in constant time.
\index{hashtable}

\end{description}


\printindex

\clearemptydoublepage
%\blankpage
%\blankpage
%\blankpage


\end{document}
